\documentclass[12pt,a4paper]{article}
\AddToHook{cmd/section/before}{\clearpage}

\usepackage[a4paper, total={6.5in, 10in}]{geometry}
\usepackage[utf8]{inputenc}
\usepackage{amsfonts, amssymb, amsmath}
\usepackage{amsthm}
\usepackage[dvipsnames]{xcolor}
\usepackage{titlesec}
\usepackage{afterpage}
\usepackage{hyperref}
\usepackage{esint}
\usepackage{diffcoeff}

\pagecolor{white}
\color{black}% set the default colour to white

\theoremstyle{definition}
\newtheorem{definition}{Definition}[subsection]
\newtheorem{theorem}[definition]{Theorem}
\newtheorem{proposition}[definition]{Proposition}
\newtheorem{corollary}[definition]{Corollary}
\newtheorem{lemma}[definition]{Lemma}
\newtheorem{example}[definition]{Example}
\newtheorem*{remark}{Remark}

\title{Complex Analysis II Course Notes}
\author{Isaac Holt}

\begin{document}

\maketitle
\tableofcontents
\newpage

\section{The complex plane and Riemann sphere}

\subsection{Complex numbers}

\begin{definition}
	A \textbf{complex number} $z$ is a number $z = x + iy$ where $x, y \in \mathbb{R}^2$ and $i$ is the \textbf{imaginary unit}. The set of all complex numbers is written as $\mathbb{C}$.
\end{definition}

\begin{definition}
	For $z_1 = x_1 + i y_1$ and $z_2 = x_2 + i y_2$, addition, subtraction and multiplication of complex numbers is defined as
	\[
		\begin{aligned}
			z_1 \pm z_2 & := (x_1 \pm x_2) + i (y_1 \pm y_2) \\
			z_1 z_2 & := (x_1 x_2 - y_1 y_2) + i (x_1 y_2 + x_2 y_1)
		\end{aligned}
	\]
\end{definition}

\begin{definition}
	For a complex number $z = x + iy$, the \textbf{real part} of $z$, $\text{Re}(z)$, is $x$ and the \textbf{imaginary part}, $\text{Im}(z)$, is $y$.
\end{definition}

\begin{definition}
	For a complex number $z = x + iy$, the \textbf{complex conjugate} of $z$, $\overline{z}$ is defined as $\overline{z} = x - iy$.
\end{definition}

\begin{definition}
	Division of complex numbers $z_1 = x_1 + i y_1$ and $z_2 = x_2 + i y_2 \ne 0$ is given by
	\[
		\frac{z_1}{z_2} = \frac{x_1 + i y_1}{x_2 + i y_2} = \frac{(x_1 + i y_1) (x_2 - i y_2)}{(x_2 + i y_2) (x_2 - i y_2)} = \frac{x_1 x_2 + y_1 y_2}{x_2^2 + y_2^2} + i \frac{x_2 y_1 - x_1 y_2}{x_2^2 + y_2^2}
	\]
	This gives a multiplicative inverse for every $z = x + iy \ne 0$:
	\[
		z^{-1} = \frac{1}{z} = \frac{x}{x^2 + y_2} - i \frac{y}{x^2 + y^2}
	\]
\end{definition}

\begin{definition}
	The \textbf{modulus} or \textbf{absolute value} of a complex number $z = x + iy$, $|z|$, is defined as $|z| = \sqrt{x^2 + y^2}$.
\end{definition}

\begin{lemma}
	\hfill
	\begin{enumerate}
		\item $\forall z_1, z_2 \in \mathbb{C}^2, z_1 z_2 \Longleftrightarrow z_1 = 0 \text{ or } z_2 = 0$.
		\item $\forall z \in \mathbb{C}, |z| = \sqrt{z \overline{z}}$.
		\item $\text{Re}(z) = \frac{z + \overline{z}}{2}$ and $\text{Im}(z) = \frac{z - \overline{z}}{2i}$.
		\item $z^{-1} = \frac{\overline{z}}{|z|^2}$.
	\end{enumerate}
\end{lemma}

\begin{proof}
	Easy.
\end{proof}

\begin{definition}
	For a complex number $z = x + iy$ plotted on an Argand diagram (where $z$ is at the point $(x, y)$), the \textbf{argument} of $z$, $\arg(z)$, is the anticlockwise angle $\theta$ from the real axis to $z$.
\end{definition}

\begin{definition}
	For a complex number $z$ with $|z| = r$ and $\arg(z) = \theta$, $z$ can be written in polar coordinates:
	\[
		z = r(\cos(\theta) + i \sin(\theta)) = r e^{i \theta}
	\]
\end{definition}

\begin{definition}
	$\arg(z)$ is only defined up to multiples of $2 \pi$. The \textbf{principal value} of $\arg(z)$ is the value of $\arg(z)$ in the interval $\left(-\pi, \pi \right]$, written as $\text{Arg}(z)$.
\end{definition}

\begin{lemma}
	\hfill
	\begin{enumerate}
		\item $\arg(z_1 z_2) = \arg(z_1) + \arg(z_2) \pmod{2 \pi}$.
		\item $\arg(1 / z) = -\arg(z) \pmod{2 \pi}$.
		\item $\arg(\overline{z}) = -\arg(z) \pmod{2 \pi}$.
	\end{enumerate}
\end{lemma}

\begin{proof}
	Easy.
\end{proof}

\begin{lemma}
	Multiplication in $\mathbb{C}$ can be geometrically described as a dilated rotation: if $z_1 = r_1 e^{i \theta_1}$ and $z_2 = r_2 e^{i \theta_2}$ then
	\[
		z_1 z_2 = r_1 r_2 e^{i (\theta_1 + \theta_2)}
	\]
\end{lemma}

\begin{proof}
	\[
		\begin{aligned}
			z_1 z_2
				& = r_1 r_2 (\cos(\theta_1) + i \sin(\theta_1)) (\cos(\theta_2) + i \sin(\theta_2)) \\
				& = r_1 r_2 ((\cos(\theta_1) \cos(\theta_2) - \sin(\theta_1) \sin(\theta_2)) + i (\cos(\theta_2) \sin(\theta_1) + \sin(\theta_2) \cos(\theta_1))) \\
				& = r_1 r_2 (\cos(\theta_1 + \theta_2) + i \sin(\theta_1 + \theta_2)) \\
				& = r_1 r_2 e^{i (\theta_1 + \theta_2)}
		\end{aligned}
	\]
\end{proof}

\begin{remark}
	\hfill
	\begin{itemize}
		\item Multiplying $z_1$ by $z_2$ represents a rotation of $z_1$ by $\theta_2$ anticlockwise, followed by a dilation of factor $r_2$.
		\item Addition represents to translation.
		\item Complex represents reflection in the real axis.
		\item Taking the real part represents projection onto the real axis.
		\item Taking the imaginary part represents projection onto the imaginary axis.
	\end{itemize}
\end{remark}

\begin{corollary}
	\hfill
	\begin{enumerate}
		\item $\forall z_1, z_2 \in \mathbb{C}^2, |z_1 z_2| = |z_1| |z_2|$.
		\item ${(\cos(\theta) + i \sin(\theta))}^n = \cos(n \theta) + i \sin(n \theta)$ (de Moivre's formula).
		\item $\forall z_1, z_2, |z_1 + z_2| \le |z_1| + |z_2|$.
		\item $\forall z \in \mathbb{C}, |z| \ge 0$ and $|z| = 0 \Longleftrightarrow z = 0$.
		\item $\max \{ |\text{Re}(z)|, |\text{Im}(z)| \} \le |z| \le |\text{Re}(z)| + |\text{Im}(z)|$.
	\end{enumerate}
\end{corollary}

\begin{proof}
	Easy.
\end{proof}

\begin{definition}
	The \textbf{upper half} of the complex plane, $\mathbb{H}$, is defined as
	\[
		\mathbb{H} = \{ z \in \mathbb{C}: \text{Im}(z) > 0 \}
	\]
\end{definition}

\subsection{Exponential and trigonometric functions}

\begin{definition}
	The \textbf{complex exponential function} $\exp: \mathbb{C} \rightarrow \mathbb{C}$, written as $e^z$, is defined as
	\[
		e^z = \exp(z) := e^x (\cos(y) + i \sin(y))
	\]
\end{definition}

\begin{proposition}\label{prop:expProperties}
	\hfill
	\begin{enumerate}
		\item $\forall z \in \mathbb{C}, e^z \ne 0$.
		\item $\forall z_1, z_2 \in \mathbb{C}^2, e^{z_1 + z_2} = e^{z_1} e^{z_2}$.
		\item $e^z = 1 \Longleftrightarrow \exists k \in \mathbb{Z}, z = 2 \pi i k$.
		\item $e^{-z} = 1 \ e^z$.
		\item $|e^z| = e^{\text{Re}(z)}$.
	\end{enumerate}
\end{proposition}

\begin{proof}
	Easy. For 3, $\exp(z) = 1 \Longleftrightarrow e^x \cos(y) = 1$ and $e^x \sin(y) = 0$. $e^x > 0$ so $\sin(y) = 0$ so $y = n \pi$ for some $n \in \mathbb{Z}$. So $1 = e^x \cos(n \pi) = e^x {(-1)}^n$ so $n$ is even and $x = 0$.
\end{proof}

\begin{corollary}\label{cor:EulersFormula}
	$\exp(2 \pi i) = 1$ and $\exp(\pi i) = -1$ (Euler's formula).
\end{corollary}

\begin{corollary}
	$\forall k \in \mathbb{Z}, \forall z \in \mathbb{C}, \exp(z + 2k \pi i) = \exp(z)$.
\end{corollary}

\begin{definition}
	The following functions from $\mathbb{C}$ to $\mathbb{C}$ are defined:
	\[
		\begin{aligned}
			\sin(z) & := \frac{1}{2i} \left( e^{iz} - e^{-iz} \right) \\
			\cos(z) & := \frac{1}{2} \left( e^{iz} + e^{-iz} \right) \\
			\sinh(z) & := \frac{1}{2} \left( e^{z} - e^{-z} \right) = -i \sin(iz) \\
			\cosh(z) & := \frac{1}{2} \left( e^{z} + e^{-z} \right) = \cos(iz)
		\end{aligned}
	\]
\end{definition}

\begin{remark}
	The usual trigonometric function identities hold, e.g. $\cos(z)^2 + \sin(z)^2 = 1$.
\end{remark}

\subsection{Logarithms and complex powers}

\begin{definition}
	We write the \textbf{set of non-zero complex numbers} as
	\[
		\mathbb{C}^* = \mathbb{C} - \{ 0 \}
	\]
\end{definition}

\begin{lemma}
	For every $w \in \mathbb{C}^*$, $e^z = w$ has a solution for $z$. Let $w = |w| e^{i \theta}$, $\theta = \text{Arg}(w)$. Then every solution for $z$ is given by
	\[
		z = \log(|w|) + i (\theta + 2 \pi k), \quad k \in \mathbb{Z}
	\]
\end{lemma}

\begin{proof}
	By Proposition~\ref{prop:expProperties} (part 2) and Corollary~\ref{cor:EulersFormula},
	\[
		w = |w| e^{i \theta} = e^{\log(|w|)} e^{i \theta} = e^{\log(|w|)} e^{i (\theta + 2 \pi k)} = e^{\log(|w|) + i (\theta + 2 \pi k)} = e^z
	\]
	Let $z = x + iy$, then
	\[
		e^z = e^x e^{iy} = w = |w| e^{i \theta} \Longrightarrow |e^z| = e^x = |w| \Longrightarrow x = \log(|w|)
	\]
	Hence $e^{iy} = e^{i \theta}$ so $e^{i(y - \theta)} = 1$ so $y - \theta = 2 \pi k$ for some $k \in \mathbb{Z}$ by Proposition~\ref{prop:expProperties} (part 3).
\end{proof}

\begin{definition}
	Let $\theta_1 < \theta_2$ with $\theta_2 - \theta_1 = 2 \pi$. Let $\arg$ be the argument function with values in $(\theta_1, \theta_2]$. Then
	\[
		\log(z) := \log(|z|) + i \arg(z)
	\]
	is called a \textbf{branch of logarithm}. It has a jump discontinuity on the ray $R_{\theta_1} = R_{\theta_2}$. This ray is called a \textbf{branch cut}.
\end{definition}

\begin{definition}
	Choosing $\theta_1 = -\pi$ and $\theta_2 = \pi$, so that $\arg = \text{Arg}$, gives the \textbf{principal branch of logarithm}
	\[
		\text{Log}(z) := \log(|z|) + i \text{Arg}(z)
	\]
	which has a jump discontinuity on the ray $\mathbb{R}_{\le 0}$ (the non-positive real axis).
\end{definition}

\begin{remark}
	The prinical branch, $\text{Log}$, matches the definition of $\log$ for real numbers, so is the branch that should be used, unless otherwise stated.
\end{remark}

\begin{lemma}
	\hfill
	\begin{enumerate}
		\item $\forall z \in \mathbb{C}^*, e^{\log(z)} = z$.
		\item Generally, $\log(zw) \ne \log(z) + \log(w)$.
		\item Generally, $\log(e^z) \ne z$.
	\end{enumerate}
\end{lemma}

\begin{definition}
	For a fixed $w \in \mathbb{C}^*$, we can choose any branch of log to define a \textbf{complex power function} by
	\[
		z^w = \exp(w \log(z))
	\]
\end{definition}

\begin{remark}
	The complex power function depends on the branch of log we choose.
\end{remark}

\subsection{The Riemann sphere and extended complex plane}

\begin{definition}
	The \textbf{unit sphere} $S^2$ is defined as
	\[
		S^2 := \{ (x, y, s) \in \mathbb{R}^3: x^2 + y^2 + s^2 = 1 \}
	\]
\end{definition}

\begin{definition}
	We define $N = (0, 0, 1) \in S^2$ to be the \textbf{north pole}. For every point $v \in S^2 - \{ N \}$, there is a unique straight line $L_{N, v}$ passing through $N$ and $v$. This is not parallel to the $(x, y)$-plane as $v \ne N$, hence $L_{N, v}$ intersects the $(x, y)$-plane at a unique point $(x, y, 0)$ which corresponds to the point $x + iy \in \mathbb{C}$.
\end{definition}

\begin{definition}
	The \textbf{stereographic projection} map $P: S^2 - \{ N \} \rightarrow \mathbb{C}$ is defined as
	\[
		P(v) = x + iy
	\]
	where $x + iy$ is the complex number corresponding to the point $(x, y, 0)$ where the line passing through $N$ and $v$, $L_{N, v}$ intersects the $(x, y)$-plane.
	
	The equation $L_{N, v}$ is
	\[
		\gamma(t) = N + ((x, y, s) - N) t = (0, 0, 1) + (x, y, s - 1) t
	\]
	which intersects the $(x, y)$-plane at $t = 1 / (1 - s)$. So
	\[
		P(x, y, s) = \frac{x}{1 - s} + i \frac{y}{1 - s}
	\]
\end{definition}

\begin{definition}
	The \textbf{inverse stereographic projection}, the inverse of $P$, is
	\[
		P^{-1}(z) = \frac{1}{1 + |z|^2} (2 \text{Re}(z), 2 \text{Im} (z), |z|^2 - 1)
	\]
\end{definition}

\begin{remark}
	$P$ is a bijection as it has an inverse.
\end{remark}

\begin{definition}
	The \textbf{extended complex plane} is defined as
	\[
		\hat{\mathbb{C}} = \mathbb{C} \cup \{ \infty \}
	\]
\end{definition}

\begin{remark}
	$N$ corresponds to $\infty \in \hat{\mathbb{C}}$ under the stereographic projection. So $\hat{\mathbb{C}}$ can be thought of as the entire sphere $S^2$.
\end{remark}

\begin{remark}
	The south pole, $(0, 0, -1)$ could also be used to define a different, valid projection.
\end{remark}

\begin{proposition}
	The following are correspondences between $S^2$ and $\hat{\mathbb{C}}$:
	\[
		\begin{aligned}
			N & \longleftrightarrow \infty \\
			S & \longleftrightarrow 0 \\
			\text{equator} & \longleftrightarrow \text{unit circle: } \{ z \in \mathbb{C}: |z| = 1 \} \\
			\text{open Southern hemisphere} & \longleftrightarrow \text{unit disc: } \{ z \in \mathbb{C}: |z| < 1 \} \\
			\text{open Northern hemisphere} & \longleftrightarrow \hat{\mathbb{C}} - \overline{B_1}(0) = \hat{\mathbb{C}} - \{ z \in \mathbb{C}: |z| \le 1 \} \\
		\end{aligned}
	\]
\end{proposition}

\section{Metric Spaces}

\subsection{Metric spaces}

\begin{definition}
	A \textbf{metric space} is a set $X$ together with a function $d: X \times X \rightarrow \mathbb{R}_{\ge 0}$ that satisfies, for every $x, y, z \in X^3$,
	\begin{enumerate}
		\item \textbf{(D1) Positivity}: $d(x, y) \ge 0$ and $d(x, y) = 0 \Longleftrightarrow x = y$.
		\item \textbf{(D2) Symmetry}: $d(x, y) = d(y, x)$.
		\item \textbf{(D3) Triangle inequality}: $d(x, z) \le d(x, y) + d(y, z)$.
	\end{enumerate}
	$d$ is called a \textbf{metric}. A metric space is written as $(X, d)$.
\end{definition}

\begin{example}
	Let $X = \mathbb{C}$ and $d(x, y) = |x - y|$. Then $(X, d)$ is a metric space.
\end{example}

\begin{example}
	Let $X = \mathbb{C}^n$ and
	\[
		d(\underline{x}, \underline{y}) = ||x - y||_2 = \sqrt{\sum_{i = 1}^n |x_i - y_i|^2}
	\]
	Then $(X, d)$ is a metric space.
\end{example}

\begin{example}
	Let $V$ be a finite dimensional vector space with an inner product $\langle \cdot, \cdot \rangle$, then
	\[
		d(\underline{x}, \underline{y}) = ||x - y|| = \sqrt{\langle x - y, x - y \rangle}
	\]
	is a metric.
\end{example}

\begin{definition}
	Let $V$ be a real or complex vector space. A function $||\cdot||: V \rightarrow \mathbb{R}_{\ge 0}$ is called a \textbf{norm} if it satisfies, for every $v, w \in V^2$ and $\lambda \in \mathbb{C}$ or $\mathbb{R}$:
	\begin{enumerate}
		\item \textbf{(N1) Positivity}: $||v|| \ge 0$ and $||v|| = 0 \Longleftrightarrow v = 0$.
		\item \textbf{(N2) Linearity in scalar multiplication}: $||\lambda v|| = |\lambda| ||v||$.
		\item \textbf{(N3) Triangle inequality}: $||v + w|| \le ||v|| + ||w||$.
	\end{enumerate}
\end{definition}

\begin{definition}
	Property N3 implies the \textbf{reverse triangle inequality}:
	\[
		||v - w|| \ge | \ ||v|| - ||w|| \ |
	\]
\end{definition}

\begin{definition}
	A vector space equipped with a norm is called a \textbf{normed vector space}.
\end{definition}

\begin{remark}
	A normed vector space together with $d(v, w) = ||v - w||$ is always a metric space.
\end{remark}

\begin{example}
	For every $p \ge 1$, the \textbf{$l_p$-norm} is defined on vectors in $\mathbb{C}^n$ by
	\[
		||\underline{x}||_p := \sqrt[p]{\sum_{i = 1}^{n} |x_i|^p}
	\]
	The \textbf{Taxicab norm} is $l_p$ norm when $p = 1$.
\end{example}

\begin{example}
	The \textbf{$l_{\infty}$-norm} (or \textbf{sup-norm}) is defined as
	\[
		||\underline{x}||_{\infty} := \max_{i = 1, \dots, n} |x_i|
	\]
\end{example}

\begin{example}
	The \textbf{Riemannian metric} (or \textbf{chordal metric}) is defined as
	\[
		d(z, w) := ||f(z) - f(w)||_2
	\]
	where $z, w \in \hat{\mathbb{C}}$ and $f: \hat{\mathbb{C}} \rightarrow S^2$ is the inverse stereographic projection.
\end{example}

\begin{definition}
	Let $X$ be a non-empty finite set. The \textbf{discrete metric} on $X$ is defined as
	\[
		d(x, y) := \begin{cases}
			0 & \text{if } x = y \\
			1 & \text{if } x \ne y
		\end{cases}
	\]
	$(X, d)$ is called a \textbf{discrete metric space}.
\end{definition}

\begin{example}
	Let $X = C([a, b])$ be the space of continuous functions on $[a, b]$. Then
	\[
		||f|| := \max_{x \in [a, b]} |f(x)|
	\]
	defines a norm, so is a metric.
\end{example}

\begin{example}
	Let $(X, d)$ be a metric space. Every non-empty subset $Y \subset X$ also forms a metric space with $(Y, d)$. The metric restricted to $Y$ is called the \textbf{subspace metric}.
\end{example}

\subsection{Open and closed sets}

\begin{definition}
	Let $(X, d)$ be a metric space, $x \in X$ and $r > 0$ be a real number. The \textbf{open ball $B_r(x)$ of radius $r$ centred at $x$} is defined as
	\[
		B_r(x) := \{y \in X: d(x, y) < r\}
	\]
\end{definition}

\begin{definition}
	Let $(X, d)$ be a metric space, $x \in X$ and $r > 0$ be a real number. The \textbf{closed ball $B_r(x)$ of radius $r$ centred at $x$} is defined as
	\[
		\overline{B_r}(x) := \{y \in X: d(x, y) \le r\}
	\]
\end{definition}

\begin{example}
	Let $X = \mathbb{C}$ and $d(z, w) = |z - w|$. Then $B_1(0) = \mathbb{D} = \{ z: |z| < 1 \}$, the unit disc.
\end{example}

\begin{example}
	Let $X = \mathbb{R}^2$. For $l_2$-norm, the unit ball $B_1(\underline{0})$ is the inside of the unit circle centred at the origin.

	For the $l_{\infty}$-norm, $B_1(\underline{0})$ is the inside of the square with vertices $(1, 1), (-1, 1), (-1, -1), (1, -1)$, since $\max{ |x|, |y| } < 1$ in this ball.

	For the $l_1$-norm, in $B_1(0)$, we have $|x| + |y| < 1$ so in the 1st quadrant, $y < 1 - x$, in the 2nd quadrant, $y < 1 + $, in the 3rd quadrant, $y > -1 - x$, and in the 4th quadrant, $y > -1 + x$. So the unit ball is the inside of the diamond with vertices $(1, 0), (0, 1), (-1, 0), (0, -1)$.
\end{example}

\begin{definition}
	Let $(X, d)$ be a metric space. $U \subset X$ is called \textbf{open} (in $X$) if for every $x \in U$, for some $\epsilon > 0$, $B_{\epsilon}(x) \subset U$.
\end{definition}

\begin{definition}
	Let $(X, d)$ be a metric space. $U \subset X$ is called \textbf{closed} (in $X$) its complement in $X$, $X - U$, is open.
\end{definition}

\begin{definition}
	Sets in a metric space that are both open and closed are called \textbf{clopen}.
\end{definition}

\begin{example}
	$\emptyset$ and $X$ are clopen in any metric space.
\end{example}

\begin{lemma}\label{lem:openBallsAreOpen}
	In a metric space, the open ball $B_r(x)$ is open.
\end{lemma}

\begin{proof}
	Let $y \in B_r(x)$ and let $s := d(x, y) < r$. Let $\epsilon = r - s > 0$. Then for every $z \in B_{\epsilon}(y)$,
	\[
		d(x, z) \le d(x, y) + d(y, z) < s + \epsilon = r
	\]
	So $z \in B_r(x)$.
\end{proof}

\begin{example}
	$\mathbb{H}, \mathbb{D}, \mathbb{C}^*, \mathbb{C} - \mathbb{R}_{\le 0}$ are all open. The 1st quadrant $\Omega_1 := \{ z \in \mathbb{C}: \text{Re}(z) > 0, \text{Im}(z) > 0 \}$ is open, since for every $z \in \Omega_1$, let $r = \min( \text{Re}(z), \text{Im}(z) ) > 0$, then $B_r(z) \subset \Omega_1$.
\end{example}

\begin{example}
	Let $(X, d)$ be a discrete metric space, so
	\[
		d(x, y) = \begin{cases}
			0 & \text{if } x = y \\
			1 & \text{if } x \ne y
		\end{cases}
	\]
	For every $x \in X$ and $r > 0$,
	\[
		B_r(x) = \begin{cases}
			\{x\} & \text{ if } r \le 1 \\
			X & \text{ if } r > 1
		\end{cases}
	\]
	So by Lemma~\ref{lem:openBallsAreOpen}, every singleton $\{ x \}$ is an open set with respect to the discrete metric. But also, $X - \{ x \}$ is open, since for every $y \in X - \{ x \}$ and $r < 1$, $y \ne x$ so $B_r(y) = \{ y \} \subset X - \{ x \}$.

	So all balls are clopen with respect to the discrete metric.
\end{example}

\begin{example}
	$[0, 1)$ is neither open nor closed in $\mathbb{R}$ with respect to the standard metric $|\cdot|$, since $0 \in [0, 1)$ doesn't have a ball in $[0, 1)$ containing it, but $1 \in \mathbb{R} - [0, 1) = (-\infty, 0) \cup [1, \infty)$ also doesn't have a ball in $\mathbb{R} - [0, 1)$ containing it.
\end{example}

\begin{lemma}\label{lem:unionsAndIntersectionsOpen}
	Let $(X, d)$ be a metric space.
	\begin{enumerate}
		\item Arbitrary unions of open sets are open, i.e. for every (finite or infinite) collection of open sets $U_i \subseteq X$, the union
		\[
			\bigcup_{i} U_i
		\]
		is open.
		\item Finite intersections of open sets are open, i.e. for every finite collection of open sets $U_i \subseteq X$, the intersection
		\[
			\bigcap_{i = 1}^n U_i
		\]
		is open.
	\end{enumerate}
\end{lemma}

\begin{proof}
	\hfill
	\begin{enumerate}
		\item Let $x \in \bigcup_{i} U_i$. Then for some $j$, $x \in U_j$. $U_j$ is open so for some $\epsilon > 0$, $B_{\epsilon} (x) \subseteq U_j \subseteq \bigcup_{i} U_i$.
		\item Let $x \in \bigcap_{i = 1}^n U_i$. For every $i$, $U_i$ is open so for some $r_i > 0$, $B_{r_i} (x) \subseteq U_i$. Let $\epsilon = \min \{ r_1, \ldots, r_n \} > 0$, so $B_{\epsilon}(x) \subseteq B_{r_i}(x)$ for every $i$. Then
		\[
			B_{\epsilon}(x) \subseteq \bigcap_{i = 1}^n B_{r_i}(x) \subseteq \bigcap_{i = 1}^n U_i
		\]
	\end{enumerate}
\end{proof}

\begin{corollary}
	Let $(X, d)$ be a metric space. Then
	\begin{enumerate}
		\item Finite unions of closed sets are closed.
		\item Arbitrary intersections of closed sets are closed.
	\end{enumerate}
\end{corollary}

\begin{proof}
	Use De Morgan's laws and Lemma~\ref{lem:unionsAndIntersectionsOpen}.
\end{proof}

\begin{remark}
	Infinite intersections of open sets are not always open, and infinite unions of closed sets are not always closed. For example,
	\[
		\bigcup_{n = 1}^{\infty} \left[ \frac{1}{i}, 1 - \frac{1}{i} \right] = (0, 1)
	\]
	is an infinite union of closed sets but $(0, 1)$ is open in $\mathbb{R}$.
\end{remark}

\section{Mobius Transformations}

\begin{corollary}
	Any Mobius transformation is a bijection from $\hat{\mathbb{C}}$ to $\hat{\mathbb{C}}$.
\end{corollary}

Let $T \in GL_2(\mathbb{C})$ and $M_T$ be a Mobius transformation, then a point $z$ is a fixed point of $M_T$ if $M_T(z) = z$.

\begin{lemma}
	Let $T \in GL_2(\mathbb{C})$. If $M_T: \mathbb{C} \rightarrow \mathbb{C}$ is not the identity map, then $M_T$ has at most two fixed points in $\mathbb{C}$. If a Mobius transformation has three fixed points then it is the identity map.
\end{lemma}

\begin{proof}
	Case 1: Suppose $M_T(\infty) = \infty$. From the definition, $M_T(z) = \frac{az + b}{cz + d}$, therefore $c = 0$. So $M_T(z) = \frac{a}{d}z + \frac{b}{d}$, with $a \ne 0, d \ne 0$ (since $\det T \ne 0$).

	Such an affine linear map has at most one fixed point because:
	\begin{itemize}
		\item If $a \ne d$ then $\frac{a}{d}z + \frac{b}{d} = z \Longleftrightarrow z = \frac{b}{d - a}$ so $M_T$ has a unique fixed point.
		\item If $a = d$ then $b \ne 0$ (since we assume $M_T$ is not the identity). So $M_T(z) = z + \frac{b}{a}$ is a translation which has no fixed points.
	\end{itemize}

	Case 2: Suppose $M_T(\infty) \ne \infty$. Suppose $z_0 \in \mathbb{C}$ is such that $M_T(z_0) = z_0$. We have $M_T(z_0) = z_0 \Longleftrightarrow \frac{a z_0 + b}{c z_0 + d} = z_0 \Longleftrightarrow c z_0 ^ 2 + (d - a)z_0 - b = 0$. This quadratic equation has at most two roots so there are at most two fixed points of $M_T$.
\end{proof}

\begin{definition}
	Given four distinct points $z_0, z_1, z_2, z_3 \in \mathbb{C}$, the cross-ratio of these points denoted $(z_0, z_1; z_2, z_3)$ is defined by
	\[\frac{(z_0 - z_2)(z_1 - z_3)}{(z_0 - z_3)(z_1 - z_2)}\]

	We extend the definition to the case where one of the points is $\infty$ by removing all differences involving that point e.g. $(\infty, z_0; z_2, z_3) = \frac{z_1 - z_3}{z_1 - z_2}$.
\end{definition}

\begin{theorem}
	(Three points theorem)
	Let ${z_1, z_2, z_3}$ and ${w_1, w_2, w_3}$ be two sets of three ordered points in $\hat{\mathbb{C}}$. Then there exists a unique Mobius transformation $f$ such that $f(z_i) = w_i$ for every $i \in \{1, 2, 3\}$.
\end{theorem}

\begin{proof}
	Existence:

	We consider the functions $F(z) = (z, w_1; w_2, w_3) = \frac{(z - w_2)(w_1 - w_3)}{(z - z_3)(w_1 - w_2)}$ and $G(z) = \frac{(z - z_2)(z - z_3)}{(z - z_3)(z_1 - z_2)}$. These are Mobius transformations with the properties that $F(w_1) = 1$, $F(w_2) = 0$, $F(w_3) = \infty$ and similarly, $G(z_1) = 1$, $G(z_2) = 0$, $G(z_3) = \infty$. Therefore $F^{-1} \circ G$ maps each $z_i$ to $w_i$.

	Uniqueness:

	Assume that there are two such maps, say $f_1$ and $f_2$. Then the Mobius transformation $H = f_1 ^ {-1} \circ f_2$ satisfies $H(z_i) = z_i$.

	This shows that $H$ has three fixed points so, by Three Point Theorem, it must be the identity. Thus $f_1 = f_2$.
\end{proof}

\begin{proposition}
	Mobius transformations preserve the cross ratio. That is, if $z_0, z_1, z_2, z_3$ are four distinct points in $\hat{\mathbb{C}}$ and $f$ is a Mobius transformation, then $(f(z_0), f(z_1); f(z_2), f(z_3)) = (z_0, z_1; z_2, z_3)$.
\end{proposition}

\begin{proof}
	Let $w_i = f(z_i)$ for every $i \in \{1, 2, 3\}$. Let $F(z) = (z, w_1; w_2, w_3)$ and $G(z) = (z, z_1; z_2, z_3)$. Recall $F^{-1} \circ G$ maps $z_i$ to $w_i$ like $f$ does. Since there is a unique Mobius transformation with this property, we have \[f = F^{-1} \circ G\] and \[F \circ f = G\]

	That is, $(f(z_0), w_1; w_2, w_3) = F \circ f (z_0) = G(z_0) = (z_0, z_1; z_2, z_3)$.
\end{proof}

\begin{remark}
	General strategy: to find Mobius transformation, find image of 3 points and use the fact that cross ratio is preserved. Plug known points into (*) and rearrange for $f(z_0)$.
\end{remark}

\subsection{The Riemann Sphere Revisited}

Circles in $\hat{\mathbb{C}}$ correspond to circles in $S^2$ that don't pass through $N$ (the North pole).
\\
Lines in $\hat{\mathbb{C}}$ correspond to circle in $S^2$ that pass through $N$.

\begin{remark}
	Mobius transformations give all biholomorphic maps from $S^2$ to $S^2$.
\end{remark}

\begin{remark}
	Stereographic projections are conformal.
\end{remark}

\subsection{Mobius transformations preserving the upper half plane and the unit disc}

Notation: for a domain $D \subset \mathbb{C}$, let $Mob(D)$ be the set of Mobius transformations $f$ such that $f(D) = D$.

\begin{proposition}
	(H2H) Every Mobius transformation mapping $\mathbb{H}$ to $\mathbb{H}$ ($\mathbb{H} = \{ z \in \mathbb{C}: Im(z) > 0\}$) is of the form $M_T$ with $T \in SL_2 (\mathbb{R}) := \{T = \begin{matrix}
		()
	\end{matrix}: a, b, c, d \in \mathbb{R}, \det T = 1\}$
	
	Conversely, every such Mobius transformation maps $\mathbb{H}$ to $\mathbb{H}$ and hence a biholomorphism from $\mathbb{H}$ to $\mathbb{H}$.

	i.e. H2H: $f \in \text{Mob}(\mathbb{H}) \Leftrightarrow f = M_T$ with $T \in SL_2 (\mathbb{R})$.
\end{proposition}

\begin{remark}
	$T \rightarrow M_T$ gives a group homomorphism $SL_2 (\mathbb{R}) \rightarrow Aut(\mathbb{H})$
\end{remark}

\begin{proof}
	Any Mobius transformation $f: \mathbb{H} \rightarrow \mathbb{H}$ must map $\partial \mathbb{H}$ to $\partial \mathbb{H}$. As $\partial \mathbb{H}$ is the real line, $f: \mathbb{R} \cup \infty \rightarrow \mathbb{R} \cup \infty$. So $f$ must map the ordered set $\{1, 0, \infty \}$ to $\{x_1, x_2, x_3\}$ for some $x_i \in \mathbb{R} \cup \infty$.

	We know that the cross ratio is preserved under a Mobius transformation:
	\[(f(z), x_1; x_2, x_3) = \frac{(f(z) - x_2)(x_1 - x_3)}{(f(z) - x_3)(x_1 - x_2)} = \frac{z - 0}{1 - 0} = (z, 1; 0, \infty)\]
	\[\Leftrightarrow (f(z) - x_2)(x_1 - x_3) = z (f(z) - x_3)(x_1 - x_2)\]
	\[\Leftrightarrow f(z) = \frac{x_3 (x_1 - x_2) z + x_2 (x_3 - x_1)}{(x_1 - x_2)z + x_3 - x_1}\]

	We see that the coefficients of $T$ are real.
	
	If $T \in GL_2 (\mathbb{R})$ and $z = x + iy$ then
	\[Im(M_T(z)) = Im(\frac{az + b}{cz + d}) = Im(\frac{(az + b)(c \bar{z} + d)}{|cz + d|^2})\]
	\[= Im(\frac{bc \bar{z} + adz}{(cz + d)}) = \frac{y \det T}{|cz+d|}\]

	We have $z \in \mathbb{H} \Leftrightarrow y > 0$ so $M_T(z) \in H \Leftrightarrow T \in GL_2(\mathbb{R})$, $\det T > 0$. We can therefore replace $T$ by a real matrix of determinant 1 by scaling $T$ by a real number.
\end{proof}

\begin{proposition}
	(D2D): Every Mobius transformation from the unit disc $\mathbb{D} = \{z \in \mathbb{C}: |z| < 1\}$ to $\mathbb{D}$ is of the form $T \in SU(1, 1)$

	Conversely, every such Mobius transformation maps $\mathbb{D}$ to $\mathbb{D}$ and hence gives a h
	biholopmorhpic automorphism of $\mathbb{D}$.

	i.e. $f \in Mob(\mathbb{D}) \Leftrightarrow f = M_T$, $T \in SU(1, 1)$.
\end{proposition}

\begin{proof}
	($\Rightarrow$): Let $M_T: \mathbb{D} \rightarrow \mathbb{D}$ be a Mobius transformation. The Cayley map $H_C$ maps $\mathbb{H}$ to $\mathbb{D}$. We have that $f = M_C^{-1} \circ M_T \circ M_C$ is a Mobius transformation from $\mathbb{H}$ to $\mathbb{H}$. By proposition 4.20, we have $f = M_S$ where $S \in SL_2(\mathbb{R})$.

	Hence $C^{-1} T C = S \in SL_2(\mathbb{R})$ by Lemma 4.4.

	Let $S \in \mathbb{M}_2(\mathbb{R})$, $\det S = 1$. Then $T = CSC^{-1}$. Evalutating this shows $T \in SU(1, 1)$.

	($\Leftarrow$): If $T \in SU(1, 1)$, then the same calculation in reverse shows that the matrix $S = C^{-1} T C \in SL_2(\mathbb{R})$. Then $M_S: \mathbb{H} \rightarrow \mathbb{H}$ is a Mobius transformation by proposition 4.20 (H2H), and the map $M_T := M_C \circ M_S \circ M_C^{-1}$ is a Mobius transformation from $\mathbb{D}$ to $\mathbb{D}$
\end{proof}

\begin{remark}
	$T \rightarrow M_T$ gives a group homomorphism from $SU(1, 1)$ to $Aut(\mathbb{D})$.
\end{remark}

\begin{corollary}
	(D2D*):
	\begin{enumerate}
		\item Every Mobius transformation $f$ from $\mathbb{D}$ to $\mathbb{D}$ can be written as
		\[f(z) = e^{i\theta} \frac{z - z_0}{\bar{z_0}z - 1}\]
		for some angle $\theta$ and $z_0 \in \mathbb{D}$ where $z_0$ is the unique point in $\mathbb{D}$ such that $f(z_0) = 0$.
		\item Every Mobius transformation of the unit disc $\mathbb{D}$ to $\mathbb{D}$ for which $f(0) = 0$ are rotations about $0$.
	\end{enumerate}
\end{corollary}

\begin{proof}
	\begin{enumerate}
		\item By proposition D2D, we have
		\[f(z) = \frac{az + b}{\bar{b}z + \bar{a}} = \frac{a(z + b / a)}{-\bar{a}((-\bar{b} / \bar{a}) z - 1)} = -\frac{a}{\bar{a}} \frac{z - (-b / a)}{(-\bar{b} / \bar{a}) z - 1}\]

		So $z_0 = -\frac{b}{a}$. Since $|-\frac{a}{\hat{a}} = 1$, $-\frac{a}{\hat{a}} = e^{i\theta}$ for some $\theta \in (-\pi, \pi]$.

		$|z_0|^2 - 1 = |-\frac{b}{a}|^2 - 1 = \frac{|b|^2}{|a|^2} - 1$. Now $1 = |a|^2 - |b|^2$ so $|z_0|^2 - 1 = \frac{-1}{|a|^2} < 0$ so $|z_0|^2 < 1$ and so $|z_0| < 1$.
		\item $f(0) = 0 \Leftrightarrow e^{i\theta} \frac{0 - z_0}{\bar{z_0}\cdot 0 - 1} = 0 \Leftrightarrow z_0 = 0 \Leftrightarrow f(z) = e^{i\theta}\frac{z - 0}{0 - 1} = e^{-i\theta}z$.
		
		So $f$ is a rotation.
	\end{enumerate}
\end{proof}

\begin{remark}
	The map $g(z) = \frac{z - z_0}{\bar{z_0}z - 1}$ swaps $z_0$ and $0$ and is an involution ($g \circ g = Id$). Also, $z \rightarrow e^{i\theta}z$ is a rotation.

	So every Mobius transformation from $\mathbb{D}$ to $\mathbb{D}$ is given by an involution followed by a rotation.
\end{remark}

\subsection{Finding biholomorphic maps between domains}

To find a biholomorphism $f$ between domains, we build $f$ in various stages using simpler known maps.

\begin{example}
	Find biholomorphism from $D = \{z \in \mathbb{D}: Im(z) < 0\}$ to $\mathbb{H}$.

	The Cayley Map $M_C$ is a map from $\mathbb{H}$ to $\mathbb{D}$, so $M_C^{-1}: \mathbb{D} \rightarrow \mathbb{H}$, $M_C^{-1}(z) = \frac{iz + i}{-z + 1}$.

	To find the image of $D$ under $M_C^{-1}$, consider how it acts on two segments of $\delta D$:

	\begin{itemize}
		\item Under $M_C^{-1}$, $-1 \rightarrow 0$, $0 \rightarrow i$ and $1 \rightarrow \infty$. Therefore the line segment from $-1$ to $1$ through $0$ is mapped to the positive imaginary axis.
		\item Under $M_C^{-1}$, $-i \rightarrow 1$, so the circular arc from $-1$ to $1$ through $-i$ is mapped to the positive real axis.
	\end{itemize}

	Now $-\frac{i}{2} \in D$ and $M_C^{-1}(-\frac{i}{2}) = \frac{4 + 3i}{5}$. The image of $D$ under $M_C^{-1}$ is $\Omega = \{w \in \mathbb{C}: 0 < Arg(w) < \frac{\pi}{2}\}$.

	Now we find a biholomorphic map from $\Omega$ to $\mathbb{H}$. $g(z) = z^2$ satisfies this, as it doubles the argument of $z$.

	So the map is $f = g \circ M_C^{-1}$, $f: D \rightarrow \mathbb{H}$.
\end{example}

\section{Notions of convergence in complex analysis and power series}

\subsection{Pointwise and uniform convergence}

\begin{definition}
	Let $(X, d_X)$ and $(Y, d_Y)$ be two metric spaces. A sequence of functions ${\{f_n\}}_{n \in \mathbb{N}}: X \rightarrow Y$ converges pointwise (on $X$) to $f$ if for every $x \in X$, the limit function $f(x) := \lim_{n \rightarrow \infty} f_n(x)$ exists in $Y$.

	In other words, we have for every $x \in X$ and for every $\epsilon > 0$, for some $N \in \mathbb{N}$, for every $n > N$, $d_Y(f_n(x), f_n(x)) < \epsilon$. (Not that $N$ depends on $x$).
\end{definition}

\begin{remark}
	For every $x \in X$, $f_n(x)$ is just a sequence of points in $Y$. The above definition is what we get by applying definition 2.11 (in notes) to the sequence $f_n(z)$.
\end{remark}

\begin{example}
	Let $f_n(z) = z^n$, $f_n: \mathbb{C} \rightarrow \mathbb{C}$. There are the following cases:

	\begin{enumerate}
		\item $z \in \mathbb{D}$. Let $\epsilon > 0$. Then $|z|^N < \epsilon$ for every $N > \frac{\log \epsilon}{\log |z|}$. So for every $n > N$ we have $f_n(z) - 0 = |z|^n < |z|^N \epsilon$, hence $\lim_{n \rightarrow \infty} f_n(z) = 0 \in \mathbb{D}$.
		\item $|z| = 1$. The point $z$ rotates around the unit circle $\delta \mathbb{D}$ by $Arg(z)$ anticlockwise every iteration. For $z \ne 1$, this sequence doesn't converge. But for $z = 1$, $\lim_{n \rightarrow \infty} f_n(z) = \lim_{n \rightarrow \infty} 1 = 1$.
		\item $|z| > 1$. The value of $|z|^n$ is unbounded so doesn't converge.
	\end{enumerate}

	The sequence $f_n$ doesn't converge pointwise on $\mathbb{C}$. But it is pointwise convergent on $\mathbb{D} \cup {1}$ with limit function:

	\begin{equation}
		f(z) =
		\begin{cases}
			0 & \text{if } z \in \mathbb{D}\\
			1 & \text{if } z = 1
		\end{cases}
	\end{equation}
\end{example}

\begin{definition}
	Let $(X, d_X)$ and $(Y, d_Y)$ be two metric spaces. A sequence of functions ${\{f_n\}}_{n \in \mathbb{N}}: X \rightarrow Y$ converges uniformly (on $X$) to the limit function $f$ if for every $\epsilon > 0$ for some $N \in \mathbb{N}$, for every $n > N$, $d_Y(f_n(x), f(x)) < \epsilon$ for every $x \in X$.
\end{definition}

\begin{theorem}
	Let $(X, d_X)$ and $(Y, d_Y)$ be two metric spaces and let ${\{f_n\}}_{n \in \mathbb{N}}: X \rightarrow Y$ be a sequence of functions that converges uniformly to $f$ on $X$.

	Then $f$ is continuous on $X$.
\end{theorem}

\begin{proof}
	Same as in Analysis I.
\end{proof}

\begin{lemma}
	let ${\{f_n\}}_{n \in \mathbb{N}}: X \rightarrow \mathbb{C}$ be a sequence of functions converging pointwise to a limit function $f$.

	\begin{enumerate}
		\item If $|f_n(x) - f(x)| \le s_n$ for every $x \in X$ where ${\{s_n\}}_{n \in \mathbb{N}}$ is some sequence in $\mathbb{R} > 0$ (independent of $x$) with $\lim_{n \rightarrow \infty} s_n = 0$ then $f_n$ converge uniformly to $f$ on $X$.
		\item If for some sequence $x_n \in X$, $|f_n(x_n) - f(x_n)| \ge c$ for some positive constant $c$ then $f_n$ does not converge uniformly to $f$ on $X$.
	\end{enumerate}
\end{lemma}

\begin{theorem}
	(Weierstrass M-test): Let $f_n: X \rightarrow \mathbb{C}$ be a sequence of fucntions such that $|f_n(x)| \le M_n$ for every $x \in X$ and $\sum_{n = 1}^{\infty} M_n < \infty$.

	Then $\sum_{n = 1}^{\infty} f_n$ converges uniformly on $X$ to some limit function $f: X \rightarrow \mathbb{C}$.
\end{theorem}

\begin{proof}
	Similar to Analysis I.
\end{proof}

\begin{theorem}
	Let a sequence of functions $f_n: [a, b] \rightarrow \mathbb{R}$ converge uniformly on an interval $[a, b]$ to some function $f$, such that $\{f_n\}$ are all continuous. Then

	\[\lim_{n \rightarrow \infty} \int_a^c f_n(x) dx = \int_a^c f(x) dx \text{ for every } c \in [a, b]\]
\end{theorem}

\begin{definition}
	Let ${\{f_n\}}_{n \in \mathbb{N}}$ be a sequence of functions in a metric space $X$. $f_n$ converges locally uniformly (on $X$) to the limit function $f$ if for every $x \in X$, for some open set $U \subset X$ containing $x$, $f_n$ converges uniformly to $f$ on $U$.
\end{definition}

\begin{theorem}
	Let ${\{f_n\}}_{n \in \mathbb{N}}$ be a sequence of continuous functions which converges locally uniformly on $X$ to a limit function $f$. Then $f$ is continuous on $X$.
\end{theorem}

\begin{proof}
	For every $x \in X$, $f_n$ converges uniformly on some open set $U$ containing $x$. Hence $f$ is continuous on $U$ by theorem 5.5 (in notes). So $f$ is continuous at $x$ for every $x \in X$.
\end{proof}

\begin{remark}
	The limit of a locally uniform convergent sequence of holomorphic functions is again holomorphic.
\end{remark}

\begin{example}
	For every $w \in \mathbb{D}$, for some $r < 1$, $w \in B_r(0)$ and $B_r(0)$ is open. Then for every $z \in B_r(0)$, $|z|^n < r^n$ and $\lim_{n \rightarrow \infty} r_n = 0$. So by lemma 5.6 (in notes), with $s_n = r^n$, $f_n$ converges uniformly to $f$ in $B_r(0)$.
\end{example}

\begin{remark}
	To prove that the limit function is conitnuous on all of $\mathbb{D}$, it is enough to prove locally uniform convergence on every ball $B_r(0)$, $0 < r < 1$, in $\mathbb{D}$.
\end{remark}

\begin{theorem}
	Let $X$ be a metric space and let $f_n: X \rightarrow \mathbb{C}$ be a sequence of continuous functions such that for any $y \in X$, there is an open $U \subset X$ containing $y$ and constants $M_n > 0$ with $\sum_{n = 1}^{\infty} M_n < \infty$ and $|f_n(x)| \le M_n$ for every $x \in U$. Then $\sum_{n = 1}^{\infty} f_n$ converges locally uniformly to a continuous function on $X$.
\end{theorem}

\begin{proof}
	Given $y \in X$, the hypotheses of the theorem imply that for some constants $M_n > 0$, $|f_n(y)| \le M_n$ and $\sum_{n = 1}^{\infty} M_n < \infty$.

	\[|F_k(y)| = |\sum_{n = 1}^k f_n(y)| \le \sum_{n = 1}^{\infty} |f_n(y)| \le \sum_{n = 1}^k M_n\]

	As $k \rightarrow \infty$, the RHS $\sum_{n = 1}^k M_n$ converges so it must be bounded, and let the upper bound by $L$. Thus for every $k$, $|F_k(y)| \le L$. So the sequence ${(F_k(y))}_k$ is bounded, hence it lies in some boundedd, closed ball in $\mathbb{C}$, which is compact by Heine-Borel.

	Therefore there is a subsequence ${(F_{k_j}(y))}_{k_j}$ that converges to $F(y)$.

	Now, for $k_j > k$,

	\[|F_{k_j}(y) - F_k(y)| = |\sum_{n = k + 1}^{k_j} f_n(y)| \le \sum_{n = k + 1}^{k_j} |f_n(y)| \le \sum_{n = k + 1}^{k_j} M_n\]

	Taking the limit as $j \rightarrow \infty$, both the LHS and RHS converge, and we get

	\[|F(y) - F_k(y)| \le \sum_{n = k + 1}^{\infty} M_n\]

	Now taking the limit as $k \rightarrow \infty$, we get

	\[\lim_{k \rightarrow \infty} |F(y) - F_k(y)| = 0\]

	since the RHS tends to zero.

	Repeating this for every $y$, $F_k \rightarrow F$ pointwise on $X$.

	From the hypotheses of the theorem, we have that for every $y \in X$, for some open $U \subset X$ containing $y$ and constants $M_n > 0$ with $\sum_{n = 1}^{\infty} < \infty$ and $|f_n(x)| \le M_n$ for every $x \in U$.

	Then, for every $x \in U$ and for every $L > k$,

	\[|F_L(x) - F_k(x)| = |\sum_{n = k + 1}^L f_n(x)| = \sum_{n = k + 1}^L |f_n(x)| \le \sum_{n = k + 1}^L M_n\]

	Taking the limit as $l \rightarrow \infty$:

	\[|F(x) - F_k(x)| \le \sum_{n = k + 1}^{\infty} M_n\]

	for every $x \in U$.

	$\lim_{k \rightarrow \infty} \sum_{n = k + 1}^{\infty} M_n = 0$. So by lemma 5.6 (in notes), $F_k \rightarrow F$ uniformly on $U$.
\end{proof}

\subsection{Complex power series}

\begin{theorem}
	A complex power series is an expression of the form
	\[
		\sum_{n = 0}^{\infty} a_n {(z - c)}^n, \quad a_n, c \in \mathbb{C}^2
	\]
	There are three cases:

	\begin{enumerate}
		\item $\sum_{n = 0}^{\infty} a_n {(z - c)}^n$ converges only for $z = c$ ($R = 0$).
		\item There exists $R > 0$ (radius of convergence) such that
		\begin{itemize}
			\item $\sum_{n = 0}^{\infty} a_n {(z - c)}^n$ converges absolutely for $|z - c| < R$ (We call $B_R(c)$ the disc of convergence).
			\item $\sum_{n = 0}^{\infty} a_n {(z - c)}^n$ diverges for $|z - c| > R$ (anything can happen on the circle $|z - c| = R$).
		\end{itemize}
		\item $\sum_{n = 0}^{\infty} a_n {(z - c)}^n$ converges absolutely for every $z \in \mathbb{C}$ ($R = \infty$).
	\end{enumerate}
\end{theorem}

\begin{remark}
	Radius of convergence is usually determined via ratio test or root test.
\end{remark}

\begin{theorem}
	A power series $\sum_{n = 0}^{\infty} a_n {(z - c)}^n$ with radius of convergence $0 < R < \infty$ converges uniformly on every ball $B_r(c)$ with $0 < r < R$. This implies that the power series is locally uniformly convergent on its disc of convergence.
\end{theorem}

\begin{proof}
	Follows via the M-test.
\end{proof}

\begin{remark}
	The power series do not converge uniformly in the entire disc of conergence $B_R(c)$.
\end{remark}

\begin{proposition}
	Let $\sum_{n = 0}^{\infty} a_n {(z - c)}^n$ be a power series with radius of convergence $0 < R < \infty$. Then the formal derivatives and antiderivatives

	\[\sum_{n = 0}^{\infty} n a_n {(z - c)}^{n - 1}\] and

	\[\sum_{n = 0}^{\infty} \frac{a_n}{n + 1} {(z - c)}^{n + 1}\]

	have the same radius of convergence $R$.
\end{proposition}

\begin{theorem}
	Let $\sum_{n = 0}^{\infty} a_n {(z - c)}^n$ be a power series with radius of convergence $0 < R < \infty$ and let $f: B_R(c) \rightarrow \mathbb{C}$ be the resulting limit function. Then $f$ is holomorphic on $B_R(c)$ with

	\[f'(z) = \sum_{n = 0}^{\infty} n a_n {(z - c)}^{n - 1}\] for $z \in B_R(c)$.
\end{theorem}

\begin{proof}
	Assume $c = 0$ (the general case for $c$ is analogous).

	\[f(z) - f(w) = \sum_{n = 1}^{\infty} a_n (z^n - w^n) = \sum_{n = 1}^{\infty} (z - w) q_n(z)\]
	
	where $q_n(z) = \sum_{k = 0}^{n - 1} w^k z^{n - 1 - k}$.

	So for $z \ne w$, let $h(z) := \frac{f(z) - f(w)}{z - w} = \sum_{n = 1}^{\infty} a_n q_n(z)$

	Given $z_0 \in B_R(0)$, let $r < R$ such that $w, z_0 \in B_r(0)$. To apply the local M-test, we need constants $M_n$ for this set $B_r(0)$ that bound the terms $a_n q_n(z)$ defining $h$.

	For $z \in B_r(0)$,

	\[|a_n q_n(z)| = |a_n \sum_{k = 0}^{n - 1} w^k z^{n - 1 - k}| \le |a_n| \sum_{k = 0}^{n - 1} |w|^k |z|^{n - 1 - k} < |a_n| \sum_{k = 0}^{n - 1} r^{n - 1} = n |a_n| r^{n - 1}\]

	So let $M_n = n|a_n| r^{n - 1}$, then $\sum_{n = 1}^{\infty} M_n = \sum_{n = 1}^{\infty} n|a_n| r^{n - 1}$ which converges by proposition 5.19 (in lecture notes).

	The formal derivative $\sum_{n = 1}^{\infty} n a_n r^{n - 1}$ has radius of convergence $R$ so converges absolutely on its disc of convergence $B_R(0)$. In particular, it converges at $z = R$. By the local M-test, the series defining $h$ converges locally uniformly to a continuous function on $B_R(0)$. Hence
	
	\[\lim_{z \rightarrow w} \frac{f(z) - f(w)}{z - w} = \lim_{h \rightarrow w} h(z) = h(w) = \sum_{n = 1}^{\infty} a_n q_n(w) = \sum_{n = 1}^{\infty} n a_n w^{n - 1}\]
\end{proof}

\begin{corollary}
	A power series $f$ as theorem 5.21 (in lecture notes) with positive radius of convergence $R$ can be differentiated infinitely many times and
	
	\[f^{(k)} := \sum_{n = k}^{\infty} k! {n \choose k} a_n {(z - c)}^{n - k}\] for $z \in B_R(c)$
\end{corollary}

\begin{corollary}
	A power series $f$ as in theorem 5.21 (in lecture notes) with positive radius of convergence $R$ has a holomorphic antiderivative $F: B_R(c) \rightarrow \mathbb{C}$, with $F'(z) = f(z)$, defined by

	\[F(z) := \sum_{n = 0}^{\infty} \frac{a_n}{n + 1} {(z - c)}^{n + 1}\]
\end{corollary}

\hfill

\section{Complex integration over contours}

\subsection{Definition of contour integrals}

\begin{definition}
	For a continuous function $f: [a, b] \rightarrow \mathbb{C}$, with $f(z) = u(z) + i v(z)$,

	\[\int_a^b f(t) dt := \int_a^b u(t) dt + i \int_a^b v(t) dt \in \mathbb{C}\]
\end{definition}

\begin{lemma}
	\hfill
	\begin{enumerate}
		\item Let $f_1$ and $f_2$ be continuous functions from $[a, b]$ to $\mathbb{C}$. Then $\int_a^b (f_1(t) + f_2(t)) dt = \int_a^b f_1(t) dt + \int_a^b f_2(t) dt$.
		\item For any complex number $c \in \mathbb{C}$ and continuous function $f: [a, b] \rightarrow \mathbb{C}$,
		
		\[ \int_a^b c f(t) dt = c \int_a^b f(t) dt \]
	\end{enumerate}
\end{lemma}

\begin{definition}
	A smooth curve in $\mathbb{C}$ is a continuously differentiable function $\gamma: [0, 1] \rightarrow \mathbb{C}$ (i.e. differentiable with continuous derivative). More generally we can consider continuously differentiable curves $\gamma: [a, b] \rightarrow \mathbb{C}$. We say that such curves are $C^1$.

\end{definition}

\begin{remark}
	We write $\gamma(t) = u(t) + i v(t)$ with $u, v: [a, b] \rightarrow \mathbb{R}$. Then the derivative $\gamma'$ is defined as

	\[ \gamma'(t) := u'(t) + i v'(t) \]

	At the endpoints, we demand that the one-sided derivative exists and is continuous from the one side:

	\[ \gamma'(b) := \lim_{h \rightarrow 0^-} \frac{u(b + h) - u(b)}{h} + i \lim_{h \rightarrow 0^-} \frac{v(b + h) - v(b)}{h} \] exists and

	\[ \lim_{t \rightarrow b^-} \gamma'(t) = \gamma'(b) \]
\end{remark}

\begin{definition}
	Let $U \subset \mathbb{C}$ be an open set, and $f: U \rightarrow \mathbb{C}$ be a continuous function. Let $\gamma: [a, b] \rightarrow U$ be a $C^1$ curve. The integral of $f$ along the curve $\gamma$ is defined as

	\[ \int_{\gamma} f(z) dz = \int_a^b f(\gamma(t)) \gamma'(t) dt \]
\end{definition}

\begin{corollary}
	Properties of the integral along a curve:
	\begin{enumerate}
		\item $\int_{\gamma} (f_1(z) + f_2(z)) dz = \int_{\gamma} f_1(z) dz + \int_{\gamma} f_2(z) dz$
		\item For $c \in \mathbb{C}$, $\int_{\gamma} c f(z) dz = c \int_{\gamma} f(z) dz$
	\end{enumerate}
\end{corollary}

\begin{proof}
	Easy
\end{proof}

\begin{definition}
	Given $\gamma: [a, b] \rightarrow \mathbb{C}$, the curve $(-\gamma): [-b, -a] \rightarrow \mathbb{C}$ is defined as
	
	\[ (-\gamma)(t) := \gamma(-t) \] Then we have
	
	\[ \int_{-\gamma} f(z) dz = -\int_{\gamma} f(z) dz \]
\end{definition}

\begin{lemma}
	Let $U \subset \mathbb{C}$ be an open set, $f: U \rightarrow \mathbb{C}$ be continuous and $\gamma: [a, b] \rightarrow \mathbb{C}$ be a $C^1$ curve. If $\phi: [a', b'] \rightarrow [a, b]$ with $\phi(a') = a$ and $\phi(b') = b$ is continuously differentiable and we define $\delta: [a', b'] \rightarrow \mathbb{C}$, $\delta := \gamma \circ \phi$, then

	\[ \int_{\gamma} f(z) dz = \int_{\delta} f(z) dz \]
\end{lemma}

\begin{proof}
	\[\int_{\delta} f(z) dz = \int_{a'}^{b'} f(\delta(t)) \delta'(t) dt = \int_{a'}^{b'} f(\gamma(\phi(t))) (\gamma(\phi(t)))' dt\]
	
	\[ = \int_{a'}^{b'} f(\gamma(\phi(t))) \gamma'(\phi(t)) \phi'(t) dt\]
	With a change of variables $s = \phi(t)$, $ds = \phi'(t) dt$:

	\[ \int_{a'}^{b'} f(\gamma(\phi(t))) \gamma'(\phi(t)) \phi'(t) dt = \int_a^b f(\gamma(s)) \gamma'(s) ds = \int_{\gamma} f(z) dz \]
\end{proof}

\begin{definition}
	Let $\gamma: [a, b] \rightarrow \mathbb{C}$ be a curve and suppose there exist $a = a_0 < a_1 < \cdots < a_n = b$ such that the curves $\gamma_i: [a_{i - 1}, a_i] \rightarrow \mathbb{C}$, defined by $\gamma_i(t) = \gamma(t)$ for $t \in [a_{i - 1}, a_i]$ are $C^1$ curves. Then $\gamma$ is a piecewise $C^1$ curve or contour.

	For a contour $\gamma$ above, a contour integral is defined as

	\[ \int_{\gamma} f(z) dz = \sum_{n = 1}^n \int_{\gamma_i} f(z) dz \]
\end{definition}

\begin{definition}
	If $\gamma: [a, b] \rightarrow \mathbb{C}$ and $\delta: [c, d] \rightarrow \mathbb{C}$ are two contours with $\gamma(b) = \delta(c)$ the contour $\gamma \cup \delta: [a, b + d - c] \rightarrow \mathbb{C}$ is defined as

	\[ (\gamma \cup \delta)(t) := \begin{cases}
		\gamma(t) & \text{ if } a \le t \le b \\
		\delta(t) & \text{ if } c \le t \le d
	\end{cases} \]

	Then

	\[ \int_{\gamma \cup \delta} f(z) dz = \int_{\gamma} f(z) dz + \int_{\delta} f(z) dz \]
\end{definition}

\subsection{The fundamental theorem of calculus}

\begin{theorem}
	Let $U \in \mathbb{C}$ be an open set and let $F: U \rightarrow \mathbb{C}$ be holomorphic with continuous derivative $f$. Then for every contour $\gamma: [a, b] \rightarrow U$,

	\[ \int_{\gamma} f(z) dz = F(\gamma(b)) - F(\gamma(a)) \]
	In particular, if $\gamma$ is closed, so $\gamma(a) = \gamma(b)$, then

	\[ \int_{\gamma} f(z) dz = 0 \]
\end{theorem}

\begin{proof}
	First consider the case where $\gamma$ is a $C^1$ curve. Let $F = u + iv$. Then

	\[ \int_{\gamma} f(z) dz = \int_{\gamma} F'(z) dz = \int_a^b F'(\gamma(t)) \gamma'(t) dt = \int_a^b (F(\gamma(t)))' dt\]

	\[ = \int_a^b (u(\gamma(t)))' dt + i \int_a^b (v(\gamma(t)))' dt = {[u(\gamma(t))]}_a^b + i {[v(\gamma(t))]}_a^b \]

	\[ = u(\gamma(b)) - u(\gamma(b)) + i(v(\gamma(b)) - v(\gamma(b))) = F(\gamma(b)) - F(\gamma(a)) \]
	Now extend this proof to any contour.

	Let $\gamma: [a, b] \rightarrow \mathbb{C}$ be a contour, then for some $a = a_0 < a_1 < \cdots < a_n = b$, the curves $\gamma_i: [a_{i - 1}, a_i] \rightarrow \mathbb{C}$, $i = 1, \ldots, n$, defined by $\gamma_i(t) = \gamma(t)$ for $t \in [a_{i - 1}, a_i]$ are $C^1$ curves. Then

	\[ \int_{\gamma} f(z) dz = \int_{\gamma} F'(z) dz = \sum_{i = 1}^n \int_{\gamma_i} F'(z) dz \]

	\[ = \sum_{i = 1}^n (F(\gamma(a_i)) - F(\gamma(a_{i - 1}))) = F(\gamma(a_n)) - F(\gamma(a_0)) = F(\gamma(b)) - F(\gamma(a)) \]
\end{proof}

\begin{remark}
	Under the hypotheses on $F$, the integral only depends on the endpoints of the curve.
\end{remark}

\begin{theorem}
	If $f: [a, b] \rightarrow \mathbb{R}$ is continuous,
	
	\[\int_a^b f(t) dt \le \int_a^b \max_{t \in [a, b]} f(t) dt \le (b - a) \max_{t \in [a, b]}\]
\end{theorem}

\begin{proof}
	From Analysis I.
\end{proof}

\begin{definition}
	Let $\gamma: [a, b] \rightarrow \mathbb{C}$ be a contour. The \textbf{length} of $\gamma$ is defined as

	\[L(\gamma) = \int_a^b |\gamma'(t)| dt\]
\end{definition}

\begin{lemma}
	(\textbf{The Estimation Lemma}) Let $f: U \rightarrow \mathbb{C}$ be continuous and $\gamma: [a, b] \rightarrow U$ be a contour. Then

	\[ \left | \int_{\gamma} f(z) dz \right | \le L(\gamma) \sup_{\gamma} |f| \]
	where $\sup_{\gamma} |f| := \sup \{ |f(z)|: z \in \gamma \}$.
\end{lemma}

\begin{proof}
	First prove that for a continuous function $g: [a, b] \rightarrow \mathbb{C}$,

	\[ \left | \int_a^b g(t) dt \right | \le \int_a^b |g(t)| dt \]
	If we write $\int_a^b g(t) dt = r e^{i \theta}$ with $r \ge 0$, then

	\[ \left | \int_a^b g(t) dt \right | = |r e^{i \theta}| = r = \text{Re} \left( e^{-i \theta} \int_a^b g(t) dt \right ) \]

	\[ = \text{Re}\left( \int_a^b g(t) e^{-i \theta} dt \right) = \int_a^b \text{Re}( g(t) e^{-i \theta}) dt \le \int_a^b \left |e^{-i \theta} g(t) \right | dt = \int_a^b |g(t)| dt \]

	Let $g(t) = f(\gamma(t)) \gamma'(t)$, then

	\[ \left | \int_{\gamma} g(z) dz \right | = \left | \int_a^b f(\gamma(t)) \gamma'(t) dt \right | \le \int_a^b \left | f(\gamma(t)) \gamma'(t) \right | dt \]
	Then

	\[ \int_a^b \left | f(\gamma(t)) \gamma'(t) \right | dt \le \sup_{\gamma} |f| \int_a^b |\gamma'(t)| dt = L(\gamma) \sup_{\gamma} |f| \]
\end{proof}

\begin{theorem}
	(Converse to FTC) Let $f: D \rightarrow \mathbb{C}$ be continuous on a domain $D$. If $\int_{\gamma} f(z) dz = 0$ for every closed contour $\gamma \in D$, for some $F: D \rightarrow \mathbb{C}$, $F'(z) = f(z)$.
\end{theorem}

\begin{proof}
	Let $a_0 \in D$. For every $a_0 \ne w \in D$, let $\gamma(w)$ be a contour connecting $a_0$ to $w$ and is contained in $D$.

	Since $D$ is a domain, it is path-connected, i.e. there is a smooth path $\gamma_w$ connecting $a_0$ to $w$, therefore the collection of contours contained in $D$ and connecting $a_0$ and $W$ is non-empty. Let

	\[ F(w) := \int_{\gamma(w)} f(z) dz \]

	Let $\tilde{\gamma} (w)$ be another contour that connects $a_0$ to $w$ and is contained in $D$. Then let $c(w) = \gamma(w) \cup (-\tilde{\gamma}(w))$ that is obtained by moving through $\gamma$ then through $\tilde{\gamma}$ in the opposite direction. Since $c$ is a closed contour in $D$, $\int_C f(z) dz = 0$.

	Then $0 = \int_C f(z) dz = \int_{\gamma(w) \cup (-\tilde{\gamma}(w))} f(z) dz = \int_{\gamma(w)} f(z) dz + \int_{-\tilde{\gamma}(w)} f(z) dz = \int_{\gamma(w)} f(z) dz - \int_{\tilde{\gamma}(w)} f(z) dz$. Hence

	\[ \int_{\gamma(w)} f(z) dz = \int_{\tilde{\gamma}(w)} f(z) dz \]
	Therefore $F$ does not depend on the contour chosen to join $a_0$ to $w$.

	Now we claim $F$ is holomorphic and we claim that $F$ is holomorphic and $\forall z \in D, F'(z) = f(z) \Rightarrow \lim_{h \rightarrow 0} \frac{F(w + h) - F(w)}{h} = f(w)$.

	To evaluate $F(w + h)$ we need a contour joining $a_0$ to $w + h$ contained in $D$. For every $w \in D$, let $r > 0$ such that $B_r(w) \subset D$. This ball must exist since $D$ is open. Then for every $h \in \mathbb{C}$ with $|h| < r$ consider the striaght line $\delta_h$ that connects $w$ to $w + h$.

	A parameterisation of this line is given by
	
	\[ \delta_h: [0, 1] \rightarrow D, \quad \delta_h(t) = w + t h \]
	The contour $\gamma_w \cup \delta_h$ is contained in $D$. So

	\[ F(w + h) = \int_{\gamma_w \cup \delta_h} f(z) dz = \int_{\gamma_w} f(z) dz + \int_{\delta_h} f(z) dz = F(w) + \int_{\delta_h} f(z) dz \]

	\[ \int_{\delta_h} f(w) dz = f(w) \int_{\delta_h} dz = f(w) \int_0^1 h dt = h f(w) \]
	We can rewrite the previous equation as
	\[ F(w + h) = F(w) + h f(w) + \int_{\delta_h} (f(z) - f(w)) dz \]
	For $h \ne 0$,

	\[ \left| \frac{F(w + h) - F(w)}{h} - f(w) \right| = \frac{1}{|h|} \left| \int_{\delta_h} (f(z) - f(w)) dz \right| \]
\end{proof}

\subsection{First Version of Cauchy's Theorem}

\begin{definition}
	A domain $D$ is \textbf{starlike} if for some point $a_0 \in D$, for every $b \ne a_0 \in D$, the straight line connecting $a_0$ and $b$ is contained in $D$.
\end{definition}

\begin{example}
	\hfill
	\begin{enumerate}
		\item $\mathbb{C}$ is starlike.
		\item The ball $B_r(a)$ is starlike.
		\item Any convex set is starlike.
	\end{enumerate}
\end{example}

\begin{example}
	\hfill
	\begin{enumerate}
		\item $\mathbb{C}^{\star}$ is not starlike, because a straight line between two points could pass through $0$, and $0 \notin \mathbb{C}^{\star}$.
		\item Similarly, $B_r^{\star}(a) = B_r(a) - \{ a \}$ is not starlike.
	\end{enumerate}
\end{example}

\begin{lemma}\label{lem:ctLem1}
	Let $U$ be an open set and let $f: U \rightarrow \mathbb{C}$ be holomorphic. Then
	\[
		\int_{\partial \Delta} f(z) dz = 0
	\]
	for every \textbf{triangle} $\Delta$ in $U$.
\end{lemma}

\begin{remark}
	Here $\partial \Delta$ is the boundary of $\Delta$, traversed anticlockwise.
\end{remark}

\begin{remark}
	Given any closed contour without a parameterisation given, we will assume that it is traversed anticlockwise.
\end{remark}

\begin{proof}
	First, split $\Delta$ into four triangles, $\Delta^{(1)}, \Delta^{(2)}, \Delta^{(3)}, \Delta^{(4)}$, using the midpoints of each side. Then
	\[
		\int_{\partial \Delta} f(z) dz = \sum_{i = 1}^{4} \int_{\partial \Delta^{(i)}} f(z) dz
	\]
	Let $\Delta_1$ be one of these four triangles which has the largest integral, then
	\[
		\left| \int_{\partial \Delta} f(z) dz \right| \le 4 \left| \int_{\partial \Delta_1} f(z) dz \right|
	\]
	We then continue this procedure to produce a sequence of triangles
	\[
		\Delta > \Delta_1 > \cdots > \Delta_n > \cdots
	\]
	The length of $\Delta_1$, $L(\Delta_1)$ satisfies $L(\Delta_1) = \frac{1}{2} L(\Delta)$, therefore
	\[
		L(\Delta_n) = \frac{1}{2} n L(\Delta) \Longrightarrow L(\Delta_n) \rightarrow \text{ as } n \rightarrow \infty
	\]
	Also,
	\[
		\bigcap_{n \in \mathbb{N}} \Delta_n = \{ w \}
	\] is a single point in $D$. Now, notice that
	\[
		\int_{\partial \Delta_n} 1 dz = 0 = \int_{\partial \Delta_n} z dz
	\]
	and that $w, f(w), f'(w)$ are constants. Then sneakily,
	\[
		\int_{\partial \Delta_n} f(z) dz = \int_{\partial \Delta_n} \left( f(z) - f(w) - (z - w) f'(w) \right)
	\]
	Define the auxiliary function
	\[
		g(z) = \begin{cases}
			\frac{f(z) - f(w)}{z - w} - f'(w) & \text{ if } z \in D \ \backslash \ \{ w \} \\
			0 & \text{ if } z = w
		\end{cases}
	\]
	which is continuous at $z = w$, so is continuous on $D$. So
	\[
		\int_{\partial \Delta_n} f(z) dz = \int_{\partial \Delta_n} (z - w) g(z) dz
	\]
	Now,
	\[
		\left| \int_{\partial \Delta} f(z) dz \right| \le 4^n \left| \int_{\partial \Delta_n} f(z) dz \right| = 4^n \left| \int_{\partial \Delta_n} (z - w) g(z) dz \right|
	\]
	Note
	\[
		\sup_{z \in \partial \Delta_n} |z - w| \le L(\partial \Delta_n)
	\]
	so by the Estimation Lemma,
	\[
		\begin{aligned}
			\left| \int_{\partial \Delta} f(z) dz \right| & \le 4^n L(\partial \Delta_n) \sup_{z \in \partial \Delta_n} | (z - w) g(z) | \\
			& \le 4^n L(\partial \Delta_n) \sup_{z \in \partial \Delta_n} | (z - w) | \sup_{z \in \partial \Delta_n} | g(z) | \\
			& \le 4^n {(L(\partial \Delta_n))}^2 \sup_{z \in \partial \Delta_n} | g(z) | \\
			& = {L(\Delta)}^2 \sup_{z \in \partial \Delta_n} | g(z) |
		\end{aligned}
	\]
	As $n \rightarrow \infty$, $\sup_{z \in \partial \Delta_n} | g(z) | \rightarrow g(w) = 0$. This completes the proof.
\end{proof}

\begin{lemma}\label{lem:ctLem2}
	Let $D$ be a starlike domain and $f: D \rightarrow \mathbb{C}$ be continuous. Then, if
	\[
		\int_{\partial \Delta} f(z) dz = 0
	\]
	for every $\Delta \subset D$, then for some $F: D \rightarrow \mathbb{C}$,
	\[
		F'(z) = f(z) \quad \forall z \in D
	\]
\end{lemma}

\begin{proof}
	Similar to the proof of converse of FTC.
\end{proof}

\begin{theorem}
	(\textbf{Cauchy's Theorem for Starlike Domains - CTSD}) Let $D$ be a starlike domain and let $f: D \rightarrow \mathbb{C}$ be holomorphic. Then for every closed contour $\gamma \in D$,
	\[
		\int_{\gamma} f(z) dz = 0
	\]
\end{theorem}

\begin{proof}
	By Lemma~\text{Re}f{lem:ctLem1},
	\[
		\int_{\partial \Delta} f(z) dz = 0 \quad \forall \Delta \in D
	\]
	By Lemma~\text{Re}f{lem:ctLem2}, $f$ has a holomorphic antiderivative $F$. Then, by FTC,
	\[
		\int_{\gamma} f(z) dz = 0 \quad \forall \text{ closed } \gamma \in D
	\]
\end{proof}

\begin{remark}
	The same result holds if $f$ is holomorphic on $D - S$, where $S$ is a finite set of points and $f$ is continuous on $D$. We will need this in proofs but is not used much elsewhere.
\end{remark}

\begin{example}
	Consider
	\[
		\int_{|z| = \frac{1}{2}} \frac{e^z {(\sin z)}^2}{e^{z^2}} dz
	\]
	Because the function in the integral is holomorphic and $|z| = \frac{1}{2}$ is a closed contour, by CTSD, this integral is equal to $0$.
\end{example}

\subsection{Cauchy's integral formula}

\begin{theorem}
	(\textbf{Cauchy's integral formula - CIF}) Let $B_r(a)$ be a ball in $\mathbb{C}$ and $f: B_r(a) \rightarrow \mathbb{C}$ be holomorphic. Then for every $w \in B_r(a)$,
	\[
		f(w) = \frac{1}{2 \pi i} \int_{|z - a| = \rho} \frac{f(z)}{z - w} dz
	\]
	where $\rho$ is any real number with $|w - a| < \rho < r$.
\end{theorem}

\begin{proof}
	Define an auxiliary function $g$ by
	\[
		g(z) = \begin{cases}
			\frac{f(z) - f(w)}{z - w} & \text{ if } z \ne w \\
			f'(w) & \text{ if } z = w
		\end{cases}
	\]
	Note that $g$ is continuous at $z = w$, and holomorphic elsewhere. By CTSD,
	\[
		\int_{|z - a| = \rho} g(z) dz = 0
	\]
	Therefore
	\[
		\int_{|z - a| = \rho} \frac{f(z)}{z - w} dz = \int_{|z - a| = \rho} \frac{f(w)}{z - w} dz
	\]
	Now,
	\[
		\begin{aligned}
			\frac{1}{z - w}
				& = \frac{1}{z - a + a - w} \\
				& = \frac{1}{(z - a)(1 - \frac{w - a}{z - a})} \\
				& = \frac{1}{z - a} \sum_{n = 0}^{\infty} {\left( \frac{w - a}{z - a} \right)}^n
		\end{aligned}
	\]
	which converges uniformly, since $|\frac{w - a}{z - a}| = |\frac{w - a}{\rho} < 1|$.
	So we have
	\[
		\begin{aligned}
			\int_{|z - a| = \rho} \frac{f(z)}{z - w} dz
				& = f(w) \int_{|z - a| = \rho} \sum_{n = 0}^{\infty} \frac{{(w - a)}^n}{{(z - a)}^{n + 1}} dz \\
				& = \sum_{n = 0}^{\infty} \left( f(w) {(w - a)}^n \int_{|z - a| = \rho} \frac{1}{{(z - a)}^{n + 1}} dz \right)
		\end{aligned}
	\]
	The inner integral is equal to $0$ except when $n = 0$, when it's value is $2 \pi i$. So
	\[
		\int_{|z - a| = \rho} \frac{f(z)}{z - w} dz = f(w) {(w - a)}^0 \cdot 2 \pi i = 2 \pi i \cdot f(w)
	\]
\end{proof}

\section{Features of holomorphic functions}

\begin{theorem}
	(\textbf{Cauchy-Taylor theorem}) Let $U$ be an open set and $f: U \rightarrow \mathbb{C}$ be holomorphic on $U$. Then for every $r > 0$ such that $B_r(a) \subset U$, $f$ has a power series converging on $B_r(a)$ given by
	\[
		f(z) = \sum_{n = 0}^{\infty} c_n {(z - a)}^n
	\]
	where
	\[
		c_n = \frac{1}{2 \pi i} \int_{|z - a| = \rho} \frac{f(z)}{{(z - a)}^{n + 1}} dz
	\]
	is a constant for every $0 < \rho < r$. This is the \textbf{Taylor series} of $f$ about $a$.
\end{theorem}

\begin{proof}
	By the CIF, for every $w$ with $|w - a| < \rho$,
	\[
		\begin{aligned}
			f(w)
				& = \frac{1}{2 \pi i} \int_{|z - a| = \rho} \frac{f(z)}{z - w} dz \\
				& = \frac{1}{2 \pi i} \int_{|z - a| = \rho} f(z) \sum_{n = 0}^{\infty} \frac{{(w - a)}^n}{{(z - a)}^{n + 1}} dz \\
				& = \sum_{n = 0}^{\infty} \left( \frac{1}{2 \pi i} \int_{|z - a| = \rho} \frac{f(z)}{{(z - a)}^{n + 1}} dz \right) {(w - a)}^n \\
				& = \sum_{n = 0}^{\infty} c_n {(w - a)}^n
		\end{aligned}
	\]
\end{proof}

\begin{theorem}
	(\textbf{CIF for derivatives}) Let $f: B_r(a) \rightarrow \mathbb{C}$ be holomorphic. Then for every $0 < \rho < r$,
	\[
		\int_{|z - a| = \rho} \frac{f(z)}{{(z - a)}^{n + 1}} dz = \frac{2 \pi i}{n!} f^{(n)} (a)
	\]
\end{theorem}

\begin{proof}
	By Cauchy-Taylor, we have a convergent power series such that
	\[
		f(z) = \sum_{n = 0}^{\infty} c_n {(z - a)}^n
	\]
	where
	\[
		c_n = \frac{1}{2 \pi i} \int_{|z - a| = \rho} \frac{f(z)}{{(z - a)}^{n + 1}} dz
	\]
	But we also have (corollary 5.22 in lecture notes),
	\[
		c_n = \frac{f^{(n)} (a)}{n!}
	\]
	Equating these two expressions for $c_n$ completes the proof.
\end{proof}

\begin{remark}
	Combining theorem 7.1 (lecture notes) and theorem 7.2 (lecture notes), every holomorphic function $f$ has power series
	\[
		f(z) = \sum_{n = 0}^{\infty} \frac{f^{(n)} (a)}{n!} {(z - a)}^n
	\]
\end{remark}

\begin{remark}
	Cauchy-Taylor does not hold in real analysis. Let $f: \mathbb{R} \rightarrow \mathbb{R}$ be defined as
	\[
		f(x) = \begin{cases}
			e^{-1 / x} & \text{ if } x > 0 \\
			0 & \text{ if } x \le 0
		\end{cases}
	\]
	$f$ is differentiable for $x \ne 0$. For $x = 0$,
	\[
		f'(x) = \lim_{x \rightarrow 0} \frac{f(x) - f(0)}{x - 0} = \lim_{x \rightarrow 0} \frac{f(x)}{x}
	\]
	\[
		\lim_{x \rightarrow 0^-} \frac{f(x)}{x} = \lim_{n \rightarrow 0^-} \frac{0}{x} = 0
	\]
	and
	\[
		\lim_{x \rightarrow 0^+} \frac{f(x)}{x} = \lim_{x \rightarrow 0^+} \frac{e^{-1 / x}}{x} = \lim_{x \rightarrow 0^+} \frac{1 / x}{e^{1 / x}} = \lim_{y \rightarrow \infty} \frac{y}{e^y} = 0
	\]
	so $f'(0) = 0$, hence $f$ is differentiable on $\mathbb{R}$. But if $f$ had a Taylor series at $x = 0$, then
	\[
		f(x) = \sum_{n = 0}^{\infty} \frac{f^{(n)} (0)}{n!} z^n = 0
	\]
	around $x = 0$.
\end{remark}

\begin{corollary}
	(Holomorphic functions have infinitely many derivatives) If $f: U \rightarrow \mathbb{C}$ is holomorphic on an open set $U$ then $f$ has derivatives of all orders and each derivative is also holomorphic.
\end{corollary}

\begin{proof}
	Since $U$ is open, $\exists B_r(a) \subset U$ around a point $z = a$. But then by Cauchy-Taylor, $f$ has a power series. By theorem 5.21 (lecture notes), this power series is holomorphic. By corollary 5.22 (lecture notes) we can term-by-term differentiate to get a power series for $f'(z)$. By theorem 5.21 (lecture notes), $f'(z)$ is holomorphic. This can be repeated indefinitely.
\end{proof}

\begin{remark}
	This is a huge difference between real and complex analysis. Let $f: \mathbb{R} \rightarrow \mathbb{R}$ by defined as
	\[
		f_n(x) = |x| x^n
	\]
	\[
		f_n'(0) = \lim_{x \rightarrow 0} \frac{f_n(x) - f_n(0)}{x - 0} = \lim_{x \rightarrow 0} |x| x^{n - 1} = 0
	\]
	$f_n'(x) = (n + 1) |x| x^{n - 1}$ and $f^{(n)} (x) = c|x|$ which is not differentiable.
\end{remark}

\begin{theorem}
	(\textbf{Morera's Theorem}) Let $f: D \rightarrow \mathbb{C}$ be continuous on a domain $D$. If
	\[
		\int_{\gamma} f(z) dz = 0 \quad \forall \text{ closed } \gamma \subset D
	\]
	then $f$ is holomorphic.
\end{theorem}

\begin{proof}
	By the converse FTC, $f$ has a holomorphic antiderivative $F: D \rightarrow \mathbb{C}$ such that $F'(z) = f(z) \quad \forall z \in D$. By corollary 7.6 (lecture notes), if $F$ is holomorphic, its derivative $f$ must be.
\end{proof}

\begin{example}
	Consider
	\[
		\int_{|z| = 3} \frac{e^z}{z^2 (z - 1)} dz
	\]
	We use partial fractions:
	\[
		\frac{1}{z^2 (z - 1)} = \frac{a}{z} + \frac{b}{z^2} + \frac{c}{z - 1}
	\]
	So $1 = (c + a) z^2 + (b - a) z - b$, so $b = -1, a - -1, c = 1$. Using the CIF and CIF for derivatives,
	\[
		\begin{aligned}
			\int_{|z| = 3} \frac{e^z}{z^2 (z - 1)} dz
				& = -\int_{|z| = 3} \frac{e^z}{z} dz - \int_{|z| = 3} \frac{e^z}{z^2} dz + \int_{|z| = 3} \frac{e^z}{z - 1} dz \\
				& = -2 \pi i e^0 - 2 \pi i e^0 + 2 \pi i e^1 \\
				& = 2 \pi i (e - 2)
		\end{aligned}
	\]
\end{example}

\subsection{Liouville's theorem}

\begin{definition}
	A function $f: \mathbb{C} \rightarrow \mathbb{C}$ is \textbf{entire} if $f$ is holomorphic on $\mathbb{C}$.
\end{definition}

\begin{definition}
	A function $f: \mathbb{C} \rightarrow \mathbb{C}$ is \textbf{bounded} if for some $M > 0$, $|f(z)| \le M \ \forall z \in \mathbb{C}$.
\end{definition}

\begin{theorem}
	(\textbf{Liouville's theorem}) Every bounded entire function is constant.
\end{theorem}

\begin{proof}
	Let $f$ be entire and bounded. We will show that $\forall w \in \mathbb{C}, \ f(w) = f(0)$. By the CIF, for every $\rho > |w|$,
	\[
		\begin{aligned}
			|f(w) - f(0)|
				& = \left| \frac{1}{2 \pi i} \int_{|z| = \rho} \frac{f(z)}{z - w} dz - \frac{1}{2 \pi i} \int_{|z| = \rho} \frac{f(z)}{z} dz \right| \\
				& = \frac{|w|}{2 \pi} \left| \int_{|z| = \rho} f(z) \frac{1}{z(z - w)} dz \right| \\
				& = 
		\end{aligned}
	\]
	Using the Estimation lemma, boundedness of $f$ and the reverse triangle inequality,
	\[
		\begin{aligned}
			|f(w) - f(0)|
				& \le \frac{|w|}{2 \pi} 2 \pi \rho \cdot \sup_{|z| = \rho} \frac{|f(z)|}{|z| |z - w|} \\
				& \le |w| \rho \frac{M}{\rho} \sup_{|z| = \rho} \frac{1}{|z - w|} \\
				& \le |w| M \sup_{|z| = \rho} \frac{1}{| |z| - |w| |} \\
				& = \frac{|w| M}{\rho - |w|} \\
				& \rightarrow 0 \quad \text{as } \rho \rightarrow \infty
		\end{aligned}
	\]
	and $\rho$ can be arbitrarily large.
\end{proof}

\begin{remark}
	The holomorphicity condition is essential (we can't just say that $f$ is continuous). For example,
	\[
		f(z) = f(x + iy) = \sin(x) + i \sin(y)
	\]
	is continuous and bounded on $\mathbb{C}$ but is not entire.
\end{remark}

\begin{theorem}
	(\textbf{Fundamental theorem of Algebra}) Every non-constant polynomial with complex coefficients $p(z) = a_d z^d + \cdots + a_1 z + a_0$, $a_d \ne 0$ has a complex root: for some $z_0 \in \mathbb{C}, P(z_0) = 0$.
\end{theorem}

\begin{proof}
	By assumption, $d \ge 1$, so $|p(z)| \rightarrow \infty$ as $|z| \rightarrow \infty$. In particular, $\exists R > 0, |p(z)| > 1$ if $|z| = R$. Assume the converse, that $p$ has no roots.

	Then $f(z) := \frac{1}{p(z)}$ is holomorphic on $\mathbb{C}$. On the set $|z| > R$, $f$ is bounded, since $|f(z)| = \frac{1}{|p(z)|} < 1$. But $\overline{B_R}(0) = \{ z \in \mathbb{C}: |z| \le R \}$ is compact, so by theorem 2.30 (lecture notes), $|f(z)|$ attains a maximum on $\overline{B_R}(0)$. In particular, $f$ is bounded on $\overline{B_R}(0)$. Thus $f$ is bounded and entire, so by Liouville's theorem, $f$ is constant, which is a contradiction.
\end{proof}

\begin{theorem}
	(\textbf{Local maximum modulus principle}) Let $f: B_r(a) \rightarrow \mathbb{C}$ be holomorphic. If for every $z \in B_r(a)$, $|f(z)| \le |f(a)|$ then $f$ is constant on $B_r(a)$.
\end{theorem}

\begin{proof}
	First, we show $|f|$ is constant. Pick any $w \in B_r(a)$. We will show $|f(w)| = |f(a)|$. Let $\rho = |w - a| < r$ so that the contour $\gamma(t) = a + \rho e^{2 \pi i t}, t \in [0, 1]$ passes through $w$. By the CIF,
	\[
		\begin{aligned}
			|f(a)|
				& = \left| \frac{1}{2 \pi i} \int_{|z - a| = \rho} \frac{f(z)}{z - a} dz \right| \\
				& = \frac{1}{2 \pi} \left| \int_{\gamma} \frac{f(z)}{z - a} dz \right| \\
				& = \frac{1}{2 \pi} \left| \int_{0}^{1} \frac{f(\gamma(t))}{\gamma(t) - a} \gamma'(t) dt \right| \\
				& = \frac{1}{2 \pi} \left| \int_{0}^{1} \frac{f(\gamma(t))}{\rho e^{2 \pi i t}} \rho \cdot 2 \pi i e^{2 \pi i t} dt \right| \\
				& = \left| \int_{0}^{1} f(\gamma(t)) dt \right| \\
				& \le \int_{0}^{1} |f(\gamma(t))| dt \\
				& \le \int_{0}^{1} |f(a)| dt \\
				& \le |f(a)| \int_{0}^{1} 1 dt \\
				& = |f(a)|
		\end{aligned}
	\]
	Therefore every inequality above must be an equality. In particular,
	\[
		\int_{0}^{1} |f(\gamma(t))| dt = \int_{0}^{1} |f(a)| dt
	\]
	So by continuity, $|f(\gamma(t))| = |f(a)|$. So $|f(w)| = |f(a)|$, and $w$ was arbitrary, so $|f(z)| = |f(a)| \ \forall z \in B_r(a)$.

	Now, we show that $f$ is constant. Let $|f(z)| = c \ \forall z \in B_r(a)$. If $c = 0$, the $|f(z)| = 0 \Rightarrow f(z) = 0 \forall z \in B_r(a)$. So assume $c \ne 0$. Now,
	\[
		c^2 = |f(z)|^2 = f(z) \cdot \overline{f(z)}
	\]
	So $\overline{f(z)} = \frac{c^2}{f(z)}$ is holomorphic (since $f$ is holomorphic). Let $f = u + iv$, then $\overline{f} = u - iv = u + i(-v)$. Then by the Cauchy-Riemann equations, $u_x = v_y$ but also $u_x = {(-v)}_y = -v_y$ so $u_x = v_y = 0$, and $u_y = -v_x$ but also $u_y = {(-v)}_x = v_x$ so $u_y = v_x = 0$. By proposition 3.3 (lecture notes), $f'(z) = u_x + i v_x = 0 + i \cdot 0 = 0$, hence $f$ is constant.
\end{proof}

\begin{theorem}
	(\textbf{Maximum modulus theorem}) Let $D$ be a domain and $f: D \rightarrow \mathbb{C}$. If
	\[
		\exists a \in D, |f(z)| \le |f(a)| \quad \forall z \in D
	\]
	then $f$ is constant on $D$.
\end{theorem}


\begin{proof}
	Let $U_1 = \{ z \in D: f(z) = f(a) \}$ and let $U_2 = \{ z \in D: f(z) \ne f(a) \}$. Let $U = U_1 \cup U_2$ (TODO: check this last sentence, might not be $U$). $U_1$ is non-empty, because $a \in U_1$. We will show that $U_1$ is open. Pick $z \in U_1$. Then by the openness of $U$, there exists a ball $B_r(z)$ in $U$. Pick $W \in B_r(z)$. We have $|f(w)| \le |f(a)| = |f(z)|$, so by the Local Maximum Modulus Principle, $f$ is constant on $B_r(z)$, so $f(w) = f(z) = f(a) \Longrightarrow w \in U_1 \Longrightarrow B_r(z) \subset U_1$. Hence $U_1$ is open.

	Now $U_2 = D - U_1 = f^{-1} (\mathbb{C} - \{ f(a) \})$, where $f^{-1}$ denotes the preimage. Therefore $U_2$ is open by theorem 2.17 (lecture notes), since $\mathbb{C} - \{ f(a) \}$ is open.

	So $D = U_1 \cup U_2$ where $U_1$ is non-empty and open and $U_2$ is open. By fact 7.14 (lecture notes), $U_2$ is empty hence $D = U_1$, so $f$ is constant on $D$.
\end{proof}

\begin{example}
	Find the maximum absolute value of $f(z) = z^2 + 2z - 3$ on $\overline{B_1}(0) = \{ z \in \mathbb{C}: |z| \le 1 \}$.
	
	$\overline{B_1}(0)$ is compact so by theorem 2.30 (lecture notes), $|f(z)|$ attains a maximum on it. Also, $B_1(0)$ is a domain, so by the maximum modulus theorem, the function does not attain the maximum in $B_1(0)$. So the maximum must be obtained on the boundary. So $|z| = 1$, so let $z = e^{it}$ for some $t \in [0, 2 \pi]$. Then $|f(z)|^2 = |z^2 + 2z - 3|^2 = (z^2 + 2z - 3) (\overline{z^2 + 2z - 3}) = 14 - 3e^{-2it} - 3e^{2it} - 4^{it} - 4e^{-it} = 14 - 6 \cos(2t) - 8 \cos(t) = -12 {\cos(t)}^2 - 8 \cos(t) + 20 = -12x^2 - 8x + 20$ where $x = \cos(t)$. The maximum is attained when $x = \cos(t) = -\frac{1}{3}$. This gives $|f(z)| = \frac{8}{\sqrt{3}}$.
\end{example}

\subsection{Analytic continuation and the identity theorem}

Let $f: B_r(a) \rightarrow \mathbb{C}$ be holomorphic. Then by Cauchy-Taylor, $f$ has a convergent Taylor series
\[
	f(z) = \sum_{n = 0}^{\infty} c_n {(z - a)}^n
\]
for $z \in B_r(a)$. Assume $f \not\equiv 0$ so at least one coefficient is non-zero. Let $m = \min\{ n \ge 0: c_n \ne 0 \}$. Then
\[
	f(z) = {(z - a)}^m \sum_{n = m}^{\infty} c_n {(z - a)}^{n - m} = {(z - a)}^m \sum_{k = 0}^{\infty} c_{k + m} {(z - a)}^k
\]
where $k = n - m$. Let
\[
	h(z) = \sum_{k = 0}^{\infty} c_{k + m} {(z - a)}^k
\]
Then $h$ is a convergent power series so is holomorphic by theorem 5.21 (lecture notes) and $h(a) = c_m \ne 0$. Note $f(a) = 0 \Longleftrightarrow m > 0$.

\begin{definition}
	(\textbf{Orders of zeros}) $f: B_r(a) \rightarrow \mathbb{C}$ has a \textbf{zero of order} $m$ at $a$ if for some holomorphic function $h: B_r(a) \rightarrow \mathbb{C}$ such that $f(z) = {(z - a)}^m h(z)$ and $h(a) \ne 0$.
\end{definition}

\begin{remark}
	We can show that $f$ has a zero of order $m$ at $z = a$ iff
	\[
		f(a) = f'(a) = \cdots = f^{(m - 1)} (a) = 0
	\]
\end{remark}

\begin{example}
	Let $f(z) = z (e^z - 1)$. Let $z = a = 0$. Then
	\[
		\begin{aligned}
			f(z)
				& = z \left( \sum_{n=  0}^{\infty} \frac{z^n}{n!} - 1 \right) \\
				& = z \sum_{n = 1}^{\infty} \frac{z^{n + 1}}{n!} \\
				& = z^2 \sum_{m = 0}^{\infty} \frac{z^m}{(m + 1)!}
		\end{aligned}
	\]
	so $f$ has a zero of order $2$ at $a = 0$. As a check, $f'(z) = e^z - 1 + ze^z$ and $f''(z) = e^z + e^z + ze^z$ so $f(0) = f'(0) = 0$ but $f''(0) = e^0 + e^0 + 0 = 2$.
\end{example}

\begin{remark}
	Because holomorphic functions are continuous, if $f(a) \ne 0$, we can always find $B_{\rho} (a)$ such that $f(z) \ne 0$ on $B_{\rho}(a)$.
\end{remark}

\begin{theorem}
	(\textbf{Principle of isolated zeroes}) Let $f: B_r(a) \rightarrow \mathbb{C}$ be holomorphic and $f \not\equiv 0$. Then for some $\rho > 0$,
	\[
		f(z) \ne 0 \quad \forall z \in B_r(a) - \{ a \}
	\]
	In particular, the zeros are isolated from one another.
\end{theorem}

\begin{proof}
	For $f(a) \ne 0$, we are done by continuity. If $f(a) = 0$, for some $m > 0$,
	\[
		f(z) = {(z - a)}^m h(z) \quad \forall z \in B_{\rho}(a)
	\]
	where $h: B_r(a) \rightarrow \mathbb{C}$ is holomorphic, $\rho > 0$ and $h(a) \ne 0$. Thus $h(z) \ne 0$ on $B_{\rho}(a)$ and ${(z - a)}^m \ne 0$ on $B_{\rho} (a) - \{ a \}$. So $f(z) \ne 0$ on $B_{\rho}(a)$.
\end{proof}

\begin{theorem}
	(\textbf{Uniqueness of analytic continuation}) Let $D' \subset D$ be non-empty domains and $f: D' \rightarrow \mathbb{C}$ be holomorphic. Then there exists \textbf{at most one} holomorphic $g: D \rightarrow \mathbb{C}$ such that
	\[
		g(z) = f(z) \quad \forall z \in D'
	\]
	If $g$ exists, it is called the \textbf{analytic continuation} of $f$ to $D$.
\end{theorem}

\begin{proof}
	Let $g_1, g_2: D' \rightarrow \mathbb{C}$ be analytic continuations. Let $h(z) = g_1(z) - g_2(z)$. We will show that $h(z) = 0$ on $D$. Note $h(z) = 0 \ \forall z \in D'$. Let
	\[
		D_0 = \{ w \in D: \exists r > 0, h(z) = 0 \text{ on } B_r(w) \\
		D_1 = \{ w \in D: \exists n \ge 0, h^{(n)}(w) \ne 0 \}
	\]
	We will show $D_0$ is non-empty and open, $D_1$ is open and $D$ is the disjoint union $D = D_0 \cup D_1$, so $D_0 \cap D_1 = \emptyset$.

	First we show $D_0$ is non-empty and open. Since $D' \subset D_0$, $D_0$ is non-empty. We want to show that $\forall w \in D_0, \exists r > 0, B_r(w) \subset D_0$. By the definition of $D_0$, for some $r > 0$, $h(z) = 0$ on $B_r(w)$. Pick $z \in B_r(w)$. $B_r(w)$ is open, so $\exists B_{\rho} (z)$ inside $B_r(w)$, on which $h(z) = 0$. Thus $z \in D_0$. Thus $D_0$ is open.

	Now we show $D_1$ is open.
	\[
		\begin{aligned}
			D_1
				& = \bigcup_{n = 0}^{\infty} \{ w \in D: h^{(n)}(w) \ne 0 \} \\
				& = \bigcup_{n = 0}^{\infty} {\left( h^{(n)} \right)}^{-1} (\mathbb{C} - \{ 0 \})
		\end{aligned}
	\]
	By Lemma 2.8 (lecture notes) and Theorem 2.17 (lecture notes), $D_1$ is open.

	Now we show $D = D_0 \cup D_1$. Pick $w \in D$. If $w \notin D_1$, then $h^{(n)}(w) = 0 \ \forall n \ge 0$. But by Cauchy-Taylor (lecture notes), $h$ has a Taylor series about $z = w$ with coefficients $h^{(n)}(w) / n! = 0$. Thus $h = 0$ around $w$. So $w \in D_0$.

	If $w \in D_1$, it must have a non-zero Taylor series expansion (at least coefficient $c_n$ is non-zero). By the Principle of Isolated Zeroes (lecture notes), for some $B_{\rho}(w)$, $h(z) \ne 0$ on $B_{\rho}(w) - \{ w \}$. So $w \notin D_0$. This completes the proof.
\end{proof}

\begin{corollary}\label{cor:sameOnBallImpliesSameOnDomain}
	Let $f, g$ be holomorphic on a domain $D$. If $f = g$ on some $B_r(w) \subseteq D$ then $f = g$ on $D$.
\end{corollary}

\begin{definition}
	Given a set $S \subset \mathbb{C}$, a point in $S$ is called
	\begin{itemize}
		\item \textbf{isolated} in $S$ if $\exists \epsilon > 0$, $B_{\epsilon}(w) \cap S = \{ w \}$.
		\item \textbf{non-isolated} in $S$ if $\forall \epsilon > 0$, $\exists z \in S$, $z \in B_{\epsilon}(w)$, with $\omega \ne z$.
	\end{itemize}
\end{definition}

\begin{theorem}
	(\textbf{Identity theorem}) Let $f, g: D \rightarrow \mathbb{C}$ be holomorphic on a domain $D$. If $S = \{ z \in D: f(z) = g(z) \}$ contains a non-isolated point, then
	\[
		f(z) = g(z) \quad \text{on } D
	\]
\end{theorem}

\begin{proof}
	Let $w \in S$ be non-isolated and let $h(z) = f(z) - g(z)$. Then $h(w) = 0$. By the Principle of Isolated Zeroes (lecture notes), for some $\rho$, $\forall z \in B_{\rho} (w) - \{ w \}$, $h(z) \ne 0$. But this contradicts $w$ being non-isolated. So $h(z) = 0$ on $B_{\rho} (w)$. By Corollary~\ref{cor:sameOnBallImpliesSameOnDomain}, $h(z) = 0$ on $D$.
\end{proof}

\begin{example}
	Let $f: \mathbb{C} \rightarrow \mathbb{C}$ and assume $\forall n \in \mathbb{N}, f(1 / n) = \sin(1 / n)$. Then $f(z) = \sin(z)$ on $\mathbb{C}$.

	By continuity (lemma 2.29 (lecture notes)),
	\[
		f(0) = f \left( \lim_{n \rightarrow \infty} \frac{1}{n} \right) = \lim_{n \rightarrow \infty} f \left( \frac{1}{n} \right) = \lim_{n \rightarrow \infty} \sin \left( \frac{1}{n} \right) = \sin(0)
	\]
	Since $z = 0$ is a non-isolated point in $S = \{ z \in \mathbb{C}: f(z) = \sin(z) \}$, $f(z) = \sin(z)$ on $\mathbb{C}$.
\end{example}

\begin{example}
	(Problems class) Evaluate
	\[
		I = \int_{|z| = 1} \frac{1}{(z - a)(z - b)} dz
	\]
	in the cases:
	\[
		|a| > 1, |b| > 1; \quad |a| < 1 < |b|; \quad |a| < 1, |b| < 1
	\]
	\begin{itemize}
		\item $|a| > 1, |b| > 1$: we can find a ball $B_r(0)$ with $1 < r < \min \{ |a|, |b| \}$, on which the integrand is holomorphic, so by CTSD, $I = 0$.
		\item $|a| < 1 < |b|$: Let $f(z) = 1/(z - b)$, then
		\[
			I = \int_{|z| = 1} \frac{f(z)}{z - a} dz = 2 \pi i \frac{1}{a - b}
		\]
		by Cauchy's integral formula.
		\item $|a| < 1, |b| < 1$: if $a = b$ then
		\[
			\frac{1}{(z - a)(z - b)} = \frac{1}{{(z - a)}^2} \Longrightarrow I = 2 \pi i \cdot 0 = 0
		\]
		by CIF for derivatives. If $a \ne b$, then use partial fractions.
	\end{itemize}
\end{example}

\begin{example}
	(Problems class) Evaluate
	\[
		I = \int_{|z| = 2} \frac{{\sin(z)}^2}{z^2} dz
	\]
	Let $f(z) = {\sin(z)}^2, a = 0, \rho = 2, n = 1$. Then by CIF for derivatives,
	\[
		I = \frac{2 \pi i}{1} {\left[ {\sin(z)}^2 \right]}' \Big|_{z = 0} = 2 \pi i [2 \sin(z) \cos(z)] \Big|_{z = 0} = 0
	\]
\end{example}

\begin{example}
	(Problems class) Let $f(z) = 1/{(1 - z)}^2$. Find a Taylor series of $f$.

	An antiderivative of $f$ is $F(z) = 1/(1 - z)$. $F$ has Taylor series
	\[
		F(z) = \sum_{n = 0}^{\infty} z^n \quad (|z| < 1)
	\]
	So by theorem 5.21 (lecture notes), $f$ has Taylor series
	\[
		f(z) = \sum_{n = 1}^{\infty} n z^{n - 1} = \sum_{m = 0}^{\infty} (m + 1) z^m \quad (|z| < 1)
	\]
\end{example}

\begin{example}
	(Problems class) Using the Taylor series expansion of $f(z) = e^z$ about $z = 0$, prove that $|e^z - 1| \le e^{|z|} - 1 \le |z| e^{|z|}$.

	Using the triangle inequality,
	\[
		\begin{aligned}
			|e^z - 1|
				& = \left| \sum_{n = 0}^{\infty} \frac{z^n}{n!} - 1 \right| \\
				& = \left| \sum_{n = 1}^{\infty} \frac{z^n}{n!} \right| \\
				& \le \sum_{n = 1}^{\infty} \frac{|z|^n}{n!} \\
				& = \sum_{n = 0}^{\infty} \frac{|z|^n}{n!} - 1 \\
				& = e^{|z|} - 1
		\end{aligned}
	\]
	Also,
	\[
		\begin{aligned}
			e^{|z|} - 1
				& = \sum_{n = 1}^{\infty} \frac{|z|^n}{n!} \\
				& = |z| \sum_{n = 1}^{\infty} \frac{|z|^{n - 1}}{n!} \\
				& = |z| \sum_{m = 0}^{\infty} \frac{|z|^m}{(m + 1)!} \\
				& \le |z| \sum_{m = 0}^{\infty} \frac{|z|^m}{m!} \\
				& = |z| e^{|z|}
		\end{aligned}
	\]
\end{example}

\begin{example}
	Find the Taylor series about $z_0 \in \mathbb{C}$ of $f(z) = e^z$.

	$\forall n \in \mathbb{N}, f^{(n)} (z) = e^z$. By Corollary 5.22 (lecture notes),
	\[
		f(z) = \sum_{n = 0}^{\infty} \frac{f^{(n)}(z_0)}{n!} {(z - z_0)}^n = \sum_{n = 0}^{\infty} \frac{e^{z_0}}{n!} {(z - z_0)}^n
	\]
\end{example}

\begin{example}
	Calculate the Taylor series of $f(z) = \text{Log}(1 + z)$ about $z = 0$.
	\[
		f'(z) = \frac{1}{1 + z} = \frac{1}{1 - (-z)} = \sum_{n = 0}^{\infty} {(-z)}^n
	\]
	Then by corollary 5.23 (lecture notes),
	\[
		f(z) = \sum_{n = 0}^{\infty} \frac{{(-1)}^n}{n + 1} z^{n + 1} + c \quad (|z| < 1)
	\]
	for some $c$. $\text{Log}(1 + 0) = 0$, so $c = 0$.
\end{example}

\subsection{Harmonic functions and the Dirichlet problem}

\begin{definition}
	A \textbf{harmonic function} is a real valued function $u: D \rightarrow \mathbb{R}$ on a domain $D \subset \mathbb{C}$ that has continuous second-order partial derivatives which satisfy the \textbf{Laplace equation}:
	\[
		u_{xx} + u_{yy} = 0
	\]
\end{definition}

\begin{proposition}
	Let $f = u + iv: D \rightarrow \mathbb{C}$ be holomorphic. Then $u$ and $v$ are harmonic.
\end{proposition}

\begin{proof}
	By proposition 3.3 (lecture notes) the first-order partial derivatives exist and $f' = u_x - i u_y = v_y + i v_x$. By corollary 7.6 (lecture notes), $f'$ is holomorphic and so continuous, hence $u_x, u_y, v_y, v_x$ are continuous as well. By the same argument with $f'$, the second-order partial derivatives exist and are continuous. By proposition 3.3 (lecture notes), $u_x, u_y, v_x, v_y$ satisfy the Cauchy-Riemann equations:
	\[
		\begin{aligned}
			u_x = v_y & \Longrightarrow u_{xx} = v_{yx} \\
			u_y = -v_x & \Longrightarrow u_{yy} = -v_{xy}
		\end{aligned}
	\]
	By the Schwartz-Clairault theorem, $v_{yx} = v_{xy}$, hence $u_{xx} + u_{yy} = v_{yx} - v_{yx} = 0$.
\end{proof}

\begin{example}
	Let $f(z) = e^z = e^{x + iy} = e^x (\cos(y) + i \sin(y))$. So $u(x, y) = e^x \cos(y)$ and $v(x, y) = e^x \sin(y)$. $u_x (x, y) = e^x \cos(y)$, $u_y = -e^x \sin(y)$, $u_{xx} = e^x \cos(y)$ and $u_{yy} = -e^x \cos(y)$.
\end{example}

\begin{example}
	Let $f(x + iy) = x^2 + y^2$. So $u(x, y) = x^2 + y^2$ and $v(x, y) = 0$. $u_x = 2x \Longrightarrow u_{xx} = 2$ but $u_y = 2y \Longrightarrow u_{yy} = 2$ so $u_{xx} + u_{yy} = 4$. Note $f(z) = |z|^2$. So $f$ is not holomorphic.
\end{example}

\begin{example}
	Let $u(x, y) = x^2 - y^2 + 3x$, so $u_x = 2x + 3 \Longrightarrow u_{xx} = 2$, $u_y = -2y \Longrightarrow u_{yy} = -2$ so $u$ is harmonic.
\end{example}

\begin{proposition}
	Let $f: D \rightarrow \mathbb{C}$ be holomorphic on a starlike domain $D$. Then for some $F: D \rightarrow \mathbb{C}$, $F$ is holomorphic and $F' = f$.
\end{proposition}

\begin{proof}
	By Cauchy's Starlike theorem (lecture notes),
	\[
		\int_{\gamma} f(z) dz = 0 \quad \forall \text{ closed } \gamma
	\]
	By the converse FTC (lecture notes), there exists a holomorphic antiderivative of $f$, $F$.
\end{proof}

\begin{theorem}
	(\textbf{The existence of a harmonic conjugate}) If $D$ is a starlike domain and $u: D \rightarrow \mathbb{R}$ is harmonic, then for some harmonic function $v: D \rightarrow \mathbb{R}$,
	\[
		f = u + iv
	\]
	is holomorphic on $D$. $v$ is called the \textbf{harmonic conjugate} of $u$ and is unique up to a real additive constant.
\end{theorem}

\begin{proof}
	If such an $f$ exists, then $f' = u_x - i u_y$ would also be holomorphic. We will first construct $f'$ and then construct $f$.

	Let $g(x, y) = u_x + i (-u_y)$. We want to show $g$ is holomorphic. Using theorem 3.5 (lecture notes), we will show $g$ satisfies the Cauchy-Riemann equations.
	\[
		{(u_x)}_x = u_{xx}, \quad {(-u_y)}_y = -u_{yy} = u_{xx}
	\]
	as $u$ is harmonic. Similarly,
	\[
		{(u_x)}_y = u_{xy}, \quad -{(-u_y)}_x = u_{yx}
	\]
	by Schwarz-Clairault. So by theorem 3.5 (lecture notes), $g$ is holomorphic.

	Now we will construct $f$. By proposition 7.28 (lecture notes), $g$ has a holomorphic antiderivative, $F = U + iV$, where $F'(z) = g(z)$ on $D$. $F$ satisfies the Cauchy-Riemann equations and $F' = U_x - U_y i = g = u_x - u_y i$. Hence
	\[
		U_x = u_x, \quad U_y = u_y \Longrightarrow {(U - u)}_x = 0 = {(U - u)}_y
	\]
	Therefore $U = u + c$ for some constant $c$. So let $f = F - c = (U + iV) - c = u + iV$, and let $v = V$. By proposition 7.26 (lecture notes), $v$ is harmonic.

	Finally, we show $v$ is unique (TODO).
\end{proof}

\begin{example}
	We have seen $u(x, y) = x^2 - y^2 + 3x$ is harmonic. Construct its harmonic conjugate $v$.

	We want a holomorphic $f = u + iv$. By the first Cauchy-Riemann equation, $v_y = u_x = 2x + 3$. So $v(x, y) = 2xy + 3y + g(x)$ for some function $g$. By the second Cauchy-Riemann equation,
	\[
		2y + g'(x) = v_x = -u_y = 2y
	\]
	hence $g'(x) = 0 \Longrightarrow g(x) = c$ for a constant $c \in \mathbb{R}$. So $v(x, y) = 2xy + 2y + c$. Then $f(x + iy) = x^2 - y^2 + 3x + i(2xy + 3y + c)$.
	\
	Note that ${(x + iy)}^2 = x^2 - y^2 + 2xyi$, so $f(z) = z^2 + 3z + ic$.
\end{example}

\begin{definition}
	The \textbf{Dirichlet boundary problem} states: Let $D \subseteq \mathbb{C}$ be a domain with closure $\bar{D}$ and boundary $\delta D$. Let $g: \delta D \rightarrow \mathbb{R}$ be continuous. Find a continuous function $\mu: \bar{D} \rightarrow \mathbb{R}$ such that $\mu$ is harmonic on $D$ and matches $g$ on $\delta D$.
\end{definition}

\begin{example}\label{exa:dirichletBoundaryProblemExample1}
	Let
	\[
		D = \{ x + iy \in \mathbb{C}: 2 < y < 5 \}, \quad g(x, y) = \begin{cases}
			4 & \text{ if } y = 2 \\
			13 & \text{ if } y = 5
		\end{cases}
	\]
	So $\bar{D} = \{ x + iy \in \mathbb{C}: 2 \le y \le 5 \}$ and $\delta D = \{ x + iy \in \mathbb{C}: y = 2 \text{ or } y = 5 \}$. We want a harmonic $\mu$ such that $\mu = g$ on $\delta D$.

	Note that $g$ doesn't depend on $x$, so it is possible that $\mu = 0$. If this was true, then $\mu_{xx} = 0$ and $\mu$ is harmonic so $\mu_{yy} = -\mu_{xx} = 0$, so
	\[
		\mu(x, y) = ay + b
	\]
	This guess for this solution is called an \textbf{ansatz}. Check $\mu(x, 2) = 4 = 2a + b$ and $\mu(x, 5) = 13 = 5a + b$, which gives $a = 3$ and $b = -2$. Thus $\mu(x, y) = 3y - 2$ is a solution to the Dirichlet boundary problem.
\end{example}

\begin{proposition}
	Let $f: D \rightarrow \mathbb{C}$, $f = u + iv$ be holormorphic on $D$ and $\mu$ is harmonic on $f(D)$. Then
	\[
		\tilde{\mu} := \mu \circ f = \mu(u, v)
	\]
	is harmonic on $D$.
\end{proposition}

\begin{example}
	From Example~\ref{exa:dirichletBoundaryProblemExample1}, $\mu(x, y) = 3y - 2$ was a solution to the Dirichlet boundary problem on $D_1 = \{ x + iy: 2 < y < 5 \}$ with
	\[
		g(x, y) = \begin{cases}
			4 & \text{ if } y = 2 \\
			13 & \text{ if } y = 5
		\end{cases}
	\]
	Let $D_2$ be $D_1$ rotated anticlockwise by $\pi/4$ about the origin. Solve the Dirichlet boundary problem on $D_2$ when
	\[
		g(x, y) = \begin{cases}
			4 & \text{ if } y = x + 2 \sqrt{2} \\
			13 & \text{ if } y = x + 5 \sqrt{2}
		\end{cases}
	\]
	Let $f(z) = e^{-\pi i / 4} z$. From Example~\ref{exa:dirichletBoundaryProblemExample1}, we know a solution on $f(D_2) = D_1$. So by proposition 7.31 (lecture notes), $\tilde{\mu} = \mu \circ f$ is a solution on $D_2$. Note that
	\[
		f(x, y) = e^{-\pi i / 4} (x + i y) = \frac{1}{\sqrt{2}} (x + y) + i \frac{1}{\sqrt{2}} (y - x)
	\]
	Thus $\tilde{\mu} = \mu(\frac{1}{\sqrt{2}} (x + y), \frac{1}{\sqrt{2}} (y - x)) = \frac{3}{\sqrt{2}} (y - x) - 2$.
\end{example}

\section{General form of the Cauchy-Taylor theorem and Cauchy's integral formula}

\subsection{Winding number and simply connected sets}

\begin{definition}
	Let $\gamma: [a, b] \rightarrow \mathbb{C}$ be a contour of the form
	\[
		\gamma(t) = \omega + r(t) e^{i \theta (t)}
	\]
	where $\omega \in \mathbb{C}$, $\theta(t): [a, b] \rightarrow \mathbb{C}$, $r(t): [a, b] \rightarrow \mathbb{R}^+$ are continuous, piecewise-$C^1$. The \textbf{winding number} of $\gamma$ about $\omega$ is defined as
	\[
		I(\gamma, \omega) = \frac{\theta(b) - \theta(a)}{2 \pi}
	\]
\end{definition}

\begin{example}
	Let $\gamma_1(t) = e^{2 \pi i t}$ for $t \in [0, 1]$. Then $r(t) = 1$, $\omega = 0$, $\theta(t) = 2 \pi t$. So
	\[
		I(\gamma_1, 0) = \frac{2 \pi \cdot 1 - 2 \pi \cdot 0}{2 \pi} = 1
	\]
	Let $\gamma_2(t) = e^{2 \pi i t}$ for $t \in [0, 2]$. Then
	\[
		I(\gamma_2, 0) = \frac{2 \pi \cdot 2 - 2 \pi \cdot 0}{2 \pi} = 2
	\]
\end{example}

\begin{remark}
	$w \notin \gamma$ since $r(t) > 0$, and $I(\gamma, \omega) \in \mathbb{Z}$ when $\gamma$ is closed, since if $\gamma$ is closed, then
	\[
		\begin{aligned}
			& \gamma(a) = \gamma(b) = \omega + r(a) e^{i \theta(a)} = \omega + r(b) e^{i \theta(b)} \\
			& \Longleftrightarrow r(b) = r(a) \text{ and } \theta(a) = \theta(b) + 2 \pi n \quad (n \in \mathbb{Z})	
		\end{aligned}
	\]
	Thus
	\[
		I(\gamma, \omega) = \frac{\theta(b) - \theta(a)}{2 \pi} \in \mathbb{Z}
	\]
\end{remark}

\begin{theorem}
	Let $\gamma: [a, b] \rightarrow \mathbb{C}$ be a contour. Then for every $\omega \in \mathbb{C}$ with $\omega \notin \gamma$, for some continuous, piecewise $C^1$, $\theta: [a, b] \rightarrow \mathbb{R}$ and $r: [a, b] \rightarrow \mathbb{R}^+$,
	\[
		\gamma(t) = \omega + r(t) e^{i \theta(t)}
	\]
\end{theorem}

\begin{proof}
	Omitted.
\end{proof}

\begin{lemma}\label{lem:formulaForClosedContourWindingNumber}
	Let $\gamma: [a, b] \rightarrow \mathbb{C}$ be a closed contour and $\omega \notin \gamma$. Then
	\[
		I(\gamma, \omega) = \frac{1}{2 \pi i} \int_{\gamma} \frac{1}{z - \omega} dz
	\]
\end{lemma}

\begin{proof}
	By theorem 8.2 (lecture notes),
	\[
		\gamma(t) = \omega + r(t) e^{i \theta(t)}
	\]
	By definition 6.4 (lecture notes),
	\[
		\begin{aligned}
			\int_{\gamma} \frac{1}{z - w} dz
				& = \int_{a}^{b} \frac{1}{\gamma(t) - \omega} \gamma'(t) dt \\
				& = \int_{a}^{b} \frac{1}{r(t) e^{i \theta(t)}} \left( r'(t) e^{i \theta(t)} + r(t) \cdot i \theta'(t) e^{i \theta(t)} \right) dt \\
				& = \int_{a}^{b} \left( \frac{r'(t)}{r(t)} + i \theta'(t) \right) dt \\
				& = {[\log(r(t)) + i \theta(t)]}_a^b = i (\theta(b) - \theta(a)) = 2 \pi i \cdot I(\gamma, \omega)
		\end{aligned}
	\]
\end{proof}

\begin{proposition}
	Let $D$ be a starlike domain. Then for every closed contour $\gamma$ and every $\omega \notin D$,
	\[
		I(\gamma, \omega) = 0
	\]
\end{proposition}

\begin{proof}
	By Lemma~\ref{lem:formulaForClosedContourWindingNumber},
	\[
		I(\gamma, \omega) = \frac{1}{2 \pi i} \int_{\gamma} \frac{1}{z - \omega} dz
	\]
	and $1 / (z - w)$ is holomorphic on $D$. So by CTSD, $I(\gamma, \omega) = 0$.
\end{proof}

\begin{definition}
	Let $U$ be an open set. A closed contour $\gamma$ in $U$ is \textbf{homologous to zero} if $I(\gamma, \omega) = 0$ for every $\omega \notin U$.
\end{definition}

\begin{definition}
	An open set $U$ is called \textbf{simply connected} if every closed contour in $U$ is homologous to zero.
\end{definition}

\begin{example}
	By proposition 8.4 (lecture notes), starlike domains are simply connected.
\end{example}

\begin{example}
	Let $A = \{ z \in \mathbb{C}: \alpha < |z| < \beta \}$. Let $\gamma(t) = \rho e^{2 \pi i t}$ for $t \in [0, 1]$ and $\alpha < \rho < \beta$. Pick $\omega = 0$ then $\omega 
	\notin A$ but
	\[
		\frac{1}{2 \pi i} \int_{\gamma} \frac{1}{z} dz = 1
	\]
	by CIF. Thus $A$ is not simply connected.
\end{example}

\begin{definition}
	A \textbf{cycle} $\Gamma$ defined on an open set $U$ is a finite collection of closed contours in $U$. We write
	\[
		\Gamma = \gamma_1 + \gamma_2 + \cdots + \gamma_n
	\]
\end{definition}

\begin{definition}
	Let $\Gamma$ be a cycle. $w \in \mathbb{C}$ is \textbf{not on $\Gamma$} ($\omega \notin \gamma_i$) if $w \notin \gamma_i$ for every $i$. The \textbf{winding number of $\Gamma$ around $\omega$} is defined as
	\[
		I(\Gamma, \omega) = \sum_{i = 1}^n I(\gamma_i, \omega)
	\]
	and we define
	\[
		\int_{\Gamma} f(z) dz = \sum_{i = 1}^n \int_{\gamma_i} f(z) dz
	\]
	$\Gamma$ is called \textbf{homologous to zero in $U$} if $I(\Gamma, \omega) = 0$ for every $\omega \notin U$.
\end{definition}

\begin{example}
	Let $A_{\alpha, \beta} (\omega) = \{ z \in \mathbb{C}: \alpha < |z - \omega| < \beta$ for $0 \le \alpha < \beta \le \infty \}$. Let $\Gamma$ be a cycle in $A_{\alpha, \beta} (\omega)$.

	Let $\gamma_1(t) = \omega + \rho_1 e^{-2 \pi i t}$ for $t \in [0, 1]$, $\gamma_2(t) = \omega + \rho_2 e^{2 \pi i t}$ for $t \in [0, 1]$ where $\alpha < \rho_1 < \rho_2 < \beta$. Define
	\[
		\Gamma = \gamma_1 + \gamma_2
	\]
	We claim that $\Gamma$ is homologous to zero in $A_{\alpha, \beta} (\omega)$. Let $\omega' \in A_{\alpha, \beta} (\omega)$. Consider $a \notin A_{\alpha, \beta} (\omega)$. Then
	\begin{itemize}
		\item If $|\omega - a| > \beta$ then
		\[
			\begin{aligned}
				I(\Gamma, a)
					& = I(\gamma_1, a) + I(\gamma_2, a) \\
					& = \frac{1}{2 \pi i} \int_{\gamma_1} \frac{1}{z - a} dz + \frac{1}{2 \pi i} \int_{\gamma_2} \frac{1}{z - a} dz
			\end{aligned}
		\]
		But then $1 / (z - a)$ is holomorphic. So by Cauchy's theorem both integrals vanish.
		\item If $|\omega - a| < \alpha$ then by CIF,
		\[
			\int_{\gamma_1} \frac{1}{z - a} dz = -1, \quad \int_{\gamma_2} \frac{1}{z - a} dz = 1
		\]
		so $I(\Gamma, a) = 0$.
	\end{itemize}
\end{example}

\subsection{General form of the Cauchy-Taylor theorem and CIF}

\begin{definition}
	A closed curve $\gamma: [a, b] \rightarrow \mathbb{C}$ is called \textbf{simple} if for all $t_1 < t_2$
	\[
		\gamma(t_1) = \gamma(t_2) \Longrightarrow t_1 = a \text{ and } t_2 = b
	\]
	i.e. it cannot cross itself or go back on itself.
\end{definition}

\begin{theorem}
	(\textbf{Jordan curve theorem}) Let $\gamma \subset \mathbb{C}$ be a simple closed curve. Then its complement $\mathbb{C} - \gamma$ is a disjoint union of two domains, exactly one of which is bounded.
\end{theorem}

\begin{proof}
	Beyond the scope of this course, so omitted.
\end{proof}

\begin{definition}
	The bounded domain is called the \textbf{interior} of $\gamma$ and we write $D_{\gamma}^{\text{int}}$. We say $w \in D_{\gamma}^{\text{int}}$ \textbf{lies inside} $\gamma$.
\end{definition}

\begin{definition}
	The other, non-bounded, domain is called the \textbf{exterior} of $\gamma$ and we write $D_{\gamma}^{\text{ext}}$. We say $w \in D_{\gamma}^{\text{ext}}$ \textbf{lies outside} $\gamma$.
\end{definition}

\begin{remark}
	$\mathbb{C} = D_{\gamma}^{\text{int}} \cup \gamma \cup D_{\gamma}^{\text{ext}}$ as a disjoint union.
\end{remark}

\begin{remark}
	Given a simple closed \textbf{contour}, it is always possible to place an orientation on $\gamma$ such that
	\[
		\forall w \in \mathbb{C} - \gamma, \quad I(\gamma, w) = \begin{cases}
			1 & \text{ if } w \in D_{\gamma}^{\text{int}} \\
			0 & \text{ if } w \in D_{\gamma}^{\text{ext}}
		\end{cases}
	\]
	We call $\gamma$ \textbf{positively oriented} if this equation holds.
\end{remark}

\begin{definition}
	$f$ is called \textbf{holomorphic on $D_{\gamma}^{\text{int}} \cup \gamma$} (and \textbf{holomorphic on and inside $\gamma$}) if for some domain $D$ containing $D_{\gamma}^{\text{int}} \cup \gamma$ on which $f$ is holomorphic.
\end{definition}

\begin{remark}
	For a simple closed curve, $\gamma$ is homologous to zero in $D$, since if $w \notin D$ then $w \in D_{\gamma}^{\text{ext}}$ so $I(\gamma, w) = 0$.
\end{remark}

\begin{theorem}
	Let $f: D \rightarrow \mathbb{C}$ be a holomorphic function on a domain $D$. Then for every cycle $\Gamma \in D$ such that $\Gamma$ is homologous to zero in $D$, for every $\omega \in D - \Gamma$, we have the \textbf{General form of the Cauchy-Taylor theorem}
	\[
		\int_{\Gamma} f(z) dz = 0
	\]
	and the \textbf{General form of the Cauchy integral formula}
	\[
		\int_{\Gamma} \frac{f(z)}{z - w} dz = 2 \pi i \cdot I(\Gamma, \omega) \cdot f(\omega)
	\]
\end{theorem}

\begin{example}
	By definition, if $D$ is simply connected, every cycle in $D$ is homologous to zero, so for every closed contour $\Gamma$,
	\[
		\int_{\Gamma} f(z) dz = 0
	\]
\end{example}

\begin{example}
	Let $D = B_r(a)$ and let $\gamma$ be $|z - a| = \rho$. Then for every $\omega \in B_r(a)$ with $|\omega - a| < \rho$,
	\[
		I(\gamma, \omega) = 1
	\]
	so
	\[
		\int_{\gamma} \frac{f(z)}{z - \omega} dz = 2 \pi i \cdot f(\omega)
	\]
\end{example}

\begin{theorem}
	Let $\gamma$ be a simple closed contour, positively oriented, and let $f$ be holomorphic on $D_{\gamma}^{\text{int}} \cup \gamma$. Then we have \textbf{Cauchy's theorem for simple closed curves}
	\[
		\int_{\gamma} f(z) dz = 0
	\]
	and \textbf{Cauchy's integral formula for simple closed curves}: if $w \in D_{\gamma}^{\text{int}}$ then
	\[
		\int_{\gamma} \frac{f(z)}{z - w} dz = 2 \pi i \cdot f(w)
	\]
\end{theorem}

\begin{remark}
	From here onwards, this will be the version of these two theorems which we use most often.
\end{remark}

\begin{example}
	Let $\gamma$ be the square with vertices at $1 + i, 1 - i, -1 + i, -1 - i$. Consider
	\[
		\int_{\gamma} \frac{\cos(z)}{z (z^2 + 2)} dz
	\]
	$f(z) = \cos(z) / (z^2 + 2)$ is holomorphic on $D_{\gamma}^{\text{int}} \cup \gamma$ so by CIF for simple closed curves,
	\[
		\int_{\gamma} \frac{f(z)}{z} dz = 2 \pi i \cdot f(0) = \pi i
	\]
\end{example}

\section{Holomorphic functions on punctured domains}

\subsection{Laurent series}

\begin{definition}
	A \textbf{Laurent series} is an infinite series of the form
	\[
		\sum_{n = -\infty}^{\infty} c_n {(z - a)}^n
	\]
	where $c_n \in \mathbb{C}$ and $a \in \mathbb{C}$. $a$ is called the \textbf{centre}. The sum
	\[
		\sum_{n = 0}^{\infty} c_n {(z - a)}^n
	\]
	is called the \textbf{analytic part} and the sum
	\[
		\sum_{n = -\infty}^{-1} c_n {(z - a)}^n
	\]
	is called the \textbf{principal part}.
\end{definition}

\begin{definition}
	We say a Laurent series \textbf{converges at $z \in \mathbb{C}$} iff the principal part and analytic part independently converge at $z$.
\end{definition}

\begin{remark}
	A Laurent series is a Taylor series if the principal part is $0$, otherwise, it is not defined at $z = a$.
\end{remark}

\begin{definition}
	For every $0 \le r < R \le \infty$, and $a \in \mathbb{C}$, the \textbf{annulus} of centre $a$, interior radius $r$ and external radius $R$ is defined as
	\[
		A_{r, R}(a) = \{ z \in \mathbb{C}: r < |z - a| < R \}
	\]
\end{definition}

\begin{proposition}
	Given a Laurent series with a non-zero principal part, then either
	\begin{itemize}
		\item The Laurent series converges nowhere \textbf{or}
		\item For some $r, R$, the Laurent series converges absolutely on the annulus $A_{r, R}(a)$ and does not converge for $|z - a| > R$ or $|z - a| < r$. We call $A_{r, R}(a)$ the \textbf{annulus of convergence}.
	\end{itemize}
\end{proposition}

\begin{proof}
	By definition, the Laurent series converges iff
	\[
		F_1(z) = \sum_{n = 0}^{\infty} c_n {(z - a)}^n, \quad F_2(z) = \sum_{n = -\infty}^{-1} c_n {(z - a)}^n
	\]
	both converge. By Theorem 5.15 (lecture notes), either $F_1$ converges at $z = a$ (but then the Laurent series converges nowhere as the principal is not defined at $z = a$) or for some $0 < R \le \infty$, $F_1$ converges absolutely for $|z - a| < R$.

	Now define $w = 1 / (z - a)$. Then
	\[
		F_2(z) = \sum_{n = -\infty}^{-1} c_n (z - a)^n = \sum_{n = -\infty}^{-1} c_n w^{-n} = \sum_{m = 1}^{\infty} c_{-m} w^m =: \tilde{F}(w)
	\]
	Either $\tilde{F}$ converges only at $w = 0$ (so the Laurent series converges nowhere) or for some $0 < R' \le \infty$ such that $\tilde{F}$ converges when $|w| < R$. Let
	\[
		r = \begin{cases}
			1 / R' & \text{ if } R' \ne \infty \\
			0 & \text{ if } R' = \infty
		\end{cases}
	\]
	Then $F_2$ converges when $|w| < R' \Longleftrightarrow |z - a| > $. TODO: check lecture notes. 
\end{proof}

\end{document}