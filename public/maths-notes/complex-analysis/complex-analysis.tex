\documentclass[12pt,a4paper]{article}
\AddToHook{cmd/section/before}{\clearpage}

\usepackage[a4paper, total={6.5in, 10in}]{geometry}
\usepackage[utf8]{inputenc}
\usepackage{amsfonts, amssymb, amsmath}
\usepackage{amsthm}
\usepackage[dvipsnames]{xcolor}
\usepackage{titlesec}
\usepackage{afterpage}
\usepackage{hyperref}
\usepackage{esint}
\usepackage{diffcoeff}

\pagecolor{white}
\color{black}% set the default colour to white

\theoremstyle{definition}
\newtheorem{definition}{Definition}[subsection]
\newtheorem{theorem}[definition]{Theorem}
\newtheorem{proposition}[definition]{Proposition}
\newtheorem{corollary}[definition]{Corollary}
\newtheorem{lemma}[definition]{Lemma}
\newtheorem{example}[definition]{Example}
\newtheorem*{remark}{Remark}

\title{Complex Analysis II Course Notes}
\author{Isaac Holt}

\begin{document}

\maketitle
\tableofcontents
\newpage

\newcommand{\Arg}{\text{Arg}}
\newcommand{\Log}{\text{Log}}
\newcommand{\biholo}{\mathrel{\overset{\sim}{\longrightarrow}}}
\newcommand{\Res}{\text{Res}}

\section{Summary notes}



\section{The complex plane and Riemann sphere}

\subsection{Complex numbers}
ff

\begin{definition}
	A \textbf{complex number} $z$ is a number $z = x + iy$ where $x, y \in \mathbb{R}^2$ and $i$ is the \textbf{imaginary unit}. The set of all complex numbers is written as $\mathbb{C}$.
\end{definition}

\begin{definition}
	For $z_1 = x_1 + i y_1$ and $z_2 = x_2 + i y_2$, addition, subtraction and multiplication of complex numbers is defined as
	\[
		\begin{aligned}
			z_1 \pm z_2 & := (x_1 \pm x_2) + i (y_1 \pm y_2) \\
			z_1 z_2 & := (x_1 x_2 - y_1 y_2) + i (x_1 y_2 + x_2 y_1)
		\end{aligned}
	\]
\end{definition}

\begin{definition}
	For a complex number $z = x + iy$, the \textbf{real part} of $z$, $\Re(z)$, is $x$ and the \textbf{imaginary part}, $\Im(z)$, is $y$.
\end{definition}

\begin{definition}
	For a complex number $z = x + iy$, the \textbf{complex conjugate} of $z$, $\overline{z}$ is defined as $\overline{z} = x - iy$.
\end{definition}

\begin{definition}
	Division of complex numbers $z_1 = x_1 + i y_1$ and $z_2 = x_2 + i y_2 \ne 0$ is given by
	\[
		\frac{z_1}{z_2} = \frac{x_1 + i y_1}{x_2 + i y_2} = \frac{(x_1 + i y_1) (x_2 - i y_2)}{(x_2 + i y_2) (x_2 - i y_2)} = \frac{x_1 x_2 + y_1 y_2}{x_2^2 + y_2^2} + i \frac{x_2 y_1 - x_1 y_2}{x_2^2 + y_2^2}
	\]
	This gives a multiplicative inverse for every $z = x + iy \ne 0$:
	\[
		z^{-1} = \frac{1}{z} = \frac{x}{x^2 + y_2} - i \frac{y}{x^2 + y^2}
	\]
\end{definition}

\begin{definition}
	The \textbf{modulus} or \textbf{absolute value} of a complex number $z = x + iy$, $|z|$, is defined as $|z| = \sqrt{x^2 + y^2}$.
\end{definition}

\begin{lemma}
	\hfill
	\begin{enumerate}
		\item $\forall z_1, z_2 \in \mathbb{C}^2, z_1 z_2 \Longleftrightarrow z_1 = 0 \text{ or } z_2 = 0$.
		\item $\forall z \in \mathbb{C}, |z| = \sqrt{z \overline{z}}$.
		\item $\Re(z) = \frac{z + \overline{z}}{2}$ and $\Im(z) = \frac{z - \overline{z}}{2i}$.
		\item $z^{-1} = \frac{\overline{z}}{|z|^2}$.
	\end{enumerate}
\end{lemma}

\begin{proof}
	Easy.
\end{proof}

\begin{definition}
	For a complex number $z = x + iy$ plotted on an Argand diagram (where $z$ is at the point $(x, y)$), the \textbf{argument} of $z$, $\arg(z)$, is the anticlockwise angle $\theta$ from the real axis to $z$.
\end{definition}

\begin{definition}
	For a complex number $z$ with $|z| = r$ and $\arg(z) = \theta$, $z$ can be written in polar coordinates:
	\[
		z = r(\cos(\theta) + i \sin(\theta)) = r e^{i \theta}
	\]
\end{definition}

\begin{definition}
	$\arg(z)$ is only defined up to multiples of $2 \pi$. The \textbf{principal value} of $\arg(z)$ is the value of $\arg(z)$ in the interval $\left(-\pi, \pi \right]$, written as $\Arg(z)$.
\end{definition}

\begin{lemma}
	\hfill
	\begin{enumerate}
		\item $\arg(z_1 z_2) = \arg(z_1) + \arg(z_2) \pmod{2 \pi}$.
		\item $\arg(1 / z) = -\arg(z) \pmod{2 \pi}$.
		\item $\arg(\overline{z}) = -\arg(z) \pmod{2 \pi}$.
	\end{enumerate}
\end{lemma}

\begin{proof}
	Easy.
\end{proof}

\begin{lemma}
	Multiplication in $\mathbb{C}$ can be geometrically described as a dilated rotation: if $z_1 = r_1 e^{i \theta_1}$ and $z_2 = r_2 e^{i \theta_2}$ then
	\[
		z_1 z_2 = r_1 r_2 e^{i (\theta_1 + \theta_2)}
	\]
\end{lemma}

\begin{proof}
	\[
		\begin{aligned}
			z_1 z_2
				& = r_1 r_2 (\cos(\theta_1) + i \sin(\theta_1)) (\cos(\theta_2) + i \sin(\theta_2)) \\
				& = r_1 r_2 ((\cos(\theta_1) \cos(\theta_2) - \sin(\theta_1) \sin(\theta_2)) + i (\cos(\theta_2) \sin(\theta_1) + \sin(\theta_2) \cos(\theta_1))) \\
				& = r_1 r_2 (\cos(\theta_1 + \theta_2) + i \sin(\theta_1 + \theta_2)) \\
				& = r_1 r_2 e^{i (\theta_1 + \theta_2)}
		\end{aligned}
	\]
\end{proof}

\begin{remark}
	\hfill
	\begin{itemize}
		\item Multiplying $z_1$ by $z_2$ represents a rotation of $z_1$ by $\theta_2$ anticlockwise, followed by a dilation of factor $r_2$.
		\item Addition represents to translation.
		\item Complex represents reflection in the real axis.
		\item Taking the real part represents projection onto the real axis.
		\item Taking the imaginary part represents projection onto the imaginary axis.
	\end{itemize}
\end{remark}

\begin{corollary}
	\hfill
	\begin{enumerate}
		\item $\forall z_1, z_2 \in \mathbb{C}^2, |z_1 z_2| = |z_1| |z_2|$.
		\item ${(\cos(\theta) + i \sin(\theta))}^n = \cos(n \theta) + i \sin(n \theta)$ (de Moivre's formula).
		\item $\forall z_1, z_2, |z_1 + z_2| \le |z_1| + |z_2|$.
		\item $\forall z \in \mathbb{C}, |z| \ge 0$ and $|z| = 0 \Longleftrightarrow z = 0$.
		\item $\max \{ |\Re(z)|, |\Im(z)| \} \le |z| \le |\Re(z)| + |\Im(z)|$.
	\end{enumerate}
\end{corollary}

\begin{proof}
	Easy.
\end{proof}

\begin{definition}
	The \textbf{upper half} of the complex plane, $\mathbb{H}$, is defined as
	\[
		\mathbb{H} = \{ z \in \mathbb{C}: \Im(z) > 0 \}
	\]
\end{definition}

\subsection{Exponential and trigonometric functions}

\begin{definition}
	The \textbf{complex exponential function} $\exp: \mathbb{C} \rightarrow \mathbb{C}$, written as $e^z$, is defined as
	\[
		e^z = \exp(z) := e^x (\cos(y) + i \sin(y))
	\]
\end{definition}

\begin{proposition}\label{prop:expProperties}
	\hfill
	\begin{enumerate}
		\item $\forall z \in \mathbb{C}, e^z \ne 0$.
		\item $\forall z_1, z_2 \in \mathbb{C}^2, e^{z_1 + z_2} = e^{z_1} e^{z_2}$.
		\item $e^z = 1 \Longleftrightarrow \exists k \in \mathbb{Z}, z = 2 \pi i k$.
		\item $e^{-z} = 1 \ e^z$.
		\item $|e^z| = e^{\Re(z)}$.
	\end{enumerate}
\end{proposition}

\begin{proof}
	Easy. For 3, $\exp(z) = 1 \Longleftrightarrow e^x \cos(y) = 1$ and $e^x \sin(y) = 0$. $e^x > 0$ so $\sin(y) = 0$ so $y = n \pi$ for some $n \in \mathbb{Z}$. So $1 = e^x \cos(n \pi) = e^x {(-1)}^n$ so $n$ is even and $x = 0$.
\end{proof}

\begin{corollary}\label{cor:EulersFormula}
	$\exp(2 \pi i) = 1$ and $\exp(\pi i) = -1$ (Euler's formula).
\end{corollary}

\begin{corollary}
	$\forall k \in \mathbb{Z}, \forall z \in \mathbb{C}, \exp(z + 2k \pi i) = \exp(z)$.
\end{corollary}

\begin{definition}
	The following functions from $\mathbb{C}$ to $\mathbb{C}$ are defined:
	\[
		\begin{aligned}
			\sin(z) & := \frac{1}{2i} \left( e^{iz} - e^{-iz} \right) \\
			\cos(z) & := \frac{1}{2} \left( e^{iz} + e^{-iz} \right) \\
			\sinh(z) & := \frac{1}{2} \left( e^{z} - e^{-z} \right) = -i \sin(iz) \\
			\cosh(z) & := \frac{1}{2} \left( e^{z} + e^{-z} \right) = \cos(iz)
		\end{aligned}
	\]
\end{definition}

\begin{remark}
	The usual trigonometric function identities hold, e.g. $\cos(z)^2 + \sin(z)^2 = 1$.
\end{remark}

\subsection{Logarithms and complex powers}

\begin{definition}
	We write the \textbf{set of non-zero complex numbers} as
	\[
		\mathbb{C}^* = \mathbb{C} - \{ 0 \}
	\]
\end{definition}

\begin{lemma}
	For every $w \in \mathbb{C}^*$, $e^z = w$ has a solution for $z$. Let $w = |w| e^{i \theta}$, $\theta = \Arg(w)$. Then every solution for $z$ is given by
	\[
		z = \log(|w|) + i (\theta + 2 \pi k), \quad k \in \mathbb{Z}
	\]
\end{lemma}

\begin{proof}
	By Proposition~\ref{prop:expProperties} (part 2) and Corollary~\ref{cor:EulersFormula},
	\[
		w = |w| e^{i \theta} = e^{\log(|w|)} e^{i \theta} = e^{\log(|w|)} e^{i (\theta + 2 \pi k)} = e^{\log(|w|) + i (\theta + 2 \pi k)} = e^z
	\]
	Let $z = x + iy$, then
	\[
		e^z = e^x e^{iy} = w = |w| e^{i \theta} \Longrightarrow |e^z| = e^x = |w| \Longrightarrow x = \log(|w|)
	\]
	Hence $e^{iy} = e^{i \theta}$ so $e^{i(y - \theta)} = 1$ so $y - \theta = 2 \pi k$ for some $k \in \mathbb{Z}$ by Proposition~\ref{prop:expProperties} (part 3).
\end{proof}

\begin{definition}
	Let $\theta_1 < \theta_2$ with $\theta_2 - \theta_1 = 2 \pi$. Let $\arg$ be the argument function with values in $(\theta_1, \theta_2]$. Then
	\[
		\log(z) := \log(|z|) + i \arg(z)
	\]
	is called a \textbf{branch of logarithm}. It has a jump discontinuity on the ray $R_{\theta_1} = R_{\theta_2}$. This ray is called a \textbf{branch cut}.
\end{definition}

\begin{definition}
	Choosing $\theta_1 = -\pi$ and $\theta_2 = \pi$, so that $\arg = \Arg$, gives the \textbf{principal branch of logarithm}
	\[
		\Log(z) := \log(|z|) + i \Arg(z)
	\]
	which has a jump discontinuity on the ray $\mathbb{R}_{\le 0}$ (the non-positive real axis).
\end{definition}

\begin{remark}
	The prinical branch, $\Log$, matches the definition of $\log$ for real numbers, so is the branch that should be used, unless otherwise stated.
\end{remark}

\begin{lemma}
	\hfill
	\begin{enumerate}
		\item $\forall z \in \mathbb{C}^*, e^{\log(z)} = z$.
		\item Generally, $\log(zw) \ne \log(z) + \log(w)$.
		\item Generally, $\log(e^z) \ne z$.
	\end{enumerate}
\end{lemma}

\begin{definition}
	For a fixed $w \in \mathbb{C}^*$, we can choose any branch of log to define a \textbf{complex power function} by
	\[
		z^w = \exp(w \log(z))
	\]
\end{definition}

\begin{remark}
	The complex power function depends on the branch of log we choose.
\end{remark}

\subsection{The Riemann sphere and extended complex plane}

\begin{definition}
	The \textbf{unit sphere} $S^2$ is defined as
	\[
		S^2 := \{ (x, y, s) \in \mathbb{R}^3: x^2 + y^2 + s^2 = 1 \}
	\]
\end{definition}

\begin{definition}
	We define $N = (0, 0, 1) \in S^2$ to be the \textbf{north pole}. For every point $v \in S^2 - \{ N \}$, there is a unique straight line $L_{N, v}$ passing through $N$ and $v$. This is not parallel to the $(x, y)$-plane as $v \ne N$, hence $L_{N, v}$ intersects the $(x, y)$-plane at a unique point $(x, y, 0)$ which corresponds to the point $x + iy \in \mathbb{C}$.
\end{definition}

\begin{definition}
	The \textbf{stereographic projection} map $P: S^2 - \{ N \} \rightarrow \mathbb{C}$ is defined as
	\[
		P(v) = x + iy
	\]
	where $x + iy$ is the complex number corresponding to the point $(x, y, 0)$ where the line passing through $N$ and $v$, $L_{N, v}$ intersects the $(x, y)$-plane.
	
	The equation $L_{N, v}$ is
	\[
		\gamma(t) = N + ((x, y, s) - N) t = (0, 0, 1) + (x, y, s - 1) t
	\]
	which intersects the $(x, y)$-plane at $t = 1 / (1 - s)$. So
	\[
		P(x, y, s) = \frac{x}{1 - s} + i \frac{y}{1 - s}
	\]
\end{definition}

\begin{definition}
	The \textbf{inverse stereographic projection}, the inverse of $P$, is
	\[
		P^{-1}(z) = \frac{1}{1 + |z|^2} (2 \Re(z), 2 \Im (z), |z|^2 - 1)
	\]
\end{definition}

\begin{remark}
	$P$ is a bijection as it has an inverse.
\end{remark}

\begin{definition}
	The \textbf{extended complex plane} is defined as
	\[
		\hat{\mathbb{C}} = \mathbb{C} \cup \{ \infty \}
	\]
\end{definition}

\begin{remark}
	$N$ corresponds to $\infty \in \hat{\mathbb{C}}$ under the stereographic projection. So $\hat{\mathbb{C}}$ can be thought of as the entire sphere $S^2$.
\end{remark}

\begin{remark}
	The south pole, $(0, 0, -1)$ could also be used to define a different, valid projection.
\end{remark}

\begin{proposition}
	The following are correspondences between $S^2$ and $\hat{\mathbb{C}}$:
	\[
		\begin{aligned}
			N & \longleftrightarrow \infty \\
			S & \longleftrightarrow 0 \\
			\text{equator} & \longleftrightarrow \text{unit circle: } \{ z \in \mathbb{C}: |z| = 1 \} \\
			\text{open Southern hemisphere} & \longleftrightarrow \text{unit disc: } \{ z \in \mathbb{C}: |z| < 1 \} \\
			\text{open Northern hemisphere} & \longleftrightarrow \hat{\mathbb{C}} - \overline{B_1}(0) = \hat{\mathbb{C}} - \{ z \in \mathbb{C}: |z| \le 1 \} \\
		\end{aligned}
	\]
\end{proposition}

\section{Metric Spaces}

\subsection{Metric spaces}

\begin{definition}\label{def:metricSpace}
	A \textbf{metric space} is a set $X$ together with a function $d: X \times X \rightarrow \mathbb{R}_{\ge 0}$ that satisfies, for every $x, y, z \in X^3$,
	\begin{enumerate}
		\item \textbf{(D1) Positivity}: $d(x, y) \ge 0$ and $d(x, y) = 0 \Longleftrightarrow x = y$.
		\item \textbf{(D2) Symmetry}: $d(x, y) = d(y, x)$.
		\item \textbf{(D3) Triangle inequality}: $d(x, z) \le d(x, y) + d(y, z)$.
	\end{enumerate}
	$d$ is called a \textbf{metric}. A metric space is written as $(X, d)$.
\end{definition}

\begin{example}
	Let $X = \mathbb{C}$ and $d(x, y) = |x - y|$. Then $(X, d)$ is a metric space.
\end{example}

\begin{example}
	Let $X = \mathbb{C}^n$ and
	\[
		d(\underline{x}, \underline{y}) = ||x - y||_2 = \sqrt{\sum_{i = 1}^n |x_i - y_i|^2}
	\]
	Then $(X, d)$ is a metric space.
\end{example}

\begin{example}
	Let $V$ be a finite dimensional vector space with an inner product $\langle \cdot, \cdot \rangle$, then
	\[
		d(\underline{x}, \underline{y}) = ||x - y|| = \sqrt{\langle x - y, x - y \rangle}
	\]
	is a metric.
\end{example}

\begin{definition}
	Let $V$ be a real or complex vector space. A function $||\cdot||: V \rightarrow \mathbb{R}_{\ge 0}$ is called a \textbf{norm} if it satisfies, for every $v, w \in V^2$ and $\lambda \in \mathbb{C}$ or $\mathbb{R}$:
	\begin{enumerate}
		\item \textbf{(N1) Positivity}: $||v|| \ge 0$ and $||v|| = 0 \Longleftrightarrow v = 0$.
		\item \textbf{(N2) Linearity in scalar multiplication}: $||\lambda v|| = |\lambda| ||v||$.
		\item \textbf{(N3) Triangle inequality}: $||v + w|| \le ||v|| + ||w||$.
	\end{enumerate}
\end{definition}

\begin{definition}
	Property N3 implies the \textbf{reverse triangle inequality}:
	\[
		||v - w|| \ge | \ ||v|| - ||w|| \ |
	\]
\end{definition}

\begin{definition}
	A vector space equipped with a norm is called a \textbf{normed vector space}.
\end{definition}

\begin{remark}
	A normed vector space together with $d(v, w) = ||v - w||$ is always a metric space.
\end{remark}

\begin{example}
	For every $p \ge 1$, the \textbf{$l_p$-norm} is defined on vectors in $\mathbb{C}^n$ by
	\[
		||\underline{x}||_p := \sqrt[p]{\sum_{i = 1}^{n} |x_i|^p}
	\]
	The \textbf{Taxicab norm} is $l_p$ norm when $p = 1$.
\end{example}

\begin{example}
	The \textbf{$l_{\infty}$-norm} (or \textbf{sup-norm}) is defined as
	\[
		||\underline{x}||_{\infty} := \max_{i = 1, \dots, n} |x_i|
	\]
\end{example}

\begin{example}
	The \textbf{Riemannian metric} (or \textbf{chordal metric}) is defined as
	\[
		d(z, w) := ||f(z) - f(w)||_2
	\]
	where $z, w \in \hat{\mathbb{C}}$ and $f: \hat{\mathbb{C}} \rightarrow S^2$ is the inverse stereographic projection.
\end{example}

\begin{definition}
	Let $X$ be a non-empty finite set. The \textbf{discrete metric} on $X$ is defined as
	\[
		d(x, y) := \begin{cases}
			0 & \text{if } x = y \\
			1 & \text{if } x \ne y
		\end{cases}
	\]
	$(X, d)$ is called a \textbf{discrete metric space}.
\end{definition}

\begin{example}
	Let $X = C([a, b])$ be the space of continuous functions on $[a, b]$. Then
	\[
		||f|| := \max_{x \in [a, b]} |f(x)|
	\]
	defines a norm, so is a metric.
\end{example}

\begin{example}
	Let $(X, d)$ be a metric space. Every non-empty subset $Y \subset X$ also forms a metric space with $(Y, d)$. The metric restricted to $Y$ is called the \textbf{subspace metric}.
\end{example}

\subsection{Open and closed sets}

\begin{definition}
	Let $(X, d)$ be a metric space, $x \in X$ and $r > 0$ be a real number. The \textbf{open ball $B_r(x)$ of radius $r$ centred at $x$} is defined as
	\[
		B_r(x) := \{y \in X: d(x, y) < r\}
	\]
\end{definition}

\begin{definition}
	Let $(X, d)$ be a metric space, $x \in X$ and $r > 0$ be a real number. The \textbf{closed ball $B_r(x)$ of radius $r$ centred at $x$} is defined as
	\[
		\overline{B_r}(x) := \{y \in X: d(x, y) \le r\}
	\]
\end{definition}

\begin{example}
	Let $X = \mathbb{C}$ and $d(z, w) = |z - w|$. Then $B_1(0) = \mathbb{D} = \{ z: |z| < 1 \}$, the unit disc.
\end{example}

\begin{example}
	Let $X = \mathbb{R}^2$. For $l_2$-norm, the unit ball $B_1(\underline{0})$ is the inside of the unit circle centred at the origin.

	For the $l_{\infty}$-norm, $B_1(\underline{0})$ is the inside of the square with vertices $(1, 1), (-1, 1), (-1, -1), (1, -1)$, since $\max{ |x|, |y| } < 1$ in this ball.

	For the $l_1$-norm, in $B_1(0)$, we have $|x| + |y| < 1$ so in the 1st quadrant, $y < 1 - x$, in the 2nd quadrant, $y < 1 + $, in the 3rd quadrant, $y > -1 - x$, and in the 4th quadrant, $y > -1 + x$. So the unit ball is the inside of the diamond with vertices $(1, 0), (0, 1), (-1, 0), (0, -1)$.
\end{example}

\begin{definition}\label{def:openSet}
	Let $(X, d)$ be a metric space. $U \subset X$ is called \textbf{open} (in $X$) if for every $x \in U$, for some $\epsilon > 0$, $B_{\epsilon}(x) \subset U$.
\end{definition}

\begin{definition}\label{def:closedSet}
	Let $(X, d)$ be a metric space. $U \subset X$ is called \textbf{closed} (in $X$) its complement in $X$, $X - U$, is open.
\end{definition}

\begin{definition}
	Sets in a metric space that are both open and closed are called \textbf{clopen}.
\end{definition}

\begin{example}
	$\emptyset$ and $X$ are clopen in any metric space.
\end{example}

\begin{lemma}\label{lem:openBallsAreOpen}
	In a metric space, the open ball $B_r(x)$ is open.
\end{lemma}

\begin{proof}
	Let $y \in B_r(x)$ and let $s := d(x, y) < r$. Let $\epsilon = r - s > 0$. Then for every $z \in B_{\epsilon}(y)$,
	\[
		d(x, z) \le d(x, y) + d(y, z) < s + \epsilon = r
	\]
	So $z \in B_r(x)$.
\end{proof}

\begin{example}
	$\mathbb{H}, \mathbb{D}, \mathbb{C}^*, \mathbb{C} - \mathbb{R}_{\le 0}$ are all open. The 1st quadrant $\Omega_1 := \{ z \in \mathbb{C}: \Re(z) > 0, \Im(z) > 0 \}$ is open, since for every $z \in \Omega_1$, let $r = \min( \Re(z), \Im(z) ) > 0$, then $B_r(z) \subset \Omega_1$.
\end{example}

\begin{example}
	Let $(X, d)$ be a discrete metric space, so
	\[
		d(x, y) = \begin{cases}
			0 & \text{if } x = y \\
			1 & \text{if } x \ne y
		\end{cases}
	\]
	For every $x \in X$ and $r > 0$,
	\[
		B_r(x) = \begin{cases}
			\{x\} & \text{ if } r \le 1 \\
			X & \text{ if } r > 1
		\end{cases}
	\]
	So by Lemma~\ref{lem:openBallsAreOpen}, every singleton $\{ x \}$ is an open set with respect to the discrete metric. But also, $X - \{ x \}$ is open, since for every $y \in X - \{ x \}$ and $r < 1$, $y \ne x$ so $B_r(y) = \{ y \} \subset X - \{ x \}$.

	So all balls are clopen with respect to the discrete metric.
\end{example}

\begin{example}
	$[0, 1)$ is neither open nor closed in $\mathbb{R}$ with respect to the standard metric $|\cdot|$, since $0 \in [0, 1)$ doesn't have a ball in $[0, 1)$ containing it, but $1 \in \mathbb{R} - [0, 1) = (-\infty, 0) \cup [1, \infty)$ also doesn't have a ball in $\mathbb{R} - [0, 1)$ containing it.
\end{example}

\begin{lemma}\label{lem:unionsAndIntersectionsOpen}
	Let $(X, d)$ be a metric space.
	\begin{enumerate}
		\item Arbitrary unions of open sets are open, i.e. for every (finite or infinite) collection of open sets $U_i \subseteq X$, the union
		\[
			\bigcup_{i} U_i
		\]
		is open.
		\item Finite intersections of open sets are open, i.e. for every finite collection of open sets $U_i \subseteq X$, the intersection
		\[
			\bigcap_{i = 1}^n U_i
		\]
		is open.
	\end{enumerate}
\end{lemma}

\begin{proof}
	\hfill
	\begin{enumerate}
		\item Let $x \in \bigcup_{i} U_i$. Then for some $j$, $x \in U_j$. $U_j$ is open so for some $\epsilon > 0$, $B_{\epsilon} (x) \subseteq U_j \subseteq \bigcup_{i} U_i$.
		\item Let $x \in \bigcap_{i = 1}^n U_i$. For every $i$, $U_i$ is open so for some $r_i > 0$, $B_{r_i} (x) \subseteq U_i$. Let $\epsilon = \min \{ r_1, \ldots, r_n \} > 0$, so $B_{\epsilon}(x) \subseteq B_{r_i}(x)$ for every $i$. Then
		\[
			B_{\epsilon}(x) \subseteq \bigcap_{i = 1}^n B_{r_i}(x) \subseteq \bigcap_{i = 1}^n U_i
		\]
	\end{enumerate}
\end{proof}

\begin{corollary}
	Let $(X, d)$ be a metric space. Then
	\begin{enumerate}
		\item Finite unions of closed sets are closed.
		\item Arbitrary intersections of closed sets are closed.
	\end{enumerate}
\end{corollary}

\begin{proof}
	Use De Morgan's laws and Lemma~\ref{lem:unionsAndIntersectionsOpen}.
\end{proof}

\begin{remark}
	Infinite intersections of open sets are not always open, and infinite unions of closed sets are not always closed. For example,
	\[
		\bigcup_{n = 1}^{\infty} \left[ \frac{1}{i}, 1 - \frac{1}{i} \right] = (0, 1)
	\]
	is an infinite union of closed sets but $(0, 1)$ is open in $\mathbb{R}$.
\end{remark}

\begin{definition}
	Let $(X, d)$ be a metric space and $A \subseteq X$. The \textbf{interior} of $A$, $A^0$ is defined as
	\[
		A^0 := \{ x \in A; \text{ for some open set } U \subseteq A, x \in U \}
	\]
\end{definition}

\begin{definition}
	Let $(X, d)$ be a metric space and $A \subseteq X$. The \textbf{closure} of $A$, $\bar{A}$ is defined as the complement of the interior of the complement:
	\[
		\bar{A} := \{ x \in X: \text{ for every open set } U, x \in U \Longrightarrow U \cap A \ne \emptyset \} = X - {(X - A)}^0
	\]
\end{definition}

\begin{definition}
	Let $(X, d)$ be a metric space and $A \subseteq X$. The \textbf{boundary} of $A$, $\partial A$ is defined as the closure without the interior:
	\[
		\partial A := \bar{A} - A^0 = X - (A^0 \cup {(X - A)}^0)
	\]
\end{definition}

\begin{definition}
	Let $(X, d)$ be a metric space and $A \subseteq X$. The \textbf{exterior} of $A$, $A^e$, is the defined as the complement of the closure:
	\[
		A^e := X - \bar{A} = X - (A^0 \cup \partial A) = {(X - A)}^0
	\]
\end{definition}

\begin{remark}
	The interior and exterior are open and the boundary and closure are closed.
\end{remark}

\begin{remark}
	The interior of a set contains the points that are not on its ``edge''. Adding the missing edge points to the interior gives the closure.
\end{remark}

\begin{proposition}
	For a subset $A \subset X$,
	\begin{enumerate}
		\item \[
			A \text{ is open} \Longleftrightarrow \partial A \cap A = \emptyset \Longleftrightarrow A = A^0 = \bigcup_{U \subseteq A, U \text{ open}} U
		\].
		\item \[
			A \text{ is closed} \Longleftrightarrow \partial A \subseteq A \Longleftrightarrow A = \bar{A} = \bigcap_{A \subseteq F, F \text{ closed}} F
		\]
	\end{enumerate}
\end{proposition}

\begin{remark}
	The interior $A^0$ is the largest open set contained in $A$ and the closure $\bar{A}$ is the smallest closed set containing $A$.
\end{remark}

\begin{remark}
	For simple sets in $\mathbb{R}^n$ or $\mathbb{C}^n$, the closure can be obtained simply by replacing strict inequalities with inequalities, and the interior can be obtained by replacing inequalities with strict inequalities.
\end{remark}

\begin{example}
	Let $A = \{ z \in \mathbb{C}: 1 < |z| \le 3 \}$. Then
	\[
		\begin{aligned}
			A^0 & = \{ z \in \mathbb{C}: 1 < |z| < 3 \} \\
			\bar{A} & = \{ z \in \mathbb{C}: 1 \le |z| \le 3 \} \\
			\partial A & = \{ z \in \mathbb{C}: |z| = 1 \} \cup \{ z \in \mathbb{C}: |z| = 3 \}
		\end{aligned}
	\]
\end{example}

\begin{example}
	For every ball $B_r(x)$ in $\mathbb{R}^n$ or $\mathbb{C}^n$, the closure of $B_r(x)$ is equal to the closed ball:
	\[
		\overline{B_r(x)} = \bar{B}_r(x)
	\]
	Note that this does not hold for every metric space.
\end{example}

\subsection{Convergence and continuity}

\begin{definition}
	Let $(X, d)$ be a metric space. A sequence $\{ x_n \}$ in $X$ is said to \textbf{converge} to a point $x \in X$ if
	\[
		\lim_{n \rightarrow \infty} d(x_n, x) = 0
	\]
	Equivalently,
	\[
		\forall \epsilon > 0, \ \exists N \in \mathbb{N}, \ \forall n > N, \quad d(x_n, x) < \epsilon
	\]
	We write $x_n \rightarrow x$ as $n \rightarrow \infty$ or $\lim_{n \rightarrow \infty} x_n = x$.
\end{definition}

\begin{proposition}
	Limits in the complex plane follow the Calculus of Limits Theorem (COLT) rules.
\end{proposition}

\begin{proposition}
	Let $\{ z_n \}_{n \in \mathbb{N}}$ be a sequence of complex numbers $z_n = x_n + i y_n$. Let $z_0 = x_0 + i y_0$. Then
	\[
		\lim_{n \rightarrow \infty} z_n = z_0 \Longleftrightarrow \lim_{n \rightarrow \infty} x_n = x_0 \text{ and } \lim_{n \rightarrow \infty} y_n = y_0
	\]
	So $\{ z_n \}_{n \in \mathbb{N}}$ converges iff the real sequences $\{ \Re(z_n) \}_{n \in \mathbb{N}}$ and $\{ \Im(z_n) \}_{n \in \mathbb{N}}$ converge.
\end{proposition}

\begin{lemma}\label{lem:sequenceLimitUnique}
	Let $(X, d)$ be a metric space. Then
	\begin{enumerate}
		\item A sequence $\{ x_n \}_{n \in \mathbb{N}}$ in $X$ has at most one limit.
		\item \[
			\lim_{n \rightarrow \infty} x_n = x \Longleftrightarrow \forall \text{ open } U \text{ with } x \in U, \ \exists N \in \mathbb{N}, \ \forall n > N, \ x_n \in U
		\]
	\end{enumerate}
\end{lemma}

\begin{proof}
	\hfill
	\begin{enumerate}
		\item Let $x$ and $y$ be two limits of $\{ x_n \}_{n \in \mathbb{N}}$. Then for every $n$, by the triangle inequality, $d(x, y) \le d(x, x_n) + d(x_n, y)$, so
		\[
			d(x, y) \le \lim_{n \rightarrow \infty} d(x, x_n) + \lim_{n \rightarrow \infty} d(x_n, y) = 0 + 0 = 0
		\]
		so $d(x, y) = 0 \Longleftrightarrow x = y$ by \hyperref[def:metricSpace]{property (D2)} of a metric space.
		\item ($\Rightarrow$): Let $x = \lim_{n \rightarrow \infty} x_n$, let $U$ be open and $x \in U$. By definition, for some $r > 0$, $B_r(x) \subseteq U$ and for some $N \in \mathbb{N}$, for every $n > N$, $d(x_n, x) < r$. So for every $n > N$, $x_n \in B_r(x) \subseteq U$.
		($\Leftarrow$): Let $\epsilon > 0$. We want to find an $N \in \mathbb{N}$ such that $d(x_n, x) < \epsilon$ for every $n > N$. The ball $B_{\epsilon}(x)$ is open and $x \in B_{\epsilon}(x)$. So for some $N \in \mathbb{N}$, $x_n \in B_{\epsilon}(x)$ for every $n > N$. But this means $d(x_n, x) < \epsilon$ for every $n > N$.
	\end{enumerate}
\end{proof}

\begin{definition}\label{def:continuousMap}
	A map $f: (X_1, d_1) \rightarrow (X_2, d_2)$ is called \textbf{continuous} at $x_0 \in X_1$ if
	\[
		\forall \epsilon > 0, \ \exists \delta > 0, \ \forall x \in X_1, \quad d_1(x, x_0) < \delta \Longrightarrow d_2(f(x), f(x_0)) < \epsilon
	\]
	$f$ is \textbf{continuous on $X_1$} if it is continuous at every $x_0 \in X_1$.
\end{definition}

\begin{remark}
	An equivalent definition of continuity at $x_0$ is
	\[
		\forall \epsilon > 0, \ \exists \delta > 0, \quad x \in B_{\delta}(x_0) \Longrightarrow f(x) \in B_{\epsilon}(f(x_0))
	\]
	where $B_{\delta}(x_0) \subseteq X_1$ and $B_{\epsilon}(f(x_0)) \subseteq X_2$.
\end{remark}

\begin{lemma}\label{lem:continuousFunctionsBasicProperties}
	(Basic properties of continuous functions)
	\begin{enumerate}
		\item Products, sum and quotients of real or complex valued continuous functions on a metric space $X$ are also continuous. So let $f: X \rightarrow \mathbb{C}$, $g: X \rightarrow \mathbb{C}$, then $f + g$, $fg$ and $f / g$ are continuous.
		\item Compositions of continuous functions are continuous. So let $f: X_1 \rightarrow X_2$ and $g: X_2 \rightarrow X_3$ be continuous, then $g \circ f: X_1 \rightarrow X_3$ is continuous.
	\end{enumerate}
\end{lemma}

\begin{proof}
	Same as Analysis I.
\end{proof}

\begin{example}
	The following functions are continuous:
	\begin{itemize}
		\item The identity function.
		\item Constant functions.
		\item $\Re$ and $\Im$.
		\item Complex conjugation, $z \rightarrow \overline{z}$.
		\item The modulus function, $z \rightarrow |z|$.
		\item Every polynomial.
		\item $\exp, \sin, \cos, \sinh, \cosh$.
		\item The $\arg$ function that takes values in $(\theta_1, \theta_2]$ is continuous on $\mathbb{C} - R_{\theta_1}$ where $R_{\theta_1}$ is the ray with angle $\theta_1$.
		\item The $\log$ function corresponding to the $\arg$ function above is continuous on $\mathbb{C} - R_{\theta_1}$.
	\end{itemize}
\end{example}

\begin{definition}
	Let $f: X_1 \rightarrow X_2$ and $U \subseteq X_2$. The \textbf{preimage of $U$ under $f$}, $f^{-1}(U)$, is defined as
	\[
		f^{-1}(U) := \{ x \in X_1: f(x) \in U \}
	\]
\end{definition}

\begin{theorem}\label{thm:continuityViaOpenSets}
	Let $(X_1, d_1)$ and $(X_2, d_2)$ be metric spaces. Then the following statements are all equivalent:
	\begin{enumerate}
		\item $f: X_1 \rightarrow X_2$ is continuous.
		\item $f^{-1}(U)$ is open in $X_1$ for every open set $U \subseteq X_2$.
		\item $f^{-1}(F)$ is closed in $X_1$ for every closed set $F \subseteq X_2$.
	\end{enumerate}
\end{theorem}

\begin{proof}
	(1. $\Rightarrow$ 2.): Let $U \subseteq X_2$ be open and let $x \in f^{-1}(U)$. $U$ is open and $f(x) \in U$ so by Definition~\ref{def:openSet}, for some $\epsilon > 0$, $B_{\epsilon}(f(x)) \subseteq U$. $f$ is continuous, so by Definition~\ref{def:continuousMap} for some $\delta > 0$, if $y \in B_{\delta}(x)$, $f(y) \in B_{\epsilon}(f(x)) \Longrightarrow f(y) \in U \Longrightarrow y \in f^{-1}(U)$. This holds for every $y \in B_{\delta}(x)$, so $B_{\delta}(x) \subseteq f^{-1}(U)$, hence $f^{-1}(U)$ is open by Definition~\ref{def:openSet}.

	(2. $\Rightarrow$ 1.): Let $x \in X_1$ and $\epsilon > 0$. We must find a $\delta$ such that $y \in B_{\delta}(x) \Longrightarrow f(y) \in B_{\epsilon}(f(x))$. By Lemma~\ref{lem:openBallsAreOpen}, $B_{\epsilon}(f(x))$ is open. $f(x) \in B_{\epsilon}(f(x))$ so $x \in f^{-1}(B_{\epsilon}(f(x)))$. Also, $f^{-1}(B_{\epsilon}(f(x)))$ is open by assumption, hence for some $\delta > 0$, $B_{\delta}() \subseteq f^{-1}(B_{\epsilon}(f(x)))$. This means that $y \in B_{\delta}(x) \Longrightarrow f(y) \in B_{\epsilon}(f(x))$.

	(1. $\Rightarrow$ 3.): TODO.
	
	(3. $\Rightarrow$ 1.): TODO.
\end{proof}

\begin{remark}
	The proof of the above theorem allows us to restate it more precisely, for example,
	\[
		f: X_1 \rightarrow X_2 \text{ continuous } \Longleftrightarrow f^{-1}(U) \text{ open in } X_1 \text{ for every open } U \subseteq X_2 \text{ containing } f(x)
	\]
\end{remark}

\begin{remark}
	By Theorem~\ref{thm:continuityViaOpenSets}, if $f: X_1 \rightarrow X_2$ is continuous, then $f^{-1}(\{ x \})$ is closed for every $x \in X_2$.
\end{remark}

\begin{example}
	Show that the set below is open:
	\[
		U = \{ (x, y) \in \mathbb{R}^2: (x^2 + y^2) {\cos(x^2 - y^2)}^4 > 2
	\]
	The function $f(x, y) = (x^2 + y^2) {\cos(x^2 - y^2)}^4$ is continuous by Lemma~\ref{lem:continuousFunctionsBasicProperties}. We can write $U$ as
	\[
		U = \{ (x, y) \in \mathbb{R}^2: f((x, y)) \in (2, \infty) \} = f^{-1}((2, \infty))
	\]
	$(2, \infty)$ is open in $\mathbb{R}$ so $U$ is the preimage of an open set under a continuous map so by Theorem~\ref{thm:continuityViaOpenSets}, $U$ is open.
\end{example}

\begin{proposition}
	(Useful properties of the preimage)
	\begin{itemize}
		\item $f^{-1}(A \cup B) = f^{-1}(A) \cup f^{-1}(B)$.
		\item $f^{-1}(A \cap B) = f^{-1}(A) \cap f^{-1}(B)$.
		\item $f^{-1}(A - B) = f^{-1}(A) - f^{-1}(B)$.
	\end{itemize}
\end{proposition}

\begin{example}
	Show that the set below is open:
	\[
		U = \{ (x, y) \in \mathbb{R}^2: xy > 2, x^2 + y^2 > 4 \}
	\]
	Let $f(x, y) = xy$ and $g(x, y) = x^2 + y^2$. We can write $U$ as
	\[
		U = f^{-1}((1, \infty)) \cap g^{-1}((2, \infty))
	\]
	$f$ and $g$ are both continuous by Lemma~\ref{lem:continuousFunctionsBasicProperties} and $(1, \infty)$ and $(3, \infty)$ are open in $\mathbb{R}$, so by Theorem~\ref{thm:continuityViaOpenSets}, $f^{-1}((1, \infty))$ and $g^{-1}((2, \infty))$ are open. $U$ is the intersection of two open sets so by Lemma~\ref{lem:unionsAndIntersectionsOpen}, $U$ is open.
\end{example}

\begin{example}
	Show the function below is not continuous at $(0, 0)$:
	\[
		f(x, y) = \begin{cases}
			\frac{xy}{x^2 + y^2} & \text{ if } (x, y) \ne (0, 0) \\
			0 & \text{ otherwise}
		\end{cases}
	\]
	We claim that the preimage $f^{-1}((-\epsilon, \epsilon))$ is not open for small $\epsilon$. $f((0, 0)) = 0$ hence $(0, 0) \in f^{-1}((-\epsilon, \epsilon))$. We must show that every open ball in $\mathbb{R}^2$ centred at $(0, 0)$ is not contained in $f^{-1}((-\epsilon, \epsilon))$.
	
	Let $\epsilon < 1/4$ and $\delta > 0$. Then $(\delta / 2, \delta / 2) \in B_{\delta}((0, 0))$ since $|(\delta / 2, \delta / 2) - (0, 0)| = \delta / \sqrt{2} < \delta$. But $f((\delta / 2, \delta / 2)) = 1 / 2 > \epsilon$ so $(\delta / 2, \delta / 2) \notin f^{-1}((-\epsilon, \epsilon))$. So for every $\delta > 0$, $B_{\delta}((0, 0)) \not\subseteq f^{-1}((-\epsilon, \epsilon))$, so $f^{-1}((-\epsilon, \epsilon))$. Hence by Theorem~\ref{thm:continuityViaOpenSets}, $f$ is not continuous at $(0, 0)$.
\end{example}

\subsection{Compactness}

\begin{definition}\label{def:compactSet}
	Let $(X, d)$ be a metric space. A non-empty $K \subseteq X$ is called \textbf{(sequentially) compact} if for every sequence $\{ x_n \}_{n \in \mathbb{N}}$ in $K$, there is a subsequence $\{ x_{n_k} \}_{k \in \mathbb{N}}$ which converges to a point $x \in K$.
\end{definition}

\begin{remark}
	Note that in the above definition, $\{ x_n \}_{n \in \mathbb{N}}$ does not have to converge.
\end{remark}

\begin{lemma}\label{lem:subsequenceConvergesToSameLimit}
	Let $\{ x_n \}_{n \in \mathbb{N}}$ be a sequence in a metric space $(X, d)$ which converges to $x \in X$. Then every subsequence converges to the same limit $x$.
\end{lemma}

\begin{proof}
	Let $\{ x_{n_k} \}_{k \in \mathbb{N}}$ be a subsequence of $\{ x_n \}_{n \in \mathbb{N}}$. For every $\epsilon > 0$, for some $N \in \mathbb{N}$, for every $k > N$, $x_k \in B_{\epsilon}(x)$. But $n_k > k$ so $x_{n_k} \in B_{\epsilon}(x)$. Hence $\{ x_{n_k} \}_{k \in \mathbb{N}}$ converges to $x$.
\end{proof}

\begin{proposition}\label{prop:convergenceInClosedSubset}
	Let $(X, d)$ be a metric space and $F \subset X$. Then the following are equivalent:
	\begin{itemize}
		\item $F$ is closed.
		\item Every sequence in $F$ which converges in $X$ converges in $F$, i.e. in $\{ x_n \}_{n \in \mathbb{N}} \subseteq F$ and $x_n \rightarrow x$ for some $x \in X$, then $x \in F$.
	\end{itemize}
\end{proposition}

\begin{proof}
	($\Rightarrow$): Let $F$ be closed and $\{ x_n \}_{n \in \mathbb{N}} \subseteq F$ be a sequence which converges to $x \in X$. Assume $x \notin F$, so $x \in X - F$. $X - F$ is open, so by Definition~\ref{def:openSet}, there exists an open ball $B_{\epsilon}(x) \subseteq X - F$. But $x_n \rightarrow x$ so for some $N \in \mathbb{N}$, for every $n > N$, $x_n \in B_{\epsilon}(x)$, but then $x_n \in X - F \Longrightarrow x_n \notin F$ which is a contradiction.
	
	($\Leftarrow$): We show that $X - F$ is open. Let $x \in X - F$, then we must find a ball $B_{\epsilon}(x) \in X - F$. If for some $n \in \mathbb{N}$, $B_{1 / n}(x) \subseteq X - F$, we are done. If not, then pick an $x_n \in B_{1 / n}(x) \cap F$ for each $n$. But $x_n \rightarrow x$ and $x_n \in F$ hence $x \in F$, but this contradicts the assumption that $x \notin F$.
\end{proof}

\begin{corollary}\label{cor:compactSetsClosedAndHaveCompactCloseSubsets}
	\hfill
	\begin{enumerate}
		\item Compact sets are closed.
		\item Every closed subset of a compact subset is compact.
	\end{enumerate}
\end{corollary}

\begin{proof}
	\hfill
	\begin{enumerate}
		\item Let $F$ be compact and let $\{ x_n \}_{n \in \mathbb{N}} \subseteq F$ be a sequence which converges to $x \in X$. By Definition~\ref{def:compactSet}, there exists a convergent subsequence $\{ x_{n_k} \}_{k \in \mathbb{N}}$ which converges to $x_0 \in F$. By Lemma~\ref{lem:sequenceLimitUnique}, $x_0 = x$ so $x \in F$. By Proposition~\ref{prop:convergenceInClosedSubset}, $F$ is closed.
		\item Let $K$ be compact and $F \subseteq K$ be closed. Let $\{ x_n \}_{n \in \mathbb{N}} \subseteq F$ be a sequence. Since each $x_k \in K$, by Definition~\ref{def:compactSet}, there exists a convergent subsequence $\{ x_{n_k} \}_{k \in \mathbb{N}}$ which converges to $x \in K$. By Proposition~\ref{prop:convergenceInClosedSubset}, $x \in F$. Hence $F$ is compact.
	\end{enumerate}
\end{proof}

\begin{example}
	Not all closed sets are compact, e.g. $[0, \infty)$ is closed in $\mathbb{R}$ but $x_n = n$ has no convergent subsequence. The problem arises since $[0, \infty)$ is unbounded.
\end{example}

\begin{definition}\label{def:boundedSet}
	Let $(X, d)$ be a metrix space. $A \subseteq X$ is called \textbf{bounded} if
	\[
		\exists R > 0, \ \exists x \in X, \quad A \subset B_R(x)
	\]
	i.e. $A$ is bounded if it is contained in an open ball in $X$.
\end{definition}

\begin{lemma}\label{lem:compactSetsBounded}
	Let $(X, d)$ be a metric space and $K \subseteq K$ be compact. Then $K$ is bounded.
\end{lemma}

\begin{proof}
	We will prove the contrapositive to this statement, that if $K$ is not bounded then it is not compact. Let $K$ be not bounded and let $x \in K$. So for every $k \in \mathbb{N}$, for some $x_k \in K$, $d(x_k, x) \ge k$, as $K \not\subseteq B_k(x)$. Assume $x_0$ is a limit of the subsequence $\{ x_{n_k} \}_{k \in \mathbb{N}}$, then by the \hyperref[def:metricSpace]{triangle inequality}
	\[
		d(x_{n_k}, x_0) \ge d(x_{n_k}, x) - d(x, x_0) \ge n_k - d(x, x_0) \ge k - d(x, x_0) \to \infty \text{ as} k \to \infty
	\]
	So $\{ x_n \}_{n \in \mathbb{N}}$ has no convergent subsequence, hence $K$ is not compact.
\end{proof}

\begin{theorem}
	(\textbf{Heine-Borel for $\mathbb{R}^n$}) $K \in \mathbb{R}^n$ is compact iff $K$ is closed and bounded.
\end{theorem}

\begin{remark}
	Heine-Borel does not hold for all metric spaces.
\end{remark}

\begin{theorem}
	(\textbf{Heine-Borel for $\mathbb{C}$}) $K \subseteq \mathbb{C}$ is compact iff $K$ is closed and bounded.
\end{theorem}

\begin{proof}
	($\Rightarrow$): By Corollary~\ref{cor:compactSetsClosedAndHaveCompactCloseSubsets}, $K$ is closed and by Lemma~\ref{lem:compactSetsBounded}, $K$ is bounded.

	($\Leftarrow$): By Bolzano-Weierstrass, Heine-Borel holds for subsets of $\mathbb{R}$. Let $K \subseteq \mathbb{C}$ be closed and bounded. As $K$ is bounded, for some $R > 0$, $K \subseteq B_R(0)$, so $\forall z \in K, |z| < R$. Let $\{ z_n \}_{n \in \mathbb{N}} \subseteq \mathbb{C}$ be a sequence where $z_n = x_n + i y_n$, hence $|x_n| < R$ and $|y_n| < R$. We want to show there exists a convergent subsequence of $\{ z_n \}_{n \in \mathbb{N}}$ with limit in $K$.

	By Heine-Borel for $\mathbb{R}$, $[-R, R]$ is compact as it is closed and bounded. $\{ x_n \}_{n \in \mathbb{N}}$ lies in $[-R, R]$ so there exists a convergent subsequence $\{ x_{n_k} \}_{k \in \mathbb{N}}$ with limit $x_0 \in [-R, R]$. So the subsequence $\{ z_{n_k} \}_{k \in \mathbb{N}}$, where $z_{n_k} = x_{n_k} + i y_{n_k}$, has imaginary part $\{ y_{n_k} \}_{k \in \mathbb{N}}$ in $[-R, R]$ so there exists a convergent subsequence $\{ y_{n_{m_k}} \}_{k \in \mathbb{N}}$ with limit $y_0 \in [-R, R]$. So the subsequence $\{ z_{n_{m_k}} \}_{k \in \mathbb{N}}$, where $z_{n_{m_k}} = x_{n_{m_k}} + i y_{n_{m_k}}$, has imaginary part which converges to $y_0$ and by Lemma~\ref{lem:subsequenceConvergesToSameLimit}, the real part converges to $x_0$. Hence $\{ z_{n_{m_k}} \}_{k \in \mathbb{N}}$ converges to $z_0 = x_0 + i y_0$ and $K$ is closed by assumption, so by Proposition~\ref{prop:convergenceInClosedSubset}, $x_0 + i y_0 \in K$. Hence $K$ is compact.
\end{proof}

\begin{example}
	$\mathbb{C}$ is not compact with respect to the standard metric, e.g. the sequence $\{ ik \}_{k \in \mathbb{N}}$ has no convergent subsequence.
\end{example}

\begin{lemma}\label{lem:continuityViaSequences}
	Let $X$ and $Y$ be metric spaces and $f: X \to Y$. $f$ is continuous at $x \in X$ iff
	\[
		\lim_{n \to \infty} f(x_n) = f(x)
	\]
	for every convergent sequence $\{ x_n \}_{n \in \mathbb{N}}$ in $X$ with $x_n \to x$.
\end{lemma}

\begin{proof}
	TODO.
\end{proof}

\begin{theorem}
	Let $X$ and $Y$ be metric spaces and $f: X \to Y$. Then if $K \subset X$ is compact and $f$ is continuous, then the image $f(K)$ is compact in $Y$.

	In particular, if $Y = \mathbb{R}$, then every continuous $f: X \to \mathbb{R}$ attains minima and maxima on compact sets.
\end{theorem}

\begin{proof}
	Let $\{ y_n \}_{n \in \mathbb{N}}$ be a sequence in $f(K)$, and let $y_n = f(x_n)$. We must show that $\{ y_n \}_{n \in \mathbb{N}}$ has a convergent subsequence with limit in $f(K)$. $K$ is compact so there exists a convergent subsequence $\{ x_{n_k} \}_{k \in \mathbb{N}}$ of $\{ x_n \}_{n \in \mathbb{N}}$ with limit $x \in K$. $f$ is continuous so by Lemma~\ref{lem:continuityViaSequences}, $y_{n_k} = f(x_{n_k}) \to f(x)$. So $\{ y_{n_k} \}_{k \in \mathbb{N}}$ converges to $f(x)$ and $x \in K$ so $f(x) \in f(K)$.
\end{proof}

\section{Complex Differentiation}

\subsection{Complex differentiablity}

\begin{definition}\label{def:complexDifferentiablity}
	Let $U \subseteq \mathbb{C}$ be open. $f: U \to \mathbb{C}$ is called \textbf{(complex) differentiable at $z_0 \in U$} if
	\[
		\lim_{z \to z_0} \frac{f(z) - f(z_0)}{z - z_0}
	\]
	exists. This limit is called the \textbf{derivative of $f$ at $z_0$} and we define $f'(z_0)$ as
	\[
		f'(z_0) = \lim_{h \to 0} \frac{f(z_0 + h) - f(z_0)}{h}
	\]
\end{definition}

\begin{remark}
	Here, $h$ is a complex number, so the limit must exist from every direction.
\end{remark}

\begin{remark}
	Complex differentiablity at $z$ implies continuity at $z$.
\end{remark}

\begin{example}
	Let $f(z) = z^2$. For every $z \in \mathbb{C}$,
	\[
		\lim_{h \to 0} \frac{{(z + h)}^2 - z^2}{h} = \lim_{h \to 0} \frac{z^2 + 2hz + h^2 - z^2}{h} = \lim_{h \to 0} (2z + h) = 2z
	\]
	So $f'(z) = 2z$, and $f$ is differentiable on $\mathbb{C}$.
\end{example}

\begin{example}
	Let $f(z) = \bar{z}$. For every $z \in \mathbb{C}$,
	\[
		\begin{aligned}
			\lim_{h \to 0, h \in \mathbb{R}} \frac{\overline{z + h} - \overline{z}}{h} & = \lim_{h \to 0, h \in \mathbb{R}} \frac{h}{h} = 1 \\
			\lim_{h \to 0, h \in \mathbb{R}} \frac{\overline{z + ih} - \overline{z}}{ih} & = \lim_{h \to 0, h \in \mathbb{R}} \frac{-ih}{ih} = -1 \ne 1
		\end{aligned}
	\]
	So $f$ is differentiable nowhere as the limits do not agree.
\end{example}

\begin{proposition}
	Sums, products and quotients of complex differentiable functions are complex differentiable where defined. In particular, the product and quotient rules hold for complex derivatives.
\end{proposition}

\begin{proof}
	Omitted.
\end{proof}

\begin{proposition}\label{prop:chainRule}
	Compositions of complex differentiable functions are complex differentiable where defined. In particular, the chain rule holds for complex derivatives.
\end{proposition}

\begin{proof}
	Omitted.
\end{proof}

\begin{example}
	Non-constant, purely real or imaginary functions are not complex differentiable, e.g. $\Re(z)$, $\Im(z)$, $|z|$ are nowhere differentiable.
\end{example}

\subsection{Cauchy-Riemann equations}

\begin{definition}
	Let $U \subseteq \mathbb{C}$ and $f: U \rightarrow \mathbb{C}$. We can write
	\[
		f(z) = u(x, y) + i v(x, y)
	\]
	where $z = x + iy$, $x, y \in \mathbb{R}$, $u, v: \mathbb{R}^2 \to \mathbb{R}$ are real functions. $u$ is called the \textbf{real part} of $f$ and $v$ is called the \textbf{imaginary part}.
\end{definition}

\begin{example}
	Let $f(z) = z^2$. Then $f(z) = z^2 = {(x + iy)}^2 = x^2 + 2ixy - y^2 = x^2 - y^2 + 2ixy$. So $u(x, y) = x^2 - y^2$ and $v(x, y) = 2xy$.
\end{example}

\begin{proposition}\label{prop:cauchyRiemann}
	Let $f = u + iv$ be complex differentiable at $z_0$. Then the real partial derivatives $u_x, u_y, v_x, v_y$ exist at $z_0$ and satisfy the \textbf{Cauchy-Riemann equations}:
	\[
		\begin{aligned}
			u_x(z_0) & = v_y(z_0) \\
			u_y(z_0) & = -v_x(z_0)
		\end{aligned}
	\]
	Also,
	\[
		f'(z_0) = u_x(z_0) + i v_x(z_0)
	\]
\end{proposition}

\begin{proof}
	$f$ is complex differentiable at $z_0 = x_0 + i y_0$ so the limit in Definition~\ref{def:complexDifferentiablity} exists from every direction, in particular the purely real and imaginary directions. So
	\[
		f'(z_0) = \lim_{h \to 0, h \in \mathbb{R}} \frac{f(z_0 + h) - f(z_0)}{h} = \lim_{h \to 0, h \in \mathbb{R}} \frac{f(z_0 + ih) - f(z_0)}{ih}
	\]
	When $h$ is real,
	\[
		\begin{aligned}
			f(z_0 + h) & = f((x_0 + h) + iy_0) = u(x_0 + h, y_0) + i v(x_0 + h, y_0) \\
			f(z_0 + ih) & = f(x_0 + i(y_0 + h)) = u(x_0, y_0 + h) + i v(x_0, y_0 + h)
		\end{aligned}
	\]
	hence
	\[
		\begin{aligned}
			f'(z_0) & = \lim_{h \to 0} \frac{u(x_0 + h, y_0) - u(x_0, y_0)}{h} + i \lim_{h \to 0} \frac{v(x_0 + h, y_0) - v(x_0, y_0)}{h} \\
			& = u_x(x_0, y_0) + i v_x(x_0, y_0) \\
			& = \frac{1}{i} \lim_{h \to 0} \frac{u(x_0, y_0 + h) - u(x_0, y_0)}{h} + \frac{i}{i} \lim_{h \to 0} \frac{v(x_0, y_0 + h) - v(x_0, y_0)}{h} \\
			& = -i u_y(x_0, y_0) + v_y(x_0, y_0)
		\end{aligned}
	\]
	Comparing real and imaginary parts completes the proof.
\end{proof}

\begin{example}
	Let $f(z) = z^2$ then $u(x, y) = x^2 - y^2$ and $v(x, y) = 2xy$. So
	\[
		u_x = 2x, \quad v_y = 2x, \quad u_y = -2y, \quad v_x = 2y
	\]
	so the Cauchy-Riemann equations are satisfied, as expected (since $f$ is complex differentiable).
\end{example}

\begin{theorem}\label{thm:partialsContinuousAndCauchyRiemannImpliesDifferentiable}
	Let $U \subseteq \mathbb{C}$ be open and $f = u + iv$. If the partial derivatives $u_x, u_y, v_x, v_y$ exist, are continuous, and satisfy the Cauchy-Riemann equations at $z_0 \in U$, then $f$ is complex differentiable at $z_0$.
\end{theorem}

\begin{proof}
	Omitted.
\end{proof}

\begin{example}
	Let $f(z) = \exp(z) = e^x \cos(y) + i e^x \sin(y)$. Let $u(x, y) = e^x \cos(y)$ and $v(x, y) = e^x \sin(y)$, then
	\[
		\begin{aligned}
			u_x & = e^x \cos(y) = v_y \\
			u_y = -e^x \sin(y) = -v_x
		\end{aligned}
	\]
	The partial derivative $u_x, u_y, v_x, v_y$ exist, are continuous and satisfy the Cauchy-Riemann equations for every $z_0 \in \mathbb{C}$. So by Theorem~\ref{thm:partialsContinuousAndCauchyRiemannImpliesDifferentiable}, $f$ is differentiable on $\mathbb{C}$ and by Proposition~\ref{prop:cauchyRiemann},
	\[
		f'(z) = u_x + i v_x = e^x \cos(y) + i e^x \sin(y) = \exp(z)
	\]
\end{example}

\begin{example}
	Let $f(z) = e^{iz}$ then by the \hyperref[prop:chainRule]{chain rule}, $f'(z) = i e^{iz}$. $\sin, \cos, \sinh, \cosh$ are all sums of $\exp$ so they are all differentiable on $\mathbb{C}$, and
	\[
		\sin'(z) = \cos(z), \quad \cos'(z) = -\sin(z), \quad \sinh'(z) = \cosh(z), \cosh'(z) = \sinh(z)
	\]
\end{example}

\begin{example}
	Similarly, all polynomial and rational functions are differentiable on $\mathbb{C}$ (except where not defined) and for $a_0, \dots, a_n \in \mathbb{C}$,
	\[
		(a_0 + a_1 z + \cdots + a_n z^n)' = a_1 + 2 a_2 z + \cdots + n a_n z^{n - 1}
	\]
\end{example}

\begin{example}
	The branch of $\log$ corresponding to arguments in $(\theta_1, \theta_2]$ is differentiable on $\mathbb{C} - R_{\theta_1}$, with $\log(z) = 1/z$.
\end{example}

\begin{example}\label{exa:differentiableOnCircle}
	Let $f(z) = f(x + iy) = (x^3 + 3x^2 y - y^3 - x^2 - 2y^2) + i(-x^3 + 3xy^2 - y^3 + 4xy + 3y)$, then $u_y(z) = -v_x(z)$ for every $z \in \mathbb{C}$ but $u_x = v_y$ iff $3x^2 - 2x = -3y^2 + 4x + 3 \Longleftrightarrow {(x - 1)}^2 + y^2 = 2$. So $f$ is differentiable only on the circle of radius $\sqrt{2}$ and centre $1$.
\end{example}

\begin{definition}
	Let $U \subseteq \mathbb{C}$ be open. $f: U \to \mathbb{C}$ is called \textbf{holomorphic on $U$} if it is complex differentiable at every point in $U$.

	$f$ is called \textbf{holomorphic at $z_0 \in U$} if for some $\epsilon > 0$, $f$ is holomorphic on the open ball $B_{\epsilon}(z_0)$.
\end{definition}

\begin{example}
	$\exp, \sin, \cos, \sinh, \cosh$ and polynomials are holomorphic on $\mathbb{C}$.
\end{example}

\begin{example}
	Logarithms and complex powers are holomorphic every apart from their branch cuts.
\end{example}

\begin{example}
	$f$ defined in Example~\ref{exa:differentiableOnCircle} is holomorphic nowhere, since any open ball in $\mathbb{C}$ centred at a point $z_0$ on this circle contains a point $w$ not on the circle, and $f$ is not differentiable at $w$.
\end{example}

\subsection{Connected sets and zero derivatives}

\begin{definition}
	A \textbf{path} (or \textbf{curve}) from $a \in \mathbb{C}$ to $b \in \mathbb{C}$ is a continuous function $\gamma: [0, 1] \rightarrow \mathbb{C}$ where $\gamma(0) = a$ and $\gamma(1) = b$. $\gamma$ is called \textbf{closed} if $a = b$. $\gamma$ is called \textbf{smooth} if it is continuously differentiable.
\end{definition}

\begin{definition}
	$U \subseteq \mathbb{C}$ is called \textbf{path-connected} if for every $a, b \in U$, for some smooth path $\gamma$ from $a$ to $b$, $\gamma(t) \in U$ for every $t \in [0, 1]$.
\end{definition}

\begin{definition}
	A \textbf{domain} (or \textbf{region}) $D$ is an open, path-connected subset of $\mathbb{C}$.
\end{definition}

\begin{example}
	$B_r(z)$ is open for every $z \in \mathbb{C}$ and $r > 0$. $B_r(z)$ is also path-connected, since for every $a, b \in B_r(z)$, the line-segment $\gamma(t) = a + (b - a)t$ is a smooth path, so $B_r(z)$ is a domain. The same holds for $\mathbb{C}$.
\end{example}

\begin{example}
	$\mathbb{C} - R_{\le 0}$ is path-connected. Let $a, b \in \mathbb{C} - R_{\le 0}$, then if $a$ lies on the positive real axis then a line segment in $\mathbb{C} - R_{\le 0}$ exists from $a$ to $b$. Otherwise, we can find an arc of a circle which starts at $a$ and ends at $b$, while staying in $\mathbb{C} - R_{\le 0}$.

	$\mathbb{C} - R_{\le 0}$ is also open so it is a domain.
\end{example}

\begin{example}
	$\{ z \in \mathbb{C}: |z| \ne 1 \}$ is not a domain. It is not path-connected, since if $|a| < 1$ and $|b| > 1$, then there doesn't exist a continuous path from $a$ to $b$ without crossing the circle $|z| = 1$.
\end{example}

\begin{lemma}\label{lem:chainRuleForPaths}
	(\textbf{Chain rule}) Let $U \subseteq \mathbb{C}$ be open, $f: U \to \mathbb{C}$ be holomorphic on $U$ and $\gamma: [0, 1] \to U$ be a smooth path. Then
	\[
		\forall t \in [0, 1], \quad (f \circ \gamma)'(t) = f'(\gamma(t)) \cdot \gamma'(t)
	\]
\end{lemma}

\begin{proof}
	Omitted.
\end{proof}

\begin{theorem}
	Let $D$ be a domain and $f: D \to \mathbb{C}$ be holomorphic. If $\forall z \in D, \ f'(z) = 0$, then $f$ is constant on $D$.
\end{theorem}

\begin{proof}
	$D$ is path-connected, so we can show that $f$ is constant on every smooth path $\gamma \subset D$, i.e. we show that $f \circ \gamma$ is a constant function. Let $f = u + iv$. By the \hyperref[lem:chainRuleForPaths]{chain rule},
	\[
		(f \circ \gamma)'(t) = f'(\gamma(t)) \cdot \gamma'(t) = 0 = (u \circ \gamma)'(t) + i (v \circ \gamma)'(t)
	\]
	So $u \circ \gamma$ and $v \circ \gamma$ are constant on $\gamma$ so $f$ is also.
\end{proof}

\begin{remark}
	The same result holds if, instead of $\forall z \in D, \ f'(z) = 0$:
	\begin{itemize}
		\item $f$ has constant real or imaginary part; or
		\item $f$ has constant modulus.
	\end{itemize}
\end{remark}

\subsection{The angle-preserving properties of holomorphic functions}

\begin{definition}
	Let $D$ be a domain. A real differentiable function $f: D \to \mathbb{C}$ is called \textbf{conformal at $z_0$} if it preserves the angle and orientation between any two tangent vectors at $z_0$. This is equivalent to saying that $f$ preserves the angle and orientation between any two smooth paths that pass through $z_0$.

	$f$ is called \textbf{conformal} if it is conformal at every point in $D$.
\end{definition}

\begin{lemma}\label{lem:holomorphicImpliesConformal}
	(\textbf{Holomorphic maps are conformal}) Let $f$ be holomorphic with $f'(z_0) \ne 0$. Then $f$ is conformal at $z_0$.
\end{lemma}

\begin{proof}
	Let $\gamma: [0, 1] \to \mathbb{C}$ be a smooth path passing through $z_0 = \gamma(t_0)$. Consider the tangent vector to $\gamma$ at $z_0$. The path $f \circ \gamma$ is the path produced by mapping $\gamma$ under $f$. So we calculate what happens to the tangent vector $\gamma'(t_0)$ by using the chain rule involving $(f \circ \gamma)'(t_0)$:
	\[
		(f \circ \gamma)'(t_0) = f'(\gamma(t_0)) \cdot \gamma'(t_0) = f'(z_0) \cdot \gamma'(t_0)
	\]
	So we see that $f$ multiplies the tangent vector $\gamma'(t_0)$ by the complex number $f'(z_0)$, so this is equivalent to a dilation by $|f'(z_0)|$ and a rotation by the angle $\Arg(f'(z_0))$, which both preserve and angles and orientations between vectors.
\end{proof}

\begin{example}
	Let $f(z) = z^2$, then $f'(z) = 2z$, so $f'(z) = 0 \Longleftrightarrow z = 0$. Hence $f$ is conformal on $\mathbb{C}^* = \mathbb{C} - \{ 0 \}$ since it is holomorphic everywhere. $f$ is not conformal at $0$ as $f'(0) = 0$ so $f$ sends tangent vectors at the origin to zero, so does not preserve angles.
\end{example}

\begin{corollary}
	Let $f$ be holomorphic, $f'(z_0)$. Then $f$ rotates tangent vectors at $z_0$ by $\Arg(f'(z_0))$ and dilates them by $|f'(z_0)|$.
\end{corollary}

\begin{proposition}\label{prop:conformalImpliesHolomorphic}
	({Conformal maps are holomorphic}) Let $D$ be a domain. If $f$ is conformal at $z_0 \in D$, then $f$ is complex differentiable at $z_0$ and $f'(z_0) \ne 0$. In particular, $f$ is conformal on $D$ iff $f$ is holomorphic with $f'(z) \ne 0$ for every $z \in D$.
\end{proposition}

\begin{proof}
	Non-examinable.
\end{proof}

\begin{example}
	Let $f(z) = f(x + iy) = xy + iy^2 = u(x, y) + i v(x, y)$. Then $u_x = y, v_y = 2y, u_y = x, v_x = 0$. So the Cauchy-Riemann equations hold only at $x = y = 0$. So $f$ is not differentable at $z_0 \ne 0$ so by Proposition~\ref{prop:conformalImpliesHolomorphic}, $f$ is not conformal at $z_0 \ne 0$. $f'(0) = 0$ so by Proposition~\ref{prop:conformalImpliesHolomorphic}, $f$ is not conformal at $0$, so is nowhere conformal.
\end{example}

\begin{corollary}
	Every conformal map maps orthogonal grids in the $(x, y)$-plane to orthogonal grids. (Note, these grids may be made up of arbitrary smooth curves, not just straight lines).
\end{corollary}

\begin{example}
	Let $f(z) = f(x + iy) = z^2 = x^2 - y^2 + 2xy i$ and consider the grid in the $(x, y)$-plane made of lines parallel to the real and imaginary axes, of distance $1$ from each other.

	Given a line that doesn't pass through the origin, $x = a \ne 0$, $f$ maps this line to the points $(a^2 - y^2, 2ay)$ in the $(u, v)$-plane. Let $u(y) = a^2 - y^2$ and $v(y) = 2ay$, then $v^2 = 4a^2 y^2$ hence $u(v) = a^2 - v^2/4a^2$. Similarly, the line $y = b \ne 0$ is mapped to the parabola $u = v^2 / 4b^2 - b^2$ in the $(u, v)$-plane. So the lines are mapped to parabolae, which cross at right angles.

	The level curves $u(x, y) = x^2 - y^2 = a$ and $v(x, y) = 2xy = b$ trace the curves $y^2 = x^2 - a$ and $y = b/(2x)$. These level curves are perpendicular when drawn on a graph, since $f$ is conformal on $\mathbb{C}^*$ and they map to perpendicular straight lines.
\end{example}

\subsection{Biholomorphic maps}

\begin{definition}
	Let $D$ and $D'$ be domains. $f: D \to D'$ is called \textbf{biholomorphic} if $f$ is holomorphic, a bijection, and its inverse $f^{-1}: D' \to D$ is also holomorphic. $f$ is called a \textbf{biholomorphism}, and the domains $D$ and $D'$ are called \textbf{biholomorphic}. We write $f: D \biholo D'$.
\end{definition}

\begin{example}
	$\exp: \mathbb{C} \to \mathbb{C}^*$ is not biholomorphic since it is not injective: $e^z = e^{z + 2n \pi i}$. To make it a biholomorphic map, we can restrict it to a smaller domain.
	
	We have that $\exp(z_1) = \exp(z_2)$ iff $\Re(z_1) = \Re(z_2)$ and $\Im(z_1) - \Im(z_2) \in 2 \pi \mathbb{Z}$. So let
	\[
		D = \{ z \in \mathbb{C}: \Im(z) \in (-\pi, \pi) \}
	\]
	If $z = x + iy \in D$ then $\exp(z) = e^x e^{iy}$ so $\Arg(\exp(z)) \ne \pi$ for every $z \in D$, so $\exp$ maps $D$ to $\mathbb{C} - \mathbb{R}_{\le 0}$ and is injective on $D$, so is a bijection between $D$ and $\mathbb{C} - \mathbb{R}_{\le 0}$. Its inverse is $\Log$ which maps $\mathbb{C} - \mathbb{R}_{\le 0}$ to $D$ and is differentiable on $\mathbb{C} - \mathbb{R}_{\le 0}$ so holomorphic on $\mathbb{C} - \mathbb{R}_{\le 0}$. So $\exp$ is biholomorphic on $D$.
\end{example}

\begin{example}
	Let $f(z) = z^2$. It is not injective since e.g. $f(1) = f(-1) = 1$. $f(z_1) = f(z_2)$ with $z_1 \ne z_2$ iff $z_1 = \pm z_2$. So we can make $f$ biholomorphic by restricting to the domain
	\[
		\mathbb{H}_R := \{ z \in \mathbb{C}: \Re(z) > 0
	\]
	$f$ maps $\mathbb{H}_R$ to $\mathbb{C} - \mathbb{R}_{\le 0}$. It is a bijection with inverse
	\[
		f^{-1}(z) = \exp \left( \frac{1}{2} \Log(z) \right)
	\]
	$f^{-1}$ is the composition of holomorphic functions so is holomorphic. So $f: \mathbb{H}_R \to \mathbb{C} - \mathbb{R}_{\le 0}$ is biholomorphic.
\end{example}

\begin{example}
	The affine linear maps $f(z) = az + b$, $a \in \mathbb{C}^*$, $b \in \mathbb{C}$ are biholomorphic from $\mathbb{C}$ to $\mathbb{C}$.
\end{example}

\begin{lemma}
	Let $D \subseteq \mathbb{C}$ be a domain. The set of all biholomorphic maps $f: D \biholo D$ forms a group under composition. This group is called the \textbf{automorphism group of $D$}, written $\text{Aut}(D)$.
\end{lemma}

\begin{proof}
	\hfill
	\begin{enumerate}
		\item The identity map $\id \in \text{Aut}(D)$ and is biholomorphic.
		\item Composition of functions is associative: $f \circ (g \circ h) = (f \circ g) \circ h$.
		\item $f^{-1}$ is the group inverse of $f$ as $f \circ f^{-1} = f^{-1} \circ f = \id$. $f^{-1} \in \text{Aut}(D)$ since it is biholomorphic (since $f$ is biholomorphic).
		\item $\text{Aut}(D)$ is closed, since $(f \circ g)$ is holomorphic by the chain rule, and its inverse ${(f \circ g)}^{-1} = g^{-1} \circ f^{-1}$ is holomorphic as it is the composition of holomorphic functions.
	\end{enumerate}
\end{proof}

NOTES directly from lecture notes up to here.

\section{Mobius Transformations}

\subsection{Definition and first properties of Mobius transformations}

\begin{definition}
	The set of matrices $\GL_2(\mathbb{C})$ is defined as
	\[
		\GL_2(\mathbb{C}) = \{ A \in M_2(\mathbb{C}): \det(A) \ne 0 \}
	\]
\end{definition}

\begin{definition}
	Let $T = \begin{bmatrix}
		a & b \\
		c & d
	\end{bmatrix} \in \GL_2(\mathbb{C})$. The \textbf{Mobius transformation} $M_T: \hat{\mathbb{C}} \to \hat{\mathbb{C}}$ is defined as
	\[
		M_T(z) = \begin{cases}
			\frac{az + b}{cz + d} & \text{ if } cz + d \ne 0 \\
			\infty & \text{ if } cz + d = 0
		\end{cases}
	\]
	for $z \ne \infty$ and for $z = \infty$,
	\[
		M_T(\infty) = \begin{cases}
			\frac{a}{c} & \text{ if } c \ne 0 \\
			\infty & \text{ if } c = 0
		\end{cases}
	\]
\end{definition}

\begin{remark}\label{rem:mobiusTransformationDet1}
	Let $T \in \GL_2(\mathbb{C})$. Then for some $k$, $k^2 = \det(T)$. So
	\[
		M_T(z) = \frac{az + b}{cz + d} = \frac{az/k + b/k}{cz/k + d/k} = M_{\frac{1}{k} T} (z)
	\]
	and $\det(\frac{1}{k} T) = \frac{1}{k^2} \det(T) = 1$. So every $T \in \GL_2(\mathbb{C})$ can be scaled to obtain $T' = \frac{1}{k} T \in \GL_2(\mathbb{C})$ such that $M_{T'} = M_T$ and $\det(T') = 1$.
\end{remark}

\begin{example}
	$f(z) = z^-1$ is a Mobius transformation associated with the matrix $\begin{bmatrix} 0 & 1 \\ 1 & 0 \end{bmatrix}$. $f$ maps the punctured unit ball $B_1(0) - \{ 0 \}$ to the outside of the closed unit ball $\mathbb{C} - \overline{B_1}(0)$.
\end{example}

\begin{example}
	The \textbf{Cayley map} $f(z) = \frac{z - i}{z + i}$ associated with the matrix $\begin{bmatrix} 1 & -i \\ 1 & i \end{bmatrix}$ is a Mobius transformation.

	$f(z) \in B_1(0) \Longleftrightarrow |f(z)| < 1 \Longleftrightarrow |z + i| > |z - i| \Longleftrightarrow z \in \mathbb{H} = \{ z \in \mathbb{C}: \Im(z) > 0 \}$ so $f$ maps the upper half plane to the unit ball. $f(\infty) = 1$ and $f(-i) = \infty$.
\end{example}

\begin{lemma}
	The set of Mobius transformations form a group under composition, with
	\begin{enumerate}
		\item $M_{T_1} \circ M_{T_2} = M_{T_1 T_2}$.
		\item ${(M_T)}^{-1} = M_{T^{-1}}$.
		\item $M_T = \id \Longleftrightarrow T = t \begin{bmatrix} 1 & 0 \\ 0 & 1 \end{bmatrix}$ for some $t \in \mathbb{C}^*$.
	\end{enumerate}
\end{lemma}

\begin{proof}
	Omitted.
\end{proof}

\begin{lemma}\label{lem:mobiusTransformationBiholo}
	Let $T = \begin{bmatrix} a & b \\ c & d \end{bmatrix} \in \GL_2(\mathbb{C})$. If $c = 0$, then
	\[
		M_T: \mathbb{C} \biholo \mathbb{C}
	\]
	is a biholomorphic map. If $c \ne 0$, then
	\[
		M_T: \mathbb{C} - \{ \frac{-d}{c} \} \biholo \mathbb{C} - \{ \frac{a}{c} \}
	\]
	is a biholomorphic map.
\end{lemma}

\begin{proof}
	If $c = 0$, then since $\det(T) \ne 0$, $a \ne 0$ and $d \ne 0$. So
	\[
		M_T(z) = \frac{a}{d} z + \frac{b}{d}
	\]
	is an affine linear map which is holomorphic. It is a bijection with inverse
	\[
		{(M_T)}^{-1} = \frac{d}{a} z - \frac{b}{a}
	\]
	which is also holomorphic. So $M_T: \mathbb{C} \biholo \mathbb{C}$ is biholomorphic.

	If $c \ne 0$, then
	\[
		M_T'(z) = \frac{a(cz + d) - c(az + b)}{{(cz + d)}^2} = \frac{\det(T)}{{(cz + d)}^2}
	\]
	The derivative exists on $\mathbb{C} - \{ \frac{-d}{c} \}$ and so $M_T$ is holomorphic there. It is bijective with its inverse as the inverse Mobius transformation, which similarly is also holomorphic. So $M_T: \mathbb{C} - \{ \frac{-d}{c} \} \biholo \mathbb{C} - \{ \frac{a}{c} \}$ is biholomorphic.
\end{proof}

\begin{corollary}
	$M_T$ is conformal at every $z \in \mathbb{C}$ with $M_T(z) \ne \infty$.
\end{corollary}

\begin{proof}
	By Lemma~\ref{lem:holomorphicImpliesConformal}, $M_T$ is conformal on $\mathbb{C}$ except at points that map to $\infty$.
\end{proof}

\begin{corollary}
	Any Mobius transformation is a bijection from $\hat{\mathbb{C}}$ to $\hat{\mathbb{C}}$.
\end{corollary}

\begin{proof}
	By Lemma~\ref{lem:mobiusTransformationBiholo}, $M_T$ has an inverse $M_{T^{-1}}$ with $M_T \circ M_{T^{-1}} = \id = M_{T^{-1}} \circ M_T$. Both are maps from $\hat{\mathbb{C}}$ to $\hat{\mathbb{C}}$ and so are bijections.
\end{proof}

\subsection{Fixed points, the cross-ratio, and the three points theorem}

\begin{definition}
	Let $T \in \GL_2(\mathbb{C})$ and $M_T$ be a Mobius transformation. A point $z$ is called a \textbf{fixed point} of $M_T$ if $M_T(z) = z$. 
\end{definition}

\begin{lemma}\label{lem:threeFixedPointsImpliesIdentity}
	Let $T \in \GL_2(\mathbb{C})$. If $M_T: \hat{\mathbb{C}} \rightarrow \hat{\mathbb{C}}$ is not the identity map, then $M_T$ has at most two fixed points in $\hat{\mathbb{C}}$. In particular, if a Mobius transformation has three fixed points in $\hat{\mathbb{C}}$ then it is the identity map.
\end{lemma}

\begin{proof}
	If $M_T(\infty) = \infty$, then from the definition, $M_T(z) = \frac{az + b}{cz + d}$, therefore $c = 0$. So $M_T(z) = \frac{a}{d}z + \frac{b}{d}$, with $a \ne 0, d \ne 0$ (since $\det T \ne 0$). Such an affine linear map has at most one fixed point because:
	\begin{itemize}
		\item If $a \ne d$ then $\frac{a}{d}z + \frac{b}{d} = z \Longleftrightarrow z = \frac{b}{d - a}$ so $M_T$ has a unique fixed point.
		\item If $a = d$ then $b \ne 0$ (since we assume $M_T$ is not the identity). So $M_T(z) = z + \frac{b}{a}$ is a translation which has no fixed points.
	\end{itemize}
	If $M_T(\infty) \ne \infty$, then all fixed points of $M_T$ are in $\mathbb{C}$. Suppose $z_0 \in \mathbb{C}$ is such that $M_T(z_0) = z_0$. We have
	\[
		M_T(z_0) = z_0 \Longleftrightarrow \frac{a z_0 + b}{c z_0 + d} = z_0 \Longleftrightarrow c z_0 ^ 2 + (d - a)z_0 - b = 0
	\]
	This quadratic equation has at most two roots so there are at most two fixed points of $M_T$ in $\hat{\mathbb{C}}$.
\end{proof}

\begin{definition}
	Given four distinct points $z_0, z_1, z_2, z_3 \in \mathbb{C}$, the \textbf{cross-ratio} of these points, denoted $(z_0, z_1; z_2, z_3)$, is defined as
	\[
		(z_0, z_1; z_2, z_3) = \frac{(z_0 - z_2)(z_1 - z_3)}{(z_0 - z_3)(z_1 - z_2)}
	\]
	We extend the definition to the case where one of the points is $\infty$ by removing all differences involving that point e.g.
	\[
		(\infty, z_0; z_2, z_3) = \frac{z_1 - z_3}{z_1 - z_2}
	\]
\end{definition}

\begin{theorem}\label{thm:threePointsTheorem}
	(\textbf{Three points theorem}) Let ${z_1, z_2, z_3}$ and ${w_1, w_2, w_3}$ be two sets of three distinct points in $\hat{\mathbb{C}}$. Then there exists a unique Mobius transformation $f$ such that $f(z_i) = w_i$ for every $i \in \{1, 2, 3\}$.
\end{theorem}

\begin{proof}
	Existence: define the Mobius transformations
	\[
		F(z) = (z, w_1; w_2, w_3) = \frac{(z - w_2)(w_1 - w_3)}{(z - z_3)(w_1 - w_2)}, \quad G(z) = (z, z_1; z_2, z_3) = \frac{(z - z_2)(z - z_3)}{(z - z_3)(z_1 - z_2)}
	\]
	Then $F(w_1) = 1$, $F(w_2) = 0$, $F(w_3) = \infty$ and similarly, $G(z_1) = 1$, $G(z_2) = 0$, $G(z_3) = \infty$. Therefore $F^{-1} \circ G$ maps each $z_i$ to $w_i$.

	Uniqueness: assume that there are two such maps, say $f_1$ and $f_2$. Then the Mobius transformation $H = f_1^{-1} \circ f_2$ satisfies $H(z_i) f_1^{-1}(f_2(z_i)) = f_1^{-1}(w_i) = z_i$.

	Thus $H$ has three fixed points $z_1, z_2, z_3$ so, by Lemma~\ref{lem:threeFixedPointsImpliesIdentity}, $H$ is the identity. Thus $f_1 = f_2$.
\end{proof}

\begin{proposition}\label{prop:mobiusTransformationsPreserveCrossRatio}
	(\textbf{Mobius transformations preserve the cross-ratio}) Let $z_0$, $z_1$, $z_2$, $z_3$ be four distinct points in $\hat{\mathbb{C}}$ and $f$ be a Mobius transformation. Then
	\[
		(f(z_0), f(z_1); f(z_2), f(z_3)) = (z_0, z_1; z_2, z_3)
	\]
\end{proposition}

\begin{proof}
	Let $w_i = f(z_i)$ for every $i \in \{1, 2, 3\}$. Let $F(z) = (z, w_1; w_2, w_3)$ and $G(z) = (z, z_1; z_2, z_3)$. By the \hyperref[thm:threePointsTheorem]{three points theorem}, $F^{-1} \circ G$ maps $z_i$ to $w_i$, as does $f$, hence $f = F^{-1} \circ G \Longleftrightarrow F \circ f = G$. Then
	\[
		(f(z_0), f(z_1); f(z_2), f(z_3)) = (f(z_0), w_1; w_2, w_3) = F \circ f (z_0) = G(z_0) = (z_0, z_1; z_2, z_3)
	\]
\end{proof}

\begin{example}
	Find the unique Mobius transformation $f: \hat{\mathbb{C}} \to \hat{\mathbb{C}}$ that maps the points $\{ 1, -1, i \}$ to $\{ 0, \infty, 1 \}$. By Proposition~\ref{prop:mobiusTransformationsPreserveCrossRatio},
	\[
		\begin{aligned}
			& \frac{w_1 - w_3}{f(z) - w_3} = \frac{(z - z_2)(z_1 - z_3)}{(z - z_3)(z_1 - z_2)} \\
			\Longleftrightarrow & \frac{0 - 1}{f(z) - 1} = \frac{(z - (-1))(1 - i)}{(z - i)(1 - (-1))} \\
			\Longleftrightarrow & f(z) - 1 = \frac{-2(z - i)}{(z + 1)(1 - i)} \\
			\Longleftrightarrow & f(z) = \frac{(-1 - i)z + (1 + i)}{(1 - i)z + (1 - i)} = \frac{-1 - i}{1 - i} \frac{z - 1}{z + 1} = \frac{-iz + i}{z + 1}
		\end{aligned}
	\]
\end{example}

\begin{remark}
	The general strategy to find a Mobius transformation from how it maps three points is to use Proposition~\ref{prop:mobiusTransformationsPreserveCrossRatio}, and then rearrange to find $f(z)$.
\end{remark}

\begin{remark}
	The general strategy to find the image of a region $D$ under a Mobius transformation $M_T$ is:
	\begin{enumerate}
		\item Find the image $M_T(\partial D)$ of the boundary $\partial D$.
		\item Compute $M_T(z_0)$ for $z_0 \in D$.
		\item The region $D'$ bounded by $M_T(\partial D)$ and containing $M_T(z_0)$ is the image of $D$ under $M_T$. $M_T$ is a biholomorphism from $D$ to $D'$.
	\end{enumerate}
\end{remark}

\subsection{Circles and lines}

\begin{lemma}\label{lem:equationOfCirclesAndLines}
	(\textbf{Equation of circles and lines in $\mathbb{C}$}) Let $\gamma, \beta \in \mathbb{C}$ and $\alpha \in \mathbb{C}$. Then the equation
	\[
		\gamma z \bar{z} - \alpha \bar{z} - \bar{\alpha} z + \beta = 0
	\]
	defines a circle if $\gamma = 1$ and $|\alpha|^2 - \beta > 0$, and a line if $\gamma = 0$ and $\alpha \ne 0$. Also, any circle or line can be described by this equation.
\end{lemma}

\begin{proof}
	A circle of centre $\alpha$ and radius $r$ is defined by
	\[
		|z - \alpha|^2 = r^2 = (z - a)(\overline{z - a})(\bar{z} - \bar{a}) = z\bar{z} - \alpha \bar{z} - \bar{\alpha} z + \beta = 0
	\]
	Now let $\beta = \alpha \bar{\alpha} - r^2$, so $r^2 = |\alpha|^2 - \beta > 0$, then
	\[
		z\bar{z} - \alpha\bar{z} - \bar{\alpha}z + \beta = 0
	\]
	A line can be written as a bisector between $w_1 \ne w_2 \in \mathbb{C}$:
	\[
		\begin{aligned}
			& |z - w_1| = |z - w_2| = (z - w_1)(\bar{z} - \bar{w}_1) = (z - w_2)(\bar{z} - \bar{w}_2) \\
			& = z\bar{z} - w_1 \bar{z} - \bar{w}_1 z + w_1 \bar{w}_1 = z\bar{z} - w_2 \bar{z} - \bar{w}_2 z + w_2 \bar{w}_2
		\end{aligned}
	\]
	Let $\alpha = w_1 - w_2 \ne 0$ and $\beta = w_1 \bar{w}_1 - w_2 \bar{w}_2 \in \mathbb{R}$, then
	\[
		-\alpha\bar{z} - \bar{\alpha}z + \beta = 0
	\]
\end{proof}

\begin{proposition}\label{prop:circlesAndLines}
	Mobius transformations map circles and lines in $\hat{\mathbb{C}}$ to circles and lines in $\hat{\mathbb{C}}$.
\end{proposition}

\begin{proof}
	Let $M_T$ be a Mobius transformation, $T = \begin{bmatrix} a & b \\ c & d \end{bmatrix}$. By Remark~\ref{rem:mobiusTransformationDet1}, we can assume that $\det(T) = 1$. If $c = 0$, then $M_T$ is an affine linear map and so preserves circles and lines, since rotations, dilations and translations do. So assume $c \ne 0$. Now
	\[
		M_T(z) = \frac{az + b}{cz + d} = \frac{caz + cb}{c(cz + d)} = \frac{a}{c} + \frac{cb - ad}{c(cz + d)} = \frac{a}{c} - \frac{1}{c(cz + d)} = \frac{a}{c} - \frac{1}{c^2} \frac{1}{z + d/c}
	\]
	So $M_T$ is the composition of linear maps and the function $f(z) = 1/z$. linear maps preserve circles and lines so we just need to prove that $f$ preserves circles and lines. Also $f^{-1}(z) = f(z)$ so we only need to consider one direction.

	Let $X$ be a circle or line. By Lemma~\ref{lem:equationOfCirclesAndLines}, $X$ is defined by the equation $\gamma z \bar{z} - \alpha \bar{z} - \bar{\alpha} z + \beta = 0$. If $z \in f(X)$ and $z \ne 0$, then $f(z) = f^{-1}(z) \in X$ hence
	\[
		\gamma(1 / z)\overline{(1 / z)} - \alpha \overline{(1 / z)} - \overline{\alpha}(1 / z) + \beta = 0 \Longleftrightarrow \beta z\bar{z} - \alpha \bar{z} - \bar{\alpha} z + \gamma = 0
	\]
	If $\beta = 0$ then the equation describes a line. If $\beta \ne 0$, then
	\[
		z \bar{z} - \frac{\alpha}{\beta} \bar{z} - \frac{\bar{\alpha}}{\beta} z + \frac{\gamma}{\beta} = 0
	\]
	Now $|\alpha / \beta|^2 - \gamma / \beta > 0$ since if $X$ is a line then $\gamma = 0$ and $\alpha \ne 0$, and if $X$ is a circle then $\gamma = 1$.
\end{proof}

\begin{definition}
	A \textbf{circline} is an object that is either a circle or a line.
\end{definition}

\begin{remark}
	A circle is determined by three of its points, hence to find the image of a circle under a Mobius transformation, we just need to find where the three points on the circle are mapped.
\end{remark}

\subsection{The Riemann Sphere Revisited}

Circles in $\hat{\mathbb{C}}$ correspond to circles in $S^2$ that don't pass through $N$ (the North pole).
\\
Lines in $\hat{\mathbb{C}}$ correspond to circle in $S^2$ that pass through $N$.

\begin{remark}
	Mobius transformations give all biholomorphic maps from $S^2$ to $S^2$.
\end{remark}

\begin{remark}
	Stereographic projections are conformal.
\end{remark}

\subsection{Mobius transformations preserving the upper half plane and the unit disc}

Notation: for a domain $D \subset \mathbb{C}$, let $Mob(D)$ be the set of Mobius transformations $f$ such that $f(D) = D$.

\begin{proposition}
	(H2H) Every Mobius transformation mapping $\mathbb{H}$ to $\mathbb{H}$ ($\mathbb{H} = \{ z \in \mathbb{C}: Im(z) > 0\}$) is of the form $M_T$ with $T \in SL_2 (\mathbb{R}) := \{T = \begin{matrix}
		()
	\end{matrix}: a, b, c, d \in \mathbb{R}, \det T = 1\}$
	
	Conversely, every such Mobius transformation maps $\mathbb{H}$ to $\mathbb{H}$ and hence a biholomorphism from $\mathbb{H}$ to $\mathbb{H}$.

	i.e. H2H: $f \in \text{Mob}(\mathbb{H}) \Leftrightarrow f = M_T$ with $T \in SL_2 (\mathbb{R})$.
\end{proposition}

\begin{remark}
	$T \rightarrow M_T$ gives a group homomorphism $SL_2 (\mathbb{R}) \rightarrow Aut(\mathbb{H})$
\end{remark}

\begin{proof}
	Any Mobius transformation $f: \mathbb{H} \rightarrow \mathbb{H}$ must map $\partial \mathbb{H}$ to $\partial \mathbb{H}$. As $\partial \mathbb{H}$ is the real line, $f: \mathbb{R} \cup \infty \rightarrow \mathbb{R} \cup \infty$. So $f$ must map the ordered set $\{1, 0, \infty \}$ to $\{x_1, x_2, x_3\}$ for some $x_i \in \mathbb{R} \cup \infty$.

	We know that the cross ratio is preserved under a Mobius transformation:
	\[(f(z), x_1; x_2, x_3) = \frac{(f(z) - x_2)(x_1 - x_3)}{(f(z) - x_3)(x_1 - x_2)} = \frac{z - 0}{1 - 0} = (z, 1; 0, \infty)\]
	\[\Leftrightarrow (f(z) - x_2)(x_1 - x_3) = z (f(z) - x_3)(x_1 - x_2)\]
	\[\Leftrightarrow f(z) = \frac{x_3 (x_1 - x_2) z + x_2 (x_3 - x_1)}{(x_1 - x_2)z + x_3 - x_1}\]

	We see that the coefficients of $T$ are real.
	
	If $T \in \GL_2 (\mathbb{R})$ and $z = x + iy$ then
	\[Im(M_T(z)) = Im(\frac{az + b}{cz + d}) = Im(\frac{(az + b)(c \bar{z} + d)}{|cz + d|^2})\]
	\[= Im(\frac{bc \bar{z} + adz}{(cz + d)}) = \frac{y \det T}{|cz+d|}\]

	We have $z \in \mathbb{H} \Leftrightarrow y > 0$ so $M_T(z) \in H \Leftrightarrow T \in \GL_2(\mathbb{R})$, $\det T > 0$. We can therefore replace $T$ by a real matrix of determinant 1 by scaling $T$ by a real number.
\end{proof}

\begin{proposition}
	(D2D): Every Mobius transformation from the unit disc $\mathbb{D} = \{z \in \mathbb{C}: |z| < 1\}$ to $\mathbb{D}$ is of the form $T \in SU(1, 1)$

	Conversely, every such Mobius transformation maps $\mathbb{D}$ to $\mathbb{D}$ and hence gives a h
	biholopmorhpic automorphism of $\mathbb{D}$.

	i.e. $f \in Mob(\mathbb{D}) \Leftrightarrow f = M_T$, $T \in SU(1, 1)$.
\end{proposition}

\begin{proof}
	($\Rightarrow$): Let $M_T: \mathbb{D} \rightarrow \mathbb{D}$ be a Mobius transformation. The Cayley map $H_C$ maps $\mathbb{H}$ to $\mathbb{D}$. We have that $f = M_C^{-1} \circ M_T \circ M_C$ is a Mobius transformation from $\mathbb{H}$ to $\mathbb{H}$. By proposition 4.20, we have $f = M_S$ where $S \in SL_2(\mathbb{R})$.

	Hence $C^{-1} T C = S \in SL_2(\mathbb{R})$ by Lemma 4.4.

	Let $S \in \mathbb{M}_2(\mathbb{R})$, $\det S = 1$. Then $T = CSC^{-1}$. Evalutating this shows $T \in SU(1, 1)$.

	($\Leftarrow$): If $T \in SU(1, 1)$, then the same calculation in reverse shows that the matrix $S = C^{-1} T C \in SL_2(\mathbb{R})$. Then $M_S: \mathbb{H} \rightarrow \mathbb{H}$ is a Mobius transformation by proposition 4.20 (H2H), and the map $M_T := M_C \circ M_S \circ M_C^{-1}$ is a Mobius transformation from $\mathbb{D}$ to $\mathbb{D}$
\end{proof}

\begin{remark}
	$T \rightarrow M_T$ gives a group homomorphism from $SU(1, 1)$ to $Aut(\mathbb{D})$.
\end{remark}

\begin{corollary}
	(D2D*):
	\begin{enumerate}
		\item Every Mobius transformation $f$ from $\mathbb{D}$ to $\mathbb{D}$ can be written as
		\[f(z) = e^{i\theta} \frac{z - z_0}{\bar{z_0}z - 1}\]
		for some angle $\theta$ and $z_0 \in \mathbb{D}$ where $z_0$ is the unique point in $\mathbb{D}$ such that $f(z_0) = 0$.
		\item Every Mobius transformation of the unit disc $\mathbb{D}$ to $\mathbb{D}$ for which $f(0) = 0$ are rotations about $0$.
	\end{enumerate}
\end{corollary}

\begin{proof}
	\begin{enumerate}
		\item By proposition D2D, we have
		\[f(z) = \frac{az + b}{\bar{b}z + \bar{a}} = \frac{a(z + b / a)}{-\bar{a}((-\bar{b} / \bar{a}) z - 1)} = -\frac{a}{\bar{a}} \frac{z - (-b / a)}{(-\bar{b} / \bar{a}) z - 1}\]

		So $z_0 = -\frac{b}{a}$. Since $|-\frac{a}{\hat{a}} = 1$, $-\frac{a}{\hat{a}} = e^{i\theta}$ for some $\theta \in (-\pi, \pi]$.

		$|z_0|^2 - 1 = |-\frac{b}{a}|^2 - 1 = \frac{|b|^2}{|a|^2} - 1$. Now $1 = |a|^2 - |b|^2$ so $|z_0|^2 - 1 = \frac{-1}{|a|^2} < 0$ so $|z_0|^2 < 1$ and so $|z_0| < 1$.
		\item $f(0) = 0 \Leftrightarrow e^{i\theta} \frac{0 - z_0}{\bar{z_0}\cdot 0 - 1} = 0 \Leftrightarrow z_0 = 0 \Leftrightarrow f(z) = e^{i\theta}\frac{z - 0}{0 - 1} = e^{-i\theta}z$.
		
		So $f$ is a rotation.
	\end{enumerate}
\end{proof}

\begin{remark}
	The map $g(z) = \frac{z - z_0}{\bar{z_0}z - 1}$ swaps $z_0$ and $0$ and is an involution ($g \circ g = Id$). Also, $z \rightarrow e^{i\theta}z$ is a rotation.

	So every Mobius transformation from $\mathbb{D}$ to $\mathbb{D}$ is given by an involution followed by a rotation.
\end{remark}

\subsection{Finding biholomorphic maps between domains}

To find a biholomorphism $f$ between domains, we build $f$ in various stages using simpler known maps.

\begin{example}
	Find biholomorphism from $D = \{z \in \mathbb{D}: Im(z) < 0\}$ to $\mathbb{H}$.

	The Cayley Map $M_C$ is a map from $\mathbb{H}$ to $\mathbb{D}$, so $M_C^{-1}: \mathbb{D} \rightarrow \mathbb{H}$, $M_C^{-1}(z) = \frac{iz + i}{-z + 1}$.

	To find the image of $D$ under $M_C^{-1}$, consider how it acts on two segments of $\delta D$:

	\begin{itemize}
		\item Under $M_C^{-1}$, $-1 \rightarrow 0$, $0 \rightarrow i$ and $1 \rightarrow \infty$. Therefore the line segment from $-1$ to $1$ through $0$ is mapped to the positive imaginary axis.
		\item Under $M_C^{-1}$, $-i \rightarrow 1$, so the circular arc from $-1$ to $1$ through $-i$ is mapped to the positive real axis.
	\end{itemize}

	Now $-\frac{i}{2} \in D$ and $M_C^{-1}(-\frac{i}{2}) = \frac{4 + 3i}{5}$. The image of $D$ under $M_C^{-1}$ is $\Omega = \{w \in \mathbb{C}: 0 < Arg(w) < \frac{\pi}{2}\}$.

	Now we find a biholomorphic map from $\Omega$ to $\mathbb{H}$. $g(z) = z^2$ satisfies this, as it doubles the argument of $z$.

	So the map is $f = g \circ M_C^{-1}$, $f: D \rightarrow \mathbb{H}$.
\end{example}

\section{Notions of convergence in complex analysis and power series}

\subsection{Pointwise and uniform convergence}

\begin{definition}
	Let $(X, d_X)$ and $(Y, d_Y)$ be two metric spaces. A sequence of functions ${\{f_n\}}_{n \in \mathbb{N}}: X \rightarrow Y$ converges pointwise (on $X$) to $f$ if for every $x \in X$, the limit function $f(x) := \lim_{n \rightarrow \infty} f_n(x)$ exists in $Y$.

	In other words, we have for every $x \in X$ and for every $\epsilon > 0$, for some $N \in \mathbb{N}$, for every $n > N$, $d_Y(f_n(x), f_n(x)) < \epsilon$. (Not that $N$ depends on $x$).
\end{definition}

\begin{remark}
	For every $x \in X$, $f_n(x)$ is just a sequence of points in $Y$. The above definition is what we get by applying definition 2.11 (in notes) to the sequence $f_n(z)$.
\end{remark}

\begin{example}
	Let $f_n(z) = z^n$, $f_n: \mathbb{C} \rightarrow \mathbb{C}$. There are the following cases:

	\begin{enumerate}
		\item $z \in \mathbb{D}$. Let $\epsilon > 0$. Then $|z|^N < \epsilon$ for every $N > \frac{\log \epsilon}{\log |z|}$. So for every $n > N$ we have $f_n(z) - 0 = |z|^n < |z|^N \epsilon$, hence $\lim_{n \rightarrow \infty} f_n(z) = 0 \in \mathbb{D}$.
		\item $|z| = 1$. The point $z$ rotates around the unit circle $\delta \mathbb{D}$ by $Arg(z)$ anticlockwise every iteration. For $z \ne 1$, this sequence doesn't converge. But for $z = 1$, $\lim_{n \rightarrow \infty} f_n(z) = \lim_{n \rightarrow \infty} 1 = 1$.
		\item $|z| > 1$. The value of $|z|^n$ is unbounded so doesn't converge.
	\end{enumerate}

	The sequence $f_n$ doesn't converge pointwise on $\mathbb{C}$. But it is pointwise convergent on $\mathbb{D} \cup {1}$ with limit function:

	\begin{equation}
		f(z) =
		\begin{cases}
			0 & \text{if } z \in \mathbb{D}\\
			1 & \text{if } z = 1
		\end{cases}
	\end{equation}
\end{example}

\begin{definition}
	Let $(X, d_X)$ and $(Y, d_Y)$ be two metric spaces. A sequence of functions ${\{f_n\}}_{n \in \mathbb{N}}: X \rightarrow Y$ converges uniformly (on $X$) to the limit function $f$ if for every $\epsilon > 0$ for some $N \in \mathbb{N}$, for every $n > N$, $d_Y(f_n(x), f(x)) < \epsilon$ for every $x \in X$.
\end{definition}

\begin{theorem}
	Let $(X, d_X)$ and $(Y, d_Y)$ be two metric spaces and let ${\{f_n\}}_{n \in \mathbb{N}}: X \rightarrow Y$ be a sequence of functions that converges uniformly to $f$ on $X$.

	Then $f$ is continuous on $X$.
\end{theorem}

\begin{proof}
	Same as in Analysis I.
\end{proof}

\begin{lemma}
	let ${\{f_n\}}_{n \in \mathbb{N}}: X \rightarrow \mathbb{C}$ be a sequence of functions converging pointwise to a limit function $f$.

	\begin{enumerate}
		\item If $|f_n(x) - f(x)| \le s_n$ for every $x \in X$ where ${\{s_n\}}_{n \in \mathbb{N}}$ is some sequence in $\mathbb{R} > 0$ (independent of $x$) with $\lim_{n \rightarrow \infty} s_n = 0$ then $f_n$ converge uniformly to $f$ on $X$.
		\item If for some sequence $x_n \in X$, $|f_n(x_n) - f(x_n)| \ge c$ for some positive constant $c$ then $f_n$ does not converge uniformly to $f$ on $X$.
	\end{enumerate}
\end{lemma}

\begin{theorem}
	(Weierstrass M-test): Let $f_n: X \rightarrow \mathbb{C}$ be a sequence of fucntions such that $|f_n(x)| \le M_n$ for every $x \in X$ and $\sum_{n = 1}^{\infty} M_n < \infty$.

	Then $\sum_{n = 1}^{\infty} f_n$ converges uniformly on $X$ to some limit function $f: X \rightarrow \mathbb{C}$.
\end{theorem}

\begin{proof}
	Similar to Analysis I.
\end{proof}

\begin{theorem}
	Let a sequence of functions $f_n: [a, b] \rightarrow \mathbb{R}$ converge uniformly on an interval $[a, b]$ to some function $f$, such that $\{f_n\}$ are all continuous. Then

	\[\lim_{n \rightarrow \infty} \int_a^c f_n(x) dx = \int_a^c f(x) dx \text{ for every } c \in [a, b]\]
\end{theorem}

\begin{definition}
	Let ${\{f_n\}}_{n \in \mathbb{N}}$ be a sequence of functions in a metric space $X$. $f_n$ converges locally uniformly (on $X$) to the limit function $f$ if for every $x \in X$, for some open set $U \subset X$ containing $x$, $f_n$ converges uniformly to $f$ on $U$.
\end{definition}

\begin{theorem}
	Let ${\{f_n\}}_{n \in \mathbb{N}}$ be a sequence of continuous functions which converges locally uniformly on $X$ to a limit function $f$. Then $f$ is continuous on $X$.
\end{theorem}

\begin{proof}
	For every $x \in X$, $f_n$ converges uniformly on some open set $U$ containing $x$. Hence $f$ is continuous on $U$ by theorem 5.5 (in notes). So $f$ is continuous at $x$ for every $x \in X$.
\end{proof}

\begin{remark}
	The limit of a locally uniform convergent sequence of holomorphic functions is again holomorphic.
\end{remark}

\begin{example}
	For every $w \in \mathbb{D}$, for some $r < 1$, $w \in B_r(0)$ and $B_r(0)$ is open. Then for every $z \in B_r(0)$, $|z|^n < r^n$ and $\lim_{n \rightarrow \infty} r_n = 0$. So by lemma 5.6 (in notes), with $s_n = r^n$, $f_n$ converges uniformly to $f$ in $B_r(0)$.
\end{example}

\begin{remark}
	To prove that the limit function is conitnuous on all of $\mathbb{D}$, it is enough to prove locally uniform convergence on every ball $B_r(0)$, $0 < r < 1$, in $\mathbb{D}$.
\end{remark}

\begin{theorem}
	Let $X$ be a metric space and let $f_n: X \rightarrow \mathbb{C}$ be a sequence of continuous functions such that for any $y \in X$, there is an open $U \subset X$ containing $y$ and constants $M_n > 0$ with $\sum_{n = 1}^{\infty} M_n < \infty$ and $|f_n(x)| \le M_n$ for every $x \in U$. Then $\sum_{n = 1}^{\infty} f_n$ converges locally uniformly to a continuous function on $X$.
\end{theorem}

\begin{proof}
	Given $y \in X$, the hypotheses of the theorem imply that for some constants $M_n > 0$, $|f_n(y)| \le M_n$ and $\sum_{n = 1}^{\infty} M_n < \infty$.

	\[|F_k(y)| = |\sum_{n = 1}^k f_n(y)| \le \sum_{n = 1}^{\infty} |f_n(y)| \le \sum_{n = 1}^k M_n\]

	As $k \rightarrow \infty$, the RHS $\sum_{n = 1}^k M_n$ converges so it must be bounded, and let the upper bound by $L$. Thus for every $k$, $|F_k(y)| \le L$. So the sequence ${(F_k(y))}_k$ is bounded, hence it lies in some boundedd, closed ball in $\mathbb{C}$, which is compact by Heine-Borel.

	Therefore there is a subsequence ${(F_{k_j}(y))}_{k_j}$ that converges to $F(y)$.

	Now, for $k_j > k$,

	\[|F_{k_j}(y) - F_k(y)| = |\sum_{n = k + 1}^{k_j} f_n(y)| \le \sum_{n = k + 1}^{k_j} |f_n(y)| \le \sum_{n = k + 1}^{k_j} M_n\]

	Taking the limit as $j \rightarrow \infty$, both the LHS and RHS converge, and we get

	\[|F(y) - F_k(y)| \le \sum_{n = k + 1}^{\infty} M_n\]

	Now taking the limit as $k \rightarrow \infty$, we get

	\[\lim_{k \rightarrow \infty} |F(y) - F_k(y)| = 0\]

	since the RHS tends to zero.

	Repeating this for every $y$, $F_k \rightarrow F$ pointwise on $X$.

	From the hypotheses of the theorem, we have that for every $y \in X$, for some open $U \subset X$ containing $y$ and constants $M_n > 0$ with $\sum_{n = 1}^{\infty} < \infty$ and $|f_n(x)| \le M_n$ for every $x \in U$.

	Then, for every $x \in U$ and for every $L > k$,

	\[|F_L(x) - F_k(x)| = |\sum_{n = k + 1}^L f_n(x)| = \sum_{n = k + 1}^L |f_n(x)| \le \sum_{n = k + 1}^L M_n\]

	Taking the limit as $l \rightarrow \infty$:

	\[|F(x) - F_k(x)| \le \sum_{n = k + 1}^{\infty} M_n\]

	for every $x \in U$.

	$\lim_{k \rightarrow \infty} \sum_{n = k + 1}^{\infty} M_n = 0$. So by lemma 5.6 (in notes), $F_k \rightarrow F$ uniformly on $U$.
\end{proof}

\subsection{Complex power series}

\begin{theorem}
	A complex power series is an expression of the form
	\[
		\sum_{n = 0}^{\infty} a_n {(z - c)}^n, \quad a_n, c \in \mathbb{C}^2
	\]
	There are three cases:

	\begin{enumerate}
		\item $\sum_{n = 0}^{\infty} a_n {(z - c)}^n$ converges only for $z = c$ ($R = 0$).
		\item There exists $R > 0$ (radius of convergence) such that
		\begin{itemize}
			\item $\sum_{n = 0}^{\infty} a_n {(z - c)}^n$ converges absolutely for $|z - c| < R$ (We call $B_R(c)$ the disc of convergence).
			\item $\sum_{n = 0}^{\infty} a_n {(z - c)}^n$ diverges for $|z - c| > R$ (anything can happen on the circle $|z - c| = R$).
		\end{itemize}
		\item $\sum_{n = 0}^{\infty} a_n {(z - c)}^n$ converges absolutely for every $z \in \mathbb{C}$ ($R = \infty$).
	\end{enumerate}
\end{theorem}

\begin{remark}
	Radius of convergence is usually determined via ratio test or root test.
\end{remark}

\begin{theorem}
	A power series $\sum_{n = 0}^{\infty} a_n {(z - c)}^n$ with radius of convergence $0 < R < \infty$ converges uniformly on every ball $B_r(c)$ with $0 < r < R$. This implies that the power series is locally uniformly convergent on its disc of convergence.
\end{theorem}

\begin{proof}
	Follows via the M-test.
\end{proof}

\begin{remark}
	The power series do not converge uniformly in the entire disc of conergence $B_R(c)$.
\end{remark}

\begin{proposition}
	Let $\sum_{n = 0}^{\infty} a_n {(z - c)}^n$ be a power series with radius of convergence $0 < R < \infty$. Then the formal derivatives and antiderivatives

	\[\sum_{n = 0}^{\infty} n a_n {(z - c)}^{n - 1}\] and

	\[\sum_{n = 0}^{\infty} \frac{a_n}{n + 1} {(z - c)}^{n + 1}\]

	have the same radius of convergence $R$.
\end{proposition}

\begin{theorem}
	Let $\sum_{n = 0}^{\infty} a_n {(z - c)}^n$ be a power series with radius of convergence $0 < R < \infty$ and let $f: B_R(c) \rightarrow \mathbb{C}$ be the resulting limit function. Then $f$ is holomorphic on $B_R(c)$ with

	\[f'(z) = \sum_{n = 0}^{\infty} n a_n {(z - c)}^{n - 1}\] for $z \in B_R(c)$.
\end{theorem}

\begin{proof}
	Assume $c = 0$ (the general case for $c$ is analogous).

	\[f(z) - f(w) = \sum_{n = 1}^{\infty} a_n (z^n - w^n) = \sum_{n = 1}^{\infty} (z - w) q_n(z)\]
	
	where $q_n(z) = \sum_{k = 0}^{n - 1} w^k z^{n - 1 - k}$.

	So for $z \ne w$, let $h(z) := \frac{f(z) - f(w)}{z - w} = \sum_{n = 1}^{\infty} a_n q_n(z)$

	Given $z_0 \in B_R(0)$, let $r < R$ such that $w, z_0 \in B_r(0)$. To apply the local M-test, we need constants $M_n$ for this set $B_r(0)$ that bound the terms $a_n q_n(z)$ defining $h$.

	For $z \in B_r(0)$,

	\[|a_n q_n(z)| = |a_n \sum_{k = 0}^{n - 1} w^k z^{n - 1 - k}| \le |a_n| \sum_{k = 0}^{n - 1} |w|^k |z|^{n - 1 - k} < |a_n| \sum_{k = 0}^{n - 1} r^{n - 1} = n |a_n| r^{n - 1}\]

	So let $M_n = n|a_n| r^{n - 1}$, then $\sum_{n = 1}^{\infty} M_n = \sum_{n = 1}^{\infty} n|a_n| r^{n - 1}$ which converges by proposition 5.19 (in lecture notes).

	The formal derivative $\sum_{n = 1}^{\infty} n a_n r^{n - 1}$ has radius of convergence $R$ so converges absolutely on its disc of convergence $B_R(0)$. In particular, it converges at $z = R$. By the local M-test, the series defining $h$ converges locally uniformly to a continuous function on $B_R(0)$. Hence
	
	\[\lim_{z \rightarrow w} \frac{f(z) - f(w)}{z - w} = \lim_{h \rightarrow w} h(z) = h(w) = \sum_{n = 1}^{\infty} a_n q_n(w) = \sum_{n = 1}^{\infty} n a_n w^{n - 1}\]
\end{proof}

\begin{corollary}
	A power series $f$ as theorem 5.21 (in lecture notes) with positive radius of convergence $R$ can be differentiated infinitely many times and
	
	\[f^{(k)} := \sum_{n = k}^{\infty} k! {n \choose k} a_n {(z - c)}^{n - k}\] for $z \in B_R(c)$
\end{corollary}

\begin{corollary}
	A power series $f$ as in theorem 5.21 (in lecture notes) with positive radius of convergence $R$ has a holomorphic antiderivative $F: B_R(c) \rightarrow \mathbb{C}$, with $F'(z) = f(z)$, defined by

	\[F(z) := \sum_{n = 0}^{\infty} \frac{a_n}{n + 1} {(z - c)}^{n + 1}\]
\end{corollary}

\hfill

\section{Complex integration over contours}

\subsection{Definition of contour integrals}

\begin{definition}
	For a continuous function $f: [a, b] \rightarrow \mathbb{C}$, with $f(z) = u(z) + i v(z)$,

	\[\int_a^b f(t) dt := \int_a^b u(t) dt + i \int_a^b v(t) dt \in \mathbb{C}\]
\end{definition}

\begin{lemma}
	\hfill
	\begin{enumerate}
		\item Let $f_1$ and $f_2$ be continuous functions from $[a, b]$ to $\mathbb{C}$. Then $\int_a^b (f_1(t) + f_2(t)) dt = \int_a^b f_1(t) dt + \int_a^b f_2(t) dt$.
		\item For any complex number $c \in \mathbb{C}$ and continuous function $f: [a, b] \rightarrow \mathbb{C}$,
		
		\[ \int_a^b c f(t) dt = c \int_a^b f(t) dt \]
	\end{enumerate}
\end{lemma}

\begin{definition}
	A smooth curve in $\mathbb{C}$ is a continuously differentiable function $\gamma: [0, 1] \rightarrow \mathbb{C}$ (i.e. differentiable with continuous derivative). More generally we can consider continuously differentiable curves $\gamma: [a, b] \rightarrow \mathbb{C}$. We say that such curves are $C^1$.

\end{definition}

\begin{remark}
	We write $\gamma(t) = u(t) + i v(t)$ with $u, v: [a, b] \rightarrow \mathbb{R}$. Then the derivative $\gamma'$ is defined as

	\[ \gamma'(t) := u'(t) + i v'(t) \]

	At the endpoints, we demand that the one-sided derivative exists and is continuous from the one side:

	\[ \gamma'(b) := \lim_{h \rightarrow 0^-} \frac{u(b + h) - u(b)}{h} + i \lim_{h \rightarrow 0^-} \frac{v(b + h) - v(b)}{h} \] exists and

	\[ \lim_{t \rightarrow b^-} \gamma'(t) = \gamma'(b) \]
\end{remark}

\begin{definition}
	Let $U \subset \mathbb{C}$ be an open set, and $f: U \rightarrow \mathbb{C}$ be a continuous function. Let $\gamma: [a, b] \rightarrow U$ be a $C^1$ curve. The integral of $f$ along the curve $\gamma$ is defined as

	\[ \int_{\gamma} f(z) dz = \int_a^b f(\gamma(t)) \gamma'(t) dt \]
\end{definition}

\begin{corollary}
	Properties of the integral along a curve:
	\begin{enumerate}
		\item $\int_{\gamma} (f_1(z) + f_2(z)) dz = \int_{\gamma} f_1(z) dz + \int_{\gamma} f_2(z) dz$
		\item For $c \in \mathbb{C}$, $\int_{\gamma} c f(z) dz = c \int_{\gamma} f(z) dz$
	\end{enumerate}
\end{corollary}

\begin{proof}
	Easy
\end{proof}

\begin{definition}
	Given $\gamma: [a, b] \rightarrow \mathbb{C}$, the curve $(-\gamma): [-b, -a] \rightarrow \mathbb{C}$ is defined as
	
	\[ (-\gamma)(t) := \gamma(-t) \] Then we have
	
	\[ \int_{-\gamma} f(z) dz = -\int_{\gamma} f(z) dz \]
\end{definition}

\begin{lemma}
	Let $U \subset \mathbb{C}$ be an open set, $f: U \rightarrow \mathbb{C}$ be continuous and $\gamma: [a, b] \rightarrow \mathbb{C}$ be a $C^1$ curve. If $\phi: [a', b'] \rightarrow [a, b]$ with $\phi(a') = a$ and $\phi(b') = b$ is continuously differentiable and we define $\delta: [a', b'] \rightarrow \mathbb{C}$, $\delta := \gamma \circ \phi$, then

	\[ \int_{\gamma} f(z) dz = \int_{\delta} f(z) dz \]
\end{lemma}

\begin{proof}
	\[\int_{\delta} f(z) dz = \int_{a'}^{b'} f(\delta(t)) \delta'(t) dt = \int_{a'}^{b'} f(\gamma(\phi(t))) (\gamma(\phi(t)))' dt\]
	
	\[ = \int_{a'}^{b'} f(\gamma(\phi(t))) \gamma'(\phi(t)) \phi'(t) dt\]
	With a change of variables $s = \phi(t)$, $ds = \phi'(t) dt$:

	\[ \int_{a'}^{b'} f(\gamma(\phi(t))) \gamma'(\phi(t)) \phi'(t) dt = \int_a^b f(\gamma(s)) \gamma'(s) ds = \int_{\gamma} f(z) dz \]
\end{proof}

\begin{definition}
	Let $\gamma: [a, b] \rightarrow \mathbb{C}$ be a curve and suppose there exist $a = a_0 < a_1 < \cdots < a_n = b$ such that the curves $\gamma_i: [a_{i - 1}, a_i] \rightarrow \mathbb{C}$, defined by $\gamma_i(t) = \gamma(t)$ for $t \in [a_{i - 1}, a_i]$ are $C^1$ curves. Then $\gamma$ is a piecewise $C^1$ curve or contour.

	For a contour $\gamma$ above, a contour integral is defined as

	\[ \int_{\gamma} f(z) dz = \sum_{n = 1}^n \int_{\gamma_i} f(z) dz \]
\end{definition}

\begin{definition}
	If $\gamma: [a, b] \rightarrow \mathbb{C}$ and $\delta: [c, d] \rightarrow \mathbb{C}$ are two contours with $\gamma(b) = \delta(c)$ the contour $\gamma \cup \delta: [a, b + d - c] \rightarrow \mathbb{C}$ is defined as

	\[ (\gamma \cup \delta)(t) := \begin{cases}
		\gamma(t) & \text{ if } a \le t \le b \\
		\delta(t) & \text{ if } c \le t \le d
	\end{cases} \]

	Then

	\[ \int_{\gamma \cup \delta} f(z) dz = \int_{\gamma} f(z) dz + \int_{\delta} f(z) dz \]
\end{definition}

\subsection{The fundamental theorem of calculus}

\begin{theorem}
	Let $U \in \mathbb{C}$ be an open set and let $F: U \rightarrow \mathbb{C}$ be holomorphic with continuous derivative $f$. Then for every contour $\gamma: [a, b] \rightarrow U$,

	\[ \int_{\gamma} f(z) dz = F(\gamma(b)) - F(\gamma(a)) \]
	In particular, if $\gamma$ is closed, so $\gamma(a) = \gamma(b)$, then

	\[ \int_{\gamma} f(z) dz = 0 \]
\end{theorem}

\begin{proof}
	First consider the case where $\gamma$ is a $C^1$ curve. Let $F = u + iv$. Then

	\[ \int_{\gamma} f(z) dz = \int_{\gamma} F'(z) dz = \int_a^b F'(\gamma(t)) \gamma'(t) dt = \int_a^b (F(\gamma(t)))' dt\]

	\[ = \int_a^b (u(\gamma(t)))' dt + i \int_a^b (v(\gamma(t)))' dt = {[u(\gamma(t))]}_a^b + i {[v(\gamma(t))]}_a^b \]

	\[ = u(\gamma(b)) - u(\gamma(b)) + i(v(\gamma(b)) - v(\gamma(b))) = F(\gamma(b)) - F(\gamma(a)) \]
	Now extend this proof to any contour.

	Let $\gamma: [a, b] \rightarrow \mathbb{C}$ be a contour, then for some $a = a_0 < a_1 < \cdots < a_n = b$, the curves $\gamma_i: [a_{i - 1}, a_i] \rightarrow \mathbb{C}$, $i = 1, \ldots, n$, defined by $\gamma_i(t) = \gamma(t)$ for $t \in [a_{i - 1}, a_i]$ are $C^1$ curves. Then

	\[ \int_{\gamma} f(z) dz = \int_{\gamma} F'(z) dz = \sum_{i = 1}^n \int_{\gamma_i} F'(z) dz \]

	\[ = \sum_{i = 1}^n (F(\gamma(a_i)) - F(\gamma(a_{i - 1}))) = F(\gamma(a_n)) - F(\gamma(a_0)) = F(\gamma(b)) - F(\gamma(a)) \]
\end{proof}

\begin{remark}
	Under the hypotheses on $F$, the integral only depends on the endpoints of the curve.
\end{remark}

\begin{theorem}
	If $f: [a, b] \rightarrow \mathbb{R}$ is continuous,
	
	\[\int_a^b f(t) dt \le \int_a^b \max_{t \in [a, b]} f(t) dt \le (b - a) \max_{t \in [a, b]}\]
\end{theorem}

\begin{proof}
	From Analysis I.
\end{proof}

\begin{definition}
	Let $\gamma: [a, b] \rightarrow \mathbb{C}$ be a contour. The \textbf{length} of $\gamma$ is defined as

	\[L(\gamma) = \int_a^b |\gamma'(t)| dt\]
\end{definition}

\begin{lemma}
	(\textbf{The Estimation Lemma}) Let $f: U \rightarrow \mathbb{C}$ be continuous and $\gamma: [a, b] \rightarrow U$ be a contour. Then

	\[ \left | \int_{\gamma} f(z) dz \right | \le L(\gamma) \sup_{\gamma} |f| \]
	where $\sup_{\gamma} |f| := \sup \{ |f(z)|: z \in \gamma \}$.
\end{lemma}

\begin{proof}
	First prove that for a continuous function $g: [a, b] \rightarrow \mathbb{C}$,

	\[ \left | \int_a^b g(t) dt \right | \le \int_a^b |g(t)| dt \]
	If we write $\int_a^b g(t) dt = r e^{i \theta}$ with $r \ge 0$, then

	\[ \left | \int_a^b g(t) dt \right | = |r e^{i \theta}| = r = \Re \left( e^{-i \theta} \int_a^b g(t) dt \right ) \]

	\[ = \Re\left( \int_a^b g(t) e^{-i \theta} dt \right) = \int_a^b \Re( g(t) e^{-i \theta}) dt \le \int_a^b \left |e^{-i \theta} g(t) \right | dt = \int_a^b |g(t)| dt \]

	Let $g(t) = f(\gamma(t)) \gamma'(t)$, then

	\[ \left | \int_{\gamma} g(z) dz \right | = \left | \int_a^b f(\gamma(t)) \gamma'(t) dt \right | \le \int_a^b \left | f(\gamma(t)) \gamma'(t) \right | dt \]
	Then

	\[ \int_a^b \left | f(\gamma(t)) \gamma'(t) \right | dt \le \sup_{\gamma} |f| \int_a^b |\gamma'(t)| dt = L(\gamma) \sup_{\gamma} |f| \]
\end{proof}

\begin{theorem}
	(Converse to FTC) Let $f: D \rightarrow \mathbb{C}$ be continuous on a domain $D$. If $\int_{\gamma} f(z) dz = 0$ for every closed contour $\gamma \in D$, for some $F: D \rightarrow \mathbb{C}$, $F'(z) = f(z)$.
\end{theorem}

\begin{proof}
	Let $a_0 \in D$. For every $a_0 \ne w \in D$, let $\gamma(w)$ be a contour connecting $a_0$ to $w$ and is contained in $D$.

	Since $D$ is a domain, it is path-connected, i.e. there is a smooth path $\gamma_w$ connecting $a_0$ to $w$, therefore the collection of contours contained in $D$ and connecting $a_0$ and $W$ is non-empty. Let

	\[ F(w) := \int_{\gamma(w)} f(z) dz \]

	Let $\tilde{\gamma} (w)$ be another contour that connects $a_0$ to $w$ and is contained in $D$. Then let $c(w) = \gamma(w) \cup (-\tilde{\gamma}(w))$ that is obtained by moving through $\gamma$ then through $\tilde{\gamma}$ in the opposite direction. Since $c$ is a closed contour in $D$, $\int_C f(z) dz = 0$.

	Then $0 = \int_C f(z) dz = \int_{\gamma(w) \cup (-\tilde{\gamma}(w))} f(z) dz = \int_{\gamma(w)} f(z) dz + \int_{-\tilde{\gamma}(w)} f(z) dz = \int_{\gamma(w)} f(z) dz - \int_{\tilde{\gamma}(w)} f(z) dz$. Hence

	\[ \int_{\gamma(w)} f(z) dz = \int_{\tilde{\gamma}(w)} f(z) dz \]
	Therefore $F$ does not depend on the contour chosen to join $a_0$ to $w$.

	Now we claim $F$ is holomorphic and we claim that $F$ is holomorphic and $\forall z \in D, F'(z) = f(z) \Rightarrow \lim_{h \rightarrow 0} \frac{F(w + h) - F(w)}{h} = f(w)$.

	To evaluate $F(w + h)$ we need a contour joining $a_0$ to $w + h$ contained in $D$. For every $w \in D$, let $r > 0$ such that $B_r(w) \subset D$. This ball must exist since $D$ is open. Then for every $h \in \mathbb{C}$ with $|h| < r$ consider the striaght line $\delta_h$ that connects $w$ to $w + h$.

	A parameterisation of this line is given by
	
	\[ \delta_h: [0, 1] \rightarrow D, \quad \delta_h(t) = w + t h \]
	The contour $\gamma_w \cup \delta_h$ is contained in $D$. So

	\[ F(w + h) = \int_{\gamma_w \cup \delta_h} f(z) dz = \int_{\gamma_w} f(z) dz + \int_{\delta_h} f(z) dz = F(w) + \int_{\delta_h} f(z) dz \]

	\[ \int_{\delta_h} f(w) dz = f(w) \int_{\delta_h} dz = f(w) \int_0^1 h dt = h f(w) \]
	We can rewrite the previous equation as
	\[ F(w + h) = F(w) + h f(w) + \int_{\delta_h} (f(z) - f(w)) dz \]
	For $h \ne 0$,

	\[ \left| \frac{F(w + h) - F(w)}{h} - f(w) \right| = \frac{1}{|h|} \left| \int_{\delta_h} (f(z) - f(w)) dz \right| \]
\end{proof}

\subsection{First Version of Cauchy's Theorem}

\begin{definition}
	A domain $D$ is \textbf{starlike} if for some point $a_0 \in D$, for every $b \ne a_0 \in D$, the straight line connecting $a_0$ and $b$ is contained in $D$.
\end{definition}

\begin{example}
	\hfill
	\begin{enumerate}
		\item $\mathbb{C}$ is starlike.
		\item The ball $B_r(a)$ is starlike.
		\item Any convex set is starlike.
	\end{enumerate}
\end{example}

\begin{example}
	\hfill
	\begin{enumerate}
		\item $\mathbb{C}^{\star}$ is not starlike, because a straight line between two points could pass through $0$, and $0 \notin \mathbb{C}^{\star}$.
		\item Similarly, $B_r^{\star}(a) = B_r(a) - \{ a \}$ is not starlike.
	\end{enumerate}
\end{example}

\begin{lemma}\label{lem:ctLem1}
	Let $U$ be an open set and let $f: U \rightarrow \mathbb{C}$ be holomorphic. Then
	\[
		\int_{\partial \Delta} f(z) dz = 0
	\]
	for every \textbf{triangle} $\Delta$ in $U$.
\end{lemma}

\begin{remark}
	Here $\partial \Delta$ is the boundary of $\Delta$, traversed anticlockwise.
\end{remark}

\begin{remark}
	Given any closed contour without a parameterisation given, we will assume that it is traversed anticlockwise.
\end{remark}

\begin{proof}
	First, split $\Delta$ into four triangles, $\Delta^{(1)}, \Delta^{(2)}, \Delta^{(3)}, \Delta^{(4)}$, using the midpoints of each side. Then
	\[
		\int_{\partial \Delta} f(z) dz = \sum_{i = 1}^{4} \int_{\partial \Delta^{(i)}} f(z) dz
	\]
	Let $\Delta_1$ be one of these four triangles which has the largest integral, then
	\[
		\left| \int_{\partial \Delta} f(z) dz \right| \le 4 \left| \int_{\partial \Delta_1} f(z) dz \right|
	\]
	We then continue this procedure to produce a sequence of triangles
	\[
		\Delta > \Delta_1 > \cdots > \Delta_n > \cdots
	\]
	The length of $\Delta_1$, $L(\Delta_1)$ satisfies $L(\Delta_1) = \frac{1}{2} L(\Delta)$, therefore
	\[
		L(\Delta_n) = \frac{1}{2} n L(\Delta) \Longrightarrow L(\Delta_n) \rightarrow \text{ as } n \rightarrow \infty
	\]
	Also,
	\[
		\bigcap_{n \in \mathbb{N}} \Delta_n = \{ w \}
	\] is a single point in $D$. Now, notice that
	\[
		\int_{\partial \Delta_n} 1 dz = 0 = \int_{\partial \Delta_n} z dz
	\]
	and that $w, f(w), f'(w)$ are constants. Then sneakily,
	\[
		\int_{\partial \Delta_n} f(z) dz = \int_{\partial \Delta_n} \left( f(z) - f(w) - (z - w) f'(w) \right)
	\]
	Define the auxiliary function
	\[
		g(z) = \begin{cases}
			\frac{f(z) - f(w)}{z - w} - f'(w) & \text{ if } z \in D \ \backslash \ \{ w \} \\
			0 & \text{ if } z = w
		\end{cases}
	\]
	which is continuous at $z = w$, so is continuous on $D$. So
	\[
		\int_{\partial \Delta_n} f(z) dz = \int_{\partial \Delta_n} (z - w) g(z) dz
	\]
	Now,
	\[
		\left| \int_{\partial \Delta} f(z) dz \right| \le 4^n \left| \int_{\partial \Delta_n} f(z) dz \right| = 4^n \left| \int_{\partial \Delta_n} (z - w) g(z) dz \right|
	\]
	Note
	\[
		\sup_{z \in \partial \Delta_n} |z - w| \le L(\partial \Delta_n)
	\]
	so by the Estimation Lemma,
	\[
		\begin{aligned}
			\left| \int_{\partial \Delta} f(z) dz \right| & \le 4^n L(\partial \Delta_n) \sup_{z \in \partial \Delta_n} | (z - w) g(z) | \\
			& \le 4^n L(\partial \Delta_n) \sup_{z \in \partial \Delta_n} | (z - w) | \sup_{z \in \partial \Delta_n} | g(z) | \\
			& \le 4^n {(L(\partial \Delta_n))}^2 \sup_{z \in \partial \Delta_n} | g(z) | \\
			& = {L(\Delta)}^2 \sup_{z \in \partial \Delta_n} | g(z) |
		\end{aligned}
	\]
	As $n \rightarrow \infty$, $\sup_{z \in \partial \Delta_n} | g(z) | \rightarrow g(w) = 0$. This completes the proof.
\end{proof}

\begin{lemma}\label{lem:ctLem2}
	Let $D$ be a starlike domain and $f: D \rightarrow \mathbb{C}$ be continuous. Then, if
	\[
		\int_{\partial \Delta} f(z) dz = 0
	\]
	for every $\Delta \subset D$, then for some $F: D \rightarrow \mathbb{C}$,
	\[
		F'(z) = f(z) \quad \forall z \in D
	\]
\end{lemma}

\begin{proof}
	Similar to the proof of converse of FTC.
\end{proof}

\begin{theorem}
	(\textbf{Cauchy's Theorem for Starlike Domains - CTSD}) Let $D$ be a starlike domain and let $f: D \rightarrow \mathbb{C}$ be holomorphic. Then for every closed contour $\gamma \in D$,
	\[
		\int_{\gamma} f(z) dz = 0
	\]
\end{theorem}

\begin{proof}
	By Lemma~\Ref{lem:ctLem1},
	\[
		\int_{\partial \Delta} f(z) dz = 0 \quad \forall \Delta \in D
	\]
	By Lemma~\Ref{lem:ctLem2}, $f$ has a holomorphic antiderivative $F$. Then, by FTC,
	\[
		\int_{\gamma} f(z) dz = 0 \quad \forall \text{ closed } \gamma \in D
	\]
\end{proof}

\begin{remark}
	The same result holds if $f$ is holomorphic on $D - S$, where $S$ is a finite set of points and $f$ is continuous on $D$. We will need this in proofs but is not used much elsewhere.
\end{remark}

\begin{example}
	Consider
	\[
		\int_{|z| = \frac{1}{2}} \frac{e^z {(\sin z)}^2}{e^{z^2}} dz
	\]
	Because the function in the integral is holomorphic and $|z| = \frac{1}{2}$ is a closed contour, by CTSD, this integral is equal to $0$.
\end{example}

\subsection{Cauchy's integral formula}

\begin{theorem}
	(\textbf{Cauchy's integral formula - CIF}) Let $B_r(a)$ be a ball in $\mathbb{C}$ and $f: B_r(a) \rightarrow \mathbb{C}$ be holomorphic. Then for every $w \in B_r(a)$,
	\[
		f(w) = \frac{1}{2 \pi i} \int_{|z - a| = \rho} \frac{f(z)}{z - w} dz
	\]
	where $\rho$ is any real number with $|w - a| < \rho < r$.
\end{theorem}

\begin{proof}
	Define an auxiliary function $g$ by
	\[
		g(z) = \begin{cases}
			\frac{f(z) - f(w)}{z - w} & \text{ if } z \ne w \\
			f'(w) & \text{ if } z = w
		\end{cases}
	\]
	Note that $g$ is continuous at $z = w$, and holomorphic elsewhere. By CTSD,
	\[
		\int_{|z - a| = \rho} g(z) dz = 0
	\]
	Therefore
	\[
		\int_{|z - a| = \rho} \frac{f(z)}{z - w} dz = \int_{|z - a| = \rho} \frac{f(w)}{z - w} dz
	\]
	Now,
	\[
		\begin{aligned}
			\frac{1}{z - w}
				& = \frac{1}{z - a + a - w} \\
				& = \frac{1}{(z - a)(1 - \frac{w - a}{z - a})} \\
				& = \frac{1}{z - a} \sum_{n = 0}^{\infty} {\left( \frac{w - a}{z - a} \right)}^n
		\end{aligned}
	\]
	which converges uniformly, since $|\frac{w - a}{z - a}| = |\frac{w - a}{\rho} < 1|$.
	So we have
	\[
		\begin{aligned}
			\int_{|z - a| = \rho} \frac{f(z)}{z - w} dz
				& = f(w) \int_{|z - a| = \rho} \sum_{n = 0}^{\infty} \frac{{(w - a)}^n}{{(z - a)}^{n + 1}} dz \\
				& = \sum_{n = 0}^{\infty} \left( f(w) {(w - a)}^n \int_{|z - a| = \rho} \frac{1}{{(z - a)}^{n + 1}} dz \right)
		\end{aligned}
	\]
	The inner integral is equal to $0$ except when $n = 0$, when it's value is $2 \pi i$. So
	\[
		\int_{|z - a| = \rho} \frac{f(z)}{z - w} dz = f(w) {(w - a)}^0 \cdot 2 \pi i = 2 \pi i \cdot f(w)
	\]
\end{proof}

\section{Features of holomorphic functions}

\begin{theorem}
	(\textbf{Cauchy-Taylor theorem}) Let $U$ be an open set and $f: U \rightarrow \mathbb{C}$ be holomorphic on $U$. Then for every $r > 0$ such that $B_r(a) \subset U$, $f$ has a power series converging on $B_r(a)$ given by
	\[
		f(z) = \sum_{n = 0}^{\infty} c_n {(z - a)}^n
	\]
	where
	\[
		c_n = \frac{1}{2 \pi i} \int_{|z - a| = \rho} \frac{f(z)}{{(z - a)}^{n + 1}} dz
	\]
	is a constant for every $0 < \rho < r$. This is the \textbf{Taylor series} of $f$ about $a$.
\end{theorem}

\begin{proof}
	By the CIF, for every $w$ with $|w - a| < \rho$,
	\[
		\begin{aligned}
			f(w)
				& = \frac{1}{2 \pi i} \int_{|z - a| = \rho} \frac{f(z)}{z - w} dz \\
				& = \frac{1}{2 \pi i} \int_{|z - a| = \rho} f(z) \sum_{n = 0}^{\infty} \frac{{(w - a)}^n}{{(z - a)}^{n + 1}} dz \\
				& = \sum_{n = 0}^{\infty} \left( \frac{1}{2 \pi i} \int_{|z - a| = \rho} \frac{f(z)}{{(z - a)}^{n + 1}} dz \right) {(w - a)}^n \\
				& = \sum_{n = 0}^{\infty} c_n {(w - a)}^n
		\end{aligned}
	\]
\end{proof}

\begin{theorem}
	(\textbf{CIF for derivatives}) Let $f: B_r(a) \rightarrow \mathbb{C}$ be holomorphic. Then for every $0 < \rho < r$,
	\[
		\int_{|z - a| = \rho} \frac{f(z)}{{(z - a)}^{n + 1}} dz = \frac{2 \pi i}{n!} f^{(n)} (a)
	\]
\end{theorem}

\begin{proof}
	By Cauchy-Taylor, we have a convergent power series such that
	\[
		f(z) = \sum_{n = 0}^{\infty} c_n {(z - a)}^n
	\]
	where
	\[
		c_n = \frac{1}{2 \pi i} \int_{|z - a| = \rho} \frac{f(z)}{{(z - a)}^{n + 1}} dz
	\]
	But we also have (corollary 5.22 in lecture notes),
	\[
		c_n = \frac{f^{(n)} (a)}{n!}
	\]
	Equating these two expressions for $c_n$ completes the proof.
\end{proof}

\begin{remark}
	Combining theorem 7.1 (lecture notes) and theorem 7.2 (lecture notes), every holomorphic function $f$ has power series
	\[
		f(z) = \sum_{n = 0}^{\infty} \frac{f^{(n)} (a)}{n!} {(z - a)}^n
	\]
\end{remark}

\begin{remark}
	Cauchy-Taylor does not hold in real analysis. Let $f: \mathbb{R} \rightarrow \mathbb{R}$ be defined as
	\[
		f(x) = \begin{cases}
			e^{-1 / x} & \text{ if } x > 0 \\
			0 & \text{ if } x \le 0
		\end{cases}
	\]
	$f$ is differentiable for $x \ne 0$. For $x = 0$,
	\[
		f'(x) = \lim_{x \rightarrow 0} \frac{f(x) - f(0)}{x - 0} = \lim_{x \rightarrow 0} \frac{f(x)}{x}
	\]
	\[
		\lim_{x \rightarrow 0^-} \frac{f(x)}{x} = \lim_{n \rightarrow 0^-} \frac{0}{x} = 0
	\]
	and
	\[
		\lim_{x \rightarrow 0^+} \frac{f(x)}{x} = \lim_{x \rightarrow 0^+} \frac{e^{-1 / x}}{x} = \lim_{x \rightarrow 0^+} \frac{1 / x}{e^{1 / x}} = \lim_{y \rightarrow \infty} \frac{y}{e^y} = 0
	\]
	so $f'(0) = 0$, hence $f$ is differentiable on $\mathbb{R}$. But if $f$ had a Taylor series at $x = 0$, then
	\[
		f(x) = \sum_{n = 0}^{\infty} \frac{f^{(n)} (0)}{n!} z^n = 0
	\]
	around $x = 0$.
\end{remark}

\begin{corollary}
	(Holomorphic functions have infinitely many derivatives) If $f: U \rightarrow \mathbb{C}$ is holomorphic on an open set $U$ then $f$ has derivatives of all orders and each derivative is also holomorphic.
\end{corollary}

\begin{proof}
	Since $U$ is open, $\exists B_r(a) \subset U$ around a point $z = a$. But then by Cauchy-Taylor, $f$ has a power series. By theorem 5.21 (lecture notes), this power series is holomorphic. By corollary 5.22 (lecture notes) we can term-by-term differentiate to get a power series for $f'(z)$. By theorem 5.21 (lecture notes), $f'(z)$ is holomorphic. This can be repeated indefinitely.
\end{proof}

\begin{remark}
	This is a huge difference between real and complex analysis. Let $f: \mathbb{R} \rightarrow \mathbb{R}$ by defined as
	\[
		f_n(x) = |x| x^n
	\]
	\[
		f_n'(0) = \lim_{x \rightarrow 0} \frac{f_n(x) - f_n(0)}{x - 0} = \lim_{x \rightarrow 0} |x| x^{n - 1} = 0
	\]
	$f_n'(x) = (n + 1) |x| x^{n - 1}$ and $f^{(n)} (x) = c|x|$ which is not differentiable.
\end{remark}

\begin{theorem}
	(\textbf{Morera's Theorem}) Let $f: D \rightarrow \mathbb{C}$ be continuous on a domain $D$. If
	\[
		\int_{\gamma} f(z) dz = 0 \quad \forall \text{ closed } \gamma \subset D
	\]
	then $f$ is holomorphic.
\end{theorem}

\begin{proof}
	By the converse FTC, $f$ has a holomorphic antiderivative $F: D \rightarrow \mathbb{C}$ such that $F'(z) = f(z) \quad \forall z \in D$. By corollary 7.6 (lecture notes), if $F$ is holomorphic, its derivative $f$ must be.
\end{proof}

\begin{example}
	Consider
	\[
		\int_{|z| = 3} \frac{e^z}{z^2 (z - 1)} dz
	\]
	We use partial fractions:
	\[
		\frac{1}{z^2 (z - 1)} = \frac{a}{z} + \frac{b}{z^2} + \frac{c}{z - 1}
	\]
	So $1 = (c + a) z^2 + (b - a) z - b$, so $b = -1, a - -1, c = 1$. Using the CIF and CIF for derivatives,
	\[
		\begin{aligned}
			\int_{|z| = 3} \frac{e^z}{z^2 (z - 1)} dz
				& = -\int_{|z| = 3} \frac{e^z}{z} dz - \int_{|z| = 3} \frac{e^z}{z^2} dz + \int_{|z| = 3} \frac{e^z}{z - 1} dz \\
				& = -2 \pi i e^0 - 2 \pi i e^0 + 2 \pi i e^1 \\
				& = 2 \pi i (e - 2)
		\end{aligned}
	\]
\end{example}

\subsection{Liouville's theorem}

\begin{definition}
	A function $f: \mathbb{C} \rightarrow \mathbb{C}$ is \textbf{entire} if $f$ is holomorphic on $\mathbb{C}$.
\end{definition}

\begin{definition}
	A function $f: \mathbb{C} \rightarrow \mathbb{C}$ is \textbf{bounded} if for some $M > 0$, $|f(z)| \le M \ \forall z \in \mathbb{C}$.
\end{definition}

\begin{theorem}
	(\textbf{Liouville's theorem}) Every bounded entire function is constant.
\end{theorem}

\begin{proof}
	Let $f$ be entire and bounded. We will show that $\forall w \in \mathbb{C}, \ f(w) = f(0)$. By the CIF, for every $\rho > |w|$,
	\[
		\begin{aligned}
			|f(w) - f(0)|
				& = \left| \frac{1}{2 \pi i} \int_{|z| = \rho} \frac{f(z)}{z - w} dz - \frac{1}{2 \pi i} \int_{|z| = \rho} \frac{f(z)}{z} dz \right| \\
				& = \frac{|w|}{2 \pi} \left| \int_{|z| = \rho} f(z) \frac{1}{z(z - w)} dz \right| \\
				& = 
		\end{aligned}
	\]
	Using the Estimation lemma, boundedness of $f$ and the reverse triangle inequality,
	\[
		\begin{aligned}
			|f(w) - f(0)|
				& \le \frac{|w|}{2 \pi} 2 \pi \rho \cdot \sup_{|z| = \rho} \frac{|f(z)|}{|z| |z - w|} \\
				& \le |w| \rho \frac{M}{\rho} \sup_{|z| = \rho} \frac{1}{|z - w|} \\
				& \le |w| M \sup_{|z| = \rho} \frac{1}{| |z| - |w| |} \\
				& = \frac{|w| M}{\rho - |w|} \\
				& \rightarrow 0 \quad \text{as } \rho \rightarrow \infty
		\end{aligned}
	\]
	and $\rho$ can be arbitrarily large.
\end{proof}

\begin{remark}
	The holomorphicity condition is essential (we can't just say that $f$ is continuous). For example,
	\[
		f(z) = f(x + iy) = \sin(x) + i \sin(y)
	\]
	is continuous and bounded on $\mathbb{C}$ but is not entire.
\end{remark}

\begin{theorem}
	(\textbf{Fundamental theorem of Algebra}) Every non-constant polynomial with complex coefficients $p(z) = a_d z^d + \cdots + a_1 z + a_0$, $a_d \ne 0$ has a complex root: for some $z_0 \in \mathbb{C}, P(z_0) = 0$.
\end{theorem}

\begin{proof}
	By assumption, $d \ge 1$, so $|p(z)| \rightarrow \infty$ as $|z| \rightarrow \infty$. In particular, $\exists R > 0, |p(z)| > 1$ if $|z| = R$. Assume the converse, that $p$ has no roots.

	Then $f(z) := \frac{1}{p(z)}$ is holomorphic on $\mathbb{C}$. On the set $|z| > R$, $f$ is bounded, since $|f(z)| = \frac{1}{|p(z)|} < 1$. But $\overline{B_R}(0) = \{ z \in \mathbb{C}: |z| \le R \}$ is compact, so by theorem 2.30 (lecture notes), $|f(z)|$ attains a maximum on $\overline{B_R}(0)$. In particular, $f$ is bounded on $\overline{B_R}(0)$. Thus $f$ is bounded and entire, so by Liouville's theorem, $f$ is constant, which is a contradiction.
\end{proof}

\begin{theorem}
	(\textbf{Local maximum modulus principle}) Let $f: B_r(a) \rightarrow \mathbb{C}$ be holomorphic. If for every $z \in B_r(a)$, $|f(z)| \le |f(a)|$ then $f$ is constant on $B_r(a)$.
\end{theorem}

\begin{proof}
	First, we show $|f|$ is constant. Pick any $w \in B_r(a)$. We will show $|f(w)| = |f(a)|$. Let $\rho = |w - a| < r$ so that the contour $\gamma(t) = a + \rho e^{2 \pi i t}, t \in [0, 1]$ passes through $w$. By the CIF,
	\[
		\begin{aligned}
			|f(a)|
				& = \left| \frac{1}{2 \pi i} \int_{|z - a| = \rho} \frac{f(z)}{z - a} dz \right| \\
				& = \frac{1}{2 \pi} \left| \int_{\gamma} \frac{f(z)}{z - a} dz \right| \\
				& = \frac{1}{2 \pi} \left| \int_{0}^{1} \frac{f(\gamma(t))}{\gamma(t) - a} \gamma'(t) dt \right| \\
				& = \frac{1}{2 \pi} \left| \int_{0}^{1} \frac{f(\gamma(t))}{\rho e^{2 \pi i t}} \rho \cdot 2 \pi i e^{2 \pi i t} dt \right| \\
				& = \left| \int_{0}^{1} f(\gamma(t)) dt \right| \\
				& \le \int_{0}^{1} |f(\gamma(t))| dt \\
				& \le \int_{0}^{1} |f(a)| dt \\
				& \le |f(a)| \int_{0}^{1} 1 dt \\
				& = |f(a)|
		\end{aligned}
	\]
	Therefore every inequality above must be an equality. In particular,
	\[
		\int_{0}^{1} |f(\gamma(t))| dt = \int_{0}^{1} |f(a)| dt
	\]
	So by continuity, $|f(\gamma(t))| = |f(a)|$. So $|f(w)| = |f(a)|$, and $w$ was arbitrary, so $|f(z)| = |f(a)| \ \forall z \in B_r(a)$.

	Now, we show that $f$ is constant. Let $|f(z)| = c \ \forall z \in B_r(a)$. If $c = 0$, the $|f(z)| = 0 \Rightarrow f(z) = 0 \forall z \in B_r(a)$. So assume $c \ne 0$. Now,
	\[
		c^2 = |f(z)|^2 = f(z) \cdot \overline{f(z)}
	\]
	So $\overline{f(z)} = \frac{c^2}{f(z)}$ is holomorphic (since $f$ is holomorphic). Let $f = u + iv$, then $\overline{f} = u - iv = u + i(-v)$. Then by the Cauchy-Riemann equations, $u_x = v_y$ but also $u_x = {(-v)}_y = -v_y$ so $u_x = v_y = 0$, and $u_y = -v_x$ but also $u_y = {(-v)}_x = v_x$ so $u_y = v_x = 0$. By proposition 3.3 (lecture notes), $f'(z) = u_x + i v_x = 0 + i \cdot 0 = 0$, hence $f$ is constant.
\end{proof}

\begin{theorem}
	(\textbf{Maximum modulus theorem}) Let $D$ be a domain and $f: D \rightarrow \mathbb{C}$. If
	\[
		\exists a \in D, |f(z)| \le |f(a)| \quad \forall z \in D
	\]
	then $f$ is constant on $D$.
\end{theorem}


\begin{proof}
	Let $U_1 = \{ z \in D: f(z) = f(a) \}$ and let $U_2 = \{ z \in D: f(z) \ne f(a) \}$. Let $U = U_1 \cup U_2$ (TODO: check this last sentence, might not be $U$). $U_1$ is non-empty, because $a \in U_1$. We will show that $U_1$ is open. Pick $z \in U_1$. Then by the openness of $U$, there exists a ball $B_r(z)$ in $U$. Pick $W \in B_r(z)$. We have $|f(w)| \le |f(a)| = |f(z)|$, so by the Local Maximum Modulus Principle, $f$ is constant on $B_r(z)$, so $f(w) = f(z) = f(a) \Longrightarrow w \in U_1 \Longrightarrow B_r(z) \subset U_1$. Hence $U_1$ is open.

	Now $U_2 = D - U_1 = f^{-1} (\mathbb{C} - \{ f(a) \})$, where $f^{-1}$ denotes the preimage. Therefore $U_2$ is open by theorem 2.17 (lecture notes), since $\mathbb{C} - \{ f(a) \}$ is open.

	So $D = U_1 \cup U_2$ where $U_1$ is non-empty and open and $U_2$ is open. By fact 7.14 (lecture notes), $U_2$ is empty hence $D = U_1$, so $f$ is constant on $D$.
\end{proof}

\begin{example}
	Find the maximum absolute value of $f(z) = z^2 + 2z - 3$ on $\overline{B_1}(0) = \{ z \in \mathbb{C}: |z| \le 1 \}$.
	
	$\overline{B_1}(0)$ is compact so by theorem 2.30 (lecture notes), $|f(z)|$ attains a maximum on it. Also, $B_1(0)$ is a domain, so by the maximum modulus theorem, the function does not attain the maximum in $B_1(0)$. So the maximum must be obtained on the boundary. So $|z| = 1$, so let $z = e^{it}$ for some $t \in [0, 2 \pi]$. Then $|f(z)|^2 = |z^2 + 2z - 3|^2 = (z^2 + 2z - 3) (\overline{z^2 + 2z - 3}) = 14 - 3e^{-2it} - 3e^{2it} - 4^{it} - 4e^{-it} = 14 - 6 \cos(2t) - 8 \cos(t) = -12 {\cos(t)}^2 - 8 \cos(t) + 20 = -12x^2 - 8x + 20$ where $x = \cos(t)$. The maximum is attained when $x = \cos(t) = -\frac{1}{3}$. This gives $|f(z)| = \frac{8}{\sqrt{3}}$.
\end{example}

\subsection{Analytic continuation and the identity theorem}

Let $f: B_r(a) \rightarrow \mathbb{C}$ be holomorphic. Then by Cauchy-Taylor, $f$ has a convergent Taylor series
\[
	f(z) = \sum_{n = 0}^{\infty} c_n {(z - a)}^n
\]
for $z \in B_r(a)$. Assume $f \not\equiv 0$ so at least one coefficient is non-zero. Let $m = \min\{ n \ge 0: c_n \ne 0 \}$. Then
\[
	f(z) = {(z - a)}^m \sum_{n = m}^{\infty} c_n {(z - a)}^{n - m} = {(z - a)}^m \sum_{k = 0}^{\infty} c_{k + m} {(z - a)}^k
\]
where $k = n - m$. Let
\[
	h(z) = \sum_{k = 0}^{\infty} c_{k + m} {(z - a)}^k
\]
Then $h$ is a convergent power series so is holomorphic by theorem 5.21 (lecture notes) and $h(a) = c_m \ne 0$. Note $f(a) = 0 \Longleftrightarrow m > 0$.

\begin{definition}
	(\textbf{Orders of zeros}) $f: B_r(a) \rightarrow \mathbb{C}$ has a \textbf{zero of order} $m$ at $a$ if for some holomorphic function $h: B_r(a) \rightarrow \mathbb{C}$ such that $f(z) = {(z - a)}^m h(z)$ and $h(a) \ne 0$.
\end{definition}

\begin{remark}
	We can show that $f$ has a zero of order $m$ at $z = a$ iff
	\[
		f(a) = f'(a) = \cdots = f^{(m - 1)} (a) = 0
	\]
\end{remark}

\begin{example}
	Let $f(z) = z (e^z - 1)$. Let $z = a = 0$. Then
	\[
		\begin{aligned}
			f(z)
				& = z \left( \sum_{n=  0}^{\infty} \frac{z^n}{n!} - 1 \right) \\
				& = z \sum_{n = 1}^{\infty} \frac{z^{n + 1}}{n!} \\
				& = z^2 \sum_{m = 0}^{\infty} \frac{z^m}{(m + 1)!}
		\end{aligned}
	\]
	so $f$ has a zero of order $2$ at $a = 0$. As a check, $f'(z) = e^z - 1 + ze^z$ and $f''(z) = e^z + e^z + ze^z$ so $f(0) = f'(0) = 0$ but $f''(0) = e^0 + e^0 + 0 = 2$.
\end{example}

\begin{remark}
	Because holomorphic functions are continuous, if $f(a) \ne 0$, we can always find $B_{\rho} (a)$ such that $f(z) \ne 0$ on $B_{\rho}(a)$.
\end{remark}

\begin{theorem}
	(\textbf{Principle of isolated zeroes}) Let $f: B_r(a) \rightarrow \mathbb{C}$ be holomorphic and $f \not\equiv 0$. Then for some $\rho > 0$,
	\[
		f(z) \ne 0 \quad \forall z \in B_r(a) - \{ a \}
	\]
	In particular, the zeros are isolated from one another.
\end{theorem}

\begin{proof}
	For $f(a) \ne 0$, we are done by continuity. If $f(a) = 0$, for some $m > 0$,
	\[
		f(z) = {(z - a)}^m h(z) \quad \forall z \in B_{\rho}(a)
	\]
	where $h: B_r(a) \rightarrow \mathbb{C}$ is holomorphic, $\rho > 0$ and $h(a) \ne 0$. Thus $h(z) \ne 0$ on $B_{\rho}(a)$ and ${(z - a)}^m \ne 0$ on $B_{\rho} (a) - \{ a \}$. So $f(z) \ne 0$ on $B_{\rho}(a)$.
\end{proof}

\begin{theorem}
	(\textbf{Uniqueness of analytic continuation}) Let $D' \subset D$ be non-empty domains and $f: D' \rightarrow \mathbb{C}$ be holomorphic. Then there exists \textbf{at most one} holomorphic $g: D \rightarrow \mathbb{C}$ such that
	\[
		g(z) = f(z) \quad \forall z \in D'
	\]
	If $g$ exists, it is called the \textbf{analytic continuation} of $f$ to $D$.
\end{theorem}

\begin{proof}
	Let $g_1, g_2: D' \rightarrow \mathbb{C}$ be analytic continuations. Let $h(z) = g_1(z) - g_2(z)$. We will show that $h(z) = 0$ on $D$. Note $h(z) = 0 \ \forall z \in D'$. Let
	\[
		D_0 = \{ w \in D: \exists r > 0, h(z) = 0 \text{ on } B_r(w) \\
		D_1 = \{ w \in D: \exists n \ge 0, h^{(n)}(w) \ne 0 \}
	\]
	We will show $D_0$ is non-empty and open, $D_1$ is open and $D$ is the disjoint union $D = D_0 \cup D_1$, so $D_0 \cap D_1 = \emptyset$.

	First we show $D_0$ is non-empty and open. Since $D' \subset D_0$, $D_0$ is non-empty. We want to show that $\forall w \in D_0, \exists r > 0, B_r(w) \subset D_0$. By the definition of $D_0$, for some $r > 0$, $h(z) = 0$ on $B_r(w)$. Pick $z \in B_r(w)$. $B_r(w)$ is open, so $\exists B_{\rho} (z)$ inside $B_r(w)$, on which $h(z) = 0$. Thus $z \in D_0$. Thus $D_0$ is open.

	Now we show $D_1$ is open.
	\[
		\begin{aligned}
			D_1
				& = \bigcup_{n = 0}^{\infty} \{ w \in D: h^{(n)}(w) \ne 0 \} \\
				& = \bigcup_{n = 0}^{\infty} {\left( h^{(n)} \right)}^{-1} (\mathbb{C} - \{ 0 \})
		\end{aligned}
	\]
	By Lemma 2.8 (lecture notes) and Theorem 2.17 (lecture notes), $D_1$ is open.

	Now we show $D = D_0 \cup D_1$. Pick $w \in D$. If $w \notin D_1$, then $h^{(n)}(w) = 0 \ \forall n \ge 0$. But by Cauchy-Taylor (lecture notes), $h$ has a Taylor series about $z = w$ with coefficients $h^{(n)}(w) / n! = 0$. Thus $h = 0$ around $w$. So $w \in D_0$.

	If $w \in D_1$, it must have a non-zero Taylor series expansion (at least coefficient $c_n$ is non-zero). By the Principle of Isolated Zeroes (lecture notes), for some $B_{\rho}(w)$, $h(z) \ne 0$ on $B_{\rho}(w) - \{ w \}$. So $w \notin D_0$. This completes the proof.
\end{proof}

\begin{corollary}\label{cor:sameOnBallImpliesSameOnDomain}
	Let $f, g$ be holomorphic on a domain $D$. If $f = g$ on some $B_r(w) \subseteq D$ then $f = g$ on $D$.
\end{corollary}

\begin{definition}
	Given a set $S \subset \mathbb{C}$, a point in $S$ is called
	\begin{itemize}
		\item \textbf{isolated} in $S$ if $\exists \epsilon > 0$, $B_{\epsilon}(w) \cap S = \{ w \}$.
		\item \textbf{non-isolated} in $S$ if $\forall \epsilon > 0$, $\exists z \in S$, $z \in B_{\epsilon}(w)$, with $\omega \ne z$.
	\end{itemize}
\end{definition}

\begin{theorem}\label{thm:identityTheorem}
	(\textbf{Identity theorem}) Let $f, g: D \rightarrow \mathbb{C}$ be holomorphic on a domain $D$. If $S = \{ z \in D: f(z) = g(z) \}$ contains a non-isolated point, then
	\[
		f(z) = g(z) \quad \text{on } D
	\]
\end{theorem}

\begin{proof}
	Let $w \in S$ be non-isolated and let $h(z) = f(z) - g(z)$. Then $h(w) = 0$. By the Principle of Isolated Zeroes (lecture notes), for some $\rho$, $\forall z \in B_{\rho} (w) - \{ w \}$, $h(z) \ne 0$. But this contradicts $w$ being non-isolated. So $h(z) = 0$ on $B_{\rho} (w)$. By Corollary~\ref{cor:sameOnBallImpliesSameOnDomain}, $h(z) = 0$ on $D$.
\end{proof}

\begin{example}
	Let $f: \mathbb{C} \rightarrow \mathbb{C}$ and assume $\forall n \in \mathbb{N}, f(1 / n) = \sin(1 / n)$. Then $f(z) = \sin(z)$ on $\mathbb{C}$.

	By continuity (lemma 2.29 (lecture notes)),
	\[
		f(0) = f \left( \lim_{n \rightarrow \infty} \frac{1}{n} \right) = \lim_{n \rightarrow \infty} f \left( \frac{1}{n} \right) = \lim_{n \rightarrow \infty} \sin \left( \frac{1}{n} \right) = \sin(0)
	\]
	Since $z = 0$ is a non-isolated point in $S = \{ z \in \mathbb{C}: f(z) = \sin(z) \}$, $f(z) = \sin(z)$ on $\mathbb{C}$.
\end{example}

\begin{example}
	(Problems class) Evaluate
	\[
		I = \int_{|z| = 1} \frac{1}{(z - a)(z - b)} dz
	\]
	in the cases:
	\[
		|a| > 1, |b| > 1; \quad |a| < 1 < |b|; \quad |a| < 1, |b| < 1
	\]
	\begin{itemize}
		\item $|a| > 1, |b| > 1$: we can find a ball $B_r(0)$ with $1 < r < \min \{ |a|, |b| \}$, on which the integrand is holomorphic, so by CTSD, $I = 0$.
		\item $|a| < 1 < |b|$: Let $f(z) = 1/(z - b)$, then
		\[
			I = \int_{|z| = 1} \frac{f(z)}{z - a} dz = 2 \pi i \frac{1}{a - b}
		\]
		by Cauchy's integral formula.
		\item $|a| < 1, |b| < 1$: if $a = b$ then
		\[
			\frac{1}{(z - a)(z - b)} = \frac{1}{{(z - a)}^2} \Longrightarrow I = 2 \pi i \cdot 0 = 0
		\]
		by CIF for derivatives. If $a \ne b$, then use partial fractions.
	\end{itemize}
\end{example}

\begin{example}
	(Problems class) Evaluate
	\[
		I = \int_{|z| = 2} \frac{{\sin(z)}^2}{z^2} dz
	\]
	Let $f(z) = {\sin(z)}^2, a = 0, \rho = 2, n = 1$. Then by CIF for derivatives,
	\[
		I = \frac{2 \pi i}{1} {\left[ {\sin(z)}^2 \right]}' \Big|_{z = 0} = 2 \pi i [2 \sin(z) \cos(z)] \Big|_{z = 0} = 0
	\]
\end{example}

\begin{example}
	(Problems class) Let $f(z) = 1/{(1 - z)}^2$. Find a Taylor series of $f$.

	An antiderivative of $f$ is $F(z) = 1/(1 - z)$. $F$ has Taylor series
	\[
		F(z) = \sum_{n = 0}^{\infty} z^n \quad (|z| < 1)
	\]
	So by theorem 5.21 (lecture notes), $f$ has Taylor series
	\[
		f(z) = \sum_{n = 1}^{\infty} n z^{n - 1} = \sum_{m = 0}^{\infty} (m + 1) z^m \quad (|z| < 1)
	\]
\end{example}

\begin{example}
	(Problems class) Using the Taylor series expansion of $f(z) = e^z$ about $z = 0$, prove that $|e^z - 1| \le e^{|z|} - 1 \le |z| e^{|z|}$.

	Using the triangle inequality,
	\[
		\begin{aligned}
			|e^z - 1|
				& = \left| \sum_{n = 0}^{\infty} \frac{z^n}{n!} - 1 \right| \\
				& = \left| \sum_{n = 1}^{\infty} \frac{z^n}{n!} \right| \\
				& \le \sum_{n = 1}^{\infty} \frac{|z|^n}{n!} \\
				& = \sum_{n = 0}^{\infty} \frac{|z|^n}{n!} - 1 \\
				& = e^{|z|} - 1
		\end{aligned}
	\]
	Also,
	\[
		\begin{aligned}
			e^{|z|} - 1
				& = \sum_{n = 1}^{\infty} \frac{|z|^n}{n!} \\
				& = |z| \sum_{n = 1}^{\infty} \frac{|z|^{n - 1}}{n!} \\
				& = |z| \sum_{m = 0}^{\infty} \frac{|z|^m}{(m + 1)!} \\
				& \le |z| \sum_{m = 0}^{\infty} \frac{|z|^m}{m!} \\
				& = |z| e^{|z|}
		\end{aligned}
	\]
\end{example}

\begin{example}
	Find the Taylor series about $z_0 \in \mathbb{C}$ of $f(z) = e^z$.

	$\forall n \in \mathbb{N}, f^{(n)} (z) = e^z$. By Corollary 5.22 (lecture notes),
	\[
		f(z) = \sum_{n = 0}^{\infty} \frac{f^{(n)}(z_0)}{n!} {(z - z_0)}^n = \sum_{n = 0}^{\infty} \frac{e^{z_0}}{n!} {(z - z_0)}^n
	\]
\end{example}

\begin{example}
	Calculate the Taylor series of $f(z) = \Log(1 + z)$ about $z = 0$.
	\[
		f'(z) = \frac{1}{1 + z} = \frac{1}{1 - (-z)} = \sum_{n = 0}^{\infty} {(-z)}^n
	\]
	Then by corollary 5.23 (lecture notes),
	\[
		f(z) = \sum_{n = 0}^{\infty} \frac{{(-1)}^n}{n + 1} z^{n + 1} + c \quad (|z| < 1)
	\]
	for some $c$. $\Log(1 + 0) = 0$, so $c = 0$.
\end{example}

\subsection{Harmonic functions and the Dirichlet problem}

\begin{definition}
	A \textbf{harmonic function} is a real valued function $u: D \rightarrow \mathbb{R}$ on a domain $D \subset \mathbb{C}$ that has continuous second-order partial derivatives which satisfy the \textbf{Laplace equation}:
	\[
		u_{xx} + u_{yy} = 0
	\]
\end{definition}

\begin{proposition}
	Let $f = u + iv: D \rightarrow \mathbb{C}$ be holomorphic. Then $u$ and $v$ are harmonic.
\end{proposition}

\begin{proof}
	By proposition 3.3 (lecture notes) the first-order partial derivatives exist and $f' = u_x - i u_y = v_y + i v_x$. By corollary 7.6 (lecture notes), $f'$ is holomorphic and so continuous, hence $u_x, u_y, v_y, v_x$ are continuous as well. By the same argument with $f'$, the second-order partial derivatives exist and are continuous. By proposition 3.3 (lecture notes), $u_x, u_y, v_x, v_y$ satisfy the Cauchy-Riemann equations:
	\[
		\begin{aligned}
			u_x = v_y & \Longrightarrow u_{xx} = v_{yx} \\
			u_y = -v_x & \Longrightarrow u_{yy} = -v_{xy}
		\end{aligned}
	\]
	By the Schwartz-Clairault theorem, $v_{yx} = v_{xy}$, hence $u_{xx} + u_{yy} = v_{yx} - v_{yx} = 0$.
\end{proof}

\begin{example}
	Let $f(z) = e^z = e^{x + iy} = e^x (\cos(y) + i \sin(y))$. So $u(x, y) = e^x \cos(y)$ and $v(x, y) = e^x \sin(y)$. $u_x (x, y) = e^x \cos(y)$, $u_y = -e^x \sin(y)$, $u_{xx} = e^x \cos(y)$ and $u_{yy} = -e^x \cos(y)$.
\end{example}

\begin{example}
	Let $f(x + iy) = x^2 + y^2$. So $u(x, y) = x^2 + y^2$ and $v(x, y) = 0$. $u_x = 2x \Longrightarrow u_{xx} = 2$ but $u_y = 2y \Longrightarrow u_{yy} = 2$ so $u_{xx} + u_{yy} = 4$. Note $f(z) = |z|^2$. So $f$ is not holomorphic.
\end{example}

\begin{example}
	Let $u(x, y) = x^2 - y^2 + 3x$, so $u_x = 2x + 3 \Longrightarrow u_{xx} = 2$, $u_y = -2y \Longrightarrow u_{yy} = -2$ so $u$ is harmonic.
\end{example}

\begin{proposition}
	Let $f: D \rightarrow \mathbb{C}$ be holomorphic on a starlike domain $D$. Then for some $F: D \rightarrow \mathbb{C}$, $F$ is holomorphic and $F' = f$.
\end{proposition}

\begin{proof}
	By Cauchy's Starlike theorem (lecture notes),
	\[
		\int_{\gamma} f(z) dz = 0 \quad \forall \text{ closed } \gamma
	\]
	By the converse FTC (lecture notes), there exists a holomorphic antiderivative of $f$, $F$.
\end{proof}

\begin{theorem}
	(\textbf{The existence of a harmonic conjugate}) If $D$ is a starlike domain and $u: D \rightarrow \mathbb{R}$ is harmonic, then for some harmonic function $v: D \rightarrow \mathbb{R}$,
	\[
		f = u + iv
	\]
	is holomorphic on $D$. $v$ is called the \textbf{harmonic conjugate} of $u$ and is unique up to a real additive constant.
\end{theorem}

\begin{proof}
	If such an $f$ exists, then $f' = u_x - i u_y$ would also be holomorphic. We will first construct $f'$ and then construct $f$.

	Let $g(x, y) = u_x + i (-u_y)$. We want to show $g$ is holomorphic. Using theorem 3.5 (lecture notes), we will show $g$ satisfies the Cauchy-Riemann equations.
	\[
		{(u_x)}_x = u_{xx}, \quad {(-u_y)}_y = -u_{yy} = u_{xx}
	\]
	as $u$ is harmonic. Similarly,
	\[
		{(u_x)}_y = u_{xy}, \quad -{(-u_y)}_x = u_{yx}
	\]
	by Schwarz-Clairault. So by theorem 3.5 (lecture notes), $g$ is holomorphic.

	Now we will construct $f$. By proposition 7.28 (lecture notes), $g$ has a holomorphic antiderivative, $F = U + iV$, where $F'(z) = g(z)$ on $D$. $F$ satisfies the Cauchy-Riemann equations and $F' = U_x - U_y i = g = u_x - u_y i$. Hence
	\[
		U_x = u_x, \quad U_y = u_y \Longrightarrow {(U - u)}_x = 0 = {(U - u)}_y
	\]
	Therefore $U = u + c$ for some constant $c$. So let $f = F - c = (U + iV) - c = u + iV$, and let $v = V$. By proposition 7.26 (lecture notes), $v$ is harmonic.

	Finally, we show $v$ is unique (TODO).
\end{proof}

\begin{example}
	We have seen $u(x, y) = x^2 - y^2 + 3x$ is harmonic. Construct its harmonic conjugate $v$.

	We want a holomorphic $f = u + iv$. By the first Cauchy-Riemann equation, $v_y = u_x = 2x + 3$. So $v(x, y) = 2xy + 3y + g(x)$ for some function $g$. By the second Cauchy-Riemann equation,
	\[
		2y + g'(x) = v_x = -u_y = 2y
	\]
	hence $g'(x) = 0 \Longrightarrow g(x) = c$ for a constant $c \in \mathbb{R}$. So $v(x, y) = 2xy + 2y + c$. Then $f(x + iy) = x^2 - y^2 + 3x + i(2xy + 3y + c)$.
	\
	Note that ${(x + iy)}^2 = x^2 - y^2 + 2xyi$, so $f(z) = z^2 + 3z + ic$.
\end{example}

\begin{definition}
	The \textbf{Dirichlet boundary problem} states: Let $D \subseteq \mathbb{C}$ be a domain with closure $\bar{D}$ and boundary $\delta D$. Let $g: \delta D \rightarrow \mathbb{R}$ be continuous. Find a continuous function $\mu: \bar{D} \rightarrow \mathbb{R}$ such that $\mu$ is harmonic on $D$ and matches $g$ on $\delta D$.
\end{definition}

\begin{example}\label{exa:dirichletBoundaryProblemExample1}
	Let
	\[
		D = \{ x + iy \in \mathbb{C}: 2 < y < 5 \}, \quad g(x, y) = \begin{cases}
			4 & \text{ if } y = 2 \\
			13 & \text{ if } y = 5
		\end{cases}
	\]
	So $\bar{D} = \{ x + iy \in \mathbb{C}: 2 \le y \le 5 \}$ and $\delta D = \{ x + iy \in \mathbb{C}: y = 2 \text{ or } y = 5 \}$. We want a harmonic $\mu$ such that $\mu = g$ on $\delta D$.

	Note that $g$ doesn't depend on $x$, so it is possible that $\mu = 0$. If this was true, then $\mu_{xx} = 0$ and $\mu$ is harmonic so $\mu_{yy} = -\mu_{xx} = 0$, so
	\[
		\mu(x, y) = ay + b
	\]
	This guess for this solution is called an \textbf{ansatz}. Check $\mu(x, 2) = 4 = 2a + b$ and $\mu(x, 5) = 13 = 5a + b$, which gives $a = 3$ and $b = -2$. Thus $\mu(x, y) = 3y - 2$ is a solution to the Dirichlet boundary problem.
\end{example}

\begin{proposition}
	Let $f: D \rightarrow \mathbb{C}$, $f = u + iv$ be holormorphic on $D$ and $\mu$ is harmonic on $f(D)$. Then
	\[
		\tilde{\mu} := \mu \circ f = \mu(u, v)
	\]
	is harmonic on $D$.
\end{proposition}

\begin{example}
	From Example~\ref{exa:dirichletBoundaryProblemExample1}, $\mu(x, y) = 3y - 2$ was a solution to the Dirichlet boundary problem on $D_1 = \{ x + iy: 2 < y < 5 \}$ with
	\[
		g(x, y) = \begin{cases}
			4 & \text{ if } y = 2 \\
			13 & \text{ if } y = 5
		\end{cases}
	\]
	Let $D_2$ be $D_1$ rotated anticlockwise by $\pi/4$ about the origin. Solve the Dirichlet boundary problem on $D_2$ when
	\[
		g(x, y) = \begin{cases}
			4 & \text{ if } y = x + 2 \sqrt{2} \\
			13 & \text{ if } y = x + 5 \sqrt{2}
		\end{cases}
	\]
	Let $f(z) = e^{-\pi i / 4} z$. From Example~\ref{exa:dirichletBoundaryProblemExample1}, we know a solution on $f(D_2) = D_1$. So by proposition 7.31 (lecture notes), $\tilde{\mu} = \mu \circ f$ is a solution on $D_2$. Note that
	\[
		f(x, y) = e^{-\pi i / 4} (x + i y) = \frac{1}{\sqrt{2}} (x + y) + i \frac{1}{\sqrt{2}} (y - x)
	\]
	Thus $\tilde{\mu} = \mu(\frac{1}{\sqrt{2}} (x + y), \frac{1}{\sqrt{2}} (y - x)) = \frac{3}{\sqrt{2}} (y - x) - 2$.
\end{example}

\section{General form of the Cauchy-Taylor theorem and Cauchy's integral formula}

\subsection{Winding number and simply connected sets}

\begin{definition}
	Let $\gamma: [a, b] \rightarrow \mathbb{C}$ be a contour of the form
	\[
		\gamma(t) = \omega + r(t) e^{i \theta (t)}
	\]
	where $\omega \in \mathbb{C}$, $\theta(t): [a, b] \rightarrow \mathbb{C}$, $r(t): [a, b] \rightarrow \mathbb{R}^+$ are continuous, piecewise-$C^1$. The \textbf{winding number} of $\gamma$ about $\omega$ is defined as
	\[
		I(\gamma, \omega) = \frac{\theta(b) - \theta(a)}{2 \pi}
	\]
\end{definition}

\begin{example}
	Let $\gamma_1(t) = e^{2 \pi i t}$ for $t \in [0, 1]$. Then $r(t) = 1$, $\omega = 0$, $\theta(t) = 2 \pi t$. So
	\[
		I(\gamma_1, 0) = \frac{2 \pi \cdot 1 - 2 \pi \cdot 0}{2 \pi} = 1
	\]
	Let $\gamma_2(t) = e^{2 \pi i t}$ for $t \in [0, 2]$. Then
	\[
		I(\gamma_2, 0) = \frac{2 \pi \cdot 2 - 2 \pi \cdot 0}{2 \pi} = 2
	\]
\end{example}

\begin{remark}
	$w \notin \gamma$ since $r(t) > 0$, and $I(\gamma, \omega) \in \mathbb{Z}$ when $\gamma$ is closed, since if $\gamma$ is closed, then
	\[
		\begin{aligned}
			& \gamma(a) = \gamma(b) = \omega + r(a) e^{i \theta(a)} = \omega + r(b) e^{i \theta(b)} \\
			& \Longleftrightarrow r(b) = r(a) \text{ and } \theta(a) = \theta(b) + 2 \pi n \quad (n \in \mathbb{Z})	
		\end{aligned}
	\]
	Thus
	\[
		I(\gamma, \omega) = \frac{\theta(b) - \theta(a)}{2 \pi} \in \mathbb{Z}
	\]
\end{remark}

\begin{theorem}
	Let $\gamma: [a, b] \rightarrow \mathbb{C}$ be a contour. Then for every $\omega \in \mathbb{C}$ with $\omega \notin \gamma$, for some continuous, piecewise $C^1$, $\theta: [a, b] \rightarrow \mathbb{R}$ and $r: [a, b] \rightarrow \mathbb{R}^+$,
	\[
		\gamma(t) = \omega + r(t) e^{i \theta(t)}
	\]
\end{theorem}

\begin{proof}
	Omitted.
\end{proof}

\begin{lemma}\label{lem:formulaForClosedContourWindingNumber}
	Let $\gamma: [a, b] \rightarrow \mathbb{C}$ be a closed contour and $\omega \notin \gamma$. Then
	\[
		I(\gamma, \omega) = \frac{1}{2 \pi i} \int_{\gamma} \frac{1}{z - \omega} dz
	\]
\end{lemma}

\begin{proof}
	By theorem 8.2 (lecture notes),
	\[
		\gamma(t) = \omega + r(t) e^{i \theta(t)}
	\]
	By definition 6.4 (lecture notes),
	\[
		\begin{aligned}
			\int_{\gamma} \frac{1}{z - w} dz
				& = \int_{a}^{b} \frac{1}{\gamma(t) - \omega} \gamma'(t) dt \\
				& = \int_{a}^{b} \frac{1}{r(t) e^{i \theta(t)}} \left( r'(t) e^{i \theta(t)} + r(t) \cdot i \theta'(t) e^{i \theta(t)} \right) dt \\
				& = \int_{a}^{b} \left( \frac{r'(t)}{r(t)} + i \theta'(t) \right) dt \\
				& = {[\log(r(t)) + i \theta(t)]}_a^b = i (\theta(b) - \theta(a)) = 2 \pi i \cdot I(\gamma, \omega)
		\end{aligned}
	\]
\end{proof}

\begin{proposition}
	Let $D$ be a starlike domain. Then for every closed contour $\gamma$ and every $\omega \notin D$,
	\[
		I(\gamma, \omega) = 0
	\]
\end{proposition}

\begin{proof}
	By Lemma~\ref{lem:formulaForClosedContourWindingNumber},
	\[
		I(\gamma, \omega) = \frac{1}{2 \pi i} \int_{\gamma} \frac{1}{z - \omega} dz
	\]
	and $1 / (z - w)$ is holomorphic on $D$. So by CTSD, $I(\gamma, \omega) = 0$.
\end{proof}

\begin{definition}
	Let $U$ be an open set. A closed contour $\gamma$ in $U$ is \textbf{homologous to zero} if $I(\gamma, \omega) = 0$ for every $\omega \notin U$.
\end{definition}

\begin{definition}
	An open set $U$ is called \textbf{simply connected} if every closed contour in $U$ is homologous to zero.
\end{definition}

\begin{example}
	By proposition 8.4 (lecture notes), starlike domains are simply connected.
\end{example}

\begin{example}
	Let $A = \{ z \in \mathbb{C}: \alpha < |z| < \beta \}$. Let $\gamma(t) = \rho e^{2 \pi i t}$ for $t \in [0, 1]$ and $\alpha < \rho < \beta$. Pick $\omega = 0$ then $\omega 
	\notin A$ but
	\[
		\frac{1}{2 \pi i} \int_{\gamma} \frac{1}{z} dz = 1
	\]
	by CIF. Thus $A$ is not simply connected.
\end{example}

\begin{definition}
	A \textbf{cycle} $\Gamma$ defined on an open set $U$ is a finite collection of closed contours in $U$. We write
	\[
		\Gamma = \gamma_1 + \gamma_2 + \cdots + \gamma_n
	\]
\end{definition}

\begin{definition}
	Let $\Gamma$ be a cycle. $w \in \mathbb{C}$ is \textbf{not on $\Gamma$} ($\omega \notin \gamma_i$) if $w \notin \gamma_i$ for every $i$. The \textbf{winding number of $\Gamma$ around $\omega$} is defined as
	\[
		I(\Gamma, \omega) = \sum_{i = 1}^n I(\gamma_i, \omega)
	\]
	and we define
	\[
		\int_{\Gamma} f(z) dz = \sum_{i = 1}^n \int_{\gamma_i} f(z) dz
	\]
	$\Gamma$ is called \textbf{homologous to zero in $U$} if $I(\Gamma, \omega) = 0$ for every $\omega \notin U$.
\end{definition}

\begin{example}
	Let $A_{\alpha, \beta} (\omega) = \{ z \in \mathbb{C}: \alpha < |z - \omega| < \beta$ for $0 \le \alpha < \beta \le \infty \}$. Let $\Gamma$ be a cycle in $A_{\alpha, \beta} (\omega)$.

	Let $\gamma_1(t) = \omega + \rho_1 e^{-2 \pi i t}$ for $t \in [0, 1]$, $\gamma_2(t) = \omega + \rho_2 e^{2 \pi i t}$ for $t \in [0, 1]$ where $\alpha < \rho_1 < \rho_2 < \beta$. Define
	\[
		\Gamma = \gamma_1 + \gamma_2
	\]
	We claim that $\Gamma$ is homologous to zero in $A_{\alpha, \beta} (\omega)$. Let $\omega' \in A_{\alpha, \beta} (\omega)$. Consider $a \notin A_{\alpha, \beta} (\omega)$. Then
	\begin{itemize}
		\item If $|\omega - a| > \beta$ then
		\[
			\begin{aligned}
				I(\Gamma, a)
					& = I(\gamma_1, a) + I(\gamma_2, a) \\
					& = \frac{1}{2 \pi i} \int_{\gamma_1} \frac{1}{z - a} dz + \frac{1}{2 \pi i} \int_{\gamma_2} \frac{1}{z - a} dz
			\end{aligned}
		\]
		But then $1 / (z - a)$ is holomorphic. So by Cauchy's theorem both integrals vanish.
		\item If $|\omega - a| < \alpha$ then by CIF,
		\[
			\int_{\gamma_1} \frac{1}{z - a} dz = -1, \quad \int_{\gamma_2} \frac{1}{z - a} dz = 1
		\]
		so $I(\Gamma, a) = 0$.
	\end{itemize}
\end{example}

\subsection{General form of the Cauchy-Taylor theorem and CIF}

\begin{definition}
	A closed curve $\gamma: [a, b] \rightarrow \mathbb{C}$ is called \textbf{simple} if for all $t_1 < t_2$
	\[
		\gamma(t_1) = \gamma(t_2) \Longrightarrow t_1 = a \text{ and } t_2 = b
	\]
	i.e. it cannot cross itself or go back on itself.
\end{definition}

\begin{theorem}
	(\textbf{Jordan curve theorem}) Let $\gamma \subset \mathbb{C}$ be a simple closed curve. Then its complement $\mathbb{C} - \gamma$ is a disjoint union of two domains, exactly one of which is bounded.
\end{theorem}

\begin{proof}
	Beyond the scope of this course, so omitted.
\end{proof}

\begin{definition}
	The bounded domain is called the \textbf{interior} of $\gamma$ and we write $D_{\gamma}^{\text{int}}$. We say $w \in D_{\gamma}^{\text{int}}$ \textbf{lies inside} $\gamma$.
\end{definition}

\begin{definition}
	The other, non-bounded, domain is called the \textbf{exterior} of $\gamma$ and we write $D_{\gamma}^{\text{ext}}$. We say $w \in D_{\gamma}^{\text{ext}}$ \textbf{lies outside} $\gamma$.
\end{definition}

\begin{remark}
	$\mathbb{C} = D_{\gamma}^{\text{int}} \cup \gamma \cup D_{\gamma}^{\text{ext}}$ as a disjoint union.
\end{remark}

\begin{remark}
	Given a simple closed \textbf{contour}, it is always possible to place an orientation on $\gamma$ such that
	\[
		\forall w \in \mathbb{C} - \gamma, \quad I(\gamma, w) = \begin{cases}
			1 & \text{ if } w \in D_{\gamma}^{\text{int}} \\
			0 & \text{ if } w \in D_{\gamma}^{\text{ext}}
		\end{cases}
	\]
	We call $\gamma$ \textbf{positively oriented} if this equation holds.
\end{remark}

\begin{definition}
	$f$ is called \textbf{holomorphic on $D_{\gamma}^{\text{int}} \cup \gamma$} (and \textbf{holomorphic on and inside $\gamma$}) if for some domain $D$ containing $D_{\gamma}^{\text{int}} \cup \gamma$ on which $f$ is holomorphic.
\end{definition}

\begin{remark}
	For a simple closed curve, $\gamma$ is homologous to zero in $D$, since if $w \notin D$ then $w \in D_{\gamma}^{\text{ext}}$ so $I(\gamma, w) = 0$.
\end{remark}

\begin{theorem}
	Let $f: D \rightarrow \mathbb{C}$ be a holomorphic function on a domain $D$. Then for every cycle $\Gamma \in D$ such that $\Gamma$ is homologous to zero in $D$, for every $\omega \in D - \Gamma$, we have the \textbf{General form of the Cauchy-Taylor theorem}
	\[
		\int_{\Gamma} f(z) dz = 0
	\]
	and the \textbf{General form of the Cauchy integral formula}
	\[
		\int_{\Gamma} \frac{f(z)}{z - w} dz = 2 \pi i \cdot I(\Gamma, \omega) \cdot f(\omega)
	\]
\end{theorem}

\begin{example}
	By definition, if $D$ is simply connected, every cycle in $D$ is homologous to zero, so for every closed contour $\Gamma$,
	\[
		\int_{\Gamma} f(z) dz = 0
	\]
\end{example}

\begin{example}
	Let $D = B_r(a)$ and let $\gamma$ be $|z - a| = \rho$. Then for every $\omega \in B_r(a)$ with $|\omega - a| < \rho$,
	\[
		I(\gamma, \omega) = 1
	\]
	so
	\[
		\int_{\gamma} \frac{f(z)}{z - \omega} dz = 2 \pi i \cdot f(\omega)
	\]
\end{example}

\begin{theorem}
	Let $\gamma$ be a simple closed contour, positively oriented, and let $f$ be holomorphic on $D_{\gamma}^{\text{int}} \cup \gamma$. Then we have \textbf{Cauchy's theorem for simple closed curves}
	\[
		\int_{\gamma} f(z) dz = 0
	\]
	and \textbf{Cauchy's integral formula for simple closed curves}: if $w \in D_{\gamma}^{\text{int}}$ then
	\[
		\int_{\gamma} \frac{f(z)}{z - w} dz = 2 \pi i \cdot f(w)
	\]
\end{theorem}

\begin{remark}
	From here onwards, this will be the version of these two theorems which we use most often.
\end{remark}

\begin{example}
	Let $\gamma$ be the square with vertices at $1 + i, 1 - i, -1 + i, -1 - i$. Consider
	\[
		\int_{\gamma} \frac{\cos(z)}{z (z^2 + 2)} dz
	\]
	$f(z) = \cos(z) / (z^2 + 2)$ is holomorphic on $D_{\gamma}^{\text{int}} \cup \gamma$ so by CIF for simple closed curves,
	\[
		\int_{\gamma} \frac{f(z)}{z} dz = 2 \pi i \cdot f(0) = \pi i
	\]
\end{example}

\begin{example}
	(Problems class) Let $f$ and $g$ be holomorphic functions on a domain $D$ and suppose the product $\bar{f} g$ is also holomorphic on $D$.
	\begin{enumerate}
		\item Show that if $g(w) \ne 0$ for some $w \in D$ then $\bar{f}$ is holomorphic at $w$.
		\item Prove that either $f$ is constant on $D$ or $g(z) = 0 \ \forall z \in D$.
	\end{enumerate}

	\begin{enumerate}
		\item Let $h(z) = \overline{f(z)} g(z)$. By continuity (or Theorem 7.19 (lecture notes)), for some $\rho > 0$, $g(z) \ne 0$ on $B_{\rho}(w)$. Hence $\overline{f(z)} = h(z) / g(z)$ is holomorphic on $B_{\rho}(w)$ since $g$ and $h$ are holomorphic.
		\item Assume that for some $w$, $g(w) \ne 0$. By the proof of the local max mod (lecture notes), we know that if $f$ and $\bar{f}$ are holomorphic then $f$ is constant on $B_{\rho}(w)$. By analytic continuation, $f$ must be constant on $D$.
	\end{enumerate}
\end{example}

\begin{example}
	(Problems class) Let $S \subset \mathbb{C}$ be a non-empty set and let $w \in S$. Show that the first statements implies the second:
	\begin{enumerate}
		\item The point $w$ is a non-isolated point in $S$.
		\item For some sequence $\{ z_n \}_{n \in \mathbb{N}}$ with $z_n \in S - \{ w \}$, $z_n \rightarrow w$ as $n \rightarrow \infty$.
	\end{enumerate}

	Assume $w$ is non-isolated in $S$. So $\forall \epsilon > 0, \exists z \in B_{\epsilon}(w) \cap S$, $z \ne w$. In particular, let $\epsilon = 1 / n$. By definition, for some $z_n \in S$, $|z_n - w| < 1 / n$ and $z_n \ne w$. So $z_n \rightarrow w$ as $n \rightarrow \infty$.
\end{example}

\begin{example}
	(Problems class) Let $f$ be holomorphic on the unit disc $B_1(0)$ and satisfy $f(1 / n) = 1 / (n + 1)$ for every $n \ge 2$. Find the analytic continuation of $f$ to $\mathbb{C} - \{ -1 \}$.

	We have
	\[
		f(1 / n) = 1 / (n + 1) = \frac{1 / n}{1 + 1 / n}
	\]
	Therefore $f$ matches $g(z) = z / (1 + z)$ on the sequence $\{ 1 / n \}_{n \ge 2}$. The limit of this sequence is $0$. Using Lemma 2.29 (lecture notes),
	\[
		f(0) = f \left( \lim_{n \rightarrow \infty} \frac{1}{n} \right) = \lim_{n \rightarrow \infty} f \left( \frac{1}{n} \right) = \lim_{n \rightarrow \infty} \frac{1}{n + 1} = 0 = 0 / (1 + 0) = g(0)
	\]
	Hence $S = \{ z \in B_1(0): f(z) = g(z) \}$ contains $\{ 1 / n \}_{n \ge 2} \cup \{ 0 \}$ and $z = 0$ is a non-isolated point of $S$. So by the \hyperref[thm:identityTheorem]{Identity Theorem}, $f(z) = g(z) = z / (1 + z)$ on $B_1(0)$. By uniqueness of analytic continuation, the only holomorphic function $g$ on $\mathbb{C} - \{ -1 \}$ that matches $f$ on $B_1(0)$ is $g(z) = z / (1 + z)$.
\end{example}

\begin{example}
	(Problems class)
	\begin{enumerate}
		\item Show that $\Re(\Log(z)) = u(x, y) = \log (\sqrt{x^2 + y^2})$ is harmonic on $\mathbb{C} - \{ 0 \}$.
		\item Show that there does not exist a function $f$ that is holomorphic on $\mathbb{C} - \{ 0 \}$ with real part $u$.
	\end{enumerate}

	\begin{enumerate}
		\item For $(x, y) \ne (0, 0)$, $u(x, y) = \frac{1}{2} \log (x^2 + y^2)$. So
		\[
			\begin{aligned}
				u_x = \frac{1}{2} \frac{2x}{x^2 + y^2} = \frac{x}{x^2 + y^2} & \Longrightarrow u_{xx} = \frac{y^2 - x^2}{{(x^2 + y^2)}^2} \\
				& u_{yy} = \frac{x^2 - y^2}{{(x^2 + y^2)}^2}
			\end{aligned}
		\]
		Thus $u_{xx} + u_{yy} = 0$. Hence $u$ is harmonic.
		\item Assume there exists a holomorphic $f = u + iv$ on $\mathbb{C} - \{ 0 \}$ such that $u = \log(|z|)$. Then $v$ is a harmonic conjugate of $u$ on $\mathbb{C} - \{ 0 \}$. But this contains $\mathbb{C} - \mathbb{R}_{\le 0}$ so $v$ is a harmonic conjugate of $u$ on $\mathbb{C} - \mathbb{R}_{\le 0}$. But $\Log(z) = u + i \Arg(z)$ is holomorphic on this set and has real part $u$. Thus by uniqueness of the harmonic conjugate, $v = \Arg(z) + c$. But then $f = u + iv = \Log(z) + ic$ on $\mathbb{C} - \mathbb{R}_{\le 0}$. But this is a contradiction because $f$ would not then be holomorphic.
	\end{enumerate}
\end{example}

\begin{example}
	(Problems class) Let $u = 3x^2 y - y^3 - 2x^2 + 2y^2$. Find the harmonic conjugate $v$ of $u$.

	We need $u_x = v_y$ and $u_y = -v_x$. So $v_y = u_x = 6xy - 4x$ so $v(x, y) = 3xy^2 - 4xy + g(x)$. Also $v_x = 3y^2 - 4y + g'(x) = -u_y = -3x^2 + 3y^2 - 4y$. So $g'(x) = -3x^2$ so $g(x) = x^3 + c$ for a constant $c$. So $v(x, y) = 3xy^2 - 4xy - x^3 - c$.
\end{example}

\section{Holomorphic functions on punctured domains}

\subsection{Laurent series}

\begin{definition}
	A \textbf{Laurent series} is an infinite series of the form
	\[
		\sum_{n = -\infty}^{\infty} c_n {(z - a)}^n
	\]
	where $c_n \in \mathbb{C}$ and $a \in \mathbb{C}$. $a$ is called the \textbf{centre}. The sum
	\[
		\sum_{n = 0}^{\infty} c_n {(z - a)}^n
	\]
	is called the \textbf{analytic part} and the sum
	\[
		\sum_{n = -\infty}^{-1} c_n {(z - a)}^n
	\]
	is called the \textbf{principal part}.
\end{definition}

\begin{definition}
	We say a Laurent series \textbf{converges at $z \in \mathbb{C}$} iff the principal part and analytic part independently converge at $z$.
\end{definition}

\begin{remark}
	A Laurent series is a Taylor series if the principal part is $0$, otherwise, it is not defined at $z = a$.
\end{remark}

\begin{definition}
	For every $0 \le r < R \le \infty$, and $a \in \mathbb{C}$, the \textbf{annulus} of centre $a$, interior radius $r$ and external radius $R$ is defined as
	\[
		A_{r, R}(a) = \{ z \in \mathbb{C}: r < |z - a| < R \}
	\]
\end{definition}

\begin{proposition}
	Given a Laurent series with a non-zero principal part, then either
	\begin{itemize}
		\item The Laurent series converges nowhere \textbf{or}
		\item For some $r, R$, the Laurent series converges absolutely on the annulus $A_{r, R}(a)$ and does not converge for $|z - a| > R$ or $|z - a| < r$. We call $A_{r, R}(a)$ the \textbf{annulus of convergence}.
	\end{itemize}
\end{proposition}

\begin{proof}
	By definition, the Laurent series converges iff
	\[
		F_1(z) = \sum_{n = 0}^{\infty} c_n {(z - a)}^n, \quad F_2(z) = \sum_{n = -\infty}^{-1} c_n {(z - a)}^n
	\]
	both converge. By Theorem 5.15 (lecture notes), either $F_1$ converges at $z = a$ (but then the Laurent series converges nowhere as the principal is not defined at $z = a$) or for some $0 < R \le \infty$, $F_1$ converges absolutely for $|z - a| < R$.

	Now define $w = 1 / (z - a)$. Then
	\[
		F_2(z) = \sum_{n = -\infty}^{-1} c_n (z - a)^n = \sum_{n = -\infty}^{-1} c_n w^{-n} = \sum_{m = 1}^{\infty} c_{-m} w^m =: \tilde{F}(w)
	\]
	Either $\tilde{F}$ converges only at $w = 0$ (so the Laurent series converges nowhere) or for some $0 < R' \le \infty$ such that $\tilde{F}$ converges when $|w| < R$. Let
	\[
		r = \begin{cases}
			1 / R' & \text{ if } R' \ne \infty \\
			0 & \text{ if } R' = \infty
		\end{cases}
	\]
	Then $F_2$ converges when $|w| < R' \Longleftrightarrow |z - a| > $. TODO: check lecture notes. 
\end{proof}

TODO: notes from lecture.

\begin{theorem}
	(\textbf{Holomorphic functions on annuli have convergent Laurent series}) Let $A_{r, R}(a)$ be an annulus and $f: A_{r, R}(a) \rightarrow \mathbb{C}$ be holomorphic. Then for some $c_n \in \mathbb{C}$,
	\[
		f(z) = \sum_{n = -\infty}^{\infty} c_n {(z - a)}^n \quad (z \in A_{r, R}(a))
	\]
	and the annulus of convergence of this Laurent series contains $A_{r, R}(a)$.
\end{theorem}

\begin{proof}
	TODO: first part of proof.

	For $z \in \gamma_2$,
	\[
		\begin{aligned}
			\frac{1}{z - w} = \frac{1}{(z - a)(1 - \frac{w - a}{z - a})} & = \frac{1}{z - a} \sum_{n = 0}^{\infty} {\left( \frac{w - a}{z - a} \right)}^n \\
			& = \sum_{n = 0}^{\infty} \frac{{(w - a)}^n}{(z - a)^{n + 1}}
		\end{aligned}
	\]
	so
	\[
		\begin{aligned}
			f_2(w)
				& = \frac{1}{2 \pi i} \int_{\gamma_2} \sum_{n = 0}^{\infty} \frac{{(w - a)}^n}{(z - a)^{n + 1}} dz \\
				& = \sum_{n = 0}^{\infty} \left( \frac{1}{2 \pi i} \int_{\gamma_2} \frac{f(z)}{{(z - a)}^{n + 1}} dz \right) {(w - a)}^n \\
				& = \sum_{n = 0}^{\infty} c_n {(w - a)}^n
		\end{aligned}
	\]
	For $z \in \gamma_1$,
	\[
		\begin{aligned}
			\frac{1}{z - w} = -\frac{1}{w - z} & = \frac{-1 / (w - a)}{1 - \frac{z - a}{w - a}} \\
			& = -\frac{1}{w - a} \sum_{n = 0}^{\infty} {\left( \frac{z - a}{w - a} \right)}^n
		\end{aligned}
	\]
	Hence
	\[
		\begin{aligned}
			f_1(w)
				& = -\frac{1}{2 \pi i} \int_{\gamma_1} f(z) \sum_{n = 0}^{\infty} \frac{{(z - a)}^n}{{(w - a)}^{n + 1}} dz \\
				& = \sum_{n = 0}^{\infty} \left( -\frac{1}{2 \pi i} \int_{\gamma_1} f(z) {(z - a)}^n \right) \frac{1}{{(w - a)}^{n + 1}} \\
				& = \sum_{m = -\infty}^{-1} \left( -\frac{1}{2 \pi i} \int_{\gamma_1} \frac{f(z)}{{(z - a)}^{m + 1}} dz \right) {(w - a)}^m \quad \text{ where } m = -n - 1
		\end{aligned}
	\]
\end{proof}

\subsection{Classification of isolated singularities}

\begin{definition}
	We define the \textbf{punctured ball} of radius $R$ centred at $a$ as
	\[
		B_R^*(a) = \{ z \in \mathbb{C}: 0 < |z - a| < R \} = B_R(a) - \{ a \} = A_{0, R}(a)
	\]
\end{definition}

\begin{definition}\label{def:isolatedSingularity}
	$f$ is said to have an \textbf{isolated singularity} at $z = a$ if $f$ is holomorphic on $B_R^*(a)$ and one of the following conditions holds:
	\begin{enumerate}
		\item $\forall n \le -1, c_n = 0$ where $c_n$ are the coefficients of the Laurent series of $f$ (so $f$ has zero principal part). In this case, $f$ is said to have a \textbf{removable singularity} at $z = a$, and
		\[
			f(z) = \sum_{n = 0}^{\infty} c_n {(z - a)}^n
		\]
		\item For some $k > 0$, $c_{-k} \ne 0$ but $\forall n > k, c_{-n} = 0$. In this case, $f$ is said to have a \textbf{pole of order $k$} at $z = a$, and
		\[
			f(z) = \sum_{n = -k}^{\infty} c_n {(z - a)}^n
		\]
		So the principal part has finitely many non-zero terms.
		\item There are infinitely many negative terms in the Laurent series of $f$. In this case, $f$ is said to have a \textbf{essential singularity} at $z = a$.
	\end{enumerate}
\end{definition}

\begin{definition}
	A pole of order $1$ is called \textbf{simple}.
\end{definition}

\begin{example}
	Let $f(z) = (e^z - 1) / z$ on $B_1^*(0)$.
	\[
		f(z) = \frac{\sum_{n = 0}^{\infty} \frac{z^n}{n!} - 1}{z} = \sum_{n = 1}^{\infty} \frac{z^n}{n!} = \sum_{m = 0}^{\infty} \frac{z^m}{(m + 1)!}
	\]
	Hence $f$ has a removable singularity at $z = 0$. Note we can set $f(0) = 1$. Then the new $f$ is holomorphic on $B_1(0)$.
\end{example}

\begin{example}
	Let $f(z) = (e^z - 1) / z^2$ on $B_1^*(0)$.
	\[
		f(z) = \frac{\sum_{n = 0}^{\infty} \frac{z^n}{n!} - 1}{z^2} = \sum_{n = 1}^{\infty} \frac{z^{n - 2}}{n!} = \sum_{m = -1}^{\infty} \frac{z^m}{(m + 2)!}
	\]
\end{example}

\begin{example}
	Let $f(z) = e^{1 / z}$ be defined on $B_1^*(0)$.
	\[
		f(z) = \sum_{n = 0}^{\infty} \frac{{(1 / z)}^n}{n!} = \sum_{n = 0}^{\infty} \frac{1}{n!} \frac{1}{z^n} = \sum_{m = -\infty}^{0} \frac{z^m}{(-m)!}
	\]
	Hence $f$ has an essential singularity at $z = 0$.
\end{example}

\subsection{Removable singularities}

\begin{lemma}\label{lem:removableSingularityIffExtendsToHolomorphic}
	Let $f$ be holomorphic on $B_R^*(a)$. Then $f$ has a removable singularity at $z = a$ iff $f$ extends to a holomorphic function on $B_R(a)$.
\end{lemma}

\begin{proof}
	($\Leftarrow$): If $f$ extends to a holomorphic function on $B_R(a)$, for some $g: B_R(a) \rightarrow \mathbb{C}$, $f(z) = g(z)$ on $B_R^*(a)$. By Cauchy-Taylor (lecture notes), $g$ has a Taylor series on $B_R(a)$; in particular this Taylor series matches $f$ on $B_R^*(a)$. Hence $f$ has a removable singularity at $z = a$.
	
	($\Rightarrow$): If $f$ has a removable singularity at $z = a$, then by Definition~\ref{def:isolatedSingularity}, the Laurent series of $f$ is a Taylor series on $B_R^*(a)$. This Taylor series must converge on $B_R(a)$ by theorem 5.15 (lecture notes). Hence $f$ is equal to its Taylor series on $B_R^*(a)$ and the Taylor series is holomorphic on $B_R(a)$ and extends $f$ to $B_R(a)$.
\end{proof}

\begin{proposition}\label{prop:removableSingularityIffLimitIsZero}
	Let $f: B_R^*(a) \rightarrow \mathbb{C}$ is holomorphic. Then $f$ has a removable singularity at $z = a$ iff
	\[
		\lim_{z \to a} (z - a) f(z) = 0
	\]
\end{proposition}

\begin{proof}
	($\Rightarrow$): Assume $f$ has a removable singularity at $z = a$. By Lemma~\ref{lem:removableSingularityIffExtendsToHolomorphic}, $f$ can be extended to a holomorphic function $\tilde{f}$ at $z = a$. Then
	\[
		\lim_{z \to a} (z - a) f(z) = \lim_{z \to a} (z - a) \tilde{f}(z) = (a - a) \tilde{f} (a) = 0
	\]
	($\Leftarrow$): Assume $\lim_{z \to a} (z - a) f(z) = 0$. Since $f$ is holomorphic on $B_R^*(a)$, by Theorem 9.7 (lecture notes), $f$ has a Laurent series and by Proposition 9.5 (lecture notes), the coefficients of
	\[
		f(z) = \sum_{n = -\infty}^{\infty} c_n {(z - a)}^n
	\]
	are given by
	\[
		c_n = \frac{1}{2 \pi i} \int_{|z - a| = \rho} \frac{f(z)}{{(z - a)}^{n + 1}} dz, \quad \rho \in (0, R)
	\]
	We need to show that $\forall n > 0, c_{-n} = 0$. By the Estimation Lemma (lecture notes),
	\[
		\begin{aligned}
			0 \le |c_n| & = \left| \frac{1}{2 \pi i} \int_{|z - a| = \rho} \frac{f(z)}{{(z - a)}^{n + 1}} dz \right| \\
			& \le \frac{1}{2 \pi} \cdot 2 \pi \rho \cdot \sup_{|z - a| = \rho} \left| \frac{f(z)}{{(z - a)}^{n + 1}} \right| \\
			& = \rho \cdot \sup_{|z - a| = \rho} \left| \frac{|(z - a) f(z)|}{|z - a|^{n + 2}} \right| \\
			& = \frac{\rho}{\rho^{n + 2}} \cdot \sup_{|z - a| = \rho} \left| (z - a) f(z) \right| \\
			& = \frac{1}{\rho^{n + 1}} \cdot \sup_{|z - a| = \rho} \left| (z - a) f(z) \right|
			& \to 0 \quad \text{as} \quad \rho \to 0, \forall n \le -1
		\end{aligned}
	\]
	Thus $c_n = 0$ by squeezing.
\end{proof}

\begin{example}
	Let $f(z) = (e^z - 1) / z$ on $B_1^*(0)$.

	\[
		\lim_{z \rightarrow 0} z f(z) = \lim_{z \to 0} (e^z - 1) = e^0 - 1 = 0
	\]
	So by Proposition~\ref{prop:removableSingularityIffLimitIsZero}, $f$ has a removable singularity.
\end{example}

\subsection{Poles}

\begin{proposition}
	Let $f: B_R^*(a) \rightarrow \mathbb{C}$ be holomorphic. Then the following statements are all equivalent:
	\begin{enumerate}
		\item $f$ has a pole of order $k$ at $z = a$.
		\item $f(z) = g(z) / {(z - a)}^k$ for some holomorphic function $g: B_R(a) \rightarrow \mathbb{C}$, with $g(a) \ne 0$.
		\item For some $0 < r < R$ and some holomorphic $h: B_r(a) \rightarrow \mathbb{C}$, $h$ has a zero of order $k$ at $z = a$ and $f(z) = 1 / h(z)$ on $B_r^*(a)$.
	\end{enumerate}
\end{proposition}

\begin{proof}
	\hfill
	\begin{itemize}
		\item (1. $\Rightarrow$ 2.): If $f$ has a pole of order $k$ at $z = a$, then its Laurent series is of the form
		\[
			f(z) = \sum_{n = -k}^{\infty} c_n {(z - a)}^n
		\]
		where $c_{-k} \ne 0$. Define $g_1: B_R^*(a) \rightarrow \mathbb{C}$ by
		\[
			\begin{aligned}
				g_1(z) := f(z) \cdot {(z - a)}^k
					& = {(z - a)}^k \sum_{n = -k}^{\infty} c_n {(z - a)}^n \\
					& = \sum_{n = -k}^{\infty} c_n {(z - a)}^{n + k} \\
					& = \sum_{m = 0}^{\infty} c_{m - k} {(z - a)}^m
			\end{aligned}
		\]
		This is a Taylor series so $g_1$ has a removable singularity at $z = a$. By Lemma 9.11 (lecture notes), $g_1$ can be extended to a holomorphic function $g: B_R(a) \rightarrow \mathbb{C}$ on $B_R(a)$. Then $f(z) = g_1(z) / {(z - a)}^k = g(z) / {(z - a)}^k$ on $B_R^*(a)$. Finally, $g(a) = c_{-k} \ne 0$.
	
		\item (2. $\Rightarrow$ 3.): If $f(z) = g(z) / {(z - a)}^k$ for some holomorphic function $g: B_R(a) \rightarrow \mathbb{C}$, with $g(a) \ne 0$, then by the Principle of Isolated Zeros (lecture notes), for some $0 < r \le R$, $g(z) \ne 0$ on $B_r^*(a)$ and so as $g(a) \ne 0$, $g(z) \ne 0$ on $B_r(a)$. Define
		\[
			h(z) := \frac{{(z - a)}^k}{g(z)}
		\]
		which is holomorphic on $B_r(a)$. So $h(z) = {(z - a)}^k \frac{1}{g(z)}$ so $h$ has a zero of order $k$ at $z = a$. Finally,
		\[
			f(z) = \frac{g(z)}{{(z - a)}^k} = \frac{{(z - a)}^k}{h(z) {(z - a)}^k} = \frac{1}{h(z)}
		\]
		\item (3. $\Rightarrow$ 2.): By definition, for some holomorphic $h_1(z)$ on $B_r(a)$, $h(z) = {(z - a)}^k h_1(z)$ and $h_1(a) \ne 0$. Also, $f(z) = 1 / h(z) = 1 / ({(z - a)}^k h_1(z))$ which is holomorphic on $B_r^*(a)$. So $h_1(z) \ne 0$ on $B_r(a)$, hence
		\[
			g_1(z) := 1 / h_1(z)
		\]
		is holomorphic on $B_r(a)$ and $g_1(a) \ne 0$. Finally,
		\[
			{(z - a)}^k f(z) = \frac{{(z - a)}^k}{{(z - a)}^k h_1(z)} = g_1(z)
		\]
		on $B_r(a)$. By analytic continuation, $g_1$ can be extended to a holomorphic function $g$ on $B_R(a)$ which satisfies the same condition.
		\item (2. $\Rightarrow$ 1.): By Cauchy-Taylor, $g$ has a Taylor series on $B_R(a)$:
		\[
			g(z) = \sum_{n = 0}^{\infty} a_n {(z - a)}^n
		\]
		with $a_0 = g(a) \ne 0$. $f$ has Laurent series
		\[
			f(z) = \frac{g(z)}{{(z - a)}^k} = \sum_{n = 0}^{\infty} a_n {(z - a)}^{n - k} = \sum_{m = -k}^{\infty} a_{m + k} {(z - a)}^m
		\]
		Here, $c_{-k} = a_0 \ne 0$ so $f$ has a pole of order $k$ at $z = a$.
	\end{itemize}
\end{proof}

\begin{example}
	Classify the isolated singularities and zeroes of the following function:
	\[
		f(z) = \frac{\sin(z)}{z \cos(z)} = \frac{\tan(z)}{z}
	\]
	We look at the zeros of the numerator and the denominator. For the zeroes of the numerator,
	\[
		\sin(z) = 0 = \frac{1}{2i} \left( e^{iz} - e^{-iz} \right) \Longleftrightarrow e^{iz} = e^{-iz} \Longleftrightarrow e^{2iz} = 1
	\]
	Since by Proposition 1.9 (3.) (lecture notes) $e^z \Longleftrightarrow z = 2n \pi i$ for some $n \in \mathbb{Z}$, so $2iz = 2n \pi i$ so $z = n \pi$ for some $n \in \mathbb{Z}$. For the zeroes of the denominator,
	\[
		\begin{aligned}
			z \cos(z) = 0 & \Longleftrightarrow z = 0 \text{ or } (\cos(z) = 0 \\
			& \Longleftrightarrow e^{iz} = -e^{iz} \Longleftrightarrow e^{2iz} = -1 \Longleftrightarrow z = \frac{\pi}{2} + n \pi \text{ for some } n \in \mathbb{Z})	
		\end{aligned}
	\]
	$f(a) = 0$ when the numerator is zero and the denominator is non-zero, so $f(z) = 0$ for $z = n \pi$, $n \ne 0$. We have
	\[
		f'(z) = \frac{\cos(z)}{z \cos(z)} - \frac{\sin(z) (\cos(z) - z \sin(z))}{{(z \cos(z))}^2}
	\]
	so $f'(n \pi) = 1 / (n \pi) \ne 0$ so $n \pi$ for $n \ne 0$ is zero of order $1$.
	
	By Proposition 9.15 (lecture notes), to find the poles of $f$, consider the zeroes of
	\[
		h(z) = \frac{z \cos(z)}{\sin(z)}
	\]
	$h(z) \ne 0$ when the numerator is zero and the denominator is non-zero, so when $z = n\pi + \pi / 2$. We have
	\[
		h'(z) = \frac{\cos(z) - z \sin(z)}{\sin(z)} - \frac{z {\cos(z)}^2 }{{(\sin(z))}^2}
	\]
	so $h'(n \pi + \pi / 2) = -n \pi - \pi / 2 \ne 0$ so $h$ has a zero of order $1$ at $z = n \pi + \pi / 2$. By Proposition 9.15 (lecture notes), $f$ has a pole of order $1$ at $z = n \pi + \pi / 2$.

	To determine what happens at $z = 0$,
	\[
		\lim_{z \rightarrow 0} z f(z) = \lim_{z \rightarrow 0} \frac{\sin(z)}{\cos(z)} = 0
	\]
	so by Proposition 9.12 (lecture notes), $f$ has a removable singularity at $z = 0$.
\end{example}

\begin{proposition}
	Let $f: B_R^*(a) \rightarrow \mathbb{C}$ be holomorphic. Then $f$ has a pole at $z = a$ iff
	\[
		\lim_{z \rightarrow a} |f(z)| = \infty \Longleftrightarrow (\forall M > 0, \exists \delta > 0, |z - a| < \delta \Longrightarrow |f(z)| \ge M)
	\]
\end{proposition}

\begin{proof}
	($\Rightarrow$): Assume $f$ has a pole of order $k$. By Proposition 9.15 (3.) (lecture notes),
	\[
		|f(z)| = |g(z)| \cdot |{(z - a)}^{-k}|
	\]
	for some holomorphic $g$. But $g(a) \ne 0$ and $g$ is holomorphic so
	\[
		\lim_{z \to a} |g(z)| \cdot |z - a|^{-k} = \infty
	\]
	($\Leftarrow$): If the limit holds, then for some $r > 0$, $f(z) \ne 0$ on $B_r^*(a)$. So $h(z) = 1 / f(z)$ is holomorphic on $B_r^*(a)$. Now
	\[
		0 \le |(z - a) h(z)| = |(z - a) / f(z)| = 0
	\]
	so $h$ has a removable singularity at $z = a$ by Lemma 9.11 (lecture notes). So $h$ can be extended to a holomorphic function on $B_r(a)$. Finally,
	\[
		|h(a)| = \lim_{z \to a} |h(z)| = \lim_{z \to a} 1 / |f(z)| = 0
	\]
	So $h(a) = 0$ so $h$ has a zero at $z = a$.
\end{proof}

\subsection{Essential singularities}

\begin{theorem}
	(\textbf{Casoratti-Weierstrass Theorem}) Let $f: B_R^*(a) \to \mathbb{C}$ be holomorphic and let $f$ have an essential singularity at $z = a$. Then
	\[
		\forall w \in \mathbb{C}, \forall 0 < r < R, \forall \epsilon > 0, \exists z \in B_r^*(a), \quad f(z) \in B_{\epsilon}(w)
	\]
\end{theorem}

\begin{proof}
	Assume the converse, that
	\[
		\exists w \in \mathbb{C}, \exists 0 < r < R, \exists \epsilon > 0, \forall z \in B_r^*(a), \quad f(z) \notin B_{\epsilon}(w)
	\]
	Define
	\[
		g(z) := \frac{1}{f(z) - w}
	\]
	Since $|f(z) - w| \ge \epsilon$, $|g(z)| \le 1 / \epsilon$ on $B_r^*(a)$ and $g$ is holomorphic on $B_r^*(a)$. By Theorem 9.14 (lecture notes), $g$ has a removable singularity at $z = a$ so by Lemma 9.11 (lecture notes), $g$ can be extended to a holomorphic function on $B_r(a)$.
	
	Now $f(z) = w + 1 / g(z)$. If $g(a) \ne 0$, then
	\[
		\lim_{z \to a} (z - a) f(z) = \lim_{z \to a} (z - a) \frac{w g(z) + 1}{g(z)} = 0 \cdot \frac{w g(a)}{g(a)} = 0
	\]
	so $f$ has a removable singularity at $z = a$. Otherwise if $g(a) = 0$, define
	\[
		h(z) = 1 / f(z) = \frac{g(z)}{w g(z) + 1}
	\]
	$h$ is holomorphic and $h(a) = 0$. So $a$ is a zero of $h$, so by Proposition 9.15 (lecture notes), $f$ has a pole at $z = a$. So either $f$ has a removable singularity at $z = a$ or a pole at $z = a$, so it does not have an essential singularity at $z = a$. Contradiction.
\end{proof}

\begin{theorem}\label{thm:bigPicard}
	(\textbf{Big Picard Theorem}) Let $f: B_R^*(a) \to \mathbb{C}$ be holomorphic and let $f$ have an essential singularity at $z = a$. Then
	\[
		\exists b \in \mathbb{C}, \forall 0 < r < R, \quad \mathbb{C} - \{ b \} \subseteq f(B_r^*(a))
	\]
\end{theorem}

\begin{proof}
	Omitted.
\end{proof}

\begin{example}
	Let $f(z) = e^{1 / z}$. By Theorem~\ref{thm:bigPicard}, $\mathbb{C} - \{ 0 \} \subseteq f(B_r^*(0))$. Pick $w \in \mathbb{C} - \{ 0 \}$. Then for some $z$, $e^{1 / z} = w$.
\end{example}

\begin{example}
	(Problems class) Compute the principal part of the Laurent Series about $z = 0$ of $f(z) = e^z / z^4$.

	We have
	\[
		e^z = \sum_{n = 0}^{\infty} \frac{z^n}{n!}
	\]
	So
	\[
		\begin{aligned}
			f(z)
				& = \sum_{n = 0}^{\infty} \frac{z^{n - 4}}{n!} \\
				& = \sum_{m = -4}^{\infty} \frac{z^m}{(m + 4)!}
		\end{aligned}
	\]
	So the principal part is
	\[
		\sum_{m = -4}^{-1} \frac{z^m}{(m + 4)!} = \frac{1}{z^4} + \frac{1}{z^3} + \frac{1}{2z^2} + \frac{1}{6z}
	\]
\end{example}

\begin{example}
	(Problems class) Compute the principal part of the Laurent Series about $z = 0$ of
	\[
		f(z) = \frac{e^{\sinh(z^2 + iz - 17)}}{e^{\cosh(z^2 + iz + 17)}}
	\]
	$f$ is holomorphic everywhere so the principal part is $0$.
\end{example}

\begin{example}
	(Problems class) Let $f(z) = z / (1 + z^2)$. Find the Laurent series of $f$ on $0 < |z| < 1$ and $|z| > 1$.
	For $0 < |z| < 1$:
	\[
		f(z) = z \frac{z}{1 - {(-z)}^2} = z \sum_{n = 0}^{\infty} {(-z^2)}^n = \sum_{n = 0}^{\infty} {(-1)}^n z^{2n + 1}
	\]
	For $|z| > 1$:
	\[
		\begin{aligned}
			f(z) = \frac{1/z}{1/z^2 + 1} & = \frac{1}{z} \frac{1}{1 - -(1/z^2)} \\
			& = \frac{1}{z} \sum_{n = 0}^{\infty} \left( -\frac{1}{z^2} \right)^n = \sum_{n = 0}^{\infty} {(-1)}^n \frac{1}{z^{2n + 1}} \\
			& = \sum_{m = -\infty}^{0} {(-1)}^m z^{2m - 1}
		\end{aligned}
	\]
\end{example}

\begin{example}
	(Problems class) Show that the Laurent series
	\[
		f(z) = \sum_{n = -\infty}^{\infty} z^n
	\]
	converges nowhere on $\mathbb{C}$.

	Consider the principal part and analytic part of the $f$. For $f$ to converge, both parts must converge. The analytic part is
	\[
		\sum_{n = 0}^{\infty} z^n
	\]
	which diverges for $|z| > 1$. The principal part is
	\[
		\sum_{n = -\infty}^{-1} z^n = \sum_{m = 1}^{\infty} z^{-m} = \sum_{m = 1}^{\infty} {(\frac{1}{z})}^m
	\]
	which diverges for $|1 / z| > 1 \Longleftrightarrow |z| < 1$. Every open annulus which contains the unit disc $|z| = 1$ must contain points $z$ with $|z| \ne 1$, hence by Proposition 9.3 (lecture notes), $f$ diverges for $|z| = 1$, hence it diverges on $\mathbb{C}$.
\end{example}

\begin{example}
	(Problems class) Let $f: B_r^*(a) \to \mathbb{C}$ be holomorphic with a pole of order $m$ at $z = a$. Show that $f'(z)$ has a pole of order $m + 1$ at $z = a$. What is the coefficient of $1 / (z - a)$ in the Laurent series of $f'$?

	We have
	\[
		f(z) = \sum_{n = -m}^{\infty} c_n {(z - a)}^n
	\]
	where $c_{-m} \ne 0$. By theorem 5.21 (lecture notes) and the fact that the principal part has finitely many terms (as $f$ has a pole), we can term-by-term differentiate. The derivative of the analytic part of $f$ is
	\[
		\sum_{n = 1}^{\infty} n c_n {(z - a)}^{n - 1}
	\]
	The principal part is
	\[
		\sum_{n = -m}^{-1} c_n {(z - a)}^n
	\]
	so its derivative is
	\[
		\sum_{n = -m}^{-2} n c_n {(z - a)}^{n - 1}
	\]
	So the Laurent series of $f'$ is
	\[
		\sum_{n = -m}^{-1} n c_n {(z - a)}^{n - 1} + \sum_{n = 1}^{\infty} n c_n {(z - a)}^{n - 1}
	\]
	so the coefficient of $1 / (z - a)$ is $0$.
\end{example}

\begin{example}
	(Problems class) For every $k, m > 0$, give a holomorphic $f$ on $\mathbb{C} - \{ \pm 1, \pm i \}$ with a zero of order $k$ at $z = -i$, a pole of order $m$ at $z = i$, a removable singularity at $z = 1$ and an essential singularity at $z = -1$.

	$f_1(z) = \frac{1}{{(z - i)}^m}$ has a pole of order $m$ at $z = i$. $f_2(z) = \frac{e^{z - 1} - 1/e}{z - 1}$ has a removable singularity at $z = 1$. $f_3(z) = e^{1/(z + 1)}$ has an essential singularity at $z = -1$.
	\[
		f(z) = (f_1(z) + f_2(z) + f_3(z)) {(z + i)}^k
	\]
	is therefore a possible solution.
\end{example}

\section{Cauchy's Residue Theorem}

\subsection{Cauchy's residue theorem}

\begin{definition}
	Let $D$ be a domain. $f: D \to \mathbb{C}$ is called \textbf{meromorphic} on $D$ if for some $S \subset D$,
	\begin{itemize}
		\item $f$ is holomorphic on $D - S$ and
		\item $f$ has a pole at every point in $S$ and
		\item all points in $S$ are isolated.
	\end{itemize}
\end{definition}

\begin{remark}
	If $\gamma$ is a simple closed contour, it has finitely many poles in $D_{\gamma}^{\text{int}}$. $f$ is called meromorphic on $D_{\gamma}^{\text{int}} \cup \gamma$ if for some domain $D$ where $D_{\gamma}^{\text{int}} \cup \gamma \subset D$, $f$ is meromorphic on $D$.
\end{remark}

\begin{definition}
	Let $D$ be a domain and $f$ be holomorphic on $D$ with a pole at $z = a$ with Laurent series
	\[
		f(z) = \sum_{n = -k}^{\infty} c_n {(z - a)}^n
	\]
	Then the \textbf{residue of $f$ at $z = a$}, denoted $\Res_{z = a}(f)$, is defined as
	\[
		\Res_{z = a}(f) = c_{-1}
	\]
\end{definition}

\begin{theorem}\label{thm:cauchysResidueTheorem}
	(\textbf{Cauchy's Residue Theorem}) Let $\gamma$ be a positively-oriented simple closed contour and let $f$ be meromorphic on 
	$D_{\gamma}^{\text{int}} \cup \gamma$. If $f$ has no poles on $\gamma$, then
	\[
		\int_{\gamma} f(z) dz = 2 \pi i \cdot \sum_{j = 1}^{m} \Res_{z = a_j}(f)
	\]
	where $a_1, \dots, a_n$ are the poles of $f$ \textbf{inside} $\gamma$.
\end{theorem}

\begin{proof}
	Let $a_1, \dots, a_n$ be the poles of $f$ in $D_{\gamma}^{\text{int}}$. Consider the Laurent series of $f$ about each pole $a_j$. Define
	\[
		f_j(z) = \sum_{m = -k_j}^{-1} c_{n, j} {(z - a_j)}^n
	\]
	as the principal part of the Laurent series about $z = a_j$, where $c_{n, j}$ is the $c_n$ coefficient in the Laurent series about $z - a_j$. Each $f_j(z)$ is a finite sum of reciprocals of ${(z - a_j)}^n$ hence $f_j(z)$ is holomorphic on $\mathbb{C} - \{ a_j \}$ so $f_j(z)$ can be extended to a holomorphic function on $\mathbb{C} - \{ a_j \}$. In particular, the contour integral of $f_j(z)$ around $\gamma$ is well-defined. For $n \le -1$, by CIF for simple closed contours and the fundamental theorem of calculus,
	\[
		\int_{\gamma} {(z - a_j)}^n dz = \begin{cases}
			2 \pi i & \text{ if } n = -1 \\
			0 & \text{ if } n \ne -1
		\end{cases}
	\]
	So
	\[
		\begin{aligned}
			\int_{\gamma} f_j(z) dz & = \int_{\gamma} \sum_{n = -k_j}^{-1} c_{n, j} {(z - a_j)}^n dz \\
			& = \sum_{n = -k_j}^{-1} c_{n, j} \int_{\gamma} {(z - a_j)}^n dz = 2 \pi i c_{-1, j} = \Res_{z = a_j}(f) \cdot 2 \pi i
		\end{aligned}
	\]
	We want to show
	\[
		\begin{aligned}
			& \int_{\gamma} f(z) dz = 2 \pi i \sum_{j = 1}^{m} \Res_{z = a_j}(f) = \sum_{j = 1}^{m} \int_{\gamma} f_j(z) dz \\
			\Longleftrightarrow & \int_{\gamma} \left( f(z) - \sum_{j = 1}^{m} f_j(z) \right) dz = 0
		\end{aligned}
	\]
	Let
	\[
		g(z) = f(z) - \sum_{j = 1}^{m} f_j(z)
	\]
	Since $f_j(z)$ is holomorphic at $z = a_j$ for $j \ne i$ and $f_j(z)$ is the principal part of the Laurent series of $f$ at $z = a_j$, $g$ has a Taylor series expansion about every $z = a_j$. Thus $g$ can be extended to a holomorphic function $\tilde{g}$ at every $z = a_j$, then $\tilde{g}$ is holomorphic on $D_{\gamma}^{\text{int}} \cup \gamma$. Finally,
	\[
		\int_{\gamma} \left( f(z) - \sum_{j = 1}^{m} f_j(z) \right) dz = \int_{\gamma} g(z) dz = \int_{\gamma} \tilde{g}(z) dz = 0
	\]
	by Cauchy's theorem for simple closed contours.
\end{proof}

\begin{proposition}\label{prop:calculatingResidues}
	(\textbf{Rules for calculating residues})
	\begin{enumerate}
		\item (\textbf{Linear combinations}) Let $f$ and $g$ both have a pole at $z = a$ and $A, B \in \mathbb{C}$. Then
		\[
			\Res_{z = a}(Af + Bg) = A \cdot \Res_{z = a}(f) + B \cdot \Res_{z = a}(g)
		\]
		\item (\textbf{Cover-up rule for poles of order $1$}) Let $f$ have a simple pole (order $1$) at $z = a$. Then
		\[
			\Res_{z = a}(f) = \lim_{z \to a} (z - a) f(z)
		\]
		\item (\textbf{Simple zero on the denominator}) If $f(z) = g(z) / h(z)$ with $g$ and $h$ holomorphic at $z = a$ and $g(a) \ne 0$ and $h$ has a zero of order $1$, then
		\[
			\Res_{z = a}(f) = \frac{g(a)}{h'(a)}
		\]
		\item (\textbf{Poles of higher order}) If $f(z) = g(z) / {(z - a)}^k$ where $g$ is holomorphic, then
		\[
			\Res_{z = a}(f) = \frac{g^{(k - 1)} (a)}{(k - 1)!}
		\]
	\end{enumerate}
\end{proposition}

\begin{proof}
	\hfill
	\begin{enumerate}
		\item Omitted.
		\item \[
			f(z) = \sum_{n = -1}^{\infty} c_n {(z - a)}^n
		\]
		with $c_{-1} \ne 0$, as the pole has order $1$. Then
		\[
			\lim_{z \to a} f(z) = \lim_{z \to a} \sum_{n = -1}^{\infty} c_n {(z - a)}^{n + 1} = \lim_{z \to a} \sum_{m = 0}^{\infty} c_{m - 1} {(z - a)}^m = c_{-1}
		\]
		\item By Proposition 9.15 (lecture notes), $f$ has a pole of order $1$ at $z = a$ hence
		\[
			f(z) = \sum_{n = -1}^{\infty} c_n {(z - a)}^n
		\]
		Since $h$ is holomorphic and $h(a) = 0$, by Cauchy-Taylor,
		\[
			h(z) = \sum_{n = 1}^{\infty} d_n {(z - a)}^n
		\]
		$h$ has a zero of order $1$ so $h(z) = (z - a) h_1(z)$ for some $h_1(z)$ with $h_1(a) \ne 0$ and so
		\[
			h_1(z) = \sum_{n = 0}^{\infty} d_n {(z - a)}^n
		\]
		and $h_1(a) = d_1 \ne 0$. Also,
		\[
			h'(z) = \sum_{n = 1}^{\infty} n d_n {(z - a)}^{n - 1}
		\]
		so $h'(a) = d_1 = h_1(a)$. We have
		\[
			\begin{aligned}
				& (z - a) \frac{g(z)}{h(z)} = (z - a) f(z) = \sum_{n = -1}^{\infty} c_n {(z - a)}^{n + 1} \\
				& \Longleftrightarrow \frac{g(z)}{h_1(z)} = \sum_{m = 0}^{\infty} c_{m - 1} {(z - a)}^m \\
				& \Longrightarrow \frac{g(a)}{h_1(a)} = c_{-1} = \Res_{z = a}(f)
			\end{aligned}
		\]
		\item $g$ is holomorphic so by Cauchy-Taylor,
		\[
			g(z) = \sum_{n = 0}^{\infty} c_n {(z - a)}^n
		\]
		So
		\[
			f(z) = \frac{1}{{(z - a)}^k} g(z) = \sum_{n = 0}^{\infty} c_n {(z - a)}^{n - k} = \sum_{m = -k}^{\infty} c_{m + k} {(z - a)}^m
		\]
		Hence by Remark 7.3 (lecture notes)
		\[
			\Res_{z = a}(f) = c_{k - 1} = \frac{g^{(k - 1)}(a)}{(k - 1)!}
		\]
	\end{enumerate}
\end{proof}

\begin{example}
	Let $f(z) = 1 / (z^2 - 9)$. It has simple poles at $z = \pm 3$ and
	\[
		\Res_{z = 3}(f) = \lim_{z \to 3} (z - 3) \frac{1}{(z + 3)(z - 3)} = \frac{1}{6}
	\]
\end{example}

\begin{example}
	Let $f(z) = 1 / \sin(z)$. Then $g(z) = 1$, $h(z) = \sin(z)$. $h$ has a zero of order $1$ at $z = 0$. Then
	\[
		\Res_{z = a}(f) = \frac{1}{\cos(0)} = 1
	\]
\end{example}

\begin{example}
	Let $f(z) = 1 / \sin(z)$, so $\Res_{z = 0}(f) = 1$ by the above example. So by Cauchy's residue theorem, for every $0 < \rho < \pi$,
	\[
		\int_{|z| = \rho} \frac{1}{\sin(z)} dz = 2 \pi i \cdot \Res_{z = 0}(f) = 2 \pi i
	\]
\end{example}

\begin{example}
	Compute $\int_{\gamma} e^z / (z^2 + z^3) dz$ where $\gamma$ is any simple closed contour with $0 \in D_{\gamma}^{\text{int}}$ and $-1 \in D_{\gamma}^{\text{ext}}$.
	\[
		f(z) = \frac{e^z}{z^2 (z + 1)}
	\]
	So $f$ has poles of order $2$ at $z = 0$ and order $1$ at $z = -1$. Let $g(z) = e^z / (1 + z)$. By Cauchy's residue theorem,
	\[
		\int_{\gamma} f(z) dz = 2 \pi i \cdot \Res_{z = 0}(f) = 2 \pi i \cdot g'(0) = 0
	\]
\end{example}

\subsection{Calculation of real integrals}

\begin{example}\label{exa:integralRationalTrigFunctions}
	(\textbf{Integral of rational functions of $\sin$ and $\cos$}) Show that for every $-1 < a < 1$,
	\[
		\int_{0}^{2 \pi} \frac{1}{1 + a \sin(\theta)} d\theta = \frac{2 \pi}{\sqrt{1 - a^2}}
	\]
	Assume $a \ne 0$. We have
	\[
		\sin(\theta) = \frac{e^{i\theta} - e^{-i\theta}}{2i}
	\]
	Note that $\gamma(\theta) = e^{i\theta}$ for $\theta \in [0, 2 \pi]$ is the unit circle and $\gamma'(\theta) = i e^{i\theta}$. So
	\[
		\begin{aligned}
			\int_{0}^{2 \pi} \frac{1}{1 + a \sin(\theta)} d\theta & = \int_{0}^{2 \pi} \frac{1}{1 + \frac{a}{2i} (e^{i\theta} - e^{-i\theta})} d\theta \\
			& = \int_{0}^{2 \pi} \frac{2i}{2i + a e^{i\theta} - a e^{-i\theta}} \frac{i e^{i \theta}}{i e^{i \theta}} d\theta \\
			& = \int_{0}^{2 \pi} \frac{2i}{2i + a \gamma(\theta) - a / \gamma(\theta)} \frac{\gamma'(\theta)}{i \gamma(\theta)} d\theta \\
			& = \int_{\gamma} \frac{2i}{(2i + az - a/z) iz} dz \\
			& = \int_{\gamma} \frac{2}{a} \frac{1}{z^2 + i \frac{2}{a} z - 1} dz \\
		\end{aligned}
	\]
	The denominator of the integrand has roots $z_1, z_2 = \left( \frac{-1 \pm \sqrt{1 - a^2}}{a} \right) i$. Since $|a| < 1$, $|z_2| = \frac{1 + \sqrt{1 - a^2}}{|a|} > 1$. Since $z_1 z_2 = -1$, $|z_1| < 1$. By Cauchy's residue theorem,
	\[
		\int_{0}^{2 \pi} \frac{1}{1 + a \sin(\theta)} d\theta = 2 \pi i \cdot \Res_{z = z_1}(f)
	\]
	Since $z = z_1$ is a pole of order $1$, by the cover up rule (lecture notes),
	\[
		\int_{0}^{2 \pi} \frac{1}{1 + a \sin(\theta)} d\theta = 2 \pi i \cdot \lim_{z \to z_1} (z - z_1) f(z) = 2 \pi i \frac{2}{a} \frac{1}{z_1 - z_2} = \frac{2 \pi}{\sqrt{1 - a^2}}
	\]
\end{example}

\begin{remark}
	We can use the method from Example~\ref{exa:integralRationalTrigFunctions} to compute
	\[
		\int_{0}^{2\pi} F(\sin(\theta), \cos(\theta)) d\theta
	\]
	where $F = P(x, y) / Q(x, y)$ is the quotient of polynomials $P$ and $Q$ in two variables. The general method is the \textbf{change of variables}: write
	\[
		z = e^{i \theta}, \quad \sin(\theta) = \frac{z - z^{-1}}{2i}, \quad \cos(\theta) = \frac{z + z^{-1}}{2}
	\]
	which gives
	\[
		\int_{0}^{2\pi} F(\sin(\theta), \cos(\theta)) d\theta = \int_{|z| = 1} F \left( \frac{z - z^{-1}}{2i}, \frac{z + z^{-1}}{2} \right) \frac{1}{iz} dz
	\]
	This method works as long as $F(z)$ does not have a pole on $|z| = 1$.
\end{remark}

\begin{example}\label{exa:integratingRationalFunctions}
	(\textbf{Integrating rational functions}) Evaluate
	\[
		\int_{0}^{\infty} \frac{x^2}{x^6 + 1} dx = \lim_{R \to \infty} \int_{0}^{R} \frac{x^2}{x^6 + 1} dx
	\]
	Let $f(z) = \frac{z^2}{z^6 + 1}$. The denominator is zero when $z^6 + 1 = 0 \Longleftrightarrow z^6 = e^{i\pi} = e^{\pi i + 2k\pi i}$ hence $f$ has poles at $z = e^{i\pi / 6 + k\pi i / 3}$ for $k = 0, 1, 2, 3, 4, 5$. The numerator is non-zero for all these $z_k$, so they are all poles of $f$. Since
	\[
		f(z) = \frac{z^2}{\prod_{k = 0}^{5} (z - z_k)}
	\]
	each pole is of order $1$. By Proposition~\ref{prop:calculatingResidues} (3) with $g(z) = z^2$, $h(z) = z^6 + 1$,
	\[
		\Res_{z = z_k}(f) = \frac{g(z_k)}{h'(z_k)} = \frac{z_k^2}{6 z_k^5} = \frac{1}{6 z_k^3}
	\]
	Consider the ``D'' shaped contour $\gamma_R$ defined by
	\[
		\gamma_R = L_R \cup C_R
	\]
	where $L_R$ is the line joining $-R$ to $R$ and $C_R$ is the semicircle of radius $R$ in the upper half plane rejoining $R$ to $-R$. By \hyperref[thm:cauchysResidueTheorem]{Cauchy's residue theorem}, for every $R > 1$,
	\[
		\int_{\gamma_R} f(z) dz = 2 \pi i \sum_{k = 0}^2 \Res_{z = z_k}(f) = \left( \frac{1}{6 z_0^3} + \frac{1}{6 z_1^3} + \frac{1}{6 z_2^3} \right) 2 \pi i = \frac{2 \pi i}{6} \left( \frac{1}{i} - \frac{1}{i} + \frac{1}{i} \right) = \frac{\pi}{3}
	\]
	Note that $f$ is even and so
	\[
		\int_{L_R} f(z) dz = \int_{-R}^{R} f(x) dx = 2 \int_{0}^{R} \frac{x^2}{x^6 + 1} dx
	\]
	Also,
	\[
		\int_{\gamma_R} f(z) dz = \int_{L_R} f(z) dz + \int_{C_R} f(z) dz
	\]
	Hence
	\[
		\begin{aligned}
			\lim_{R \to \infty} \int_{0}^{R} \frac{x^2}{x^6 + 1} dx & = \lim_{R \to \infty} \frac{1}{2} \int_{L_R} f(z) dz \\
			& = \frac{1}{2} \lim_{R \to \infty} \left( \int_{\gamma_R} f(z) dz - \int_{C_R} f(z) dz \right) = \frac{\pi}{6} - \lim_{R \to \infty} \int_{C_R} f(z) dz
		\end{aligned}
	\]
	We claim that
	\[
		\lim_{R \to \infty} \int_{C_R} f(z) dz = 0
	\]
	By the estimation lemma (lecture notes),
	\[
		\left| \int_{C_R} f(z) dz \right| \le L(C_R) \cdot \sup_{z \in C_R} |f(z)|
	\]
	$L(C_R) = \pi R$ and for $|z| = R$,
	\[
		|f(z)| = \left| \frac{z^2}{z^6 + 1} \right| = \frac{|z|^2}{|z^6 + 1|} \le \frac{|z|^2}{||z|^6 - 1|} = \frac{R^2}{R^6 - 1}
	\]
	Hence
	\[
		\left| \int_{C_R} f(z) dz \right| \le \pi R \frac{R^2}{R^6 - 1} = \frac{\pi R^3}{R^6 - 1} \to 0 \text{ as } R \to \infty
	\]
	Thus
	\[
		\int_{0}^{\infty} \frac{x^2}{x^6 + 1} dx = \frac{\pi}{6}
	\]
\end{example}

\begin{remark}
	The method in Example~\ref{exa:integratingRationalFunctions} works for integrals of the form
	\[
		\lim_{R \to \infty} \int_{-R}^{R} \frac{p(x)}{q(x)} dx
	\]
	for polynomials $p$ and $q$ which satisfy:
	\begin{itemize}
		\item $q$ has no real roots.
		\item $\deg(q) \ge \deg(p) + 2$.
	\end{itemize}
\end{remark}

\subsection{Principal value integrals}

\begin{definition}
	The \textbf{Cauchy principal value} of $\int_{-\infty}^{\infty} f(x) dx$ is defined as
	\[
		\text{P.V. } \int_{-\infty}^{+\infty} f(x) dx := \lim_{r \to \infty} \int_{-r}^{r} f(x) dx
	\]
	Note if $\int_{-\infty}^{\infty} f(x) dx$ exists then it equals the Cauchy principal value. If $f$ is even they also match.
\end{definition}

\begin{lemma}\label{lem:jordansLemma}
	(\textbf{Jordan's lemma}) If for some $R > 0$, a function $f$ is holomorphic on $D = \{ z \in \mathbb{C}: |z| > r \}$ and $z f(z)$ is bounded on $D$, then
	\[
		\forall \alpha > 0, \quad \lim_{R \to \infty} \int_{C_R} f(z) e^{i \alpha z} dz = 0
	\]
	where $C_R(\theta) = R e^{i \theta}$ for $\theta \in [0, \pi]$.
\end{lemma}

\begin{proof}
	$z f(z)$ is bounded, so for some $M \le 0$, $|z f(z)| \le M$ on $D$. Now
	\[
		\begin{aligned}
			\left| \int_{C_R} f(z) e^{i \alpha z} dz \right| & = \left| \int_{0}^{\pi} f(C_R(\theta)) e^{i \alpha C_R(\theta)} C_R'(\theta) d\theta \right| = \left| \int_{0}^{\pi} f(R e^{i\theta}) e^{i \alpha R e^{i\theta}} i R e^{i\theta} d\theta \right| \\
			& \le \int_{0}^{\pi} \left| f(R e^{i\theta}) R e^{i\theta} \right| \cdot |e^{i \alpha R e^{i\theta}}| d\theta \le M \int_{0}^{\pi} |e^{i \alpha R e^{i\theta}}| d\theta \\
			& = M \int_{0}^{\pi} \left| e^{i \alpha R (\cos(\theta) + i \sin(\theta))} \right| d\theta = M \int_{0}^{\pi} \left| e^{-\alpha R \sin(\theta)} \cdot e^{i \alpha R \cos(\theta)} \right| d\theta \\
			& = M \int_{0}^{\pi} e^{- \alpha R \sin(\theta)} d\theta = 2M \int_{0}^{\pi / 2} e^{- \alpha R \sin(\theta)} d\theta \\
			& \le 2M \int_{0}^{\pi / 2} e^{-\alpha R \cdot 2 \theta / \pi} d\theta = 2M \left[ -\frac{\pi}{2\alpha R} e^{-\alpha R \cdot 2 \theta \ pi} \right]_0^{\pi / 2} \\
			& = \frac{\pi M}{\alpha R} \left[ -e^{-\alpha R \cdot 2\theta / \pi} \right]_0^{\pi / 2} = \frac{\pi M}{\alpha R} \left[ 1 - e^{-\alpha R} \right] \\
			& \le \frac{\pi M}{\alpha R} \to 0 \text{ as } R \to \infty
		\end{aligned}
	\]
\end{proof}

\begin{example}
	(\textbf{Principal value integrals and Jordan's lemma}) Compute
	\[
		\int_{-\infty}^{\infty} \frac{x \sin(x)}{x^2 + 1} dx
	\]
	The integrand is even, so
	\[
		\int_{-\infty}^{\infty} \frac{x \sin(x)}{x^2 + 1} dx = \lim_{R \to \infty} \int_{-R}^{R} \frac{x \sin(x)}{x^2 + 1} dx
	\]
	Let $f(z) = z / (z^2 + 1)$. We will compute
	\[
		\int_{\gamma _R} f(z) e^{iz} dz
	\]
	where $L_R \cup C_R$ is $D$-shaped. We have
	\[
		\begin{aligned}
			\int_{L_R} f(z) e^{iz} dz & = \int_{L_R} f(z) (\cos(z) + i \sin(z)) dz \\
			& = \int_{-R}^{R} f(x) \cos(x) dx + i \int_{-R}^{R} f(x) \sin(x) dx
		\end{aligned}
	\]
	Hence
	\[
		\int_{-R}^{R} \frac{x \sin(x)}{x^2 + 1} dx = \Im\left( \int_{L_R} f(z) e^{iz} dz \right)
	\]
	Also,
	\[
		\int_{L_R} f(z) e^{iz} dz = \int_{\gamma_R} f(z) e^{iz} dz - \int_{C_R} f(z) e^{iz} dz
	\]
	and
	\[
		f(z) e^{iz} = \frac{z e^{iz}}{z^2 + 1} = \frac{z e^{iz}}{(z + i)(z - i)}
	\]
	So by Cauchy's residue theorem, and Proposition~\ref{prop:calculatingResidues} (2.),
	\[
		\begin{aligned}
			\int_{\gamma_R} f(z) e^{iz} dz & = 2 \pi i \cdot \Res_{z = i} (f(z) e^{iz}) = 2 \pi \\
			& = 2 \pi i \cdot \lim_{z \to i} (z - i) f(z) e^{iz} \\
			& = 2 \pi i \cdot \lim_{z \to i} \frac{z e^{iz}}{z + i} = \frac{i e^{-1}}{2i} \cdot 2 \pi i = \frac{i\pi}{e}
		\end{aligned}
	\]
	Thus
	\[
		\int_{-R}^{R} \frac{x \sin(x)}{x^2 + 1} dx = \frac{\pi}{e} - \Im\left( \int_{C_R} f(z) e^{iz} dz \right)
	\]
	By \hyperref[lem:jordansLemma]{Jordan's lemma} with $\alpha = 1$, for $|z| = R$,
	\[
		|z f(z)| = \left| \frac{z^2}{z^2 + 1} \right| \le \frac{|z|^2}{|z|^2 - 1} = \frac{R^2}{R^2 - 1} = \frac{1}{1 - 1/R^2}
	\]
	so for $r = \sqrt{2}$, $|z f(z)| \le 2$ hence
	\[
		\lim_{R \to \infty} \int_{C_R} f(z) e^{iz} dz = 0
	\]
	Finally,
	\[
		\int_{-\infty}^{\infty} \frac{x \sin(x)}{x^2 + 1} dx = \lim_{R \to \infty} \left( \frac{\pi}{e} - \Im \left(\int_{C_R} f(z) e^{iz} dz \right) \right) = \frac{\pi}{e}
	\]
\end{example}

\begin{remark}
	The same method works for
	\[
		\text{P.V. } \int_{-\infty}^{\infty} f(x) \sin(\alpha x) dx
	\]
	or
	\[
		\text{P.V. } \int_{-\infty}^{\infty} f(x) \cos(\alpha x) dx
	\]
	where $f$ is meromorphic satisfying Jordan's lemma, e.g. $f(x) = p(x) / q(x)$ where $p$ and $q$ are polynomials, $q$ has no real roots and $\deg(q) \le \deg(p) + 1$.
\end{remark}

\begin{lemma}\label{lem:indentationLemma}
	(\textbf{Indentation lemma}) Let $g$ be meromorphic on $\mathbb{C}$ with a simple pole at $z = 0$. Then
	\[
		\lim_{\epsilon \to 0} \int_{C_{\epsilon}} g(z) dz = \pi i \cdot \Res_{z = 0}(g)
	\]
	where $C_{\epsilon}(\theta) = \epsilon e^{i\theta}$, $\theta \in [0, \pi]$.
\end{lemma}

\begin{proof}
	Since $g$ has a simple pole at $z = 0$, it has a Laurent series
	\[
		g(z) = \sum_{n = -1}^{\infty} c_n z^n
	\]
	where $c_1 = \Res_{z = 0}(g)$. Note $g(z) = \frac{c_{-1}}{z} + h(z)$ where $h$ is holomorphic on $B_R(0)$, and
	\[
		\int_{C_{\epsilon}} g(z) dz = c_{-1} \int_{c_{\epsilon}} \frac{1}{z} dz + \int_{C_{\epsilon}} h(z) dz
	\]
	For every $r < R$, the function $h$ is holomorphic on the compact set $\overline{B_r}(0)$ so by Theorem 2.30 (lecture notes), $h$ attains a maximum on $\overline{B_r}(0)$ and so for some $M$, $|h(z)| \le M$ on $\overline{B_r}(0)$. Thus
	\[
		\left| \int_{C_{\epsilon}} h(z) dz \right| \le \epsilon \pi M \to 0 \text{ as } \epsilon \to 0
	\]
	Finally,
	\[
		\int_{C_{\epsilon}} \frac{1}{z} dz = \int_{0}^{\pi} \frac{1}{\epsilon e^{i\theta}} i e^{i\theta} d\theta = i \pi
	\]
	hence
	\[
		\lim_{\epsilon \to 0} \int_{C_{\epsilon}} g(z) dz = c_{-1} i \pi + 0 = \pi i \cdot \Res_{z = 0}(g)
	\]
\end{proof}

\begin{example}
	(\textbf{Use of indented contours}) Evaluate the integral
	\[
		\int_{0}^{\infty} \frac{\sin(x)}{x} dx
	\]
	Define $\gamma = \gamma_{R, \rho} = L_2 \cup (-C_{\rho}) \cup L_1 \cup C_R$ where $L_1$ is the line from $\rho$ to $R$, $L_2$ is the line from $-R$ to $-\rho$, $C_R$ is the semicircle from $-R$ to $R$ and $C_{\rho}$ is the semicircle from $-\rho$ to $\rho$.

	Note $g(z) := e^{iz} / z$ is holomorphic on $D_{\gamma}^{\text{int}} \cup \gamma$. By Theorem 8.15 (lecture notes),
	\[
		\int_{\gamma} g(z) dz = 0
	\]
	Note that
	\[
		\int_{\gamma} g(z) dz = \int_{L_2} g(z) dz - \int_{C_{\rho}} g(z) dz + \int_{L_1} g(z) dz + \int_{C_R} g(z) dz
	\]
	hence
	\[
		\begin{aligned}
			\int_{L_1} g(z) dz + \int_{L_2} g(z) dz & = \int_{C_{\rho}} g(z) dz - \int_{C_R} g(z) dz \\
			& = \int_{L_1} \frac{e^{iz}}{z} dz + \int_{L_2} \frac{e^{iz}}{z} dz \\
			& = \int_{\rho}^{R} \frac{e^{ix}}{x} dx + \int_{-R}^{-\rho} \frac{e^{ix}}{x} dx \\
			& = \int_{\rho}^{R} \frac{e^{ix}}{x} dx + \int_{\rho}^{R} -\frac{e^{-ix}}{x} dx \\
			& = \int_{\rho}^{R} \frac{e^{ix} - e^{-ix}}{x} dx \\
			& = 2i \int_{\rho}^{R} \frac{\sin(x)}{x} dx
		\end{aligned}
	\]
	Thus
	\[
		\int_{0}^{\infty} \frac{\sin(x)}{x} dx = \lim_{R \to \infty, \rho \to 0} \frac{1}{2i} \left( \int_{C_{\rho}} g(z) dz - \int_{C_R} g(z) dz \right)
	\]
	Now let $f(z) = 1/z$ so $g(z) = f(z) e^{iz}$. Clearly $|z f(z)| = 1$ is bounded outside any $\overline{B_r}(0)$ so by \hyperref[lem:jordansLemma]{Jordan's lemma},
	\[
		\lim_{R \to \infty} \int_{C_R} g(z) dz = 0
	\]
	By the \hyperref[lem:indentationLemma]{indentation lemma},
	\[
		\lim_{\rho \to 0} \int_{C_{\rho}} g(z) dz = \pi i \cdot \Res_{z = 0}(g) = \pi i \cdot \lim_{z \to 0} z g(z) = \pi i
	\]
	by Proposition~\ref{prop:calculatingResidues} (2). Thus
	\[
		\int_{0}^{\infty} \frac{\sin(x)}{x} dx = \frac{1}{2i} \pi i = \frac{\pi}{2}
	\]
\end{example}

\begin{example}
	Show that
	\[
		\int_{0}^{\infty} \frac{\log(x)}{{(x^2 + 4)}^2} dx = \frac{\pi}{32} (\log(2) - 1)
	\]
	Let $f(z) = \log(z) / {(z^2 + 4)}^2$, where $\log(z) = \log(|z|) + i \arg(z)$, where $-\pi / 2 < \arg(z) < 3\pi / 2$. Consider the indented contour $\gamma = \gamma_{R, \rho}$, then
	\[
		\int_{\gamma} f(z) dz = \int_{L_2} f(z) dz - \int_{C_{\rho}} f(z) dz + \int_{L_1} f(z) dz + \int_{C_R} f(z) dz
	\]
	Note
	\[
		\int_{L_1} f(z) dz = \int_{\rho}^{R} \frac{\log(x)}{{(x^2 + 4)}^2} dx
	\]
	If $x < 0$, then $x = |x| e^{i\pi}$, so $\log(x) = \log(|x|) + i \pi$. Hence
	\[
		\int_{L_2} f(z) dz = \int_{-\rho}^{-R} \frac{\log(x) + i \pi}{(x^2 + 4)^2} dx
	\]
	So
	\[
		\int_{L_1} f(z) dz + \int_{L_2} f(z) dz = 2 \int_{\rho}^{R} \frac{\log(x)}{{(x^2 + 4)}^2} dx + i \pi \int_{\rho}^{R} \frac{1}{(x^2 + 4)^2} dx
	\]
	We have
	\[
		\int_{0}^{\infty} \frac{\log(x)}{{(x^2 + 4)}^2} dx = \lim_{R \to \infty, \rho \to 0} \frac{1}{2} \Re \left( \int_{\gamma} f(z) - \int_{C_R} f(z) dz + \int_{C_{\rho}} f(z) dz \right)
	\]
	On $C_R$, we have
	\[
		|f(z)| = \left| \frac{\log(z)}{{(z^2 + 4)}^2} \right| \le \frac{\log(z) + |\arg(z)|}{||z|^2 - 4|^2} \le \frac{\log(R) + \pi}{{(R^2 - 4)}^2}
	\]
	so by the Estimation Lemma,
	\[
		\left| \int_{C_R} f(z) dz \right| \le L(C_R) \sup_{z \in C_R} |f(z)| \le \frac{\pi R (\log(R) + \pi)}{{(R^2 - 4)}^2} = \frac{\pi \log(R) / R + \pi^2 / R^2}{(R^2 - 4)^2} \to 0 \text{ as } R \to \infty
	\]
	Similarly, on $C_{\rho}$, we have
	\[
		|f(z)| \le \frac{|\log(|z|)| + |\arg(z)|}{||z|^2 - 4|^2} = \frac{-\log(\rho) + \pi}{{(\rho^2 - 4)}^2}
	\]
	Note that $\lim_{\rho \to 0} \rho \log(\rho) = \lim_{\rho \to 0} \frac{\log(\rho)}{1 / \rho} = \lim_{\rho \to 0} \frac{1/p}{-1/p^2} = 0$ by L'Hopital's rule. So by the Estimation lemma,
	\[
		\left| \int_{C_{\rho}} f(z) dz \right| \le \frac{\pi \rho \log(\rho) + \pi^2 \rho}{{(4 - \rho)}^2} \to 0 \text{ as } \rho \to 0
	\]
	By Cauchy's Residue theorem, we have
	\[
		\int_{\gamma} f(z) dz = 2 \pi i \cdot \Res_{z = 2i} (f) = 2\pi i \frac{g'(2i)}{(2 - 1)!} = \frac{\pi}{16} (\log(2) - 1) + \frac{i\pi^2}{32}
	\]
	where $g(z) = \log(z) / {(z + 2i)}^2$. So finally,
	\[
		\int_{0}^{\infty} \frac{\log(x)}{{(x^2 + 4)}^2} dx = \lim_{R \to \infty, \rho \to 0} \frac{1}{2} \Re \left( \int_{\gamma} f(z) - \int_{C_R} f(z) dz + \int_{C_{\rho}} f(z) dz \right) = \frac{\pi}{32} (\log(2) - 1)
	\]
\end{example}

\begin{example}
	(Problems class) Determine all poles of the following function and calculate the residue of each pole:
	\[
		f(z) = \frac{1}{z^2} + \frac{1}{z^2 + 1}
	\]
	We have
	\[
		f(z) = \frac{z^2 + 1 + z^2}{z^2 (z^2 + 1)} = \frac{2z^2 + 1}{z^2 (z + i)(z - i)}
	\]
	So $i$ and $-i$ are poles of order $1$ and $0$ is a pole of order $2$. $f(z) = 1/z^2 + h(z)$ where $h(z) = 1/(z^2 + 1)$ is holomorphic at $0$, so $h$ has a Taylor series at $z = 0$, so $\Res_{z = 0}(f) = 0$.

	For $z = i$, by the cover-up rule,
	\[
		\Res_{z = i}(f) = \lim_{z \to i} (z - i) f(z) = \frac{2i^2 + 1}{i^2 (2i)} = \frac{1}{2i}
	\]
	Similarly for $z = -i$.
\end{example}

\begin{example}
	(Problems class) Use Cauchy's theorem to evaluate the integral
	\[
		\int_{|z| = 8} \frac{z - 1}{e^z - 1} dz
	\]
	Let $f(z) = \frac{z - 1}{e^z - 1}$, then the reciprocal of $f$ has a zero when $e^z = 1 \Longleftrightarrow z = 2m \pi i$ for $m \in \mathbb{Z}$. So these are the poles of $f$. Their orders are $1$ since the derivative of $1 / f$ at $z = 2n\pi i$ is not zero. So we consider the poles $0, -2\pi i, 2 \pi i$ for Cauchy's residue theorem. By Proposition~\ref{prop:calculatingResidues} (3) with $g(z) = z - 1$ and $h(z) = e^z - 1$,
	\[
		\Res_{z = 2m\pi i} (f) = \frac{g(2m\pi i)}{h'(2m \pi i)} = 2m \pi i - 1
	\]
	So by Cauchy's residue theorem,
	\[
		\int_{|z| = 8} f(z) dz = 2 \pi i \cdot \sum_{m = -1}^{1} 2m \pi i = 2 \pi i ((-2\pi i - 1) + (0 - 1) + (2\pi i - 1)) = -6 \pi i
	\]
\end{example}

\begin{example}
	(Problems class) By converting into a complex contour integral, evaluate
	\[
		\int_{0}^{\pi} \frac{1}{4 + \sin(\theta)^2} d\theta
	\]
	We have $\sin(\theta) = \frac{e^{i\theta} - e^{i\theta}}{2i}$. So
	\[
		\begin{aligned}
			\int_{0}^{\pi} \frac{1}{4 + \sin(\theta)^2} d\theta & = \int_{0}^{\pi} \frac{1}{4 - {(e^{i\theta} - e^{-i\theta})}^2 / 4} d\theta \\
			& = \int_{0}^{\pi} \frac{1}{4 - \frac{1}{4} (e^{2i\theta} + e^{-2i\theta} - 2)} d\theta \\
			& = \int_{0}^{\pi} \frac{4e^{2i\theta}}{16e^{2i\theta} - e^{4i\theta} - 1 + 2e^{2i\theta}} d\theta \\
			& = 2i \int_{0}^{\pi} \frac{2i e^{2i\theta}}{e^{4i\theta} - 18e^{2i\theta} + 1} d\theta \\
			& = 2i \int_{0}^{\pi} \frac{\gamma'(\theta)}{\gamma(\theta)^2 - 18 \gamma(\theta) + 1} d\theta, \quad \gamma(\theta) = e^{2i\theta} \\
			& = 2i \int_{|z| = 1} \frac{1}{z^2 - 18z + 1} dz
		\end{aligned}
	\]
	There are poles at $z = 9 \pm 4\sqrt{5}$ which are both simple. $|9 - 4\sqrt{5}| < 1$ and $|9 + 4\sqrt{5}| > 1$ so we just need to calculate
	\[
		\Res_{z = 9 - 4\sqrt{5}} (f) = \frac{1}{2z - 18}\Big|_{z = 9 - 4\sqrt{5}} = -\frac{1}{8\sqrt{5}}
	\]
	Thus
	\[
		\int_{0}^{\pi} \frac{1}{4 \sin(\theta)^2} d\theta = 2i \cdot 2 \pi i \cdot -\frac{1}{8\sqrt{5}} = \frac{\pi}{2\sqrt{5}}
	\]
\end{example}

\begin{example}
	(Problems class) Evaluate
	\[
		\int_{0}^{\infty} f(x) dx = \int_{0}^{\infty} \frac{1}{(x^2 + a^2)(x^2 + b^2)} dx
	\]
	The integrand is even so
	\[
		\int_{0}^{\infty} f(x) dx = \lim_{R \to \infty} \int_{0}^{R} f(x) dx = \frac{1}{2} \lim_{R \to \infty} \int_{-R}^R f(x) dx
	\]
	Let $L_R$ be the line from $-R$ to $R$, so
	\[
		\int_{-R}^{R} f(x) dx = \int_{L_R} f(z) dz
	\]
	and $C_R$ be the semicircle from $R$ to $-R$ and let $\gamma_R = L_R \cup C_R$. So
	\[
		\int_{0}^{\infty} f(x) dx = \frac{1}{2} \lim_{R \to \infty} \left( \int_{\gamma_R} f(z) dz - \int_{C_R} f(z) dz \right)
	\]
	For every $z \in C_R$, $|z| = R$, so we have
	\[
		|f(z)| = \frac{1}{|z^2 + a^2| \cdot |z^2 + b^2|} \le \frac{1}{(|z|^2 - |a|^2)(|z|^2 - |b|^2)} = \frac{1}{(R^2 - |a|^2)(R^2 - |b|^2)}
	\]
	$f$ has poles at $ia, -ia, ib, -ib$, so hence when $R > \max(|a|, |b|)$, $\gamma_R$ contains the poles $ia$ and $ib$. By the estimation lemma,
	\[
		\begin{aligned}
			\left| \int_{C_R} f(z) dz \right| & \le \pi R \cdot \sup_{z \in C_R} |f(z)| \le \frac{\pi R}{(R^2 - |a|^2)(R^2 - |b|^2)} \to 0 \quad \text{as} \quad R \to \infty
		\end{aligned}
	\]
	By Proposition~\ref{prop:calculatingResidues} (2),
	\[
		\Res_{z = ai} (f) = \frac{1}{2ai (b^2 - a^2)}, \quad \Res_{z = bi} (f) = \frac{1}{2bi (a^2 - b^2)}
	\]
	So
	\[
		\int_{0}^{\infty} f(x) dx = \frac{1}{2} 2 \pi i \cdot (\Res_{z = ai} (f) + \Res_{z = bi} (f)) = \frac{\pi}{2ab(a + b)}
	\]
\end{example}

\subsection{The argument principle}

\begin{lemma}
	Let $f$ be meromorphic on a domain $D$ with a zero or a pole of order $k$ at $z = a$. Then
	\[
		\frac{f'(z)}{f(z)}
	\]
	has a simple pole at $z = a$ and
	\[
		\Res_{z = a} \left( \frac{f'(z)}{f(z)} \right) = \begin{cases}
			k & \text{ if f has a zero} \\
			-k & \text{ if f has a pole}
		\end{cases}
	\]
\end{lemma}

\begin{proof}
	By definition, if $f$ has a zero, for some $g$ holomorphic on $B_R(a)$,
	\[
		f(z) = {(z - a)}^k g(z), \quad g(a) \ne 0
	\]
	Then
	\[
		\frac{f'(z)}{f(z)} = \frac{k {(z - a)}^{k - 1} g(z) + {(z - a)}^k g'(z)}{{(z - a)}^k g(z)} = \frac{k}{z - a} + \frac{g'(z)}{g(z)}
	\]
	Similarly if $f$ has a pole, for some $g$ holomorphic on $B_R(a)$,
	\[
		f(z) = \frac{g(z)}{{(z - a)}^k}, \quad g(a) \ne 0
	\]
	Then
	\[
		\int_{f'(z)}^{f(z)} = \frac{g'(a) {(z - a)}^{-k} - k g(z) {(z - a)}^{-k - 1}}{g(z) {(z - a)}^{-k}} = \frac{-k}{z - a} + \frac{g'(z)}{g(z)}
	\]
	Since $g' / g$ is holomorphic at $z = a$, by definition of Laurent series, we arrive at the result.
\end{proof}

\begin{example}
	Let $f(z) = e^z / z$, so $f$ has a simple pole at $z = 0$. Then
	\[
		f'(z) = e^z / z - e^z / z^2 \Longrightarrow \frac{f'(z)}{f(z)} = \frac{-1}{z} + 1
	\]
	So $\Res_{z = 0}(f) = -1$.
\end{example}

\begin{remark}
	$f' / f$ has a pole if
	\begin{itemize}
		\item $f$ has a zero or a pole.
		\item $f'$ has a pole.
	\end{itemize}
	But $f'$ has a pole if and only if $f$ has a pole. So the only poles of $f' / f$ are at the zeros and poles of $f$.
\end{remark}

\begin{theorem}
	(\textbf{The argument principle}) Let $\gamma$ be a positively-oriented simple closed contour and let $f$ be meromorphic on and inside $\gamma$ ($D_{\gamma}^{\text{int}} \cup \gamma$). Then if $f$ has no poles or zeroes on $\gamma$, then
	\[
		\frac{1}{2 \pi i} \int_{\gamma} \frac{f'(z)}{f(z)} dz = Z_f - P_f
	\]
	where $Z_f$ is the number of zeros of $f$ and $P_f$ is the number of poles of $f$ in $D_{\gamma}^{\text{int}}$ where we count with multiplicities (so a zero/pole of order $3$ counts as $3$).
\end{theorem}

\begin{proof}
	Let $h(z) = f'(z) / f(z)$. Then $h$ is meromorphic and its pole are at the zeros/poles of $f$ which are all simple by Lemma 10.19 (lecture notes). So by Cauchy's Residue theorem,
	\[
		\int_{\gamma} h(z) dz = 2 \pi i \sum_{i = 1}^m \Res_{z = a_i} (h)
	\]
	where $a_i$ are the poles of $h$ inside $\gamma$. By Lemma 10.19 (lecture notes), those residues are
	\[
		\begin{cases}
			k_i & \text{ if } a_i \text{ was a zero of order } k_i \text{ of f} \\
			-k_i & \text{ if } a_i \text{ was a pole of order } k_i \text{ of f}
		\end{cases}
	\]
\end{proof}

\begin{example}
	Let
	\[
		f(z) = \frac{{(z - 3)}^3 {(z - 1)}^7 z^3}{{(z - i)}^4 {(z - 4)}^5 {(z - 3i)}^7}
	\]
	Calculate $\int_{\gamma} f'(z) / f(z) dz$ where $\gamma(\theta) = \frac{7}{2} e^{i \theta}$, $\theta \in [0, 2\pi]$.

	By the argument principle,
	\[
		\int_{\gamma} f'(z) / f(z) dz = 2 \pi i (Z_f - P_f) = 2 \pi i ((3 + 7 + 3) - (4 + 7)) = 4 \pi i
	\]
\end{example}

\begin{theorem}
	(\textbf{Rouche's theorem}) Let $\gamma$ be a simple closed contour and let $f$ and $g$ be holomorphic on $D_{\gamma}^{\text{int}} \cup \gamma$. If $\forall z \in \gamma$,
	\[
		|f(z) - g(z)| < |g(z)|
	\]
	then $f$ and $g$ have the same number of zeros inside $\gamma$ (counted with multiplicity).
\end{theorem}

\begin{proof}
	If $f(z_0) = 0$ for some $z_0 \in \gamma$, then $|g(z_0)| = |0 - g(z_0)| = |f(z_0) - g(z_0)| < |g(z_0)|$, so we have a contradiction. Similarly, if $g(z_0) = 0$ for some $z_0 \in \gamma$, then $|f(z_0)| = |f(z_0) - 0| = |f(z_0) - g(z_0)| < |g(z_0)| = 0$. Hence $f$ and $g$ have no zeros or poles on $\gamma$.

	Let $h(z) = f(z) / g(z)$. Since $f$ and $g$ are holomorphic and $g(z) \ne 0$ on $\gamma$, then $h(z)$ has no poles on $\gamma$ and since $f(z) \ne 0$ on $\gamma$, then $h(z)$ has no zeros on $\gamma$. Note that inside $\gamma$, all poles of $h$ must occur from zeros of $g$, which are isolated, by the Principle of Isolated Zeros. So $h$ is meromorphic.

	Let $Z_h$, $Z_f$, $Z_g$ be the number of zeros of $h$, $f$ and $g$ respectively. Then by the argument principle,
	\[
		\begin{aligned}
			Z_h & = \frac{1}{2\pi i} \int_{\gamma} \frac{h'(z)}{h(z)} dz \\
			& = \frac{1}{2\pi i} \int_{\gamma} \frac{f'(z) / g(z) - f(z) g'(z) / g(z)^2}{f(z) / g(z)} dz \\
			& = \frac{1}{2 \pi i} \frac{f'(z)}{f(z)} dz - \frac{1}{2 \pi i} \frac{g'(z)}{g(z)} dz = Z_f - Z_g
		\end{aligned}
	\]
	So we need to show $Z_h = 0$. By observation, $Z_h = I(h \circ \gamma; 0)$. We will show $\Gamma_h = h \circ \gamma$ lies in $B_1(1)$, i.e. for $z \in \gamma$, $|h(z) - 1| < 1$. We have
	\[
		|h(z) - 1| = \left| \frac{f(z)}{g(z)} - \frac{g(z)}{g(z)} \right| = \frac{|f(z) - g(z)|}{|g(z)|} < 1
	\]
	therefore $h \circ \gamma \subset B_1(1)$ and $Z_h = 0$.
\end{proof}

\begin{example}
	Let $P(z) = z^4 + 6z + \pi$. Let $g(z) = z^4$, then for $|z| = 2$,
	\[
		|g(z)| = |z^4| = |z|^4 = 16 > 12 + \pi = 6|z| + \pi \le |6z + \pi| = |P(z) - g(z)|
	\]
	Thus $P$ has the same number of roots as $z^4$ inside $|z| = 2$ by Rouche's theorem, i.e. $P$ has $4$ roots inside $|z| = 2$. Now let $g(z) = 6z$, then for $|z| = 1$,
	\[
		|g(z)| = |6z| = 6|z| = 6 > 1 + \pi = |z|^4 + \pi \ge |z^4 + \pi| = |P(z) - g(z)|
	\]
	$g$ has one zero inside $|z| = 1$ so $P$ has one zero  of size $|z| < 1$ and three zeros of size $1 < |z| < 2$.
\end{example}

\begin{example}
	Let $f(z) = \cos(\pi z) - \pi^{\pi} z^m$, $m \in \mathbb{Z}$.
\end{example}

\end{document}