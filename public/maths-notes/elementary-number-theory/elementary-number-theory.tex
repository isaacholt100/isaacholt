\documentclass[12pt,a4paper]{article}
\AddToHook{cmd/section/before}{\clearpage}

\usepackage[a4paper, total={6in, 10in}]{geometry}
\usepackage[utf8]{inputenc}
\usepackage{amsfonts, amsmath, amssymb, amsthm}

\theoremstyle{definition}
\newtheorem{definition}{Definition}[subsection]
\newtheorem{theorem}[definition]{Theorem}
\newtheorem{proposition}[definition]{Proposition}
\newtheorem{corollary}[definition]{Corollary}
\newtheorem{lemma}[definition]{Lemma}
\newtheorem{example}[definition]{Example}
\newtheorem*{remark}{Remark}

\title{Elementary Number Theory Course Notes}
\author{Isaac Holt}

\begin{document}

\maketitle

\section{Quadratic Residues and Non-Residues}

Consider the equation $x^2 \equiv a \pmod{p}$.

\begin{definition}
	Let $a \in \mathbb{Z}$, $p$ be an odd prime, $p \not | a$. $a \pmod{p}$ is a \textbf{quadratic residue (QR) mod $p$} if for some $x \in \mathbb{Z}$, $x^2 \equiv a \pmod{p}$.

	If there doesn't exist such an $x$, $a \pmod{p}$ is a \textbf{quadratic non-residue (NQR)}.
\end{definition}

\begin{lemma}
	For $p$ an odd prime, there are $\frac{p - 1}{2}$ QRs and $\frac{p - 1}{2}$ NQRs.
\end{lemma}

\begin{proof}
	Define the map $f: \{1, \dots, \frac{p - 1}{2}\} \rightarrow Q$, $f(x) := x^2 \pmod{p}$, where $Q := \{x^2 \pmod{p}\}$ is the set of all QRs.

	$f$ is clearly surjective, since $\{x^2 \pmod{p}: 1 \le x \le p - 1\} = \{x^2 \pmod{p}: 1 \le x \le \frac{p - 1}{2}\}$, since if $\frac{p + 1}{2} \le x \le p - 1$, $-x \pmod{p} \in \{1, \dots, \frac{p - 1}{2}\}$ and $x^2 \equiv (-x)^2 \pmod{p}$.

	Suppose that $f(a) = f(b)$, so $a^2 \equiv b^2 \pmod{p} \Rightarrow (a - b)(a + b) \equiv \pmod{p}$. $2 \le a + b \le p - 1$ so $a + b \not\equiv 0 \pmod{p}$, hence $a \equiv b \pmod{p} \Rightarrow a = b$.

	So $f$ surjective and injective so is bijective, so $|Q| = \frac{p - 1}{2}$. The remaining $\frac{p - 1}{2}$ elements are the NQRs.
\end{proof}

\begin{lemma}
	Let $a \in \mathbb{Z}$, $a \in \mathbb{Z}$, $p$ be an odd prime, $p \not| ab$. Let $Q$ denote the QRs mod $p$ and $N$ denote the NQRs mod $p$.

	\begin{enumerate}
		\item If $a \in Q$ and $b \in Q$ then $ab \in Q$.
		\item If $a \in Q$ and $b \in N$, then $ab \in N$.
		\item If $a \in N$ and $b \in N$, then $ab \in Q$.
	\end{enumerate}
\end{lemma}

\begin{proof}
	\hfill
	\begin{enumerate}
		\item If $a \in Q$ and $b \in Q$, for some $x \in \mathbb{Z}$ and $y \in \mathbb{Z}$, $x^2 \equiv a \pmod{p}$ and $y^2 \equiv b \pmod{p}$, so $ab \equiv x^2 y^2 \pmod{p} \equiv (xy)^2 \pmod{p}$ so for some $z$, $z^2 \equiv ab \pmod{p}$ ($z = xy$). So $ab \in Q$.
		\item Suppose $ab \notin N$, for $a \in Q$, $ \in N$. Since $ab \not\equiv 0 \pmod{p}$, $ab \in Q$. So for some $w \in \mathbb{Z}$, $ab \equiv w^2 \pmod{p}$. Since $a \in Q$, for some $t \in \mathbb{Z}$, $a \equiv t^2 \pmod{p}$ so $t^2 b \equiv w^2 \pmod{p}$. Cancelling $t^2$ on both sides, $b \equiv w^2 \bar{t^2} \pmod{p} \equiv (w \bar{t})^2 \pmod{p}$. But $b \in N$, so we have a contradiction.
		\item We write $a^{-1} \cdot Q := \{1 \le b \le p - 1: a \cdot b \in Q\} = \{a^{-1}x: x \in Q$ ($a^{-1}$ is such that $a^{-1}a \equiv 1 \pmod{p}$).
		
		As $a \in N$, $a^{-1} \in N$ (if $a^{-1} \in Q$ then as $a \in N$, 2. implies that $a^{-1}a \equiv 1 \in N \pmod{p}$ which is not true since $1 \equiv 1^2 \pmod{p}$).

		Thus for every $x \in Q$, $a^{-1}x \in N \Rightarrow a^{-1}Q \subseteq N$.
		
		$a^{-1}x \equiv a^{-1}y \pmod{p} \Rightarrow x \equiv y \pmod{p}$. AS $1 \le x, y \le p - 1$, $x = y$. Thus, the map $Q \rightarrow a^{-1}Q$ given by $x \rightarrow a^{-1}x$ is injective and bijective.

		Therefore $|a^{-1}Q| = |Q| = |N| \Rightarrow a^{-1}Q = N$ so if $b \in N$, $b \in a^{-1}Q$ so $ab \in Q$.
	\end{enumerate}
\end{proof}

\begin{definition}
	Let $p$ be an odd prime. The \textbf{Legendre symbol} written as $(\frac{a}{p})$ is defined for $a \in \mathbb{Z}$ as

	\begin{equation}
		(\frac{a}{p}) :=
		\begin{cases}
			0 & \text{if } p | a\\
			1 & \text{if } p \in Q\\
			-1 & \text{if } p \in N
		\end{cases}
	\end{equation}
\end{definition}

Properties of the Legendre symbol:
\begin{itemize}
	\item (multiplicativity): if $a, b \in \mathbb{Z}$ then
	\[(\frac{ab}{p}) = (\frac{a}{p})(\frac{b}{p})\]
	\item (periodicity mod $p$): if $a \equiv b \pmod{p}$ then
	\[(\frac{a}{p}) = (\frac{b}{p})\]
\end{itemize}

\begin{theorem}
	(Euler's criterion): if $p$ is an odd prime and $a \in \mathbb{Z}$ with $p \not | a$ then

	\[a^{\frac{p - 1}{2}} \equiv (\frac{a}{p}) \pmod{p}\]
\end{theorem}

\begin{proof}
	Let $g$ be a primitive root mod $p$.
	
	$\{g^r \pmod{p}: 1 \le r \le p - 1\} = {1, \dots, p - 1} \Rightarrow \{g^{2r}: 1 \le r \le \frac{p - 1}{2}\}$ gives the QRs uniquely. There are the following cases:

	\begin{enumerate}
		\item $a$ is a QR. Then for some $1 \le r \le \frac{p - 1}{2}$, $g^{2r} \equiv a \pmod{p}$. Then
		
		\[a^{\frac{p - 1}{2}} \equiv (g^{2r})^{\frac{p - 1}{2}} \equiv (g^r)^{p - 1} \equiv (g^{p - 1})^r \equiv 1^r \equiv 1 \equiv (\frac{a}{p}) \pmod{p}\]
		\item $a$ is not a QR. Then for some $1 \le r \le \frac{p - 1}{2}$, $a \equiv g^{2r - 1} \pmod{p}$. So $a^{\frac{p - 1}{2}} \equiv (g^{2r})^{\frac{p - 1}{2}} g^{\frac{p - 1}{2}}$.
		
		But $x = g^{-\frac{p - 1}{2}} \equiv -1 \pmod{p}$, since $x^2 \equiv 1 \pmod{p} \Rightarrow x \equiv \pm 1 \pmod{p}$ and since $g$ is primitive, $x \not\equiv 1 \pmod{p}$.

		So $a^{\frac{p - 1}{2}} \equiv -1 \pmod{p} \equiv (\frac{a}{p}) \pmod{p}$
	\end{enumerate}
\end{proof}

\begin{remark}
	Euler's crtierion is hard to use if $p$ is large.
\end{remark}

\begin{corollary}
	$-1$ is a QR mod $p$ iff $p \equiv 1 \pmod{4}$.
\end{corollary}

\begin{proof}
	$(-1)^{\frac{p - 1}{2}} \equiv (\frac{-1}{p}) \pmod{p}$ by Euler's criterion.
	The power $\frac{p - 1}{2}$ is even iff $p \equiv 1 \pmod{4} \Rightarrow (-1)^{\frac{p - 1}{2}} = 1$ iff $p \equiv 1 \pmod{4}$.
\end{proof}

\begin{theorem}
	(Law of quadratic reciprocity - QRL): Let $p, q$ be distinct odd primes. Then
	
	\[(\frac{p}{q}) (\frac{q}{p}) = (-1)^{\frac{p - 1}{2} \cdot \frac{q - 1}{2}}\]
\end{theorem}

\begin{proof}
	TODO
\end{proof}

\begin{corollary}
	\[(\frac{2}{p}) = (-1)^{\frac{p^2 - 1}{8}}\]
\end{corollary}

\subsection{Algorithm for computing $(\frac{a}{p})$}

$p$ is an odd prime, $a \in \mathbb{Z}$. TODO: make this clearer.

\begin{enumerate}
	\item Use the division algorithm to divide $a = kp + r$, $0 \le r \le p - 1$, hence $(\frac{a}{p}) = (\frac{r}{p})$.
	\item If $r = 0$ or $r = 1$, $(\frac{0}{p}) = 0$, $(\frac{1}{p}) = 1$ so we are done.
	\item If $r \ne 0$ and $r \ne 1$, factor $r = {p_1}^{a_1} \dots {p_k}^{a_k}$, then $(\frac{r}{p}) = (\frac{p_1}{p})^{a_1} \dots (\frac{p_k}{p})^{a_k}$
	\item If $2 | a_i$, then $(\frac{p_i}{p})^{a_i} = 1$.
	\item If $2 \not | a_i$, $(\frac{p_i}{p})^{a_i} = (\frac{p_i}{p})$
	\item If $p_i = 2$, use the above corollary: $(\frac{2}{p}) = (-1)^{\frac{p^2 - 1}{8}}$.
	\item If $p_i \ne 2$, use QRL to write $(\frac{p_i}{p}) = (\frac{p}{p_i}) (-1)^{\frac{p - 1}{2} \cdot \frac{q - 1}{2}}$ and go to step 1 to calculate $(\frac{p}{p_i})$
\end{enumerate}

\subsection{Application of Legendre Symbols}

\begin{theorem}
	There are infinitely many primes of the form $4n + 1$.
\end{theorem}

\begin{proof}
	Assume the contrary, so let $p_1 < \dots < p_k$ be a finite list of primes, with $p_i \equiv 1 \pmod{4}$ for every $i$.

	Let $N = (2p_1 \dots p_k)^2 + 1$. Since $N > 1$, for some prime $p$, $p | N$. $p \ne p_i$ for every $i$. $N \equiv 0 \pmod{p}$, hence $(2p_1 \dots p_k)^2 \equiv -1 \pmod{p}$. Thus $-1$ is a QR mod $p$.

	By Euler's criterion, $(-1)^{\frac{p - 1}{2}} \equiv 1 \pmod{p}$, so $p \equiv 1 \pmod{4}$.

	But $p \notin \{p_1, \dots, p_k\}$ and $p \equiv 1 \pmod{4}$ so we have a contradiction.
\end{proof}

\break

\section{Sums of two squares}

\subsection{Sums of two squares}

Given $n \in \mathbb{N}_0$, can we represent $n$ as a sum of two squares, i.e. do there exist $a, b \in \mathbb{Z}$ such that $a^2 + b^2 = n$.

Equivalently, find solutions $x, y \in \mathbb{Z}$ to the equation

\[x^2 + y^2 = n\]

\begin{lemma}
	If $n, m$ are both sums of two squares, so is $n \cdot m$.
\end{lemma}

\begin{proof}
	Let $n = a^2 + b^2$, $m = c^2 + d^2$, $a, b, c, d \in \mathbb{Z}$. Then $nm = (a^2 + b^2)(c^2 + d^2) = (a^2 c^2 + b^2 d^2) + (b^2 c^2 + a^2 d^2) = (ac + bd)^2 - b^2 c^2 + a^2 d^2 - 2acbd = (ac + bd)^2 - (ad - bc)^2$
\end{proof}

\begin{corollary}
	If $n = {p_1}^{e_1} \cdots {p_k}^{e_k}$ and all the powers ${p_i}^{e_i}$ are sums of two squares then $n$ is also.
\end{corollary}

We focus on prime powers: $n = p^a$.

If $a = 2b$, $b \in \mathbb{N}$, then $n = p^{2b} = (p^b)^2 = (p^b)^2 + 0^2$ so $n$ is a sum of two squares.

If $a = 2b + 1$, $n = (p^b)^2 \cdot p$.

If $n = p$ is a prime, is $n$ a sum of two squares.

\begin{theorem}
	A prime $p$ is a sum of two squares iff either $p = 2$ or $p \equiv 1 \pmod{4}$.
\end{theorem}

\begin{proof}
	($\Rightarrow$):
	For every $n$, $n^2 \equiv 0 \text{ or } 1 \pmod{4}$

	Therefore if $p = x^2 + y^2$, $p = x^2 + y^2 \text{ mod } 4 \in \{0, 1, 2\}$. The only $p$ equivalent to $0$ or $2 \pmod{4}$ is $p = 2$, otherwise, $p \equiv 1 \pmod{4}$.

	($\Leftarrow$):
	Suppose $p = 2$ or $p \equiv 1 \pmod{4}$. If $p = 2$, $p = 1^2 + 1^2$. If $p \equiv 1 \pmod{4}$, $\left(\frac{-1}{p}\right) = 1$, so we can solve $u^2 + 1 \equiv 0 \pmod{4}$, $1 \le u \le \frac{p - 1}{2}$. We will find small $A, B \in \mathbb{N}_0$ suvh that $A^2 + B^2 \equiv 0 \pmod{p}$ using $u$. If $0 < A^2 + B^2 < 2p$, $A^2 + B^2 = p$.

	Let $k = \text{floor}(\sqrt{p})$, so $k \in \mathbb{N}$ and $k < \sqrt{p} < k + 1$. Consider the set $\{a + b \cdot u \pmod{p}: 0 \le a, b \le k\}$. There are $(k + 1)^2$ pairs $(a, b)$. Since $(k + 1)^2 > (\sqrt{p})^2 = p$. By the pigeon-hole principle, we can find two pairs $(a_1, b_1) \ne (a_2, b_2)$ such that $a_1 + b_1 u \equiv a_2 + b_2 u \pmod{p}$.

	So $(b_2 - b_1) u \equiv a_1 - a_2 \pmod{p} \Rightarrow B u \equiv \pm A \pmod{p}$ where $B = |b_2 - b_1| \le k < \sqrt{p}$, $A = |a_1 - a_2| \le k < \sqrt{p}$ and at least one of $A$ and $B$ is $> 0$.

	So $A^2 + B^2 \equiv (B u)^2 + B^2 \equiv B^2 (u^2 + 1) \equiv 0 \pmod{p}$

	Since at least one of $A$ and $B$ is $> 0$, $A^2 + B^2 > 0$. Since $A, B < \sqrt{p}$, $A^2 + B^2 < 2p$. Also, $p | (A^2 + B^2)$, hence $A^2 + B^2 = p$.
\end{proof}

\begin{corollary}
	A positive integer $n > 1$ written as $n = m^2 p_1 \cdots p_k$, with $p_1, \cdots p_k$ distinct primes ($n$ can always be written in this way) is a sum of two squares iff for every $p_i$ either $p_i = 2$ or $p_i \equiv 1 \pmod{4}$.
\end{corollary}

\begin{remark}
	There is a theorem due to Lagrange that says that every $n \in \mathbb{N}_0$ can be represented as the sum of four squares.
\end{remark}

\section{Continued Fractions}

\subsection{Pell equations}

\begin{definition}
	A \textbf{Pell equation} is an equation of the form $x^2 - dy^2 = \pm 1$, where $d \ge 1$ is not a square.
\end{definition}

\begin{remark}
	If $x, y \ne 0$ and both are large, then as $(x - \sqrt{d}y)(x + \sqrt{d}y) = x^2 - dy^2 = \pm 1$,
	
	\[ \left| \frac{x}{y} - \sqrt{d} \right| \left| \frac{x}{y} + \sqrt{d} \right| = \left| \left( \frac{x}{y} \right)^2 - d \right| = \frac{1}{y^2} \]
	So if $x^2 - dy^2 = \pm 1$ has a solution $(x, y) \in \mathbb{N}_0^2$, then $\frac{x}{y}$ approximates $\pm \sqrt{d}$.
\end{remark}

\subsection{Continued fractions}

\begin{definition}
	A \textbf{finite continued fraction (finite CF)} is an expression of the form
	\[ [a_0; a_1, \ldots, a_n] = a_0 + \frac{1}{a_1 + \frac{1}{a_n}} \]
	where $a_j \in \mathbb{R}$, $n \ge 0$.

	Mostly, $a_0 \in \mathbb{Z}$ and $a_1, \ldots, a_n \in \mathbb{N}$. In this case, $[a_0, \ldots, a_n]$ is called an \textbf{ellipse}.
\end{definition}

\begin{proposition}
	Any $\frac{a}{b} \in \mathbb{Q}$ can be expressed as a finite CF.
\end{proposition}

\begin{proof}
	(Not a full proof). Suppose for simplicity that $a \ge b$ (if not, take $a_0 = 0$). By the division algorithm, $a = a_0 b + r_1$, $0 \le r_1 < b$ hence $\frac{a}{b} = a_0 + \frac{r_1}{b} = a_0 + \frac{1}{b / r_1}$.

	Now divide $b$ by $r_1$: $b = a_1 r_1 + r_2$, $0 \le r_2 < r_1$, so $\frac{b}{r_1} = a_1 + \frac{r_2}{r_1}$ so
	
	\[ \frac{a}{b} = a_0 + \frac{1}{a_1 + \frac{1}{r_1 / r_2}} \]

	We continue with this: $r_i = a_{i + 1} r_{i + 1} + r_{i + 2}$ until $r_{i + 1}$ divides $r_1$ (i.e. $r_{i + 2} = 0$). This must occur as $0 \le r_{i + 1} < r_i$.

	The continued fraction is $[a_0; a_1, \ldots, a_n]$ where $r_{n + 1} = 0$.
\end{proof}

\begin{definition}
	Given a finite CF $\alpha = [a_0; a_1, \ldots, a_n]$, the $a_i$ are called \textbf{partial quotients} of $\alpha$.

	The truncated CF's $[a_0; a_1, \ldots a_j] = \frac{p_j}{q_j}$, with $0 \le j \le n$, $p_j \in \mathbb{Z}$, $q_j \in \mathbb{N}$, are called the \textbf{convergents} of $\alpha$.

	For $j = 0, j = 1$ we have $\frac{p_0}{q_0} = [a_0] = a_0 \Rightarrow p_0 = a_0, q_0 = 1$.

	$\frac{p_1}{q_1} = a_0 + \frac{1}{a_1} = \frac{a_1 a_0 + 1}{a_1} \Rightarrow p_1 = q_1 a_0 + 1, q_1 = a_1$.
\end{definition}

\begin{proposition}
	Given a finite CF, $[a_0; a_1, \ldots, a_n]$, $n \ge 1$, $[[p_k, p_{k - 1}], [q_k, q_{k - 1}]] = [[a_1, 1], [1, 0]] \cdots [[a_k, 1], [1, 0]]$ TODO: make these matrices.

	Hence $p_0 = a_0$, $q_0 = 1$, $p_1 = a_0 a_1 + 1$, $q_1 = a_1$, $p_k = a_k p_{k - 1} + p_{k - 2}$, $q_k = a_k q_{k - 1} + q_{k - 2}$.
\end{proposition}

\begin{lemma}
	Let $\alpha = [a_0; a_1, \ldots, a_n]$ be a finite CF with convergents $\frac{p_k}{q_k}$, $0 \le k \le n$.

	For every $k \ge 0$, $q_{k + 1} \ge q_k$ and if $k \ge 1$ then $q_{k + 1} > q_k$.
\end{lemma}

\begin{proof}
	If $k = 0$, $q_1 = a_1 \ge 1 = q_0$. Inductively, if $q_{k - 1} > 0$ for $k \ge 1$ then $q_{k + 1} = a_{k + 1} q_k + q_{k - 1} \ge a_{k + 1} q_k \ge q_k$ since $a_{k + 1} \ge 1$.
\end{proof}

\begin{lemma}
	For every $k \ge 1$, $p_k q_{k - 1} - p_{k - 1} q_k = {(-1)}^{k + 1}$.
\end{lemma}

\begin{proof}
	By the previous proposition,
	
	\[ [[p_k, p_{k - 1}], [q_k, q_{k - 1}]] = [[a_1, 1], [1, 0]] \cdots [[a_k, 1], [1, 0]] \]
	
	$p_k q_{k - 1} - q_k p_{k - 1} = \det [[p_k, p_{k - 1}], [q_k, q_{k - 1}]] = \det [[a_1, 1], [1, 0]] \cdots \det [[a_k, 1], [1, 0]] = {(-1)}^{k + 1}$.
\end{proof}

\begin{corollary}
	$\frac{p_k}{q_k} - \frac{p_{k - 1}}{q_{k - 1}} = \frac{p_k q_{k - 1} - p_{k - 1} q_k}{q_k q_{k - 1}} = \frac{{(-1)}^{k + 1}}{q_k q_{k - 1}}$

	So the convergents get closer as $k$ increases.
\end{corollary}

\begin{proposition}
	The even-numbered convergents are growing: $\frac{p_0}{q_0} < \frac{p_2}{q_2} < \cdots$ and the odd-numbered convergents are decreasing: $\frac{p_1}{q_1} > \frac{p_3}{q_3} > \cdots$.

	Moreover, for every $k \ge 1$ such that $2k + 1 \le n$,
	
	\[ \frac{p_{2k}}{q_{2k}} \le \alpha \le \frac{p_{2k + 1}}{q_{2k + 1}} \]
	and

	\[ \left| \alpha - \frac{p_m}{q_m} \right| \le \frac{1}{q_m q_{m - 1}} \]
	for every $m \le n - 1$.
\end{proposition}

\begin{proof}
	TODO
\end{proof}

\begin{definition}
	In general, if $\alpha \in \mathbb{R}$ (not necessarily rational), for $j > 0$:
	\begin{enumerate}
		\item $a_j := \text{floor}(\alpha_j)$ where $ \{ a_j \} := \alpha_j - a_j$
		\item Define $\alpha_{j + 1} := \frac{1}{\{\alpha_j\}}$. ($\alpha_0 = \alpha$)
	\end{enumerate}

	The continued fraction for $\alpha$ is $[a_0; a_1, a_2, \ldots]$.

	This could continue indefinitely if $a \notin \mathbb{Q}$.
\end{definition}

\begin{definition}
	An \textbf{infinite CF} is the limit, if it exists, of a sequence of finite CF's: $\{[a_0; a_1, \ldots, a_n]\}_{n \ge 0}$ given a $\{a_i\}_{i \ge 0}$ with $\forall i, a_i \ge 1$.
\end{definition}

\begin{proposition}
	If $a_0 \in \mathbb{Z}$ and $\forall i ge 1, a_i \in \mathbb{N}$, then $\{[a_0; a_1, \ldots, a_n]\}_{n \ge 0} \subset \mathbb{Q}$ converges.
\end{proposition}

\begin{proof}
	Use the Cauchy criterion: $[a_0; a_1, \ldots, a_n] = \frac{p_n}{q_n}$ are the convergents. $\forall m \ge 1, q_{m + 1} > q_m$, $q_m \in \mathbb{N}$. Let $\alpha_n = \frac{p_n}{q_n}$. If $m \le n$,

	\[ \left| \alpha_n - \frac{p_m}{q_m} \right| \le \frac{1}{q_m q_{m + 1}} \]

	Let $\epsilon > 0$. Then for some $N$, if $m \ge N$, $q_{m + 1} > q_m > \frac{1}{\sqrt{2}}$. Then with $n \ge m \ge N$,

	\[ \left| \frac{p_n}{q_n} - \frac{p_m}{q_m} \right| \le \frac{1}{q_m q_{m + 1}} < \sqrt{\epsilon} \sqrt{\epsilon} = \epsilon \]

	Thus $\{ \frac{p_k}{q_k} \}_k$ is a Cauchy sequence.
\end{proof}

\begin{definition}
	An infinite CF $\alpha = [a_0; a_1, a_2, \ldots]$ is (eventually) periodic if for some $m \in \mathbb{N}_0$ and $k \ge 1$, if $n > m$, $\forall j \in \mathbb{N}_0$, $a_{n + jk} = a_n$. That is,

	\[ \alpha = [a_0; a_1, \ldots, a_m, a_{m + 1}, \ldots, a_{m + k}, \ldots] = [a_0; \ldots, a_m, \overline{a_{m + 1}, \ldots, a_{m + k}}] \]
	$k$ is the \textbf{period} of the CF of $\alpha$.
\end{definition}

\begin{lemma}
	If $d \in \mathbb{N}$, $d$ is not a square, the CF of $\sqrt{d}$ is eventually periodic with initial part of length 1.
\end{lemma}

\begin{theorem}
	The eventually periodic $\alpha \notin \mathbb{Q}$ are preicsely of the form $a + b \sqrt{d}$, $a, b \in \mathbb{Q}$, $d \in \mathbb{N}$, $d$ is not a square.
\end{theorem}

\begin{example}
	Find simplified expression for $\alpha = [1; 3, \overline{4, 2}] = [1; 3, \beta]$.

	$\beta = [4; 2, \beta]$ so

	\[ \beta = 4 + \frac{1}{2 + \frac{1}{\beta}} = 4 + \frac{\beta}{2 \beta + 1} \]
	so simplifying, we get $ 2\beta^2 - 8\beta - 4 = 0 \Leftrightarrow \beta^2 - 4\beta - 2 = 0$, which has a positive root $2 + \sqrt{6}$ ($\beta$ must be positive).

	This can be used to simplify the expression for $\alpha$.
\end{example}

\subsection{Application to Pell Equations}

\begin{theorem}
	Let $x^2 - d y^2 = \pm 1$, $d \in \mathbb{N}$, $d$ is not a square. Suppose the CF of $\sqrt{d}$ has period $k$.

	If $\{\frac{p_m}{q_m}\}_{m \ge 0}$ are the convergents of $\sqrt{d}$. Then for every $n \in \mathbb{N}$,

	\[ p_{kn - 1}^2 - d q_{kn - 1}^2 = {(-1)}^{k n} \]
	In particular, if $k$ is even then $x^2 - d y^2 = 1$ has an infinite collection of solutions

	\[ (x, y) = (p_{kn - 1}, q_{kn - 1}), n \in \mathbb{N}_0 \]
	If $k$ is odd then $x^2 - d y^2 = -1$ has soltuions

	\[ (x, y) = (p_{(2n - 1)k - 1}, q_{(2n - 1)k - 1}) \]
	and $x^2 - dy^2 = 1$ has soltuions

	\[ (x, y) = (p_{2kn - 1}, q_{2kn - 1}) \]
\end{theorem}

\end{document}