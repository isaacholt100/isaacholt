\documentclass[12pt,a4paper]{article}
\AddToHook{cmd/section/before}{\clearpage}

\usepackage[a4paper, total={6in, 10in}]{geometry}
\usepackage[utf8]{inputenc}
\usepackage{amsfonts}
\usepackage{amsthm}

\theoremstyle{definition}
\newtheorem{definition}{Definition}[subsection]
\newtheorem{theorem}[definition]{Theorem}
\newtheorem{proposition}[definition]{Proposition}
\newtheorem{corollary}[definition]{Corollary}
\newtheorem{lemma}[definition]{Lemma}
\newtheorem{example}[definition]{Example}
\newtheorem*{remark}{Remark}

\title{<module> Course Notes}
\author{Isaac Holt}

\begin{document}

\section{Hamiltonian Formalism}

\begin{definition}
	The classical \textbf{state} of a system at a given instant in time is a \textbf{complete} set of data that fully specifies the future evolution of the system.
\end{definition}

\begin{remark}
	\textbf{Any} set of data that fully fixes future evolution is valid.
\end{remark}

\begin{definition}
	The \textbf{phase (or state) space} is the set of all possible states for a system at a given time.
\end{definition}

\begin{example}
	A free particle moving in $\mathbb{R}$. The phase space is $\mathbb{R}^2$ ($\mathbb{R}$ for position, $\mathbb{R}$ for velocity).
\end{example}

\begin{definition}
	The \textbf{Hamiltonian formalism} studies dynamics in a phase space, parameterised by $\underline{q}(t)$ and $\underline{p}(t)$, where $p_i = \frac{\partial L}{\partial \dot{\underline{q_i}}}$, the momentum.
\end{definition}

\begin{example}
	A particle moving in $\mathbb{R}$, with $L(x, \dot{x}) = \frac{1}{2} m \dot{x}^2$.

	Then $p_x = \frac{\partial L}{\partial \dot{x}} = m \dot{x}$ so $\dot{x}(x, p_x) = \frac{p_x}{m}$.

	In the Hamltonian formalism, $L(x, p_x) = \frac{p_x^2}{2m}$.
\end{example}

\begin{example}
	A particle moving in $\mathbb{R}^2$ (in polar coordinates).

	$L(r, \theta, \dot{r}, \dot{\theta}) = \frac{1}{2} m (\dot{r}^2 + r^2 \dot{\theta}^2)$. So $p_r = m\dot{r}$ and $p_{\theta} = m r^2 \dot{\theta}$.

	So $\dot{r}(r, \theta, p_r, p_{\theta}) = \frac{p_r}{m}$, $\dot{\theta}(r, \theta, p_r, p_{\theta}) = \frac{p_{\theta}}{m r^2}$.

	$L(r, \theta, \dot{r}, \dot{\theta}) = L(r, \theta, p_r, p_{\theta}) = \frac{1}{2} (\frac{p_r^2}{m} + \frac{p_{\theta}^2}{m r^2})$.
\end{example}

\begin{definition}
	Given two functions $f(\underline{q}, \underline{p}, t)$ and $g(\underline{q}, \underline{p}, t)$ in phase space their \textbf{Poisson bracket} is:

	\[ \{f, g\} := \sum_{i = 1}^n \left( \frac{\partial f}{\partial q_i} \frac{\partial g}{\partial p_i} - \frac{\partial f}{\partial p_i} \frac{\partial g}{\partial q_i} \right)\] where $n$ is the dimension of the configuration space.
\end{definition}

\begin{remark}
	In the Hamiltonian formalism, $\frac{\partial q_i}{\partial p_j} = \frac{\partial p_j}{\partial q_i} = 0$.

	Similarly, $\frac{\partial q_i}{\partial q_j} = \frac{\partial p_i}{\partial p_j} = \delta_{i, j}$
\end{remark}

\begin{example}
	Let $f = q_i$, $g = q_j$. $\{q_i, q_j\} = 0$, and $\{p_i, p_j\} = 0$. $\{q_i, p_j\} = \sum_{k = 1}^n \delta_{i, j} \delta_{j, k} = \delta_{i, j}$.
\end{example}

\begin{definition}
	Let $\mathbb{F}$ be the set functions from a phase space $P$ to $\mathbb{R}$
\end{definition}

\begin{definition}
	The Hamiltonian flow $\Phi_f^{(s)}$, with $(s) \in \mathbb{R}$, $f \in F$ operator maps $\mathbb{F}$ to $\mathbb{F}$ and is defined as

	\[ \Phi_f^{(s)} (g) := e^{s \{\cdot, f\}} g := g + s \{g, f\} + \frac{s^2}{2} \{ \{g, f\}, f\} + \cdots \]
\end{definition}

\begin{remark}
	The transformation generated by $f$ has generator $a_i = \{q_i, f\}$ where $q_i \rightarrow q_i + \epsilon a_i$.

	Infinitesimally, $\Phi_f^{(s)} (g) := g + \epsilon \{g, f\} + O(\epsilon^2)$
\end{remark}

TODO: properties on poisson bracket

\begin{example}
	(Rotation in $\mathbb{R}^2$ in Cartesian coordinates) As a guess, choose $f = q_1 \dot{q_2} - \dot{q_1} q_2$, the angular momentum.

	$L = \frac{1}{2} (\dot{q_1}^2 + \dot{q_2}^2) - V(q_1, q_2)$ so $p_1 = \frac{\partial L}{\partial \dot{q_1}} = \dot{q_1}$ and $p_2 = \frac{\partial L}{\partial \dot{q_2}} = \dot{q_2} \Rightarrow f = q_1 p_2 - q_2 p_1$.

	Then $q_1 \rightarrow q_1 + \epsilon \{ q_1, f \} + O(\epsilon^2) = q_1 + \epsilon \{ q_1, q_1 p_2 - q_2 p_1 \} = q_1 + \epsilon \{ q_1, q_1 p_2 \} - \epsilon \{ q_1, q_2 p_1 \} = q_1 + \epsilon \{q_1, q_1\} p_2 + \epsilon \{q_1, p_2\} q_1 - \epsilon \{q_1, q_2\} p_1 - \epsilon \{q_1, p_1\} q_2 = q_1 - \epsilon q_2$

	Similarly, $q_2 \rightarrow q_2 + \epsilon q_1$ so $(q_1, q_2) \rightarrow (q_1, q_2) + \epsilon ((0, -1), (1, 0)) (q_1, q_2)$ TODO make into matrices and column vectors.
\end{example}

\begin{definition}
	The \textbf{Hamiltonian} is the energy expressed in Hamiltonian coordinates:

	\[ H = \sum_{i = 1}^n \dot{q_i(\underline{q}, \underline{p})} p_i - L(\underline{q}, \dot{\underline{q}} (\underline{q}, \underline{p})) \]
\end{definition}

\begin{example}
	(Harmonic oscillator in one dimension) Let $\frac{1}{2} m \dot{x}^2 - \frac{1}{2} k x^2 \Rightarrow p = m\dot{x} \Rightarrow \dot{x} = \frac{p}{m}$.

	$H = \dot{x} p - L = \frac{p^2}{m} - (\frac{1}{2} \frac{p^2}{m} - \frac{1}{2} k x^2) = \frac{1}{2} \frac{p^2}{m} + \frac{1}{2} k x^2$
\end{example}

\begin{theorem}
	The time evolution of the phase space coordinates $\underline{q}, \underline{p}$ is generated by Hamiltonian flow $\Phi_H$:

	\[ q_i(t + a) = \Phi_H^{(a)} q_i(t), p_i(t + a) = \Phi_H^{(a)} p_i(t) \]
	Infinitesimally, $q_i(t) + \epsilon \dot{q_i}(t) + O(\epsilon^2) = q_i(t + \epsilon) = q_i(t) + \epsilon \{ q_i, H \} + O(\epsilon^2) \Leftrightarrow \dot{q_i} = \{q_i, H \} = \frac{\partial H}{\partial p_i}$ and similarly, $\dot{p_i} = \{ p_i, H \} = -\frac{\partial H}{\partial q_i}$.
	
	These equations are called \textbf{Hamilton's equations}.
\end{theorem}

\begin{proof}
	$\frac{\partial H}{\partial q_i}$. TODO: complete this proof, finish rest of notes from lecture.
\end{proof}

\begin{corollary}
	The time evolution of any function $f(\underline{q}, \underline{p})$ in phase space is generated by $\Phi_H$:

	\[ \frac{df}{dt} = \{ f, H \} \]
	If $f(\underline{q}, \underline{p}, t)$ depends explicitly on time then

	\[ \frac{df}{dt} = \{ f, h \} + \frac{\partial f}{\partial t} \]
\end{corollary}

\begin{proof}
	$\frac{df}{dt} = \sum_{i = 1}^n \left( \frac{\partial f}{\partial q_i} \dot{q_i} + \frac{\partial f}{\partial p_i} \dot{p_i} \right) + \frac{\partial f}{\partial t} = \sum_{i = 1}^n \left( \frac{\partial f}{\partial q_i} \frac{\partial H}{\partial p_i} - \frac{\partial f}{\partial p_i} \frac{\partial H}{\partial q_i} \right) + \frac{\partial f}{\partial t} = \{ f, H \} + \frac{\partial f}{\partial t}$.
\end{proof}

\end{document}