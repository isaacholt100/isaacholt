\documentclass[12pt,a4paper]{article}
\AddToHook{cmd/section/before}{\clearpage}

\usepackage[a4paper, total={6.5in, 10in}]{geometry}
\usepackage[utf8]{inputenc}
\usepackage{amsfonts, amssymb, amsmath}
\usepackage{amsthm}
\usepackage[dvipsnames]{xcolor}
\usepackage{titlesec}
\usepackage{afterpage}
\usepackage{hyperref}
\usepackage{esint}
\usepackage{diffcoeff}

\pagecolor{white}
\color{black}% set the default colour to white

\theoremstyle{definition}
\newtheorem{definition}{Definition}[subsection]
\newtheorem{theorem}[definition]{Theorem}
\newtheorem{proposition}[definition]{Proposition}
\newtheorem{corollary}[definition]{Corollary}
\newtheorem{lemma}[definition]{Lemma}
\newtheorem{example}[definition]{Example}
\newtheorem*{remark}{Remark}

\title{Algebra II Course Notes}
\author{Isaac Holt}

\begin{document}

\maketitle
\tableofcontents
\newpage

\section{Rings and fields}

\subsection{Rings, subrings and fields}

\begin{definition}
	A \textbf{ring} $(R, +, \cdot)$ is a set $R$ with two binary opertaions: addition ($+$) and multiplication ($\cdot$), such that $(R, +)$ is an abelian group and these conditions hold:
	\begin{enumerate}
		\item (\textbf{Identity}) for some element $1 \in R$, $\forall x \in R, \ 1 \cdot x = x \cdot 1 = x$.
		\item (\textbf{Associativity}) $\forall (x, y, z) \in R^3, \ x \cdot (y \cdot z) = (x \cdot y) \cdot z$.
		\item (\textbf{Distributivity}) $\forall (x, y, z) \in R^3, \ x \cdot (y + z) = x \cdot y + x \cdot z \text{ and } (y + z) \cdot x = y \cdot x + z \cdot x$. 
	\end{enumerate}
\end{definition}

\begin{remark}
	Often we write $R$ to mean the entire ring instead of just the set of the ring.
\end{remark}

\begin{definition}
	A ring $R$ is \textbf{commutative} if $\forall x, y \in R^2, \ x \cdot y = y \cdot x$ and is \textbf{non-commutative} otherwise.
\end{definition}

\begin{example}
	Let $V$ be a finite dimensional vector space over $\mathbb{C}$. The set of \textbf{linear endomorphisms} is defined as
	\[
		\text{End}(V) = \{ f: V \rightarrow V : f \text{ is a linear map} \}
	\]
	For $f \in \text{End}(V)$ and $g \in \text{End}(V)$, addition is defined as
	\[
		(f + g)(v) := f(v) + g(v)
	\]
	Multiplication is defined as function composition:
	\[
		f \cdot g := f \circ g
	\]
	where $(f \circ g)(v) := f(g(v))$. $\text{End}(v)$ is an abelian group under addition, and it forms a ring with the addition and multiplication operations defined as above:
	\begin{enumerate}
		\item The identity element is defined as the identity map $\text{id}: V \rightarrow V$, $\text{id}(v) := v$.
		\item Associativity: $f \circ (g \circ h) (v) = f((g \circ h)(v)) = f(g(h(v)))$ and $((f \circ g) \circ h) (v) = (f \circ g) (h(v)) = f(g(h(v))) = f \circ (g \circ h) (v)$.
		\item Distributivity is similarly easy to check.
	\end{enumerate}
\end{example}

\begin{definition}
	For $n \in \mathbb{N}$, the set of remainders modulo $n$ is
	\[
		\mathbb{Z} / n := \{ \bar{0}, \bar{1}, \dots, \overline{n - 1} \}
	\]
	The elements of $\mathbb{Z} / n$ are called \textbf{residue classes}.
\end{definition}

\begin{definition}
	\hfill
	\begin{itemize}
		\item Addition in $\mathbb{Z} / n$ is defined as $\overline{a} + \overline{b} = \overline{a + b}$.
		\item Subtraction in $\mathbb{Z} / n$ is defined as $\overline{a} - \overline{b} = \overline{a - b}$.
		\item Multiplication in $\mathbb{Z} / n$ is defined as $\overline{a} \cdot \overline{b} = \overline{a \cdot b}$.
	\end{itemize}
\end{definition}

\begin{example}
	$\mathbb{Z} / n$ is a commutative ring.

	\begin{itemize}
		\item Commutativity: $\overline{a} \cdot \overline{b} = \overline{ab} = \overline{ba} = \overline{b} \cdot \overline{a} \quad \forall \overline{a}, \overline{b} \in {(\mathbb{Z} / n)}^2$, by commutativity of $\mathbb{Z}$.
		\item Identity: $\overline{1} \cdot \overline{a} = \overline{1 \cdot a} = \overline{a \cdot 1} = \overline{a} \cdot \overline{1} \quad \forall \overline{a} \in \mathbb{Z} / n$ so $\overline{1}$ is the identity element.
		\item Associativity: $\overline{a} (\overline{a} \overline{c}) = \overline{a} (\overline{bc}) = \overline{a(bc)} = \overline{(ab)c} = (\overline{ab}) \overline{c} = (\overline{a} \overline{b}) \overline{c} \quad \forall \overline{a}, \overline{b}, \overline{c} \in {(\mathbb{Z} / n)}^3$.
	\end{itemize}
\end{example}

\begin{definition}
	A \textbf{subring} $S$ of a ring $R$ is a set $S \subset R$ that satisfies:
	\begin{enumerate}
		\item $0 \in S$ and $1 \in S$.
		\item $\forall a, b \in S^2, a + b \in S$.
		\item $\forall a, b \in S^2, a \cdot b \in S$,
		\item $\forall a \in S, -a \in S$.
	\end{enumerate}
	Note that the addition and multiplication operations for $S$ are the same as those for $R$.
\end{definition}

\begin{example}
	$\mathbb{Q}$ is a subring of $\mathbb{Q}[x]$. For every $a \in \mathbb{Q}$, $a$ is a constant polynomial in $\mathbb{Q}[x]$. $0 \in \mathbb{Q}$ and $1 \in \mathbb{Q}$. $\forall a, b \in \mathbb{Q}^2, a + b \in \mathbb{Q} \text{ and } -a \in \mathbb{Q} \text{ and } ab \in \mathbb{Q}$.
\end{example}

\begin{example}
	$\mathbb{Z}[\sqrt{2}]  \{ a + b \sqrt{2}: a, b \in \mathbb{Z}^2 \}$ is a ring. Instead of proving this using the definition of a ring, we can prove that it is a subring of $\mathbb{R}$, which requires less work.
\end{example}

\begin{example}
	A subset of a ring can be a ring without being a subring. For example, $R = \{ \overline{0}, \overline{2}, \overline{4} \} \subset \mathbb{Z} / 6$ but $R$ is not a subring of $\mathbb{Z} / 6$ since $\overline{1} \notin R$. However, $R$ is a ring itself, with identity $\overline{4}$.
\end{example}

\begin{definition}
	A ring $R$ is a \textbf{field} if
	\begin{enumerate}
		\item $R$ is commutative.
		\item $0 \in R$ and $1 \in R$, with $0 \ne 1$, so $R$ has at least two elements.
		\item $\forall a \in R$ with $a \ne 0$, for some $b \in R$, $ab = 1$. $b$ is called the \textbf{inverse} of $a$.
	\end{enumerate}
\end{definition}

\begin{remark}
	For a field $F$, if $a, b \in F^2$ satisfy $ab = 0$, then if $b \ne 0$, $a = ab b^{-1} = 0 b^{-1} = 0$. Similarly, if $a \ne 0$, then $b = 0$. So $ab = 0 \Longleftrightarrow a = 0 \text{ or } b = 0$.

	This is not true in all rings, and if a ring doesn't satisfy this property, then it can't be a field.
\end{remark}

\begin{definition}
	Let $R$ be a ring and let $a \in R$ such that for some $b \ne 0$, $ab = 0$. Then $a$ is called a \textbf{zero divisor}.
\end{definition}

\subsection{Integral domains}

\begin{definition}\label{def:integralDomain}
	A ring $R$ is called an \textbf{integral domain} if it is commutative, has at least two elements ($0 \ne 1$), and has no zero divisors except for $0$ ($\forall a, b \in R^2, ab = 0 \Longrightarrow a = 0 \text{ or } b = 0$).
\end{definition}

\begin{remark}
	Every ring that is a subring of a field is an integral domain.
\end{remark}

\begin{example}
	$\mathbb{Z} / 3$ is an integral domain, because $\forall a, b \in {(\mathbb{Z} / 3)}^2, a \ne 0 \text{ and } b \ne 0 \Longrightarrow ab \ne 0$. $\mathbb{Z} / 4$ is not an integral domain, because $\overline{2} \cdot \overline{2} = \overline{0}$ in $\mathbb{Z} / 4$.
\end{example}

\begin{proposition}
	If a ring $R$ is an integral domain, then the ring of polynomials $R[x] := \{ a_0 + a_1 x + \cdots + a_n x^n : \underline{a} \in R^n \}$ is an integral domain as well.
\end{proposition}

\begin{proof}
	$R[x]$ is obviously commutative, and $0 \in R[x], 1 \in R[x], 0 \ne 1$, as this is true for $R$. To show that the only zero divisor is $0$, assume the opposite, so for some $f(x), g(x) \in {(R[x])}^2, f(x) g(x) = 0$. Let
	\[
		\begin{aligned}
			f(x) & = a_0 + \cdots + a_m x^m, a_m \ne 0 \\
			g(x) & = b_0 + \cdots + b_n x_n, b_n \ne 0
		\end{aligned}
	\]
	Then
	\[
		f(x) g(x) = a_m b_n x^{m + n} + \cdots + a_0 b_0 = 0
	\]
	so $a_m b_n = 0$. But $a_m \in R$ and $b_n \in R$ and $R$ is an integral domain, so $a_m = 0$ or $b_n = 0$, so we have a contradiction.
\end{proof}

\begin{definition}
	For a ring $R$, $a \in R$ is called a \textbf{unit} if for some $b \in R$, $ab = ba = 1$, so $b = a^{-1}$ is the inverse of $a$.
\end{definition}

\begin{proposition}
	The inverse of $a \in R$ is unique.
\end{proposition}

\begin{proof}
	Assume that for some $b_1, b_2 \in R^2$, with $b_1 \ne b_2$, $a b_1 = b_1 a = 1$ and $a b_2 = b_2 a = 1$. But then
	\[
		b_1 (a b_1) = (b_1 a) b_1 = b_1 = b_1 a b_2 = b_2
	\]
	so we have a contradiction.
\end{proof}

\begin{definition}
	The \textbf{set of all units} of a ring $R$ is written as $R^{\times}$.
\end{definition}

\begin{definition}
	For a ring $R$, $R^{\times}$ is a group under multiplication from $R$.
\end{definition}

\begin{proof}
	\hfill
	\begin{enumerate}
		\item Closure: if $a, b \in {(R^{\times})}^2$, for some $c, d \in R^2$, $ac = 1$ and $bd = 1$ so $(ab)(dc) = a(bd)c = ac = 1$ so $ab \in R^{\times}$.
		\item Identity: $1 \cdot 1 = 1$ so $1 \in R^{\times}$ is the identity.
		\item Associativity: this is automatically satisfied by associativity in $R$.
		\item Inverse element: every $a \in R^{\times}$ has an inverse by definition.
	\end{enumerate}
\end{proof}

\begin{example}
	For a field $F$, $F^{\times} = F - \{ 0 \}$ since every $a \ne 0 \in F$ is a unit.
\end{example}

\begin{example}
	$\mathbb{Z}^{\times} = \{ 1, -1 \}$.
\end{example}

\begin{example}\label{exa:constantPolynomialUnitInField}
	For a field $F$, ${F[x]}^{\times} = F^{\times} = F - \{ 0 \}$, since if $f(x), g(x) \in {(F[x])}^2$ and $f(x) g(x) = 1$, then $\deg(f) = \deg(g) = 0$, otherwise $\deg(f g) \ge 1$. Therefore if $f$ is a unit, it is a constant non-zero polynomial, so $f \in F$.
\end{example}

\begin{example}
	${M_n(\mathbb{Q})}^{\times} = \{ A \in M_n(\mathbb{Q}): \det(A) \ne 0 \}$.
\end{example}

\begin{proposition}\label{prop:unitCoPrimeEquivalence}
	Let $\overline{a} \in \mathbb{Z} / n$. $\overline{a}$ is a unit iff $\gcd(a, n) = 1$.
\end{proposition}

\begin{proof}
	Let $d = \gcd(a, n)$, so $d \mid a$ and $d \mid n$. Assume $\overline{a}$ is a unit, so let $\overline{b} = \overline{a}^{-1}$, so $\overline{a} \overline{b} = \overline{1} \Rightarrow ab \equiv 1 \pmod{n} \Rightarrow \exists x \in \mathbb{Z}, ab = xn + 1$. Now $d \mid (ab)$ and $d \mid xn$ so $d \mid (ab + xn)$, hence $d \mid 1 \Rightarrow d = 1$.

	Now assume that $d = 1$, then by the Euclidean algorithm, $\exists x, y \in \mathbb{Z}^2, xa + ny = d = 1$. So $xa \equiv 1 \pmod{n} \Rightarrow \overline{a} \overline{x} = \overline{1}$, so $\overline{a}$ is a unit, with $\overline{a}^{-1} = \overline{x}$.
\end{proof}

\begin{corollary}
	${(\mathbb{Z} / n)}^{\times} = \{ \overline{a} \in \mathbb{Z} / n: \gcd(a, n) = 1 \}$.
\end{corollary}

\begin{proof}
	It's pretty much already there.
\end{proof}

\begin{corollary}\label{cor:primeRemaindersField}
	$\mathbb{Z} / p$ is a field iff $p$ is prime.
\end{corollary}

\begin{proof}
	If $p$ is prime, then $\overline{1}, \overline{2}, \dots, \overline{p - 1}$ are all units by Proposition~\ref{prop:unitCoPrimeEquivalence}, so $\mathbb{Z} / p$ is a field.

	If $\mathbb{Z} / p$ is a field, then every $\overline{0} \ne \overline{a} \in \mathbb{Z} / p$ is a unit, hence $\gcd(a, p) = 1 \ \forall 1 \le a \le p - 1$ by Proposition~\ref{prop:unitCoPrimeEquivalence}. This means $p$ must be prime.
\end{proof}

\begin{proposition}
	$\mathbb{Z} / p$ is an integral domain iff $p$ is prime (iff $\mathbb{Z} / p$ is a field).
\end{proposition}

\begin{proposition}
	If $p$ is prime, $\mathbb{Z} / p$ is a field by Corollary~\ref{cor:primeRemaindersField}, and every field is an integral domain.

	If $p$ is not prime, $\exists a, b \in \mathbb{Z}^2, p = ab$, with $2 \le a, b \le n - 1$. But then $\overline{a} \overline{b} = \overline{p} = \overline{0}$, meaning that $\overline{a}$ and $\overline{b}$ are zero divisors in $\mathbb{Z} / p$, so $\mathbb{Z} / p$ is not an integral domain. The contrapositive of this statement completes the proof.
\end{proposition}

\subsection{Polynomials over a field}

\begin{definition}\label{def:deg}
	For a field $F$ and $f(x) = a_0 + \cdots + a_n x_n \in F[x]$, the \textbf{degree} of $f$ is defined as
	\[
		\deg(f) = \begin{cases}
			\max \{ i: a_i \ne 0 \} & \text{ if } f(x) \ne 0 \\
			-\infty & \text{ if } f(x) = 0
		\end{cases}
	\]
	It satisfies the following properties for every $f(x), g(x) \in {(F[x])}^2$:
	\begin{itemize}
		\item $\deg(fg) = \deg(f) + \deg(g)$
		\item $\deg(f + g) \le \max \{ \deg(f), \deg(g) \}$ with equality if $\deg(f) \ne \deg(g)$.
	\end{itemize}
	The degree of the zero polynomial is $-\infty$ for the following reason:
	\begin{itemize}
		\item Let $f$ be the zero polynomial and let $g, h \in {(F[x])}^2$, with $\deg(g) \ne \deg(h)$. So $f = f g = f h$.
		\item By the first property, $\deg(g) + \deg(f) = \deg(g f) = \deg(f) = \deg(h f) = \deg(h) + \deg(f)$, but $\deg(g) \ne \deg(h)$. So for this equality to be true, $\deg(f) = \pm \infty$. But by the second property, $\deg(f + g) = \max \{ \deg(f), \deg(g) \}$ when $\deg(g) \ne 0$, which would not hold if $\deg(f) = \infty$. So $\deg(f) = -\infty$.
	\end{itemize}
\end{definition}

\begin{proposition}\label{prop:divisionAlgorithm}
	Let $f(x), g(x) \in {(F[x])}^2$ and $g(x) \ne 0$. Then there are unique polynomials $q(x), r(x) \in {(F[x])}^2$, where $\deg(r) < \deg(g)$, such that
	\[
		f(x) = q(x) g(x) + r(x)
	\]
\end{proposition}

\begin{proof}
	First we show the existence of $q(x)$ and $r(x)$. If $\deg(g) > \deg(f)$, $q(x) = 0$ and $r(x) = f(x)$. If $\deg(g) \le \deg(f)$, let
	\[
		\begin{aligned}
			f(x) & = a_0 + \cdots + a_m x^m, \quad a_m \ne 0 \\
			g(x) & = b_0 + \cdots + b_n x^n, \quad b_n \ne 0
		\end{aligned}
	\]
	Use induction on $d = m - n \ge 0$.
	\begin{itemize}
		\item When $d = 0$, $m = n$, then let $q(x) = a_m / b_n$ and let
		\[
			r(x) = f(x) - q(x) g(x)
		\]
		which satisifes $\deg(r) < m = \deg(g) \le \deg(f)$.
		\item Assume $q(x)$ and $r(x)$ exist for every $0 \le d < k$ for some $k \ge 1$.
		\item When $d = k$, $m = n + k$ and let
		\[
			f_1(x) = f(x) - \frac{a_m}{b_n} x^{m - n} g(x)
		\]
		so $\deg(f_1) < \deg(f)$. By the inductive assumption, for some $q_1(x)$ and $r(x)$,
		\[
			f_1(x) = q_1(x) g(x) + r(x)
		\]
		which gives
		\[
			\begin{aligned}
				f(x)
					& = f_1(x) + \frac{a_m}{b_n} x^{m - n} g(x) \\
					& = \left( q_1(x) + \frac{a_m}{b_n} x^{m - n} \right) g(x) + r(x)
					& = q(x) g(x) + r(x)
			\end{aligned}
		\]
		where we let $q(x) = q_1(x) + \frac{a_m}{b_n} x^{m - n}$. So the result holds for $d = k$, and this completes the induction.
	\end{itemize}
	Now we show the uniqueness of $q(x)$ and $r(x)$. Let $f(x) = q_1(x) g(x) + r_1(x) = q_2(x) g(x) + r_2(x)$, where $\deg(r_1) < \deg(g)$ and $\deg(r_2) < \deg(g)$, so $\deg(R - r) < \deg(g)$. Then
	\[
		r_2(x) - r_1(x) = (q_1(x) - q_2(x)) g(x)
	\]
	so by the \hyperref[def:deg]{properties of $\deg$},
	\[
		\deg(q_1 - q_2) + \deg(g) = \deg(r_2 - r_1) < \deg(g)
	\]
	Hence $\deg(q_1 - q_2) < 0$ so $q_1(x) = q_2(x)$, and since $r_2(x) - r_1(x) = (q_1(x) - q_2(x)) g(x)$, $r_1(x) = r_2(x)$.
\end{proof}

\subsection{Divisibility and greatest common divisor in a ring}

\begin{definition}
	Let $R$ be a commutative ring and $a, b \in R^2$. $a$ \textbf{divides} $b$ if for some $r \in R$, $b = ra$ and we write $a \mid b$.
\end{definition}

\begin{definition}\label{def:gcd}
	Let $R$ be a commutative ring and $a, b \in R^2$. $d \in R$ is a \textbf{greatest common divisor}, written $d = \gcd(a, b)$, if
	\begin{itemize}
		\item $d \mid a$ and $d \mid b$.
		\item For every $e \in R$, if $e \mid a$ and $e \mid b$, $e \mid d$.
	\end{itemize}
\end{definition}

\begin{remark}
	This definition does not require that $\gcd(a, b)$ be unique. For example, by this definition $1$ and $-1$ are greatest common divisors of $4$ and $5$ in $\mathbb{Z}$. $\mathbb{Z}$ has a total ordering so in this case we can define the \textbf{greatest} common divisor to be the larger of the two. But in some rings, a total ordering does not exist, so multiple gcd's exist. Some rings exist where a gcd of two elements does not exist at all.
\end{remark}

\begin{lemma}
	For every ring $R$, $\gcd(0, 0) = 0$.
\end{lemma}

\begin{proof}
	$\forall x \in R, 0 = 0 \cdot x$ so every element divides $0$, so the first property is satisfied. By the second property, every element that divides $0$ must also divide $\gcd(0, 0)$. But every $x \in R$ divides $0$, so in particular $0 \in R$ divides $0$, so $0$ must divide $\gcd(0, 0)$ hence
	\[
		\exists m \in R, \ \gcd(0, 0) = 0 \cdot m = 0
	\]
	so $\gcd(0, 0) = 0$, which is unique.
\end{proof}

\begin{lemma}\label{lem:gcdUniqueUpToUnits}
	Let $R$ be an integral domain. Let $a, b \in R^2$ and assume $d = \gcd(a, b)$ exists. Then for every unit $u \in R^{\times}$, $ud$ is also a gcd of $a$ and $b$. Also, for any two gcd's $d_1$ and $d_2$ of $a$ and $b$, for some unit $u \in R^{\times}$, $d_1 = d_2 u$. So the gcd is unique up to units.
\end{lemma}

\begin{proof}
	We first prove that $ud$ is a gcd of $a$ and $b$. $d \mid a$ so for some $m \in R$, $dm = a$, hence
	\[
		du (u^{-1} m) = a \Longrightarrow du \mid a
	\]
	Similarly, $du \mid b$.

	For every $e \in R$ such that $e \mid a$ and $e \mid b$, $e \mid d \Longrightarrow \exists k \in R, ek = d$. Then $eku = du \Rightarrow e \mid du$. So by Definition~\ref{def:gcd}, $du$ is a gcd.

	Now we prove that the gcd is unique up to units. Let $d_1$ and $d_2$ be gcd's. Then by Definition~\ref{def:gcd}, $d_1$ and $d_2$ divide $a$ and $b$ and both divide each other. Hence
	\[
		\exists u, v \in R^2, \quad d_1 = d_2 u, \quad d_2 = d_1 v
	\]
	So $d_1 = d_1 uv$. If $d_1 = 0$ then $d_2 = 0$ so let $u = 1$. If $d_1 \ne 0$, since $R$ is an integral domain, $uv = 1$, hence $u$ and $v$ are units.
\end{proof}

\begin{definition}
	Let $F$ be a field. A polynomial
	\[
		p(x) = a_0 + \cdots + a_n x^n \in F[x]
	\]
	is called \textbf{monic} if its leading coefficient $a_n = 1$.
\end{definition}

\begin{corollary}
	Let $F$ be a field. Then for every $p_1(x), p_2(x) \in {(F[x])}^2$, there is a unique monic gcd.
\end{corollary}

\begin{proof}
	Let $g(x) = a_0 + \cdots + a_n x^n$ be a gcd of $p_1$ and $p_2$. $a_n$ is a unit in $F[x]$ by Example~\ref{exa:constantPolynomialUnitInField}, so $\frac{1}{a_n} g(x)$ is a gcd and is monic. Now assume
	\[
		h(x) = b_0 + \cdots x^m
	\]
	is another monic gcd. Then by Lemma~\ref{lem:gcdUniqueUpToUnits}, for some unit $u \in F[x]^{\times} = F^{\times}$,
	\[
		u h(x) = u(b_0 + \cdots + x_m) = \frac{1}{a_n} g(x)
	\]
	Then $u x^m = x^n$ so $u = 1$ and $m = n$. Hence $h(x) = \frac{1}{a_n} g(x)$.
\end{proof}

\begin{theorem}\label{thm:gcdExistsAndCanBeComputed}
	Let $R$ be either $\mathbb{Z}$ or $F[x]$, and $a, b \in R^2$. Then
	\begin{enumerate}
		\item A $\gcd$ of $a$ and $b$ exists.
		\item If $a \ne 0$ and $b \ne 0$, a gcd can be computed by the \textbf{Euclidean algorithm} (the algorithm is shown in the proof).
		\item If $d$ is a $\gcd(a, b)$, then for some $x, y \in R^2$, $ax + by = d$.
	\end{enumerate}
\end{theorem}

\begin{proof}
	The proof is shown for $R = F[x]$. For $R = \mathbb{Z}$, the proof is the same, but $\deg(r_i(x)) < \deg(r_{i - 1}(x))$ is replaced with just $r_i < r_{i - 1}$ and so on.

	Let $r_{-1} (x) = a$ and $r_0(x) = b$. We have
	\[
		\begin{aligned}
			\exists q_1(x), r_1(x) \in {(F[x])}^2, r_{-1}(x) & = q_1(x) r_0(x) + r_1(x), \quad \deg(r_1(x)) < \deg(r_0(x)) \\
			\vdots \\
			\exists q_i(x), r_i(x) \in {(F[x])}^2, r_{i - 2}(x) & = q_i(x) r_{i - 1}(x) + r_i(x), \quad \deg(r_i(x)) < \deg(r_{i - 1}(x)) \\
			\vdots \\
			\exists q_n(x), r_n(x) \in {(F[x])}^2, r_{n - 2}(x) & = q_n(x) r_{n - 1}(x) + r_n(x), \quad \deg(r_n(x)) < \deg(r_{n - 1}(x)) \\
			\exists q_{n + 1} \in F[x], r_{n - 1}(x) & = q_{n + 1}r_n(x) + 0
		\end{aligned}
	\]
	This process must terminate after a finite number of iterations, since the degree of $r_i(x)$ is a non-negative integer and it decreases by at least $1$ each time.

	The last non-zero remainder, $r_n(x)$ divides $r{n - 1}(x)$, hence divides $r_{n - 2}(x)$, and so on, so divides $r_{-1}(x)$ and $r_0(x)$. Now for every divisor $d(x)$ of $r_{-1}(x)$ and $r_0(x)$, $d(x)$ must divide $r_1(x)$, so also divides $r_2(x)$, and so on, so divides $r_n(x)$. Therefore $r_n(x)$ satisfies the \hyperref[def:gcd]{properties of a gcd}, so is a gcd of $a$ and $b$.

	To prove part 3 of the theorem, start from $r_n(x) = r_{n - 2}(x) - q_n(x) r_{n - 1}(x)$ and replace $r_{n - 1}(x)$ with $r_{n - 3}(x) - q_{n - 1}(x) r_{n - 2}(x)$ from the equation above. So we have
	\[
		r_n(x) = h(x) r_{n - 2}(x) + g(x) r_{n - 3}(x)
	\]
	for some $h(x), g(x)$. Continuing this process from bottom to top, we get
	\[
		r_n(x) = a(x) r_{-1} (x) + b(x) r_0 (x)
	\]
	for some $a(x), b(x) \in {(F[x])}^2$.
\end{proof}

\begin{example}
	Find a gcd of $f(x) = x^2 + 1$ and $g(x) = x^2 + 3x + 1$ in $\mathbb{Q}[x]$. Using the Euclidean algorithm, we obtain
	\[
		\begin{aligned}
			f(x) & = g(x) - 3x \\
			g(x) = (-\frac{1}{3}x - 1)(-3x) + 1 \\
			-3x = 1 (-3x) + 0
		\end{aligned}
	\]
	The algorithm terminates as the remainder is now $0$. A gcd is the last non-zero remainder, in this case $1$. Now we write $1$ as a linear combination of $f(x)$ and $g(x)$:
	\[
		\begin{aligned}
			1
				& = g(x) - \left( -\frac{1}{3}x - 1 \right)(-3x) \\
				& = g(x) + \left( \frac{1}{3}x + 1 \right) (f(x) - g(x)) \\
				& = \left( \frac{1}{3}x + 1 \right) f(x) - \frac{1}{3} x g(x)
		\end{aligned}
	\]
\end{example}

\section{Factorisations in rings}

\subsection{Irreducible polynomials over a field}

\begin{definition}
	Let $R$ be a commutative ring. $0 \ne r \in R$ is called \textbf{irreducible} if:
	\begin{enumerate}
		\item $r$ is not a unit and
		\item if for some $a, b \in R^2, r = ab$, then $a$ is a unit or $b$ is a unit.
	\end{enumerate}
\end{definition}

\begin{example}
	For a field $F$, a non-zero polynomial in $F[x]$ is irreducible if it is not constant and cannot be written as the product of two non-constant polynomials in $F[x]$.
\end{example}

\begin{example}
	$x^2 + 1 \in \mathbb{R}$ is irreducible in $\mathbb{R}[x]$, but is not irreducible in $\mathbb{C}$, since $x^2 + 1 = (x - i)(x + i)$.
\end{example}

\begin{example}
	The irreducible elements in $\mathbb{Z}$ are the prime numbers.
\end{example}

\begin{definition}
	For a field $F$, let $f(x) \in F[x]$. $a \in F$ is called a \textbf{root} (or a \textbf{zero}) of $f$ in $F$ if $f(a) = 0$.
\end{definition}

\begin{proposition}\label{prop:polynomialDegreesIrreducible}
	\hfill
	\begin{itemize}
		\item If $\deg(f) = 1$, then $f$ is irreducible in $F[x]$.
		\item If $\deg(f) = 2 \text{ or } 3$, then $f$ is irreducible in $F[x]$ iff $f$ has no roots in $F$.
		\item If $\deg(f) = 4$, then $f$ is irreducible iff $f$ has no roots in $F$ and $f$ is not the product of two quadratic polynomials.
	\end{itemize}
\end{proposition}

\begin{proof}
	\hfill
	\begin{itemize}
		\item If $\deg(f) = 1$, then $f(x) = ax + b$ for some $a, b \in F^2$, $a \ne 0$. By Example~\ref{exa:constantPolynomialUnitInField}, $f$ is not a unit. Now let $f(x) = g(x) h(x)$ for some $g(x), h(x) \in {(F[x])}^2$. But $1 = \deg(f) = \deg(g) + \deg(h)$ so either $\deg(g) = 1$ and $\deg(h) = 0$ or $\deg(g) = 0$ and $\deg(h) = 1$. Therefore one of $g$ and $h$ has degree $0$ so is a constant polynomial, therefore is a unit.
		\item If $\deg(f) = 2 \text{ or } 3$, let $\alpha \in F$ be a root of $f$, so $f(x) = q(x) (x - \alpha) + r(x)$ for some $q(x), r(x)$ where $\deg(r) \le 0$, by Proposition~\ref{prop:divisionAlgorithm}. Hence $r(x)$ is constant.
		
		Now, $0 = f(\alpha) = r(\alpha)$ but since $r(x)$ is constant, $r(x) = 0$ so $f(x) = q(x) (x - \alpha)$. Therefore $f$ is not irreducible as $\deg(q) \ge 1$.

		Conversely, if $f$ is not irreducible, $f(x) = g(x) h(x)$ for some $g(x), h(x)$ where $\deg(g) \ge 1$ and $\deg(h) \ge 1$. $\deg(f) = \deg(g) + \deg(h)$ so either $\deg(g) = 1$ or $\deg(h) = 1$. WLOG, assume $\deg(g) = 1$, then $g(x) = ax + b$ for some $a, b \in F^2$, $a \ne 0$.

		Then $g(-b/a) = 0 = f(-b/a)$ so $f$ has a root.
		\item The proof when $\deg(f) = 4$ is similar to the proof for $\deg(f) = 2 \text{ or } 3$.
	\end{itemize}
\end{proof}

\begin{proposition}
	Let $f(x) = a_0 + a_1 x + \cdots + a_n x^n \in \mathbb{Z}[x]$, where $\deg(f) \ge 1$. For every $p, q \in \mathbb{Z}^2$, if $f(p / q) = 0$ and $\gcd(p, q) = 1$, then
	\[
		p \mid a_0 \quad \text{and} \quad q \mid a_n
	\]
\end{proposition}

\begin{proof}
	\[
		\begin{aligned}
			0
				& = f(p/q) \\
				& = a_0 + a_1 (p/q) + \cdots + a_n {(p/q)}^n \\
				& = a_0 q^n + p \left( a_1 q^{n - 1} + a_2 p q^{n - 2} + \cdots + a_n p^{n - 1} \right) = 0 \\
				& = q(a_0 q^{n - 1} + a_1 p q^{n - 2} + \cdots + a_{n - 1} p^{n - 1}) + a_n p^n = 0
		\end{aligned}
	\]
	So $p \mid a_0 q^n$ and $q \mid a_n p^n$ and since $\gcd(p, q) = 1$, $p \mid a_0$ and $q \mid a_n$.
\end{proof}

\begin{example}
	$f(x) = x^3 - 3x + 2$ is irreducible in $\mathbb{Q}[x]$.
	
	By Proposition~\ref{prop:polynomialDegreesIrreducible}, it is sufficient to show that $f$ has no roots in $\mathbb{Q}$. Let $p/q \in \mathbb{Q}$, and assume $\gcd(p, q) = 1$ (if $\gcd(p, q) \ne 1$, then the fraction can be cancelled by $\gcd(p, q)$ to leave the same value). If $p/q$ was a root of $f$, then $q \mid 1$ and $p \mid 2$. So the only possible values of $p$ and $q$ are $\{ \pm 1, \pm 2 \}$ but none of these values is a root, so $f$ is irreducible in $\mathbb{Q}[x]$.
\end{example}

\begin{lemma}\label{lem:GaussLemma}
	(\textbf{Gauss's lemma}) Let $f(x) = a_0 + a_1 x + \cdots + a_n x^n \in \mathbb{Z}[x]$ be a non-constant polynomial. Then $f(x)$ is irreducible in $\mathbb{Z}[x]$ iff $f(x)$ is irreducible in $\mathbb{Q}[x]$ and $\gcd(a_0, \dots, a_n) = 1$.
\end{lemma}

\begin{proof}
	($\Longleftarrow$): Let $f(x)$ be irreducible in $\mathbb{Q}[x]$ and $\gcd(a_0, \dots, a_n) = 1$. Let $f(x) = g(x) h(x)$ for some $g(x), h(x) \in {(\mathbb{Z})}^2$. If $\deg(g) \ge 1$ and $\deg(h) \ge 1$ then $f$ would have a proper factorisation in $\mathbb{Q}[x]$, contradicting the fact that it is irreducible in $\mathbb{Q}[x]$. So $\deg(g) = 0$ or $\deg(h) = 0$. WLOG, assume that $\deg(g) = 0$, hence $g(x) \in \mathbb{Z}$. If $g(x) \ne \pm 1$, for some prime number $p \in \mathbb{Z}$, $p \mid g(x) \Longrightarrow p \mid f(x)$, but $\gcd(a_0, \dots, a_n) = 1$. Hence $g(x) = \pm 1$ which is a unit, so $f(x)$ is irreducible in $\mathbb{Z}[x]$.

	($\Longrightarrow$): Omitted.
\end{proof}

\begin{corollary}\label{cor:coprimeCoeffsPolynomialIrreducibleInZAndQ}
	Let $f(x) = a_0 + a_1 x + \cdots + a_n x^n \in \mathbb{Z}[x]$ and $\gcd(a_0, \dots, a_n) = 1$. Then $f(x)$ is irreducible in $\mathbb{Z}[x]$ iff $f(x)$ is irreducible in $\mathbb{Q}[x]$.
\end{corollary}

\begin{lemma}
	Let $f(x) = a_0 + a_1 + \cdots + a_n x^n \in \mathbb{Z}[x]$, $a_n = 1$, be a monic polynomial. If $f(x)$ factors in $\mathbb{Q}[x]$, then $f(x)$ factors into integer monic polynomials.
\end{lemma}

\begin{proof}
	Clearly $\gcd(a_0, \dots, a_n) = 1$, so by Corollary~\ref{cor:coprimeCoeffsPolynomialIrreducibleInZAndQ}, if $f$ factors in $\mathbb{Q}[x]$, $f$ factors in $\mathbb{Z}[x]$. Hence
	\[
		f(x) = g(x) h(x) = (b_0 + \dots + b_m x^m) (c_0 + \dots + c_m x^m)
	\]
	where $m + l = n$ and $\forall i, \ b_i, c_i \in \mathbb{Z}^2$. Equating the coefficients of $x^n$ on both sides gives $b_m c_l = 1$ so $b_m = c_l = 1$, or $b_m = c_l = -1$. So either $g$ and $h$ are monic or $-g$ and $-h$ are monic, and $f(x) = (-g(x)) (-h(x))$.
\end{proof}

\begin{lemma}\label{lem:irreducibleTimesUnitIsIrreducible}
	Let $R$ be a commutative ring, let $x \in R$ be irreducible and let $u \in R^{\times}$. Then $ux$ is irreducible.
\end{lemma}

\begin{proof}
	$x \ne 0$ so $ux \ne 0$. If $ux$ is a unit, then for some $b \in R$, $b (ux) = 1 = (bu) x \Longrightarrow x \in R^{\times}$, which is a contradiction, hence $x$ is not a unit.

	Let $ux = ab$ for some $a, b \in R^2$, then we must show that $a$ or $b$ is a unit. $x = ab u^{-1} = a(b u^{-1})$ and as $x$ is irreducible, $a \in R^{\times}$ or $b u^{-1} \in R^{\times}$. And $b u^{-1} \in R^{\times} \Rightarrow b \in \mathbb{R}^{\times}$, as units form a group under multiplication, hence either $a$ or $b$ is a unit.
\end{proof}

\begin{proposition}
	(\textbf{Eisenstein's criterion}) Let $f(x) = a_0 + a_1 x + \cdots + a_n x^n \in \mathbb{Z}[x]$ and let $p$ be a prime with $p \mid a_0, \dots, p \mid a_{n - 1}$, $p \nmid a_n$ and $p^2 \nmid a_0$. Then $f$ is irreducible in $\mathbb{Q}[x]$.
\end{proposition}

\begin{proof}
	Let $d = \gcd(a_0, \dots, a_n)$, $b_i = a_i / d$ and
	\[
		F(x) = \frac{1}{d} f(x) = b_0 + b_1 x + \cdots + b_n x^n \in \mathbb{Z}[x]
	\]
	Then $\gcd(b_0, \dots, b_n) = 1$. Note that $p \nmid a_n$ so $p \nmid d$, so $p \mid b_0, p \mid b_1, \dots, p \mid b_{n - 1}, p \nmid b_n$ and $p^2 \nmid b_0$. By Lemma~\ref{lem:irreducibleTimesUnitIsIrreducible}, if $F$ is irreducible in $\mathbb{Q}[x]$, then $f$ is also irreducible in $\mathbb{Q}[x]$.

	Assume that $F$ is not irreducible in $\mathbb{Q}[x]$. By \hyperref[lem:GaussLemma]{Gauss's lemma}, $F(x) = g(x) h(x)$ for some $g(x), h(x) \in {(\mathbb{Z})}^2$ with $\deg(g) \ge 1$ and $\deg(h) \ge 1$. Reducing this by modulo $p$ gives
	\[
		\bar{g}(x) \bar{h}(x) = \bar{F}(x) = \bar{b_0} + \bar{b_1} x + \cdots + \bar{b_n} x^n = \bar{b_n} x^n
	\]
	Let
	\[
		\begin{aligned}
			\bar{g}(x) = \bar{\alpha_0} + \bar{\alpha_1} x + \cdots + \bar{\alpha_m} x^m \\
			\bar{h}(x) = \bar{\beta_0} + \bar{\beta_1} x + \cdots + \bar{\beta_k} x^k
		\end{aligned}
	\]
	$\deg(\bar{g}) = \deg(g)$ and $\deg(\bar{h}) = \deg(h)$, otherwise $p \mid b_n$. This gives
	\[
		\overline{\alpha_0} \overline{\beta_0} + \overline{\alpha_0} \overline{\beta_1} x + \cdots + \overline{\alpha_0} \overline{\beta_k} x^k + \overline{\alpha_1} \overline{\beta_0} x + \cdots + \overline{\alpha_m} \overline{\beta_k} x^{m + k} = \overline{b_n} x^n
	\]
	hence $\overline{\alpha_0} \overline{\beta_0} = \overline{0}$. $p$ is prime so $\mathbb{Z} / p$ is a field, so this implies $\overline{\alpha_0} = \overline{0}$ or $\overline{\beta_0} = \overline{0}$. WLOG, let $\overline{\alpha_0} = \overline{0}$, then we have $\overline{\beta_0} \overline{\alpha_m} = \overline{0}$ and $\overline{\alpha_m} \ne \overline{0}$ so $\overline{\beta_0} = \overline{0}$.

	So $p \mid \alpha_0$ and $p \mid \beta_0$, thus $p^2 \mid \alpha_0 \beta_0 = b_0$ so $p^2 \mid b_0$ which is a contradiction. Hence $F$ is irreducible, so $f$ is also.
\end{proof}

\subsection{Unique factorisation in $F[x]$}

\begin{lemma}\label{lem:pDividesProductImpliesPDividesOneFactor}
	Let $F$ be a field. If $p(x) \in F[x]$ is irreducible and $p(x) \mid a(x) b(x)$ for some $a(x), b(x) \in {(F[x])}^2$, then $p(x) \mid a(x)$ or $p(x) \mid b(x)$.
\end{lemma}

\begin{proof}
	If $p(x) \nmid a(x)$, then $\gcd(p(x), a(x)) = 1$ so by Theorem~\ref{thm:gcdExistsAndCanBeComputed} (3.), for some $A(x), B(x) \in {(F[x])}^2$,
	\[
		A(x) p(x) + B(x) a(x) = 1 \Longrightarrow A(x) p(x) b(x) + B(x) a(x) b(x) = b(x)
	\]
	But $p(x) \mid B(x) a(x) b(x)$ and $p(x) \mid A(x) p(x) b(x)$ hence $p(x) \mid b(x)$. Hence $p(x) \mid a(x)$ or $p(x) \mid b(x)$.
\end{proof}

\begin{theorem}\label{thm:uniqueFactorisationInPolynomialOverField}
	Let $F$ be a field and let $f(x) \in F[x]$ with $\deg(f) \ge 1$. Then $f(x)$ can be uniquely factorised into a product of irreducible elements, up to order of the factors and multiplication by units.
\end{theorem}

\begin{proof}
	\begin{itemize}
		\item First we prove the existence of a factorisation. Use induction on $\deg(f)$. If $\deg(f) = 1$, then $f$ is irreducible already. Assume now that we have such a factorisation for $f'(x) \in F[x]$ with $\deg(f') < n$, for some $n \in \mathbb{N}$. Let $\deg(f) = n$. If $f$ is irreducible we are done. If not, then $f(x) = g(x) h(x)$ for some $g(x), h(x) \in {(F[x])}^2$ with $1 \le \deg(f) < n$ and $1 \le \deg(h) < n$. By the induction hypothesis, $g$ and $h$ have factorisations into irreducible elements, hence $f$ also does.
		\item Now we prove the uniqueness. Let
		\[
			f(x) = p_1 (x) \cdots p_m(x) = q_1(x) \cdots q_n(x)
		\]
		where for every $i$, $p_i$ and $q_i$ are irreducible. Then $p_1(x) \mid q_1(x) \cdots q_n(x)$ so by Lemma~\ref{lem:pDividesProductImpliesPDividesOneFactor}, $p_1$ must divide one of the $q_i$. WLOG, assume $p_1 \mid q_1$. So $q_1(x) = u_1(x) p_1(x)$ for some $u_1(x) \in F[x]$, but $p_1$ and $q_1$ are irreducible so $u_1$ is a unit. Hence
		\[
			\begin{aligned}
				f(x) = p_1 (x) \cdots p_m(x) & = u_1(x) p_1(x) q_2(x) \cdots q_n(x) \\
				\Longrightarrow p_2 (x) \cdots p_m(x) & = u_1(x) q_2(x) \cdots q_n(x)
			\end{aligned}
		\]
		Repeat these steps for $p_2, p_3, \dots$ until all factors are cancelled. This gives $m = n$ and $q_i = u_i p_i$ for every $i$ and some unit $u_i$. This completes the proof.
	\end{itemize}
\end{proof}

\begin{definition}
	Let $R$ be a commutative ring. $x \in R$ is called \textbf{prime} if these conditions hold:
	\begin{enumerate}
		\item $x \ne 0$ and $x \notin R^{\times}$ and
		\item $\forall a, b \in R^2, \ x \mid ab \Longrightarrow a \mid a$ or $x \mid b$.
	\end{enumerate}
\end{definition}

\begin{example}
	$p \in \mathbb{Z}$ is prime iff $p$ is irreducible.
\end{example}

\begin{example}
	For a field $F$, $f(x) \in F[x]$ is prime iff it is irreducible.
\end{example}

\begin{lemma}
	Let $R$ be an integral domain. Let $x \in R$ be prime. Then $x$ is irreducible.
\end{lemma}

\begin{proof}
	The ring $\mathbb{Z}[\sqrt{-5}] := \{ a + b \sqrt{-5}: a, b \in \mathbb{Z}^2$ is a subring of $\mathbb{C}$. Define
	\[
		N: \mathbb{Z}[\sqrt{-5}] \rightarrow \mathbb{Z}, quad N(a + b \sqrt{-5}) = a^b + 5 b^2
	\]
	$N(z) = z \bar{z}$ where $\bar{z}$ is the complex conjugate of $z$. So $N(z) N(w) = z \bar{z} w \bar{w} = z w \bar{z} \bar{w} = N(zw)$. We show that $2$ is irreducible.

	Assume $2 = (x + y \sqrt{-5}) (z + w \sqrt{-5})$ for some $x, y, z, w \in \mathbb{Z}^4$. Then
	\[
		N(2) = N(x + y \sqrt{-5}) N(z + w \sqrt{-5})
	\]
	So $N(x + y \sqrt{-5}) \mid 4$, so $N(x + y \sqrt{-5}) \in \{ \pm 1, \pm 2, \pm 4 \}$. The only possibilities from these are $1$ and $4$. If $x^2 + 5y^2 = 1$ then $y = 0$ and $x = \pm 1$ so $x + y \sqrt{-5}$ is a unit. If $x^2 + 5y^2 = 4$ then $y = 0$ and $x = \pm 2$ so $2 = \pm 2 (z + w \sqrt{-5})$, hence $(z + w \sqrt{-5})$ is a unit. Hence $2$ is irreducible.

	However, $2$ is not prime, since
	\[
		2 \mid (1 - \sqrt{-5}) (1 + \sqrt{-5}) = 6
	\]
	but $2 \nmid (1 - \sqrt{-5})$ and $2 \nmid (1 + \sqrt{-5})$, since $2(x + y \sqrt{-5}) = 1 \pm \sqrt{-5}$ for some $x, y \in \mathbb{Z}^2$ then $2x = 1$, a contradiction.
\end{proof}

\begin{definition}
	Let $R$ be an integral domain. $R$ is called a \textbf{unique factorization domain} (\textbf{UFD}) if every non-zero non-unit element of $R$ has a unique factorization into a product of irreducible elements, which is unique up to order of factors and multiplication by units.
\end{definition}

\begin{example}
	$\mathbb{Z}$ is a UFD.
\end{example}

\begin{example}
	For a field $F$, $F[x]$ is a UFD by Theorem~\ref{thm:uniqueFactorisationInPolynomialOverField}.
\end{example}

\begin{example}
	$\mathbb{Z}[-\sqrt{5}]$ is not a UFD, as $6$ has two factorisations:
	\[
		(1 + \sqrt{-5}) (1 - \sqrt{-5}) = 2 \cdot 3 = 6
	\]
\end{example}

\section{Homomorphisms between Rings}	

Let $R$ and $S$ be two rings. A map $f: R \rightarrow S$ is called a (ring)-homomorphism if:
\begin{enumerate}
	\item $f(1) = 1$
	\item $f(a + b) = f(a) + f(b)$
	\item $f(ab) = f(a)f(b)$
\end{enumerate}

\begin{lemma}
	$f(0) = 0$ and $f(-a) = -f(a)$
\end{lemma}

\begin{proof}
	$f(0) = f(0 + 0) = f(0) + f(0)$

	$0 = f(0) = f(a + (-a)) = f(a) + f(-a)$

	Hence $-f(a) = f(-a)$
\end{proof}

\begin{definition}
	Two rings $R$ and $S$ are \textbf{isomorphic} if there exists a bijective homomorphism between $R$ and $S$. The map between them is an \textbf{isomorphism}. We write $R \cong S$.
\end{definition}

\begin{lemma}
	A homomorphism $f: R \rightarrow S$ is injective iff $\ker f = {0}$.
\end{lemma}

\begin{proof}
	If $f$ is injective, $f(x) = f(y) \Rightarrow x = y$. Assume $f$ is injective. $\ker f = {a \in \mathbb{R}: f(a) = 0}$ so $f(a) = 0 \Rightarrow f(a) = f(0) \Rightarrow a = 0$.

	For the other direction: assume $\ker f = {0}$. $f(x) = f(y) \Rightarrow f(x) - f(y) = 0 \Rightarrow f(x) + f(-y) = 0 \Rightarrow f(x - y) = 0 \Rightarrow x - y \in \ker f$. Since $\ker f = {0}$, $x - y = 0$ and so $x = y$.
\end{proof}

\begin{definition}
	Let $R$ and $S$ be two rings.
	\begin{itemize}
		\item The \textbf{product} of $R$ and $S$ is defined as $R \times S := \{(r, s): r \in R, s \in S\}$ which is itself a ring.
		\item \textbf{Addition} is defined as $(r_1, s_1) + (r_2, s_2) := (r_1 + r_2, s_1 + s_2)$.
		\item \textbf{Multiplication} is defined as $(r_1, s_1) \cdot (r_2, s_2) := (r_1 r_2, s_1 s_2)$
		\item The multiplicative identity is $(1, 1)$.
	\end{itemize}
\end{definition}

\begin{definition}
	We have two ring homomorphisms:
	\begin{enumerate}
		\item $p_1: R \times S \rightarrow R = (r, s) \rightarrow r$
		\item $p_2: R \times S \rightarrow S = (r, s) \rightarrow s$
	\end{enumerate}

	$\ker p_1 = \{(r, s) \in R \times S: p_1((r, s)) = 0\} = \{(r, s) \in R \times S: r = 0\} = \{(0, s): s \in S\}$
\end{definition}

\begin{remark}
	Note $\ker p_1$ is not a subring of $R \times S$ since $(1, 1) \notin \ker p_1$.

	But we can consider $\ker p_1$ as a ring by taking $(0, 1)$ as the multiplicative identity.

	Then $\ker p_1 \cong S$ as we map $(0, s) \rightarrow s$.

	Similarly, $\ker p_2 \cong R$ and so $\ker p_1 \times \ker p_2 \cong S \times R \cong R \times S$.
\end{remark}

\begin{lemma}
	Let $f: R \rightarrow S$ be a ring homomorphism. Then $\ker f$ has the following two properties:
	\begin{enumerate}
		\item $\ker f$ is closed under addition.
		\item For every $r \in R$ and $x \ker f$ we have $r \cdot x \in \ker f$ and $x \cdot r \in \ker f$.
	\end{enumerate}
\end{lemma}

\begin{proof}
	\hfill
	\begin{enumerate}
		\item If $x, y \in \ker f$ then $f(x + y) = f(x) + f(y) = 0 + 0 = 0$. That is $x + y \in \ker f$.
		\item For every $r \in R$ and $x \ker f$, $f(r \cdot x) = f(r) \cdot f(x) = f(r) \cdot 0 = 0$. Thus $r \cdot x \in \ker f$. Similarly for $x \cdot r$.
	\end{enumerate}
\end{proof}

\begin{definition}
	Let $I$ be an ideal in a ring $R$. Then for an element $x \in R$, the \textbf{coset} of $I$ generated by $x$ to be the set $\bar{x} := x + I := \{ x + r: r \in I \} \subset R$.

	$x$ is said to be a representative of this coset.
\end{definition}

\begin{lemma}
	Let $x \in R$ and $y \in R$. Then the following statements are equivalent
	\begin{enumerate}
		\item $x + I = y + I$
		\item $x + I \cap y + I \ne \emptyset$
		\item $x - y \in I$
	\end{enumerate}
\end{lemma}

\begin{proof}
	($(1) \Rightarrow (2)$) is obvious

	($(2) \Rightarrow (3)$): if $x + I \cap y + I \ne \emptyset$, for some $r_1 \in I, r_2 \in I$, $x + r_1 = y + r_2$ and so $x - y = r_2 - r_1 \in I$.

	($(3) \Rightarrow (1)$): since $x - y \in I$, for some $r' \in I$, $x = y + r'$. Then $x + I = \{x + r: r \in I\} = \{y + r' + r: r \in I\} \subseteq y + I$ as ideals are closed under addition, and $r' + r \in I$. $y + I = \{y + r: r \in I\} = {x - r' + r: r \in I} \subseteq x + I$ and so $x + I = y + I$.
\end{proof}

Notation: $\bar{x} = \bar{y} \Leftrightarrow x + I = y + I \Leftrightarrow x \equiv y \pmod I \Leftrightarrow x - y \in I$

\begin{definition}
	$R / I := \{\bar{x}: x \in R \} = \{x + I: x \in R\}$ is the set of all distinct cosets of $R \pmod I$ 
\end{definition}

\begin{remark}
	If $R = \mathbb{Z}$ and $I = (n)$, $n \in \mathbb{N}$, $R / I = \mathbb{Z} / n = \{\bar{0}, \dots, \bar{n - 1}\}$.
\end{remark}

\begin{definition}
	\hfill
	\begin{itemize}
		\item Addition: $(x + I) + (y + I) = x + y + I$
		\item Multiplication: $(x + I) \cdot (y + I) = xy + I$
	\end{itemize}
\end{definition}

A coset $x + I$ has many representatives, for example $x + r$ with $r \in I$ gives the same coset, since $x + r - x = r \in I$.

Assume $x, x' \in R$ such that $x + I = x' + I$ and $y, y' \in R$ such that $y + I = y' + I$.

\begin{proof}
	\begin{itemize}
		\item Addition: $x + I = x' + I \Leftrightarrow x - x' \in I$ and similarly $y - y' \in I$. $I$ is closed under addition so $(x - x') + (y - y') \in I \Leftrightarrow (x + y) - (x' + y') \in I \Leftrightarrow x + y + I = x' + y' + I$.
		\item $x - x' \in I$ and $y - y' \in I$, so $(x - x')y \in I$ and $x(y - y') \in I$. $(x - x')y + x(y - y') = xy - x'y' \in I \Leftrightarrow xy + I = x'y' + I$.
	\end{itemize}
\end{proof}

$R / I$ with the two binary operations of addition and multiplication is a ring:
\begin{itemize}
	\item The zero element is $0 + I$ as $(x + I) + (0 + I) = x + I$.
	\item The multiplicative identity is $1 + I$.
	\item All properties follow from the corresponding properties of $R$:
	\item e.g. distributivity: $\bar{x} = x + I$, $\bar{y} = y + I$, $\bar{z} = z + I$.
	$\bar{x}(\bar{y} + \bar{z}) = \bar{x}(\overline{y + z}) = \overline{x(y + z)} = \overline{xy + xz} = \overline{xy} + \overline{xz} = \overline{x}\overline{y} + \overline{x}\overline{z}$.
\end{itemize}

\begin{definition}
	Let $R$ be a ring, and $I \subseteq R$ be an ideal of $R$. Then the ring $R / I$ is called the \textbf{quotient} of $R$ by $I$ ($R$ mod $I$). Its elements, $x + I$, $x \in R$ are called cosets (or residue classes or equivalence classes) and we denote them $\bar{x}$.
	
	$R / I$ may be commutative or non-commutative, but if $R$ is commutative, so is $R / I$.

	If $I = R$, then $R / R$ consists of a single element, since for every $x \in R$, $y \in R$, we have $x - y \in R$ and hence $x + R = y + R$.

	If $I = 0 = {0}$ is the zero ideal, if $x \in R$, $x + I = x + 0 = x$. Hence $R / I = R$.
\end{definition}

\begin{definition}
	Given $R$, $I \subseteq R$ an ideal, the \textbf{quotient map} (or \textbf{canonical homomorphism}) is defined as $\Pi: R \rightarrow R / I = x \rightarrow \overline{x} = x + I$ and is a ring hoomomorphism.

	$\ker \Pi = \{r \in R: \overline{r} = \overline{0}\} = \{r \in R: r - 0 = r \in I\} = I$.
\end{definition}

Hence, given a ring $R$ and an ideal $I \subseteq R$, there exists a ring homomorphism ($\Pi$) such that $\ker \Pi = I$.

\begin{theorem}
	(First Isomorphism Theorem or FIT) Let $\phi: R \rightarrow S$ be a ring homomorphism. The map $\bar{\phi}: R / \ker \phi \rightarrow \text{Im } \phi = \bar{x} \rightarrow \phi(x)$ is well-defined and it is a ring isomorphism: $R / \ker \phi \cong \text{Im } \phi$.
\end{theorem}

\begin{proof}
	Let $x, x' \in R$ such that $\overline{x} = \overline{x'}$, i.e. $x + \ker \phi = x' + \ker \phi$. So $x - x' \in \ker \phi$, hence $\phi(x - x') = 0 \Leftrightarrow \phi(x) - \phi(x') = 0 \Leftrightarrow \phi(x) = \phi(x')$. Hence $\overline{\phi}$ is well-defined.

	\begin{enumerate}
		\item $\overline{\phi}(\bar{1}) = \phi(1) = 1$
		\item $\overline{\phi}(\bar{x} + \bar{y}) = \overline{\phi}(\overline{x + y}) = \phi(x + y) = \phi(x) + \phi(y) = \bar{\phi}(\bar{x}) + \bar{\phi}(\bar{y})$.
		\item Similarly, $\bar{\phi}(\bar{x}\cdot \bar{y}) = \bar{\phi}(\bar{x}) \cdot \bar{\phi}(\bar{y})$.
	\end{enumerate}

	Hence $\bar{\phi}$ is a ring homomorphism.

	$\bar{\phi}(\bar{x}) = 0 \Leftrightarrow \phi(x) = 0 \Leftrightarrow x \in \ker \phi \Leftrightarrow \bar{x} = 0$, hence $\ker \bar{\phi} = \{\bar{0}\}$.
	Let $y \in \text{Im } \phi \Leftrightarrow$ for some $x \in R$, $\phi(x) = y$. Hence $\bar{\phi}(\bar{x}) = \phi(x) = y$, hence $\bar{\phi}$ is also surjective, hence it is bijective.
\end{proof}

\begin{definition}
	Let $R$ be a commutative ring. An ideal $I \subseteq R$ is a \textbf{prime ideal} if $I \ne R$ ($I$ is proper) and for every $a, b \in R$ such that $a \cdot b \in I$ then $a \in I$ or $b \in I$.

	The ideal $I \ne R$ is \textbf{maximal} if the only ideals that contain $I$ is $I$ itself and $R$. i.e. there is no ideal $J$ such that $I \subsetneq J \subsetneq R$.
\end{definition}

\begin{theorem}
	Recall $x \in R$ is prime if $0 \ne x \notin R^{\times}$ and $x | ab \Rightarrow x | a$ or $x | b$.

	If $x$ is a prime element then $(x)$ is a prime ideal.
\end{theorem}

\begin{proof}
	$ab \in (x) \Rightarrow$ for some $r \in R$, $ab = rx \Rightarrow x | ab$ so because $x$ is prime, $x | a$ or $x | b$ so $a \in (x)$ or $b \in (x)$.
\end{proof}

\begin{lemma}
	Let $(x)$ be a non-zero prime ideal. The $x$ is a prime element.
\end{lemma}

\begin{proof}
	If $x | ab$, $ab \in (x)$, so because $(x)$ is a prime ideal, $a \in (x)$ or $b \in (x)$, so $x | a$ or $x | b$.
\end{proof}

\begin{remark}
	$x | a \Leftrightarrow a \in (x) \Leftrightarrow (a) \subseteq (x)$.

	This can be described as ``to divide is to contain''.
\end{remark}

\begin{corollary}
	The zero ideal $(0) = 0$ is a prime ideal iff $R$ is an integral domain, since an integral means $ab = 0 \Rightarrow a = 0 \text{ or } b = 0$.
\end{corollary}

\begin{theorem}
	Let $R$ be a commutative ring and $I \subseteq R$ an ideal.

	\begin{enumerate}
		\item $I$ is prime iff $R / I$ is an integral domain.
		\item $I$ is maximal iff $R / I$ is a field.
	\end{enumerate}
\end{theorem}

\begin{proof}
	\hfill
	\begin{enumerate}
		\item Assume $I$ is prime. Assume $\bar{a}\bar{b} = \bar{0}$ with $a, b \in R$, $\bar{a}, \bar{b} \in R / I$. $\bar{a}\bar{b} = \bar{0} \Rightarrow \overline{ab} = \bar{0} \Rightarrow ab \in I \Rightarrow a \in I \text{ or } b \in I \Rightarrow \bar{a} = \bar{0} \text{ or } \bar{b} = {0}$, hence $R / I$ is an integral domain.
		
		Now assume $R / I$ is an integral domain. $ab \in I \Rightarrow \overline{ab} = \bar{0}$. Since $R / I$ is an integral domain, $\bar{a} = \bar{0}$ or $\bar{b} = \bar{0} \Rightarrow a \in I \text{ or } b \in I$.

		\item ($\Rightarrow$): Assume that $I$ is maximal. Let $\bar{x} \ne \bar{0}$, $\bar{x} \in R / I$, then $x \in R$ with $x \notin I$. Consider $(I, x) := \{ r + r'x: r \in I, r' \in R \}$. This is an ideal, as $r_1 + r_1' x + r_2 + r_2'x = (r_1 + r_2) + (r_1' + r_2')x \in R$, and $r'' (r + r'x) = r''r + r''r'x \in R$.
		
		$I \subsetneq (I, x) \subseteq R$. $I$ is maximal so $(I, x) = R \Rightarrow 1 \in (I, x)$. Hence for some $y \in R$, $yx + m = 1$ for some $m \in I$.

		Hence $yx - 1 \in I \Rightarrow \overline{yx} = \bar{y}\bar{x} = \overline{1}$ hence $\bar{x}$ is invertible, so $R / I$ is a field.

		($\Leftarrow$): Assume $R / I$ is a field. If $\bar{0} \ne \bar{x} \in R / I$, then for some $y \in R / I$, $\bar{x} \bar{y} = 1 \Rightarrow xy - 1 \in I \Rightarrow xy = 1 + m$ for some $m \in I$. That is, $1 = xy - m$ hence $1 \in (I, x) \Rightarrow (I, x) = R$.

		Now let $J$ be an ideal such that $I \subsetneq J \subseteq R$. Since $I \subsetneq J$, for some $x \in J$, $x \notin I$. Then $I \subsetneq (I, x) \subseteq J \subseteq R$. But $(I, x) = R$, hence $J = R$. Hence there is no ideal $J$ such that $I \subsetneq J \subsetneq R$, hence $I$ is maximal.
	\end{enumerate}
\end{proof}

\begin{corollary}
	If $I$ is maximal then $I$ is prime.
\end{corollary}

\begin{proof}
	$I$ is maximal $\Rightarrow R / I$ is a field $\Rightarrow R / I$ is an integral domain $\Rightarrow I$ is a prime ideal.
\end{proof}

\subsection{Principal Ideal Domains (PIDs)}

\begin{example}
	Let $a, b \in \mathbb{Z}$. Then let $d = (a, b) = \gcd(a, b)$. $(a, b) \subseteq (d)$ since $d | a$ and $d | b \Leftrightarrow a = d r_1$ and $b = d r_2$, $r_1, r_2 \in \mathbb{Z} \Rightarrow a \in (d)$ and $b \in (d)$.

	Moreover, for some $r_1, r_2 \in \mathbb{Z}$, $d = r_1 + r_2 b \Rightarrow d \in (a, b) \Rightarrow (d) \subseteq (a, b)$.


	The same argument holds for $F[x]$ with $F$ a field.

	i.e. $(f(x), g(x)) = (\gcd(f(x), g(x)))$.
\end{example}

\begin{definition}
	An integral domain in which \textbf{all} ideals are principle is called a \textbf{principle ideal domain (PID)}.
\end{definition}

\begin{theorem}
	Let $R$ be a either $\mathbb{Z}$ or $F[x]$ with $F$ a field. Then $R$ is a PID.
\end{theorem}

\begin{proof}
	Define the following ``degree'' function $d: R \backslash \{0\} \rightarrow \mathbb{N}$ by
	\[
		d(a) := \begin{cases}
			|a| & \text{ if } a \in \mathbb{Z} \\
			\deg(a) & \text{ if } a \in F[x]
		\end{cases}
	\]

	By division, for every $a, m \in R \backslash \{0\}$, we can find unique $q, R \in R$ such that $a = qm + r$ with $r = 0$ of $d(r) < d(m)$.

	Let $I \subseteq R$ be an ideal. If $I = 0 = \{0\}$ we are done. So now let $I \ne 0$. Let $0 \ne m \in I$ such that $d(m)$ is minimal among elements of $I$. We claim that $I = (m)$.

	Let $a \in I$. $a \in (m) \Leftrightarrow m | a$. Dividing $a$ by $m$, we get $a = qm + r$, with $r = 0$ or $d(r) < d(m)$. But since $r = a - qm \in I$, $d(r) < d(m)$ would contradict the minimality of $d(m)$. Hence $r = 0$, so $m | a \Leftrightarrow a \in (m)$. $(m) \subseteq I$ so $a \in I \Leftrightarrow a \in (m)$.
\end{proof}

\begin{theorem}
	(Stated without proof) Any PID is a UFD.
\end{theorem}

\begin{remark}
	There are integral domains which are not PIDs, e.g. $\mathbb{Z}[\sqrt{-5}]$ which is not a UFD and hence not a PID.
\end{remark}

\begin{proposition}
	Let $R$ be a PID and $a, b \in R$. Then $\gcd(a, b)$ exists and $(a, b) = (\gcd(a, b))$.
\end{proposition}

\begin{proof}
	Since $R$ is a PID, for some $d \in R$, $(a, b) = (d)$. We claim that $d = \gcd(a, b)$.

	$(a, b) = (d) \Rightarrow a \in (d) \text{ and } b \in (d) \Rightarrow d | a \text{ and } d | b$. Suppose $e \in R$ such that $e | a \Rightarrow a \in (e)$ and $e | b \Rightarrow b \in (e)$. $(d) = (a, b) \subseteq (e) \Rightarrow e | d$. Therefore $d = \gcd(a, b)$.
\end{proof}

\begin{theorem}
	(Stated without proof): $\mathbb{Z}[i], \mathbb{Z}[\pm \sqrt{2}]$ are PID's.
\end{theorem}

\begin{lemma}
	Let $R$ be a PID and let $a \in R$ be irreducible. Then the principle ideal genereated by $a$ is a maximal ideal.
\end{lemma}

\begin{proof}
	Suppose $(a) \subseteq I$, with $I$ an ideal. We must show $I = (a)$ or $I = R$. Since $R$ is a PID, for some $t \in R$, $I = (t)$. So $(a) \subseteq (t)$ so for some $m \in R$, $a = t m$. But $a$ is irreducible, so either $t$ is a unit or $m$ is a unit.

	If $t \in R^{\times}$ then $I = (t) = R$. If $m \in R^{\times}$ then $(a) = (t) = I$ (last question of assignment 3).
\end{proof}

\subsection{Fields on quotients}

\begin{theorem}
	Let $F$ be a field and $f(x) \in F[x]$, with $f(x)$ irreducible. Then $F[x] / (f(x))$ is a field and a vector space over $F$ with basis

	\[ B := \{\bar{1}, \bar{x}, \bar{x}^2, \ldots, \bar{x}^{n - 1} \} \] where $n = \deg f$.

	That is, every element of $F[x] / (f(x))$ can be uniquely written as

	\[ \overline{a_0 1 + a_1 x + \cdots + a_{n - 1} x^{n - 1}} \]
\end{theorem}

\begin{proof}
	Since $f(x)$ is irreducible, $F[x] / (f(x))$ is a field. $F[x]/(f(x))$ is a vector space over $F$ and an abelian group with respect to addition and scalar multiplication with elements of $F$: if $\overline{g(x)} \in F[x] / (f(x))$ and $\alpha \in F$ then $\alpha \overline{g(x)} = \overline{\alpha g(x)} \in F[x] / (f(x))$.

	We must prove $B$ spans $F[x] / (f(x))$. For every $\overline{g(x)} \in F[x] / (f(x))$, $g(x) = q(x) f(x) + r(x)$ with $\deg(r) < \deg(f) = n \Rightarrow g(x) - r(x) = q(x) f(x) \in (f(x)) \Rightarrow \overline{g(x)} = \overline{r(x)}$, $\deg(r) < n$. Hence $\overline{g(x)} = \overline{r(x)} = a_0 + a_1 \bar{x} + \cdots + a_{n - 1} \bar{x}^{n - 1}$ with $a_i \in F$. Hence $B$ spans $F[x] / (f(x))$.

	We must show $B$ is linearly independent over F, i.e. show if $\sum_{i = 0}^{n - 1} a_i \bar{x}^i = \bar{0}$ then $\forall i, a_i = 0$.

	$\sum_{i = 0}^{n - 1} a_i \bar{x}^i = \bar{0} \Leftrightarrow \sum_{i = 0}^{n - 1} a_i x^i \in (f(x)) \Rightarrow f(x) | \sum_{i = 0}^{n - 1} a_i x^i$. But $\deg(f) = n$ and $\deg(\sum_{i = 0}^{n - 1} a_i x^i) < n$ so $\sum_{i = 0}^{n - 1} a_i x^i$ is the zero polynomial so $\forall i, a_i = 0$. Therefore $B$ is linearly independent.

	So $B$ is a basis.
\end{proof}

\section{Finite fields}

\begin{theorem}
	For every prime $p$ and $n \in \mathbb{N}$, for some irreducible polynomial $f(x) \in (\mathbb{Z} / p)[x]$, $\deg(f) = n$. Thus $(\mathbb{Z} / p)[x] / (f(x))$ is a field with $p^n$ elements (since there are $p$ choices for each $a_i$ in $a_0 + a_1 \bar{x} + \cdots + a_{n - 1}\bar{x}^{n - 1}$).

	Any two such fields are isomorphic and we denote the unique, up to isomorphism, field with $p^n$ elements with $\mathbb{F}_{p^n}$.
\end{theorem}

\begin{proof}
	Not examinable.
\end{proof}

\begin{remark}
	If $n = 1$ then $\mathbb{F}_p \cong \mathbb{Z} / p$ with $p$ prime. However if $n > 1$ then $\mathbb{F}_{p^n} \not\cong \mathbb{Z} / p^n$ since $\mathbb{Z} / p^n$ is not a field.
\end{remark}

\begin{example}
	Find an irreducible polynomial $f$ in $(\mathbb{Z} / 3)[x]$ of degree $3$.
	
	$f(x) = x^3 + x^2 + x + \bar{2}$. This has no roots in $\mathbb{Z} / 3$ so $f(x)$ is irreducible since $\deg(f) = 3$. Then $\mathbb{F}_{27} = \mathbb{F}_{3^3} \cong (\mathbb{Z} / 3) [x] / (f(x))$. All elements can be written as $a_0 + a_1 \bar{x} + a_2 \bar{x}^2$, $a_i \in \mathbb{Z} / 3$.

	$\overline{f(x)} = \bar{0} = \overline{x^3 + x^2 + x + \bar{2}} \Rightarrow \bar{x}^3 = - \bar{x}^2 - \bar{x} - \bar{2}$.
\end{example}

\subsection{The Chinese Remainder Theorem (CRT)}

\begin{definition}
	Let $a, b \in R$. $a$ and $b$ are \textbf{coprime} if $\not\exists r$ irreducible in $R$ such that $r | a$ and $r | b$.
\end{definition}

\begin{lemma}
	Let $R$ be a PID and $a, b \in R$ be coprime. Then $(a, b) = R$ and hence $\exists x, y \in R$ such that $xa + yb = 1$.
\end{lemma}

\begin{proof}
	Since $R$ is a PID, $(a, b) = (r)$ for some $r \in R$. So $a, b \in (r) \Rightarrow r | a \text{ and } r | b$. So $a = rn$ and $b = rm$ for some $n, m \in R$. $r$ must be a unit in $R$ since otherwise, $r = p_1 \cdots p_k$ for some $p_i$ irreducible, but then $a = p_1 \cdots p_k n$, $b = p_k \cdot p_k m$, which would contradict $a$ and $b$ being coprime.

	So $r \in R^{\times} \Rightarrow (r) = R \Rightarrow (a, b) = R$.
\end{proof}

\begin{corollary}
	For $a, b \in R$ coprime, any $\gcd(a, b) \in R^{\times}$.
\end{corollary}

\begin{proof}
	In a PID, $(a, b) = (\gcd(a, b))$. By the lemma above, if $a$ and $b$ are coprime, $(a, b) = R \Rightarrow (\gcd(a, b)) = R = (1) \Rightarrow \gcd(a, b) \in R^{\times}$.
\end{proof}

\begin{theorem}
	(CRT for PID's) Let $R$ be a PID and let $a_1, \ldots, a_k \in R$ be pairwise coprime elements. Then the map from $R / (a_1, \ldots, a_k) \rightarrow R / (a_1) \times \cdots \times R / (a_k)$ given by $r + (a_1, \dots, a_k) \rightarrow (r + (a_1), \dots, r + (a_k))$ is a ring isomorphism.
\end{theorem}

\begin{proof}
	Let $\psi: R \rightarrow R / (a_1) \times \cdots \times R / (a_k)$, $\psi(r) = (r + (a_1), \dots, r + (a_k))$. Clearly, $\psi$ is a ring homomorphism. 
	
	For every $i = 1, 2, \dots, k$, the elements $a_i$ and $a_1 \dots a_{i - 1} a_{i + 1} \dots a_k$ are coprime. (If not, there exists an irreducible $p$ such that $p | a_i$ and $p | a_1 \dots a_{i - 1} a_{i + 1} \dots a_k$. But then $p \text{irreducible} \Leftrightarrow p \text{ prime}$ hence $p | a_j$ for some $j \ne i$, but this contradicts that $a_i$ and $a_j$ are coprime).

	By the above lemma, for some $x_i, y_i \in R$, $x_i a_i + y_i (a_1 \dots a_{i - 1} a_{i + 1} \dots a_k) = 1$. Set $e_i := 1 - a_i x_i$ for each $i = 1, \dots, k$. Then $e_i = 1 + (a_i)$ and $e_i = 0 + (a_j)$ for $j \ne i$, since $e_i = 1 - a_i x_i = y_i (a_1 \dots a_{i - 1} a_{i + 1} \dots a_k)$.

	Let $(r_1 + (a_1), \dots, r_k + (a_k))$ be any element in $R / (a_1) \times \cdots \times R / (a_k)$. We claim that

	\[ \psi \left( \sum_{i = 1}^k r_i e_i \right) = (r_1 + (a_1), \dots, r_k + (a_k)) \]

	\[ \psi \left( \sum_{i = 1}^k r_i e_i \right) = \sum_{i = 1}^k \psi(r_i e_i) = \sum_{i = 1}^k \psi(r_i) \psi(e_i) \]

	\[ \psi(e_1) = (0 + (a_1), \dots, 1 + (a_i), 0 + (a_{i + 1}), \dots, 0 + (a_k)) \]
	since $e_i = 1 + (a_i)$ and $e_i = 0 + (a_j)$ for $j \ne i$ and

	\[ \psi(r_i) = (r_i + (a_1), \dots r_i + (a_k)) \] so

	\[ \psi(e_i) \psi(r_i) = TODO finish and check this proof \]
	Thus $\psi$ is surjective.
	$\ker \psi = \{ r \in R: r \in (a_i), i = 1, \dots, k \} = \{ r \in R: a_i | r, i = 1, \dots, k \} = \{ r \in R: a_1 \dots a_k | r \}$ since $a_i$ and $a_j$ are coprime.
	$\ker \psi = (a_1 a_2 \dots a_k)$. Then by the FIT, $R / \ker \psi \cong R / (a_1) \times \cdots \times R / (a_k)$.
\end{proof}

\section{Group Theory}

\begin{definition}
	A \textbf{group} is a pair $(G, \circ)$ where $G$ is a set and $\circ$ is a map
	\[
		\circ: G \times G \rightarrow G, \quad \circ (g, h) = g \circ h
	\]
	Satisfying these properties:
	\begin{enumerate}
		\item \textbf{Closure}: $g, h \in G \Rightarrow g \circ h \in G$.
		\item \textbf{Associativity}: $x, y, z \in G \Rightarrow (x \circ y) \circ z = x \circ (y \circ z)$.
		\item \textbf{Identity element}: $\exists e \in G, \ \forall g \in G, \ e \circ g = g \circ e = g$.
		\item \textbf{Existence of inverse}: $\forall g \in G, \ \exists h \in G, \ g \circ h = h \circ g = e$. $h$ is called the \textbf{inverse} of $g$ and is written as $g^{-1}$.
	\end{enumerate}
\end{definition}

\begin{definition}
	A group $(G, \circ)$ is an \textbf{Abelian group} if $\forall g, h \in G, \ g \circ h = h \circ g$. Otherwise, it is called \textbf{non-Abelian}.
\end{definition}

\begin{remark}
	Often, $G$ is written to refer to a group, not just the set of a group.
\end{remark}

\begin{lemma}
	Let $(R, +, \cdot)$ be a ring. Then $(G, \circ) = (R, +)$ is a group.
\end{lemma}

\begin{proof}
	Properties 1 and 2 of a group are automatically satisfied. The identity element is $0 \in R$. The inverse element for any element will be the same inverse element in the ring.
\end{proof}

\begin{lemma}
	Let $(F, +, \cdot)$ be a field. Then $(G, \circ) = (R, \cdot)$ is a group.
\end{lemma}

\begin{proof}
	Again, group properties 1 and 2 are automatic. The identity element is $1 \in F$. The inverse element for any element will be the same inverse element in the field.
\end{proof}

\begin{example}
	(\textbf{Symmetries of a square}): The following are all symmetries of a square:
	\begin{itemize}
		\item Rotation by $\frac{\pi}{2}$.
		\item Reflection about the $y$-axis, $x$-axis, $y = x$ axis, $y = -x$ axis.
		\item Any of the above symmetries can be combined to form a new symmetry.
	\end{itemize}
	Define the group $G(, \circ)$ where $G$ is the symmetries of the square and $\circ$ is composition of the symmetries. The identity $e$ is the map which does nothing to the square. The inverse of a rotation is rotation in the opposite direction, and the inverse of a reflection is the same reflection.
\end{example}

\begin{definition}
	The group in the above example is the \textbf{dihedral group}.
\end{definition}

\begin{definition}
	The \textbf{general linear group} is defined as the set $GL_2 (\mathbb{R}) := \{ A \in M_2 (\mathbb{R}) : \det A \ne 0 \}$ together with $\circ$ being matrix multiplication.
\end{definition}

\begin{lemma}
	The general linear group is a group.
\end{lemma}

\begin{proof}
	\hfill
	\begin{enumerate}
		\item $\det (A B) = \det A \det B \ne 0$ so $A, B \in GL_2(\mathbb{R}) \Rightarrow AB \in GL_2(\mathbb{R})$.
		\item Matrix multiplication is associative.
		\item The identity is $I_2$.
		\item The inverse of $A \in GL_2(\mathbb{R})$ is $A^{-1}$, which exists since $\det A \ne 0$.
	\end{enumerate}
\end{proof}

\begin{remark}
	$GL_2(\mathbb{R})$ is non-abelian.
\end{remark}

\subsection{Subgroups}

\begin{definition}
	A subset $H \subseteq G$ is a \textbf{subgroup} of $(G, \circ)$ if $(H, \circ)$ is also a group. We write $H \le G$.
\end{definition}

\begin{remark}
	$H = G$ is a subgroup of a group $G$.
\end{remark}

\begin{definition}
	Every group $(G, \circ)$ has a \textbf{trivial subgroup}, $H = \{ e \}$, where $e \in G$ is the identity element.
\end{definition}

\begin{definition}
	A subgroup $H$ of $G$ is \textbf{proper} if $H \ne \{ e \}$ and $H \ne G$. We write $H < G$.
\end{definition}

\begin{proposition}
	(\textbf{Subgroup criteria}) Let $(G, \circ)$ be a group. Then $H \subseteq G$ is a subgroup iff all these conditions hold:
	\begin{enumerate}
		\item $H \ne \emptyset$
		\item $h_1, h_2 \in H \Rightarrow h_1 \circ h_2 \in H$.
		\item $h \in H \Rightarrow h^{-1} \in H$.
	\end{enumerate}
\end{proposition}

\begin{proof}
	We only need to show that $H$ contains an identity: $h \in H \Rightarrow h^{-1} \in H \Rightarrow e = h \circ h^{-1} \in H$.
\end{proof}

\begin{example}
	If $(S, +, \cdot)$ is a subring, then $(S, +)$ is a subgroup.
\end{example}

\begin{proposition}
	Let $I \subseteq R$ be a non-empty ideal of a ring $(R, +, \cdot)$. Then $(I, +)$ is a subgroup of $(R, +)$.
\end{proposition}

\begin{proof}
	Criteria 1 and 2 are satisfied by definition. Now we must show that $x \in I \Rightarrow -x \in I$: if $x \in I$, then $(-1_R) x = -x \in I$ where $-1_R + 1_R = 0_R$.
\end{proof}

\begin{definition}
	The \textbf{special linear group} is defined as $SL_2(\mathbb{R}) = \{ A \in M_2(\mathbb{R}): \det A = 1 \}$, which satisfies $(SL_2(\mathbb{R}), \cdot) \le (Gl_2(\mathbb{R}), \cdot)$.
\end{definition}

\begin{example}
	Let $q \in \mathbb{N}$, then $q \mathbb{Z} = \{ m q: m \in \mathbb{Z} \}$ is an ideal in $\mathbb{Z}$. For example, the even numbers, $2 \mathbb{Z}$, is a subgroup.

	However, the odd numbers are not subgroup, as they do not contain $0$, nor is $\bar{a} = \{ a + mq: m \in \mathbb{Z} \}$ for $1 \le a \le q - 1$.
\end{example}

\subsection{Cosets}

\begin{definition}
	Let $(G, \circ)$ be a group and $H \le G$. A \textbf{left coset} of $H$ is a set of the form
	\[
		g \circ H := \{ g \circ h: h \in H \} \quad \text{for } g \in G
	\]
	A \textbf{right coset} of $H$ is a set of the form
	\[
		H \circ g := \{ h \circ g: h \in H \} \quad \text{for } g \in G
	\]
\end{definition}

\begin{remark}
	$x \in g \circ H \Longleftrightarrow g^{-1} \circ x \in H$.
\end{remark}

\begin{remark}
	If $G$ is Abelian, then $g \circ H = H \circ g$, but this isn't true in general for non-Abelian groups.
\end{remark}

\begin{proposition}
	Let $(G, \circ)$ be a group and $H \le G$. Then:
	\begin{enumerate}
		\item For every $g \in G$, $g \circ H$ and $H$ are in bijection. (So $|H| < \infty \Rightarrow |g \circ H| = |H|$).
		\item If $g \in G$, then $g \in H \Longleftrightarrow g \circ H = H$.
		\item If $g_1, g_2 \in G$, then either $g_1 \circ H = g_2 \circ H$ or $(g_1 \circ H) \cap (g_2 \circ H) = \emptyset$.
	\end{enumerate}
\end{proposition}

\begin{proof}
	\hfill
	\begin{enumerate}
		\item Let $g \in G$. Define $\phi_g: H \rightarrow g \circ H$ as
		\[
			\phi_g(h) := g \circ h
		\]
		$\forall x \in g \circ H, \exists h_x \in H, x = g \circ h_x = \phi_g(h_x)$ so $\phi_g$ is surjective. Let $h_1, h_2 \in H$ such that $\phi_g(h_1) = \phi_g(h_2) \Leftrightarrow g \circ h_1 = g \circ h_2 \Rightarrow h_1 = e \circ h_1 = (g^{-1} \circ g) \circ h_1 = g^{-1} \circ (g \circ h_1)$. Similarly, $h_2 = e \circ h_2 = (g^{-1} \circ g) \circ h_2 = g^{-1} \circ (g \circ h_2)$. Hence $h_1 = h_2$, so $\phi_g$ is injective, and so also bijective.
		\item ($\Rightarrow$) Let $g \in H$. If $h \in H$, then $g \circ h \in H \Longrightarrow g \circ H \subseteq H$. To show that $H \subseteq g \circ H$, we will show that if $h \in H$, then $\exists h' \in H, h = g \circ h' \in g \circ H \Longleftrightarrow h' = g^{-1} \circ h \in H \Longleftrightarrow h = g \circ (g^{-1} \circ h) \in g \circ H \Longleftrightarrow H \subseteq g \in H$.
		($\Leftarrow$) If $g \circ H = H$, $g = g \circ e \in g \circ H$ since $e \in H$, hence $g \in H$.
		\item Let $(g_1, g_2) \in G^2$ and assume that $g_1 \circ H \ne g_2 \circ H$, and that $(g_1 \circ H) \cap (g_2 \circ H) \ne \emptyset$. Let $x \in (g_1 \circ H) \cap (g_2 \circ H)$, then $\exists (h_1, h_2) \in H^2$, $x = g_1 \circ h_1 = g_2 \circ h_2 \Longleftrightarrow g_2^{-1} \circ g_1 = h_2 \circ h_1^{-1} \in H$. By part 2, $(g_2^{-1} \circ g_1) \circ H = H \Longrightarrow g_1 \circ H = g_2 \circ H$, but this is a contradiction, which completes the proof.
	\end{enumerate}
\end{proof}
		
\begin{theorem}
	(Lagrange) If $G$ is a \textbf{finite} group and $H \le G$, then $|H|$ divides $|G|$. So if $|H| \nmid |G|$ then $H \not\le G$.
\end{theorem}

\begin{proof}
	Let $G_0 = G$ and let $G_1 = G_0 \backslash H$. If $|G_1| = 0$, we are done, otherwise for some $g_1 \in G$, $H \cap g_1 \circ H = \emptyset$. Then set $G_2 = G_1 \backslash G_1 \backslash (g_1 \circ H)$. If $|G_2| = 0$, we are done, otherwise for some $g_2 \in G$, $(H \cup (g_1 \circ H)) \cap (g_2 \circ H) = \emptyset$, and set $G_3 = G_2 \backslash (g_2 \circ H)$.

	This process must terminate since $|g_i \circ H| = |H| \ge 1$ elements are removed each time. At the end of this process, for some $S \subseteq G$,
	\[
		G = \bigcup_{g \in S} (g \circ H)
	\]
	and for $g, g' \in S$, $g \circ H \cap g' \circ H = \emptyset$. So
	\[
		|G| = \left| \bigcup_{g \in S} (g \circ H) \right| = \sum_{g \in S} | g \circ H |
	\]
	Since $|g \circ H| = |H| \forall g \in S$, $|G| = |S| |H| \Longrightarrow |H| \mid |G|$.
\end{proof}

\subsection{Normal subgroups}

\begin{definition}
	A subgroup $H \le G$ is \textbf{normal} if $\forall g \in G, \ g \circ H = H \circ g$. Equivalently, $H$ is normal if either:
	\begin{enumerate}
		\item $\forall g \in G, \ g \circ H \circ g^{-1} \subseteq H$.
		\item $\forall g \in G, h \in H, \ g \circ h \circ g^{-1} \in H$.
	\end{enumerate}
	We write $H \triangleleft G$.
\end{definition}

\begin{remark}
	This means that $\forall h \in H, \ \exists h' \in H, g \circ h = h' \circ g$, but $h \ne h'$ in general.
\end{remark}

\begin{example}
	If $G$ is \textbf{abelian}, then every subgroup $H \le G$ is normal, since if $g \in G, h \in H$, then $g \circ h \circ g^{-1} = g \circ (g^{-1} \circ h) = h \in H$.
\end{example}

\begin{definition}
	For a group $G$ and $g \in G$, $g^k$ for $k \in \mathbb{Z}$ is defined as
	\[
		g^k = \begin{cases}
			g \circ g \circ \cdots \circ g \quad (k \text{ times}) & \text{ if } k \ge 1 \\
			g^{-1} \circ g^{-1} \circ \cdots \circ g^{-1} \quad (-k \text{ times}) & \text{ if } k < 0 \\
			e & \text{ if } k = 0
		\end{cases}
	\]
\end{definition}

\begin{definition}
	For a group $G$ and $g \in G$, the \textbf{group generated by $g$}, $H$, is defined as
	\[
		H := \langle g \rangle = \left\{ g^k: k \in \mathbb{Z} \right\}
	\]
\end{definition}

\begin{proposition}
	$H$ is a Abelian group.
\end{proposition}

\begin{proof}
	\hfill
	\begin{enumerate}
		\item $g^{n + m} = g^n \circ g^m = g^m \circ g^n$.
		\item $g^{-n} = {\left( g^n \right)}^{-1}$.
	\end{enumerate}
\end{proof}

\begin{definition}
	Let $S \subseteq G$ be finite, so $S = \{ g_1, \dots, g_k \}$. The \textbf{subgroup of $G$ generated by $S$} is defined as
	\[
		H := \langle S \rangle = \{ g_1^{a_1} \circ \dots \circ g_k^{a_k} \circ g_1^{b_1} \circ \dots \circ g_k^{b_k}: a_i, b_j \in \mathbb{Z}^2 \}
	\]
	$H$ is the set of finite products of $g_i$ and $g_j^{-1}$, for $1 \le i, j \le k$.
\end{definition}

\begin{example}
	Let $q \in \mathbb{N}$ be odd, so $\bar{2} \in \mathbb{Z} / q$. Then $\langle \bar{2} \rangle = \mathbb{Z} / q$, since every $\bar{a} \in \mathbb{Z} / q$ is of the form $\bar{2} \cdot x, x \in \mathbb{Z}$.
\end{example}

\begin{example}
	Let $q = p^2$ for $p$ prime. Then $\langle \bar{p} \rangle = \{ \bar{p}, \overline{2p}, \dots, \overline{p(p - 1)}, \bar{0} \}$.
\end{example}

\begin{example}
	Let $(G, \circ) = (\mathbb{R}^{\times}, \cdot)$ and $S = \{ \sqrt{2}, \pi \}$. Then $\langle S \rangle = \{ \sqrt{2}^a \cdot \pi^b: a, b \in \mathbb{Z}^2 \}$. Since $(\mathbb{R}^{\times}, \cdot)$ is Abelian.
\end{example}

\begin{definition}
	Let $G$ be a group, and let $g \in G$. The \textbf{order} of $g$ in $G$, written as $\text{ord}_G (g)$ or $\text{ord}(g)$ is the smallest $d \in \mathbb{N}$ such that $g^d = e$.

	If $d$ does not exist, $\text{ord}_G(g) = \infty$. If $\text{ord}_G(g) < \infty$, $g$ has \textbf{finite order}, otherwise, $g$ has \textbf{infinite order}.
\end{definition}

\begin{example}
	For $(G, \circ) = (\mathbb{Z}, +)$, every $x \in \mathbb{Z} - \{ 0 \}$ has infinite order, because $x + \cdots + x = dx = 0$, and since $\mathbb{Z}$ is an integral domain, $d = 0$, but $d \in \mathbb{N}$.
\end{example}

\begin{example}
	In $D_4$, the symmetries of a square,
	\begin{itemize}
		\item The rotation by $\frac{\pi}{2}$, $r$, has $\text{ord} (r) = 4$.
		\item Reflection, $s$, has $\text{ord}(s) = 2$.
	\end{itemize}
\end{example}

\subsection{Cyclic groups}

\begin{definition}
	A group $G$ is \textbf{cyclic} if $\exists g \in G, G = \langle g \rangle$.
\end{definition}

\begin{theorem}
	Let a group $G$ be finite and let $|G| = p$ for $p$ prime. Then $G$ is cyclic.
\end{theorem}

\begin{proof}
	Since $|G| = p > 1$, $\exists g \in G, g \ne e$. Let $H = \langle g \rangle$, so $H \le G$. By Lagrange's theorem, $|H| \mid |G|$. Since $|G|$ is prime, $|H| = 1$ or $|H| = p$. Since $\{ e, g \} \subset H$, $|H| \ge 2$, so $|H| = p$. $H \subseteq G$, so $G = H = \langle g \rangle$.
\end{proof}

\begin{remark}
	For every $g \ne e$ in $G$ of prime order, $G = \langle g \rangle$, and $\text{ord}_G(g) = p$.
\end{remark}

\subsection{Permutation groups}

\begin{definition}
	A \textbf{permutation} of a non-empty set $X$ is a bijection from $X$ to itself. We define $S_X$ to be the set of all bijections from $X$ to itself. For $n \ge 1$, we write
	\[
		S_n = S_{\{ 1, \dots, n \}}
	\]
\end{definition}

\begin{lemma}
	$(S_X, \circ)$ is a group where $\circ$ is the composition of bijections.
\end{lemma}

\begin{proof}
	Associativity and closure are automatic due to the associativity and closure of composition of functions. The identity element is the identity function on $X$. The inverse of a bijection is given by reversing it: for a permutation $\sigma(x): X \rightarrow X$ where $\sigma(x) = y$ its inverse is given by $\sigma^{-1}$ where
	\[
		\sigma^{-1}(y) = x
	\]
\end{proof}

\begin{lemma}
	$\forall n \ge 1, |S_n| = n!$.
\end{lemma}

\begin{proof}
	There are $n$ choices to map $1$ to, then $n - 1$ choices to map $2$ to, etc., and $1$ choice to map $n$ to. So there are $n (n - 1) \cdots 1$ choices in total.
\end{proof}

\begin{definition}
	$(S_n, \circ)$ is called the \textbf{symmetric group of degree $n$} (or \textbf{symmetric group on $n$ letters}).
\end{definition}

\begin{definition}
	For a permutation $\phi: \{ 1, \dots, n \} \rightarrow \{ 1, \dots, n \}$, we can write $\phi$ as
	\[
		\begin{pmatrix}
			1 & 2 & \cdots & n \\
			\phi(1) & \phi(2) & \cdots & \phi(n)
		\end{pmatrix}
	\]
\end{definition}

\begin{definition}
	Some permutations in $S_n$ can be subdivided into simpler units called \textbf{cycles}. Let $n \ge 1$ and $1 \le k \le n$. A \textbf{$k$-cycle} is an element $\sigma \in S_n$ which satisfies, with $I = \{ i_1, i_2, \dots, i_k \} \subseteq \{ 1, \dots, n \}$:
	\begin{enumerate}
		\item $\sigma(i_1) = i_2, \sigma(i_2) = i_3, \dots, \sigma(i_{k - 1}) = i_k$ and $\sigma(i_k) = i_1$ and
		\item if $i \notin I$, $\sigma(i) = i$.
	\end{enumerate}
	We often denote a $k$-cycle $\sigma \in S_n$ as
	\[
		(i_1 \ i_2 \ \dots \ i_k)
	\]
	or equivalently,
	\[
		(i_2 \ i_3 \ \dots \ i_k \ i_1)
	\]
	etc.
\end{definition}

\begin{definition}
	A $2$-cycle is called a \textbf{transposition}.
\end{definition}

\begin{definition}
	Let $n \ge 1$, and $\sigma, \tau \in {(S_n)}^2$ be cycles. $\sigma$ and $\tau$ are called \textbf{disjoint} if their associated index sets, $\{ i_1, \dots, i_k \} = I$ for $\sigma$ and $\{ j_1, \dots, j_l \} = J$ for $\tau$ are disjoint, so
	\[
		I \cap J = \emptyset
	\]
\end{definition}

\begin{example}
	$(1 \ 3 \ 5)$ and $(2 \ 4)$ are disjoint, while $(1 \ 3 \ 5)$ and $(1 \ 2 \ 4)$ are not.
\end{example}

\begin{remark}
	$S_n$ is not an abelian group. For example, if $\sigma = (1 \ 2)$ and $\tau = (2 \ 3)$,
	\[
		\begin{aligned}
			(\sigma \circ \tau)(1) & = 2 \quad (\tau \circ \sigma)(1) = 3 \\
			(\sigma \circ \tau)(2) & = 3 \quad (\tau \circ \sigma)(2) = 1 \\
			(\sigma \circ \tau)(3) & = 1 \quad (\tau \circ \sigma)(3) = 2
		\end{aligned}
	\]
\end{remark}

\begin{lemma}
	If $\sigma, \tau \in {(S_n)}^2$ are disjoint cycles then $\sigma \circ \tau = \tau \circ \sigma$.
\end{lemma}

\begin{proof}
	Let $1 \le k \le n$. Let $T$ be the set of indices changed by $\tau$ and $S$ be the set of indices changed by $\sigma$.
	\begin{itemize}
		\item Let $k \in T$, so $k \notin S$. Then $\tau(k) \notin S$, so $(\sigma \circ \tau)(k) = \tau(k)$ and $(\tau \circ \sigma)(k) = \tau(k)$.
		\item Similarly, if $k \in S$, then $k \notin T$. So $(\tau \circ \sigma)(k) = \sigma(k)$ and $(\sigma \circ \tau)(k) = \sigma(k)$.
		\item The remaining case is that $k \notin S \cup T$. Then $\tau(k) = \sigma(k) = k$ so $(\sigma \circ \tau)(k) = \sigma(k) = k$ and $(\tau \circ \sigma)(k) = \tau(k) = k$.
	\end{itemize}
\end{proof}

\begin{example}
	The permutation
	\[
		\begin{pmatrix}
			1 & 2 & 3 & 4 & 5 & 6 & 7 \\
			6 & 3 & 2 & 1 & 5 & 4 & 7
		\end{pmatrix}
	\]
	can be written as $(2 \ 3) \circ (1 \ 6 \ 4) \circ (5) \circ (7)$.
\end{example}

\begin{proposition}\label{ref:permutationIsProductOfDisjointCycles}
	Let $n \ge 1$. Any $\sigma \in S_n$ can be written as a composition of disjoint cycles, which is unique up to rearrangement of cycles and shifts within cycles.
\end{proposition}

\begin{proof}
	TODO.
\end{proof}

\begin{lemma}\label{lem:cycleIsProductOfTranspositions}
	Let $\sigma = (i_1 \ \dots \ i_k)$ be a $k$-cycle in $S_n$, $1 \le k \le n$. Then $\sigma$ can be written in one of the following forms:
	\begin{enumerate}
		\item $\sigma = (i_1 \ \dots \ i_k) = (i_1 \ i_2) \circ (i_2 \ i_3) \circ \dots \circ (i_{k - 1} \ i_k)$
		\item $\sigma = (i_1 \ \dots \ i_k) = (i_1 \ i_k) \circ (i_1 \ i_{k - 1}) \circ \dots \circ (i_1 \ i_2)$
	\end{enumerate}
\end{lemma}

\begin{proof}
	TODO.
\end{proof}

\begin{remark}
	Often, when it is clearly what $n$ is, $1$-cycles are omitted when writing cycles. For example, $(1 \ 3 \ 5)(2)(4)$ in $S_5$ can be written as $(1 \ 3 \ 5)$.
\end{remark}

\begin{example}
	Let
	\[
		\sigma = \begin{bmatrix}
			1 & 2 & 3 & 4 & 5 \\
			4 & 3 & 1 & 5 & 2
		\end{bmatrix},
		\quad
		\tau = \begin{bmatrix}
			1 & 2 & 3 & 4 & 5 \\
			2 & 3 & 4 & 1 & 5
		\end{bmatrix}
	\]
	To determine $\sigma \circ \tau$, write $\tau$ on top of $\sigma$, but rearranging the columns of $\sigma$ to line up with the columns of $\tau$:
	\[
		\begin{bmatrix}
			1 & 2 & 3 & 4 & 5 \\
			2 & 3 & 4 & 1 & 5 \\
			2 & 3 & 4 & 1 & 5 \\
			3 & 1 & 5 & 4 & 2
		\end{bmatrix}
	\]
	so
	\[
		\sigma \circ \tau = \begin{bmatrix}
			1 & 2 & 3 & 4 & 5 \\
			3 & 1 & 5 & 4 & 2
		\end{bmatrix}
	\]
\end{example}

\begin{example}
	Let $\sigma = (1 \ 2)$ and $\tau = (1 \ 3)$ in $S_3$, then
	\[
		\sigma \circ \tau = (1 \ 2) \circ (1 \ 3) = (1 \ 3 \ 2)
	\]
\end{example}

\begin{example}
	$(1 \ 2 \ 3) = (1 \ 2) \circ (2 \ 3) = (1 \ 3) \circ (1 \ 2)$.
\end{example}

\begin{definition}
	Let $G$ be a group and $g_1, g_2 \in G^2$. $g_1$ and $g_2$ are called \textbf{conjugate} in $G$ to each other if
	\[
		\exists h \in G, \quad h g_1 h^{-1} = g_2
	\]
\end{definition}

\begin{lemma}
	Let $n \ge 2$ and $G = S_n$. Every conjugate of a transposition in $G$ is also a transposition.
\end{lemma}

\begin{proof}
	By Proposition~\ref{ref:permutationIsProductOfDisjointCycles}, each permutation can be expressed as a composition of disjoint cycles, and by Lemma~\ref{lem:cycleIsProductOfTranspositions}, each cycle is a product of transpositions. So we just need to show that the conjugate of a transposition $(a \ b)$ by another transposition $(c \ d)$ is also a transposition. There are three cases:
	\begin{enumerate}
		\item If $\{ a, b \}$ and $\{ c, d \}$ are disjoint then $(a \ b)$ and $(c \ d)$ commute, hence
		\[
			(c \ d) (a \ b) {(c \ d)}^{-1} = (a \ b)
		\]
		is a transposition.
		\item If $\{ a, b \}$ and $\{ c, d \}$ have one common element, then say the common element is $b = c$ WLOG. Then
		\[
			(b \ d) (a \ b) {(b \ d)}^{-1} = (a \ d)
		\]
		is a transposition.
		\item If $\{ a, b \} = \{ c, d \}$ then clearly
		\[
			(c \ d) (a \ b) {(c \ d)}^{-1} = (a \ b)
		\]
		is a transposition.
	\end{enumerate}
\end{proof}

\begin{example}
	(Problems class) Find a normal subgroup of $S_3$.
	
	$|S_3| = 3! = 6$ so look for a subgroup of size $3$ (this will be a normal subgroup). Let $\sigma = (1 \ 2 \ 3)$ and let
	\[
		H = \langle \sigma \rangle = \{ e, \sigma, \sigma \circ \sigma \}
	\]
	$H$ is a group of order $3$ because $\sigma \circ \sigma \circ \sigma = e \ne \sigma \circ \sigma$ (for example, $(\sigma \circ \sigma)(1) = 3 \ne 1 = e(1)$). $|H| = 3$ so $H \triangleleft S_3$.
\end{example}

\begin{example}
	(Problems class) Let $G$ be a group and $H, K \le G$. Prove that $H \cap K \le G$.

	Using the criteria for subgroups, we check:
	\begin{enumerate}
		\item $H \cap K \ne \emptyset$: since $H$ and $K$ are subgroups, they both contain $e$, so $e \in H \cap K$.
		\item if $x, y \in {(H \cap K)}^2$ then $x \circ y \in H \cap K$: if $x, y \in H^2$ and $x, y \in K^2$, $x \circ y \in H$ and $x \circ y \in K$ so $x \circ y \in H \cap K$.
		\item if $x \in H \cap K$, then $x^{-1} \in H \cap K$: if $x \in H$ and $x \in K$, then $x^{-1} \in H$ and $x^{-1} \in K$ so $x^{-1} \in H \cap K$.
	\end{enumerate}
\end{example}

\begin{example}
	Let $G$ be a group, $H \triangleleft G$ and $K \triangleleft G$. Prove that $H \cap K \triangleleft G$.

	$H \cap K$ is normal iff for every $g \in G, x \in H \cap K$, $gx g^{-1} \in H \cap K$. Let $g \in G, x \in H \cap K$. Since $x \in H$ and $H \triangleleft G$, $g x g^{-1} \in H$. Similarly, $g x g^{-1} \in K$. So $g x g^{-1} \in H \cap K$.
\end{example}

\begin{example}
	(Problems class) Given a group $G$, define
	\[
		\text{Tor}(G) := \{ x \in G: \text{ord}(x) < \infty \}
	\]
	so $x \in \text{Tor}(G)$ iff $\exists d \ge 1, x^d = e$. Let $x, y \in {(\text{Tor}(G))}^2$, such that $x \circ y = y \circ x$. Prove that $x \circ y \in \text{Tor}(G)$ and $\text{ord}(x \circ y) \le \text{lcm}(\text{ord}(x), \text{ord}(y))$.

	Let $k \ge 1$. We use induction to show that
	\[
		{(x \circ y)}^k = x^k \circ y^k
	\]
	For $k = 1$, this is trivial. ${(x \circ y)}^{k + 1} = {(x \circ y)}^k \circ (x \circ y) = x^k \circ y^k \circ x \circ y$. By the induction step, $y^k \circ x = x \circ y^k$. So
	\[
		x^k \circ y^k \circ x \circ y = x^k \circ x \circ y^k \circ y = x^{k + 1} \circ y^{k + 1}
	\]
	Let $\text{ord}(x) = a$ and $\text{ord}(y) = b$, then let $d = \text{lcm}(a, b)$ ($d$ which $a$ and $b$ divide works). Then
	\[
		{(x \circ y)}^d = x^d \circ y^d = {(x^a)}^b \circ {(y^b)}^a = e^b \circ e^a = e
	\]
	So $\text{ord}(xy) \le \text{lcm}(\text{ord}(x), \text{ord}(y)) < \infty$.
\end{example}

TODO: 3.19, 3.20, 3.21, 3.22

\subsection{Even permutations and alternating groups}

\begin{definition}
	For $n \ge 2$, let $A_n$ be the subgroup of $S_n$ which contains the even permutations. $A_n$ is called the \textbf{alternating group}.
\end{definition}

\begin{example}
	For $n = 3$, $(1 \ 2 \ 3) = (1 \ 2) \circ (2 \ 3) \in A_3$. Also, $\text{sgn}(e) = 1$ so $e \in A_3$.
\end{example}

\begin{lemma}
	$A_n \le S_n$ for every $n \ge 2$.
\end{lemma}

\begin{proof}
	Using the subgroup criteria,
	\begin{enumerate}
		\item $e \in A_n$ so $A_n \ne \emptyset$.
		\item For every $\sigma_1, \sigma_2 \in {(A_n)}^2$,
		\[
			\begin{aligned}
				\sigma_1 & = \tau_1 \circ \tau_2 \circ \cdots \circ \tau_{2r} \\
				\sigma_2 & = \tau_1' \circ \tau_2' \circ \cdots \circ \tau_{2s}'
			\end{aligned}
		\]
		where $r \ge 0$, $s \ge 0$, and the $\tau_i$ and $\tau_i'$ are transpositions. So
		\[
			\sigma_1 \circ \sigma_2 = \tau_1 \circ \tau_2 \circ \cdots \circ \tau_{2r} \circ \tau_1' \circ \tau_2' \circ \cdots \circ \tau_{2s}'
		\]
		so $\sigma_1 \circ \sigma_2$ has even parity, so $\text{sgn}(\sigma_1 \circ \sigma_2) = 1 \Longrightarrow \sigma_1 \circ \sigma_2 \in A_n$.
		\item For every $\sigma \in A_n$,
		\[
			\sigma = \tau_1 \circ \tau_2 \circ \cdots \circ \tau_{2r}
		\]
		where $r \ge 0$, so
		\[
			\sigma^{-1} = {(\tau_1 \circ \tau_2 \circ \cdots \circ \tau_{2r})}^{-1} = \tau_{2r} \circ \cdots \circ \tau_1
		\]
		hence $\text{sgn}(\sigma^{-1}) = 1$ so $\sigma^{-1} \in A_n$.
	\end{enumerate}
\end{proof}

\begin{example}
	$\{ \sigma \in S_n: \text{sgn}(\sigma) = -1 \}$ is not a group since it does not contain $e$. But if $\tau \in S_n$ is a transposition,
	\[
		\tau A_n = \{ \tau \circ \sigma: \sigma = \tau_1 \circ \tau_2 \circ \cdots \circ \tau_{2r}, r \ge 0 \} = \{ \sigma \in S_n: \text{sgn}(\sigma) = -1 \}
	\]
	So $S_n = \tau A_n \cup A_n$, and $\tau A_n \cap A_n = \emptyset$.
\end{example}

\begin{proposition}
	\hfill
	\begin{enumerate}
		\item $|A_n| = |S_n| / 2 = n! / 2$.
		\item $A_n \triangleleft S_n$.
		\item $A_n$ is generated by $3$-cycles, for $n \ge 3$.
	\end{enumerate}
\end{proposition}

\begin{proof}
	\hfill
	\begin{enumerate}
		\item By the above example and the properties of cosets, $|S_n| = |\tau A_n| + |A_n| = 2 |A_n|$ for $t \in S_n$ a transposition.
		\item If $H \le G$ for a finite group $G$, then $|H| = 2 |H| \Longrightarrow H \triangleleft G$. So let $G = S_n$, $H = A_n$, then $A_n \triangleleft S_n$ by 1.
		\item Given any $\tau_1, \tau_2$ transpositions, $\tau_1 \circ \tau_2$ is either $0$, $1$ or $2$ $3$-cycles composed together. There are three cases:
		\begin{enumerate}
			\item $\tau_1 = \tau_2 = (i \ j)$, then $\tau_1 \circ \tau_2 = e$.
			\item $\tau_1 = (i \ j)$, $\tau_2 = (i \ k)$, where $j \ne k$, then $\tau_1 \circ \tau_2 = (j \ i \ k)$.
			\item $\tau_1 = (i \ j)$, $\tau_2 = (k \ l)$, where $\{ i, j \{ \cap \{ k, l \} = \emptyset$. Then
			\[
				\tau_1 \circ \tau_2 = (i \ j) \circ ((i \ k) \circ (k \ i)) \circ (k \ l) = ((i \ j) \circ (i \ k)) \circ ((k \ i) \circ (k \ l))
			\]
			and from case 2, this gives $2$ $3$-cycles.
		\end{enumerate}
		Since every $\sigma \in A_n$ can be written as
		\[
			\sigma = \alpha_1 \circ \cdots \circ \alpha_k
		\]
		where the $\alpha_i$ are pairs of transpositions, $\sigma$ can be written as a composition of $3$-cycles.
	\end{enumerate}
\end{proof}

\subsection{Dihedral groups}

TODO: 3.25, 3.26, 3.27, 3.28, 3.29

\begin{definition}
	The \textbf{dihedral group} of order $2n$, $D_n$, is the group generated by rotations $r$ by $2 \pi / n$ anticlockwise and reflections $s$ about a fixed axis, where $r^n = e$, $s^2 = e$, $s r s = r^{-1}$.
\end{definition}

\begin{proposition}
	Every $x \in D_n$ can be uniquely written as
	\[
		r^a s^b, \quad 0 \le a \le n - 1, 0 \le b \le 1
	\]
	In particular, $|D_n| = 2n$.
\end{proposition}

\begin{proof}
	If $x \in D_n$, then
	\[
		x = r^{a_1} s^{b_1} r^{a_2} s^{b_2} \cdots r^{a_k} b^{b_k}
	\]
	where $a_i \ge 0, b_j \ge 0$ and $\forall i \in \{ 1, \dots, k - 1\}, b_i \ge 1$ and $\forall i \in \{ 2, \dots, k \}, a_i \ge 1$. Suppose $k$ is minimal (so this is the shortest representation). We claim that $k = 1$.

	If $k \ne 1$, we can shorten a representation with $k \ge 2$ factors as
	\[
		y = r^{a_1} s^{b_1} r^{a_2} s^{b_2} \cdots r^{a_{k - 2}} b^{b_{k - 2}} \Longrightarrow x = y r^{a_{k - 1}} s^{b_{k - 1}} r^{a_k} s^{b_k}
	\]
	Note that we can assume that $0 \le a_i \le n - 1$ and $0 \le b_i \le 1$ for every $i$, since if $a_i > n$, then $r_{a_i} = r^{a_i - n} \circ r^n = r^{a_i - n} \circ e = r^{a_i - n}$, and let $a_i = k_i n + u_i$, for $0 \le u_i < n$, then $r^{a_i} = r^{k_i n + u_i} = {(r^n)}^{k_i} \circ r^{u_i} = r^{u_i}$. Similarly, if $b_i = 2 l_i + v_i$, for $0 \le v_i < 2$, then $s^{b_i} = {(s^2)}^{l_i} \circ s^{v_i} = s^{v_i}$.

	Hence $b_{k - 1} = 1$ and $x = y \circ r^{a_{k - 1}} \circ (s \circ r^{a_k}) \circ s^{b_k}$. Now $s \circ r^{a_k} s = r^{-a_k} \Longrightarrow s \circ r^{a_k} = r^{-a_k} s$, and so
	\[
		\begin{aligned}
			x
				& = y \circ r^{a_{k - 1}} \circ (r^{-a_k} \circ s) \circ s^{b_k} \\
				& = y \circ r^{a_{k - 1} - a_k} \circ s^{1 + b_k} \\
				& = y \circ r^{a_{k - 1}'} \circ s^{b_{k - 1}'}
		\end{aligned}
	\]
	where $a_{k - 1}' = a_{k - 1} - a_k$, $b_{k - 1}' = 1 + b_k$. This representation has $k - 1$ terms $r^{a_i} s^{b_i}$, contradicting the minimality of $k$. Hence $k = 1$.

	To prove the uniqueness, TODO.
\end{proof}

\subsection{Homomorphisms of Groups}

\begin{definition}
	Let $(G_1, \circ_1), (G_2, \circ_2)$. A map $\phi: G_1 \rightarrow G_2$ is a \textbf{group homomorphism} if
	\[
		\forall g, h \in G^2, \quad \phi(g \circ_1 h) = \phi(g) \circ_2 \phi(h)
	\]
\end{definition}

\begin{definition}
	A group homomorphism $\phi$ is a \textbf{isomorphism} if it is also a bijection.
\end{definition}

\begin{definition}
	Groups $G_1$ and $G_2$ are called \textbf{isomorphic} if there exists an isomorphism from $G_1$ to $G_2$. We write $G_1 \cong G_2$.
\end{definition}

\begin{proposition}
	Properties of a homomorphism $\phi: G_1 \rightarrow G_2$:
	\begin{enumerate}
		\item For $e_{G_1}$ the identity in $G_1$ and $e_{G_2}$ the identity in $G_2$, $\phi(e_{G_1}) = e_{G_2}$.
		\item $\forall g \in G_1, \phi(g^{-1}) = \phi(g)^{-1}$.
		\item If $G_1 = \langle \{ g_1, \dots, g_k \} \rangle$ then $\phi$ is determined by $\phi(g_1), \dots, \phi(g_k)$. In particular, if $G_1 = \langle g \rangle = \{ g^k: k \in \mathbb{Z} \}$, then $\phi: G_1 \rightarrow G_2$ gives the image $\{ \phi(g)^k: k \in \mathbb{Z} \}$.
	\end{enumerate}
\end{proposition}

\begin{proof}
	\hfill
	\begin{enumerate}
		\item $\forall g \in G_1, \phi(g) = \phi(g \circ_1 e_{G_1}) = \phi(g) \circ_2 \phi(e_{G_1}) \Longrightarrow \phi(e_{G_1}) = \phi(g)^{-1} \circ_2 \phi(g) = e_{G_2}$.
		\item $e_{G_1} = g \circ_1 g^{-1} \Rightarrow e_{G_2} = \phi(g \circ_1 g^{-1})$. Hence $e_{G_2} = \phi(g) \circ_2 \phi(g^{-1}) \Longrightarrow \phi(g)^{-1} = \phi(g)^{-1} \circ_2 \phi(g) \circ_2 \phi(g^{-1}) = \phi(g^{-1})$.
		\item TODO.
	\end{enumerate}
\end{proof}

\begin{definition}
	Let $\phi: G_1 \rightarrow G_2$ be a group homomorphism. The \textbf{kernel} of $\phi$ is defined as
	\[
		\ker(\phi) := \{ g \in G_1: \phi(g) = e_{G_2} \} \subseteq G_1
	\]
\end{definition}

\begin{definition}
	Let $\phi: G_1 \rightarrow G_2$ be a group homomorphism. The \textbf{image} of $\phi$ is defined as
	\[
		\text{im}(\phi) := \{ \phi(g): g \in G_1 \} \subseteq G_2
	\]
\end{definition}

\begin{lemma}
	Let $\phi: G_1 \rightarrow G_2$ be a group homomorphism. Then $\ker{\phi} \le G_1$, $\text{im} \le G_2$ and $\ker(\phi) \triangleleft G_1$.
\end{lemma}

\begin{example}
	Let $G$ be a group and $g \in G$. Define the homomorphism $\phi: \mathbb{Z} \rightarrow G$ by
	\[
		\phi(k) := g^k
	\]
	Then $\phi(k_1 + k_2) = g^{k_1 + k_2} = g^{k_1} \circ g^{k_2} = \phi(k_1) \circ \phi(k_2)$.
	\[
		\ker(\phi) = \{ k d: k \in \mathbb{Z} \}
	\]
	where $d = \text{ord}(g)$. If $d = \infty$ then $\ker(\phi) = \{ 0 \}$, otherwise $\ker(\phi) = d \mathbb{Z}$.
	\[
		\text{im}(\phi) = \langle g \rangle
	\]
	So if $G = D_n$ and $g = s$ then $\text{ord}(g) = 2$ so $\ker(\phi) = 2 \mathbb{Z}$ and $\text{im}(\phi) = \{ e, s \}$.
\end{example}

\begin{example}
	Consider the sign map $\text{sgn}: S_n \rightarrow \{ 1, -1 \}$. If $\sigma_1, \sigma_2 \in {(S_n)}^2$, $\sigma = \tau_1 \circ \cdots \circ \tau_r$, $\sigma = \tau_1' \circ \cdots \circ \tau_s'$, where the $\tau_i$ and $\tau_i'$ are transpositions. So
	\[
		\sigma_1 \circ \sigma_2 = \tau_1 \circ \cdots \circ \tau_r \circ \tau_1' \circ \cdots \circ \tau_s'
	\]
	which gives $\text{sgn}(\sigma_1 \circ \sigma_2) = {(-1)}^{r + s} = {(-1)}^r {(-1)}^s = \text{sgn}(\sigma_1) \text{sgn}(\sigma_2)$.
	Then
	\[
		\begin{aligned}
			\ker(\text{sgn}) & = \{ \sigma \in S_n: \text{sgn}(\sigma) = 1 \} = A_n \\
			\text{im}(\text{sgn}) = \{ 1, -1 \}
		\end{aligned}
	\]
\end{example}

\begin{example}
	Let $n \ge 1$ and let the homomorphism $\phi: \mathbb{Z} \rightarrow \mathbb{Z} / n$ be defined by
	\[
		\phi(k) = \bar{k} = k \pmod{n}
	\]
	Then
	\[
		\begin{aligned}
			\ker(\phi) & = n \mathbb{Z} = \{ n k: k \in \mathbb{Z} \} \\
			\text{im}(\phi) & = \mathbb{Z} / n
		\end{aligned}
	\]
\end{example}

\begin{proposition}
	Let $\phi: G_1 \rightarrow G_2$. Then
	\begin{enumerate}
		\item $\text{im}(\phi) \le G_2$.
		\item $\ker(\phi) \le G_1$.
		\item $\ker(\phi) \triangleleft G_1$.
	\end{enumerate}
\end{proposition}

\begin{proof}
	\hfill
	\begin{enumerate}
		\item TODO.
		\item TODO.
		\item We must show that $\forall g \in G_1, h \in \ker(\phi)$, $g h g^{-1} \in \ker(\phi)$, i.e. $\phi(g h g^{-1}) = e_{G_2}$.
		\[
			\phi(g h g^{-1}) = \phi(g) \phi(h) \phi(g)^{-1} = \phi(g) \circ_2 e_{G_2} \circ_2 \phi(g)^{-1} = \phi(g) \circ_2 \phi(g)^{-1} = e_{G_2}
		\]
	\end{enumerate}
\end{proof}

\subsection{Quotient groups}

\begin{definition}
	Let $G$ be a group and $H \le G$. The \textbf{quotient group} of $G$ by $H$, $G / H$ is defined as
	\[
		G / H := \{ g \circ H: g \in G \}
	\]
\end{definition}

\begin{proposition}
	If $H \triangleleft G$, then
	\begin{enumerate}
		\item $(g_1 \circ H) \circ_{G / H} (g_2 \circ H) = (g_1 \circ g_2) \circ_{G / H} H$.
		\item $(g \circ H) \circ_{G / H} (g^{-1} \circ H) = H$.
	\end{enumerate}
\end{proposition}

\begin{proof}
	\hfill
	\begin{enumerate}
		\item The set $(g_1 \circ H) \circ_{G / H} (g_2 \circ H)$ is $\{ g_1 \circ h_1 \circ g_2 \circ h_2: h_1, h_2 \in H^2 \}$. WE claim this is equal to $\{ g_1 \circ g_2 \circ h': h' \in H \}$. As these are both cosets, if they intersect then they are equal.
		
		Let $g_1 \circ h_1 \circ g_2 \circ h_2 \in (g_1 \circ H) \circ_{G / H} (g_2 \circ H)$. Note that $h_1 \circ g_2 \in H \circ g_2$. Since $H \triangleleft G$, $H \circ g_2 = g_2 \circ H$. So $\exists h' \in H, h_1 \circ g_2 = g_2 \circ h' \Longrightarrow g_1 \circ (h_1 \circ g_2) \circ h_2 = g_1 \circ (g_2 \circ h') h_2 = g_1 \circ g_2 \circ (h' \circ h_2) \in g_1 \circ g_2 \circ H$. So the cosets intersect and so are equal.
		\item $(g \circ H) \circ_{G / H} (g^{-1} \circ H) = (g \circ g^{-1}) \circ H = e \circ H = H$.
	\end{enumerate}
\end{proof}

\begin{remark}
	There is a natrual homomorphism $\phi: G \rightarrow G / H$, where $H \triangleleft G$, given by
	\[
		\phi(g) = g \circ H
	\]
\end{remark}

\begin{remark}
	The identity in $G / H$ is $H = e \circ H$ since $(g \circ H) \circ_{G / H} H = (g \circ e) \circ H = g \circ H$.
\end{remark}

\begin{example}\label{exa:quotientGroup1}
	Let $G = (\mathbb{Z}, +)$ and $H = n \mathbb{Z}$ for some $n \in \mathbb{N}$. Since $G$ is Abelian, $H \triangleleft G$. Then
	\[
		G / H = \mathbb{Z} / n
	\]
\end{example}

\begin{example}\label{exa:quotientGroup2}
	Let $G = S_n$, $H = A_n$. $A_n \triangleleft S_n$, so
	\[
		G / H = \{ A_n, \tau A_n \}
	\]
	where $\tau \in S_n$ is a transposition. The composition rule on $S_n / A_n$ gives
	\[
		\begin{aligned}
			A_n \circ A_n & = A_n \\
			A_n \circ (\tau \circ A_n) & = (e \circ \tau) \circ A_n = \tau \circ A_n \\
			(\tau \circ A_n) \circ (\tau \circ A_n) & = (\tau \circ \tau) \circ A_n = e \circ A_n = A_n
		\end{aligned}
	\]
\end{example}

\begin{example}\label{exa:quotientGroup3}
	Let $G = D_n$, $H = \langle r \rangle$. $H \triangleleft G$, so
	\[
		G / H = \{ s \circ \langle r \rangle, \langle r \rangle \}
	\]
	since
	\[
		\begin{aligned}
			D_n
				& = \{ r^i \circ s^j: 0 \le i \le n - 1, 0 \le j \le n - 1 \} \\
				& = \{ r^i: 0 \le i \le n - 1 \} \cup \{ r^i \circ s: 0 \le i \le n - 1 \} \\
				& = \{ s \circ r^i: 0 \le i \le n - 1 \}
		\end{aligned}
	\]
	since $s \circ r \circ s = r^{-1}$.
\end{example}

\begin{theorem}
	(\textbf{First isomorphism theorem (FIT) for groups}) Let $\phi: G_1 \rightarrow G_2$ be a group homomorphism. Then
	\[
		\text{im}(\phi) \cong G_1 / \ker(\phi)
	\]
\end{theorem}

\begin{proof}
	We construct an isomorphism $\tilde{\phi}: G / \ker(\phi) \rightarrow \text{im}(\phi)$ by
	\[
		\tilde{\phi}(g \circ \ker(\phi)) = \phi(g)
	\]
	We need to show that $\tilde{\phi}$ is an homomorphism. Let $g_1 \circ \ker(\phi), g_2 \circ \ker(\phi) \in {(G / \ker(\phi))}^2$.
	\[
		\begin{aligned}
			\tilde{\phi}((g_1 \circ \ker(\phi)) \circ (g_2 \circ \ker(\phi)))
				& = \tilde{\phi} (g_1 \circ g_2 \circ \ker(\phi)) \\
				& = \phi(g_1 \circ g_2) = \phi(g_1) \circ \phi(g_2) \\
				& = \tilde{\phi}(g_1 \circ \ker(\phi)) \circ \tilde{\phi}(g_2 \circ \ker(\phi))
		\end{aligned}
	\]
	Hence $\tilde{\phi}$ is an homomorphism. We now need to show that $\tilde{\phi}$ is injective, i.e. $\ker(\tilde{\phi}) = \{ \ker(\phi) \}$. Suppose $g \circ \ker(\phi) \in \ker(\phi)$, then
	\[
		\begin{aligned}
			\phi(g) & = \tilde{\phi}(g \circ \ker(\phi)) \\
			& = e_{G_2} \Longrightarrow g \in \ker(\phi) \\
			& \Longrightarrow g \circ \ker(\phi) = \ker(\phi) \\
			& \Longrightarrow \ker(\tilde{\phi}) = \{ \ker(\phi) \}
		\end{aligned}
	\]
	Finally, we need to show that $\tilde{\phi}$ is surjective. Since $x \in \text{im}(\phi) \Longleftrightarrow \exists g \in G_1, \ \phi(g) = x$, $x = \tilde{\phi}(g \circ \ker(\phi))$ hence $\tilde{\phi}$ is surjective.
\end{proof}

\begin{corollary}
	Let $G$ be a group and $g \in G$ with $\text{ord}(g) = n < \infty$. Then
	\[
		\langle g \rangle \cong (\mathbb{Z} / n, +)
	\]
	If $\text{ord}(g) = \infty$, then $\langle g \rangle \cong (\mathbb{Z}, +)$.
\end{corollary}

\begin{proof}
	Define a map $\phi: \mathbb{Z} \rightarrow \langle g \rangle$ by
	\[
		\phi(k) = g^k
	\]
	This is a homomorphism, and if $\text{ord}(g) = n < \infty$ then
	\[
		\ker(\phi) = n \mathbb{Z} \Longrightarrow \mathbb{Z} / \ker(\phi) = \mathbb{Z} / n \mathbb{Z} \cong \mathbb{Z} / n
	\]
	So by FIT for groups,
	\[
		\mathbb{Z} / n = \text{im}(\phi_n) \cong \mathbb{Z} / \ker(\phi_n) = \mathbb{Z} / n \mathbb{Z}
	\]
	$\text{im}(\phi) = \langle g \rangle$ by definition, so by FIT for groups, $\mathbb{Z} / n \cong \langle g \rangle$. The case $\text{ord}(g) = \infty$ is similar, except that $\ker(\phi) = \{ 0 \}$.
\end{proof}

\begin{corollary}
	Let $G$ be a finite group with $|G| = p$ where $p$ is prime. Then $G \cong (\mathbb{Z} / p, +)$.
\end{corollary}

\begin{proof}
	$G$ is a cyclic group since $|G|$ is prime. Thus, $G = \langle g \rangle$ where $\text{ord}(g) = p$. By the previous corollary, $G \cong (\mathbb{Z} / p, +)$.
\end{proof}

\begin{example}
	Let $\phi_n: \mathbb{Z} \rightarrow \mathbb{Z} / n$ by defined by
	\[
		\phi_n(k) = \bar{k} = \{ k + nm: m \in \mathbb{Z} \}
	\]
	Then $\ker(\phi_n) = n \mathbb{Z}$ and $\text{im}(\phi_n) = \mathbb{Z} / n$ since $\forall 0 \le j \le n - 1, \ \phi(j) = j$.
\end{example}

\begin{example}
	We have seen that $\ker(\text{sgn}) = A_n$ and $\text{im}(\text{sgn}) = \{ \pm 1 \}$. So by FIT for groups,
	\[
		\{ \pm 1 \} \cong S_n / A_n
	\]
	Also note that $\{ \pm 1 \} \cong \mathbb{Z} / 2$. So $S_n / A_n \cong \mathbb{Z} / 2$.
\end{example}

\begin{example}
	Let $\phi: D_n \rightarrow \mathbb{Z} / 2$ be defined as
	\[
		\phi(r^i s^j) = j \pmod{2}
	\]
	$\text{im}(\phi) = \mathbb{Z} / 2$ and $\ker(\phi) = \{ r^i: 0 \le i \le n - 1\} = \langle r \rangle$. So by FIT for groups,
	\[
		D_n / \langle r \rangle \cong \mathbb{Z} / 2
	\]
\end{example}

\subsection{Isomorphisms invariants}

\begin{lemma}
	Let $\phi: G_1 \rightarrow G_2$ be a group isomorphism. Then
	\begin{enumerate}
		\item If $g \in G$ then $\text{ord}_{G_1}(g) = \text{ord}_{G_2}(\phi(g))$. In particular, the sets $\{ \text{ord}_{G_1}(g): g \in G_1 \}$ and $\{ \text{ord}_{G_2}(g): g \in G_2 \}$ are equal.
		\item $|G_1| = |G_2|$.
		\item $G_1$ is Abelian iff $G_2$ is also Abelian.
		\item The sets $|H|: H \le G_1$ and $|H|: H \le G_2$ are equal.
	\end{enumerate}
\end{lemma}

\begin{proof}
	\hfill
	\begin{enumerate}
		\item Let $g \in G_1$ and let $d_1 := \text{ord}_{G_1}(g), d_2 := \text{ord}_{G_2}(\phi(g))$. Note that
		\[
			e_{G_2} = \phi(e_{G_1}) = \phi(g^{d_1}) = \phi(g)^{d_1} \Longrightarrow d_2 \le d_1
		\]
		But also,
		\[
			e_{G_2} = \phi(g)^{d_2} = \phi(g^{d_2})
		\]
		Since $\phi$ is injective, $g^{d_2} = e_{G_1}$, so $d_1 \le d_2$, hence $d_1 = d_2$.
		\item TODO.
		\item TODO.
		\item TODO.
	\end{enumerate}
\end{proof}

\begin{example}
	Give a reason why each of these pairs of groups are not isomorphic:
	\begin{enumerate}
		\item $D_4$ and $\mathbb{Z} / 4$: $|D_4| = 8$ and $|\mathbb{Z} / 4| = 4$.
		\item $S_3$ and $\mathbb{Z} / 6$: $S_3$ is not Abelian, but $\mathbb{Z} / 6$ is.
		\item $A_4$ and $D_6$: in $D_6$, $\text{ord}(r) = 6$ but in $A_4$, the permutations have orders of $1, 2$ or $3$.
	\end{enumerate}
\end{example}

\end{document}