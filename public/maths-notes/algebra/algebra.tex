\documentclass[12pt,a4paper]{article}
\AddToHook{cmd/section/before}{\clearpage}

\usepackage[a4paper, total={6in, 10in}]{geometry}
\usepackage[utf8]{inputenc}
\usepackage{amsfonts, amssymb, amsmath}
\usepackage{amsthm}
\usepackage[dvipsnames]{xcolor}

\pagecolor{white}
\color{black}% set the default colour to white

\theoremstyle{definition}
\newtheorem{definition}{Definition}[subsection]
\newtheorem{theorem}[definition]{Theorem}
\newtheorem{proposition}[definition]{Proposition}
\newtheorem{corollary}[definition]{Corollary}
\newtheorem{lemma}[definition]{Lemma}
\newtheorem{example}[definition]{Example}
\newtheorem*{remark}{Remark}

\title{Algebra II Course Notes}
\author{Isaac Holt}

\begin{document}

\maketitle

\section{Homomorphisms between Rings}

Let $R$ and $S$ be two rings. A map $f: R \rightarrow S$ is called a (ring)-homomorphism if:
\begin{enumerate}
	\item $f(1) = 1$
	\item $f(a + b) = f(a) + f(b)$
	\item $f(ab) = f(a)f(b)$
\end{enumerate}

\begin{lemma}
	$f(0) = 0$ and $f(-a) = -f(a)$
\end{lemma}

\begin{proof}
	$f(0) = f(0 + 0) = f(0) + f(0)$

	$0 = f(0) = f(a + (-a)) = f(a) + f(-a)$

	Hence $-f(a) = f(-a)$
\end{proof}

\begin{definition}
	Two rings $R$ and $S$ are \textbf{isomorphic} if there exists a bijective homomorphism between $R$ and $S$. The map between them is an \textbf{isomorphism}. We write $R \cong S$.
\end{definition}

\begin{lemma}
	A homomorphism $f: R \rightarrow S$ is injective iff $\ker f = {0}$.
\end{lemma}

\begin{proof}
	If $f$ is injective, $f(x) = f(y) \Rightarrow x = y$. Assume $f$ is injective. $\ker f = {a \in \mathbb{R}: f(a) = 0}$ so $f(a) = 0 \Rightarrow f(a) = f(0) \Rightarrow a = 0$.

	For the other direction: assume $\ker f = {0}$. $f(x) = f(y) \Rightarrow f(x) - f(y) = 0 \Rightarrow f(x) + f(-y) = 0 \Rightarrow f(x - y) = 0 \Rightarrow x - y \in \ker f$. Since $\ker f = {0}$, $x - y = 0$ and so $x = y$.
\end{proof}

\begin{definition}
	Let $R$ and $S$ be two rings.
	\begin{itemize}
		\item The \textbf{product} of $R$ and $S$ is defined as $R \times S := \{(r, s): r \in R, s \in S\}$ which is itself a ring.
		\item \textbf{Addition} is defined as $(r_1, s_1) + (r_2, s_2) := (r_1 + r_2, s_1 + s_2)$.
		\item \textbf{Multiplication} is defined as $(r_1, s_1) \cdot (r_2, s_2) := (r_1 r_2, s_1 s_2)$
		\item The multiplicative identity is $(1, 1)$.
	\end{itemize}
\end{definition}

\begin{definition}
	We have two ring homomorphisms:
	\begin{enumerate}
		\item $p_1: R \times S \rightarrow R = (r, s) \rightarrow r$
		\item $p_2: R \times S \rightarrow S = (r, s) \rightarrow s$
	\end{enumerate}

	$\ker p_1 = \{(r, s) \in R \times S: p_1((r, s)) = 0\} = \{(r, s) \in R \times S: r = 0\} = \{(0, s): s \in S\}$
\end{definition}

\begin{remark}
	Note $\ker p_1$ is not a subring of $R \times S$ since $(1, 1) \notin \ker p_1$.

	But we can consider $\ker p_1$ as a ring by taking $(0, 1)$ as the multiplicative identity.

	Then $\ker p_1 \cong S$ as we map $(0, s) \rightarrow s$.

	Similarly, $\ker p_2 \cong R$ and so $\ker p_1 \times \ker p_2 \cong S \times R \cong R \times S$.
\end{remark}

\begin{lemma}
	Let $f: R \rightarrow S$ be a ring homomorphism. Then $\ker f$ has the following two properties:
	\begin{enumerate}
		\item $\ker f$ is closed under addition.
		\item For every $r \in R$ and $x \ker f$ we have $r \cdot x \in \ker f$ and $x \cdot r \in \ker f$.
	\end{enumerate}
\end{lemma}

\begin{proof}
	\begin{enumerate}
		\item If $x, y \in \ker f$ then $f(x + y) = f(x) + f(y) = 0 + 0 = 0$. That is $x + y \in \ker f$.
		\item For every $r \in R$ and $x \ker f$, $f(r \cdot x) = f(r) \cdot f(x) = f(r) \cdot 0 = 0$. Thus $r \cdot x \in \ker f$. Similarly for $x \cdot r$.
	\end{enumerate}
\end{proof}

\begin{definition}
	Let $I$ be an ideal in a ring $R$. Then for an element $x \in R$, the \textbf{coset} of $I$ generated by $x$ to be the set $\bar{x} := x + I := \{ x + r: r \in I \} \subset R$.

	$x$ is said to be a representative of this coset.
\end{definition}

\begin{lemma}
	Let $x \in R$ and $y \in R$. Then the following statements are equivalent
	\begin{enumerate}
		\item $x + I = y + I$
		\item $x + I \cap y + I \ne \emptyset$
		\item $x - y \in I$
	\end{enumerate}
\end{lemma}

\begin{proof}
	($(1) \Rightarrow (2)$) is obvious

	($(2) \Rightarrow (3)$): if $x + I \cap y + I \ne \emptyset$, for some $r_1 \in I, r_2 \in I$, $x + r_1 = y + r_2$ and so $x - y = r_2 - r_1 \in I$.

	($(3) \Rightarrow (1)$): since $x - y \in I$, for some $r' \in I$, $x = y + r'$. Then $x + I = \{x + r: r \in I\} = \{y + r' + r: r \in I\} \subseteq y + I$ as ideals are closed under addition, and $r' + r \in I$. $y + I = \{y + r: r \in I\} = {x - r' + r: r \in I} \subseteq x + I$ and so $x + I = y + I$.
\end{proof}

Notation: $\bar{x} = \bar{y} \Leftrightarrow x + I = y + I \Leftrightarrow x \equiv y \pmod I \Leftrightarrow x - y \in I$

\begin{definition}
	$R / I := \{\bar{x}: x \in R \} = \{x + I: x \in R\}$ is the set of all distinct cosets of $R \pmod I$ 
\end{definition}

\begin{remark}
	If $R = \mathbb{Z}$ and $I = (n)$, $n \in \mathbb{N}$, $R / I = \mathbb{Z} / n = \{\bar{0}, \dots, \bar{n - 1}\}$.
\end{remark}

\begin{definition}
	\hfill\break
	\begin{itemize}
		\item Addition: $(x + I) + (y + I) = x + y + I$
		\item Multiplication: $(x + I) \cdot (y + I) = xy + I$
	\end{itemize}
\end{definition}

A coset $x + I$ has many representatives, for example $x + r$ with $r \in I$ gives the same coset, since $x + r - x = r \in I$.

Assume $x, x' \in R$ such that $x + I = x' + I$ and $y, y' \in R$ such that $y + I = y' + I$.

\begin{proof}
	\begin{itemize}
		\item Addition: $x + I = x' + I \Leftrightarrow x - x' \in I$ and similarly $y - y' \in I$. $I$ is closed under addition so $(x - x') + (y - y') \in I \Leftrightarrow (x + y) - (x' + y') \in I \Leftrightarrow x + y + I = x' + y' + I$.
		\item $x - x' \in I$ and $y - y' \in I$, so $(x - x')y \in I$ and $x(y - y') \in I$. $(x - x')y + x(y - y') = xy - x'y' \in I \Leftrightarrow xy + I = x'y' + I$.
	\end{itemize}
\end{proof}

$R / I$ with the two binary operations of addition and multiplication is a ring:
\begin{itemize}
	\item The zero element is $0 + I$ as $(x + I) + (0 + I) = x + I$.
	\item The multiplicative identity is $1 + I$.
	\item All properties follow from the corresponding properties of $R$:
	\item e.g. distributivity: $\bar{x} = x + I$, $\bar{y} = y + I$, $\bar{z} = z + I$.
	$\bar{x}(\bar{y} + \bar{z}) = \bar{x}(\overline{y + z}) = \overline{x(y + z)} = \overline{xy + xz} = \overline{xy} + \overline{xz} = \overline{x}\overline{y} + \overline{x}\overline{z}$.
\end{itemize}

\begin{definition}
	Let $R$ be a ring, and $I \subseteq R$ be an ideal of $R$. Then the ring $R / I$ is called the \textbf{quotient} of $R$ by $I$ ($R$ mod $I$). Its elements, $x + I$, $x \in R$ are called cosets (or residue classes or equivalence classes) and we denote them $\bar{x}$.
	
	$R / I$ may be commutative or non-commutative, but if $R$ is commutative, so is $R / I$.

	If $I = R$, then $R / R$ consists of a single element, since for every $x \in R$, $y \in R$, we have $x - y \in R$ and hence $x + R = y + R$.

	If $I = 0 = {0}$ is the zero ideal, if $x \in R$, $x + I = x + 0 = x$. Hence $R / I = R$.
\end{definition}

\begin{definition}
	Given $R$, $I \subseteq R$ an ideal, the \textbf{quotient map} (or \textbf{canonical homomorphism}) is defined as $\Pi: R \rightarrow R / I = x \rightarrow \overline{x} = x + I$ and is a ring hoomomorphism.

	$\ker \Pi = \{r \in R: \overline{r} = \overline{0}\} = \{r \in R: r - 0 = r \in I\} = I$.
\end{definition}

Hence, given a ring $R$ and an ideal $I \subseteq R$, there exists a ring homomorphism ($\Pi$) such that $\ker \Pi = I$.

\begin{theorem}
	(First Isomorphism Theorem - FIT) Let $\phi: R \rightarrow S$ be a ring homomorphism. The map $\bar{\phi}: R / \ker \phi \rightarrow \text{Im } \phi = \bar{x} \rightarrow \phi(x)$ is well-defined and it is a ring isomorphism: $R / \ker \phi \cong \text{Im } \phi$.
\end{theorem}

\begin{proof}
	Let $x, x' \in R$ such that $\overline{x} = \overline{x'}$, i.e. $x + \ker \phi = x' + \ker \phi$. So $x - x' \in \ker \phi$, hence $\phi(x - x') = 0 \Leftrightarrow \phi(x) - \phi(x') = 0 \Leftrightarrow \phi(x) = \phi(x')$. Hence $\overline{\phi}$ is well-defined.

	\begin{enumerate}
		\item $\overline{\phi}(\bar{1}) = \phi(1) = 1$
		\item $\overline{\phi}(\bar{x} + \bar{y}) = \overline{\phi}(\overline{x + y}) = \phi(x + y) = \phi(x) + \phi(y) = \bar{\phi}(\bar{x}) + \bar{\phi}(\bar{y})$.
		\item Similarly, $\bar{\phi}(\bar{x}\cdot \bar{y}) = \bar{\phi}(\bar{x}) \cdot \bar{\phi}(\bar{y})$.
	\end{enumerate}

	Hence $\bar{\phi}$ is a ring homomorphism.

	$\bar{\phi}(\bar{x}) = 0 \Leftrightarrow \phi(x) = 0 \Leftrightarrow x \in \ker \phi \Leftrightarrow \bar{x} = 0$, hence $\ker \bar{\phi} = \{\bar{0}\}$.
	Let $y \in \text{Im } \phi \Leftrightarrow$ for some $x \in R$, $\phi(x) = y$. Hence $\bar{\phi}(\bar{x}) = \phi(x) = y$, hence $\bar{\phi}$ is also surjective, hence it is bijective.
\end{proof}

\begin{definition}
	Let $R$ be a commutative ring. An ideal $I \subseteq R$ is a \textbf{prime ideal} if $I \ne R$ ($I$ is proper) and for every $a, b \in R$ such that $a \cdot b \in I$ then $a \in I$ or $b \in I$.

	The ideal $I \ne R$ is \textbf{maximal} if the only ideals that contain $I$ is $I$ itself and $R$. i.e. there is no ideal $J$ such that $I \subsetneq J \subsetneq R$.
\end{definition}

\begin{theorem}
	Recall $x \in R$ is prime if $0 \ne x \notin R^{\times}$ and $x | ab \Rightarrow x | a$ or $x | b$.

	If $x$ is a prime element then $(x)$ is a prime ideal.
\end{theorem}

\begin{proof}
	$ab \in (x) \Rightarrow$ for some $r \in R$, $ab = rx \Rightarrow x | ab$ so because $x$ is prime, $x | a$ or $x | b$ so $a \in (x)$ or $b \in (x)$.
\end{proof}

\begin{lemma}
	Let $(x)$ be a non-zero prime ideal. The $x$ is a prime element.
\end{lemma}

\begin{proof}
	If $x | ab$, $ab \in (x)$, so because $(x)$ is a prime ideal, $a \in (x)$ or $b \in (x)$, so $x | a$ or $x | b$.
\end{proof}

\begin{remark}
	$x | a \Leftrightarrow a \in (x) \Leftrightarrow (a) \subseteq (x)$.

	This can be described as ``to divide is to contain".
\end{remark}

\begin{corollary}
	The zero ideal $(0) = 0$ is a prime ideal iff $R$ is an integral domain, since an integral means $ab = 0 \Rightarrow a = 0 \text{ or } b = 0$.
\end{corollary}

\begin{theorem}
	Let $R$ be a commutative ring and $I \subseteq R$ an ideal.

	\begin{enumerate}
		\item $I$ is prime iff $R / I$ is an integral domain.
		\item $I$ is maximal iff $R / I$ is a field.
	\end{enumerate}
\end{theorem}

\begin{proof}
	\hfill
	\begin{enumerate}
		\item Assume $I$ is prime. Assume $\bar{a}\bar{b} = \bar{0}$ with $a, b \in R$, $\bar{a}, \bar{b} \in R / I$. $\bar{a}\bar{b} = \bar{0} \Rightarrow \overline{ab} = \bar{0} \Rightarrow ab \in I \Rightarrow a \in I \text{ or } b \in I \Rightarrow \bar{a} = \bar{0} \text{ or } \bar{b} = {0}$, hence $R / I$ is an integral domain.
		
		Now assume $R / I$ is an integral domain. $ab \in I \Rightarrow \overline{ab} = \bar{0}$. Since $R / I$ is an integral domain, $\bar{a} = \bar{0}$ or $\bar{b} = \bar{0} \Rightarrow a \in I \text{ or } b \in I$.

		\item ($\Rightarrow$): Assume that $I$ is maximal. Let $\bar{x} \ne \bar{0}$, $\bar{x} \in R / I$, then $x \in R$ with $x \notin I$. Consider $(I, x) := \{ r + r'x: r \in I, r' \in R \}$. This is an ideal, as $r_1 + r_1' x + r_2 + r_2'x = (r_1 + r_2) + (r_1' + r_2')x \in R$, and $r'' (r + r'x) = r''r + r''r'x \in R$.
		
		$I \subsetneq (I, x) \subseteq R$. $I$ is maximal so $(I, x) = R \Rightarrow 1 \in (I, x)$. Hence for some $y \in R$, $yx + m = 1$ for some $m \in I$.

		Hence $yx - 1 \in I \Rightarrow \overline{yx} = \bar{y}\bar{x} = \overline{1}$ hence $\bar{x}$ is invertible, so $R / I$ is a field.

		($\Leftarrow$): Assume $R / I$ is a field. If $\bar{0} \ne \bar{x} \in R / I$, then for some $y \in R / I$, $\bar{x} \bar{y} = 1 \Rightarrow xy - 1 \in I \Rightarrow xy = 1 + m$ for some $m \in I$. That is, $1 = xy - m$ hence $1 \in (I, x) \Rightarrow (I, x) = R$.

		Now let $J$ be an ideal such that $I \subsetneq J \subseteq R$. Since $I \subsetneq J$, for some $x \in J$, $x \notin I$. Then $I \subsetneq (I, x) \subseteq J \subseteq R$. But $(I, x) = R$, hence $J = R$. Hence there is no ideal $J$ such that $I \subsetneq J \subsetneq R$, hence $I$ is maximal.
	\end{enumerate}
\end{proof}

\begin{corollary}
	If $I$ is maximal then $I$ is prime.
\end{corollary}

\begin{proof}
	$I$ is maximal $\Rightarrow R / I$ is a field $\Rightarrow R / I$ is an integral domain $\Rightarrow I$ is a prime ideal.
\end{proof}

\subsection{Principal Ideal Domains (PIDs)}

\begin{example}
	Let $a, b \in \mathbb{Z}$. Then let $d = (a, b) = \gcd(a, b)$. $(a, b) \subseteq (d)$ since $d | a$ and $d | b \Leftrightarrow a = d r_1$ and $b = d r_2$, $r_1, r_2 \in \mathbb{Z} \Rightarrow a \in (d)$ and $b \in (d)$.

	Moreover, for some $r_1, r_2 \in \mathbb{Z}$, $d = r_1 + r_2 b \Rightarrow d \in (a, b) \Rightarrow (d) \subseteq (a, b)$.


	The same argument holds for $F[x]$ with $F$ a field.

	i.e. $(f(x), g(x)) = (\gcd(f(x), g(x)))$.
\end{example}

\begin{definition}
	An integral domain in which \textbf{all} ideals are principle is called a \textbf{principle ideal domain (PID)}.
\end{definition}

\begin{theorem}
	Let $R$ be a either $\mathbb{Z}$ or $F[x]$ with $F$ a field. Then $R$ is a PID.
\end{theorem}

\begin{proof}
	Define the following ``degree" function $d: R \backslash \{0\} \rightarrow \mathbb{N}$ by
	\[
		d(a) := \begin{cases}
			|a| & \text{ if } a \in \mathbb{Z} \\
			\deg(a) & \text{ if } a \in F[x]
		\end{cases}
	\]

	By division, for every $a, m \in R \backslash \{0\}$, we can find unique $q, R \in R$ such that $a = qm + r$ with $r = 0$ of $d(r) < d(m)$.

	Let $I \subseteq R$ be an ideal. If $I = 0 = \{0\}$ we are done. So now let $I \ne 0$. Let $0 \ne m \in I$ such that $d(m)$ is minimal among elements of $I$. We claim that $I = (m)$.

	Let $a \in I$. $a \in (m) \Leftrightarrow m | a$. Dividing $a$ by $m$, we get $a = qm + r$, with $r = 0$ or $d(r) < d(m)$. But since $r = a - qm \in I$, $d(r) < d(m)$ would contradict the minimality of $d(m)$. Hence $r = 0$, so $m | a \Leftrightarrow a \in (m)$. $(m) \subseteq I$ so $a \in I \Leftrightarrow a \in (m)$.
\end{proof}

\begin{theorem}
	(Stated without proof) Any PID is a UFD.
\end{theorem}

\begin{remark}
	There are integral domains which are not PIDs, e.g. $\mathbb{Z}[\sqrt{-5}]$ which is not a UFD and hence not a PID.
\end{remark}

\begin{proposition}
	Let $R$ be a PID and $a, b \in R$. Then $\gcd(a, b)$ exists and $(a, b) = (\gcd(a, b))$.
\end{proposition}

\begin{proof}
	Since $R$ is a PID, for some $d \in R$, $(a, b) = (d)$. We claim that $d = \gcd(a, b)$.

	$(a, b) = (d) \Rightarrow a \in (d) \text{ and } b \in (d) \Rightarrow d | a \text{ and } d | b$. Suppose $e \in R$ such that $e | a \Rightarrow a \in (e)$ and $e | b \Rightarrow b \in (e)$. $(d) = (a, b) \subseteq (e) \Rightarrow e | d$. Therefore $d = \gcd(a, b)$.
\end{proof}

\begin{theorem}
	(Stated without proof): $\mathbb{Z}[i], \mathbb{Z}[\pm \sqrt{2}]$ are PID's.
\end{theorem}

\begin{lemma}
	Let $R$ be a PID and let $a \in R$ be irreducible. Then the principle ideal genereated by $a$ is a maximal ideal.
\end{lemma}

\begin{proof}
	Suppose $(a) \subseteq I$, with $I$ an ideal. We must show $I = (a)$ or $I = R$. Since $R$ is a PID, for some $t \in R$, $I = (t)$. So $(a) \subseteq (t)$ so for some $m \in R$, $a = t m$. But $a$ is irreducible, so either $t$ is a unit or $m$ is a unit.

	If $t \in R^{\times}$ then $I = (t) = R$. If $m \in R^{\times}$ then $(a) = (t) = I$ (last question of assignment 3).
\end{proof}

\subsection{Fields on quotients}

\begin{theorem}
	Let $F$ be a field and $f(x) \in F[x]$, with $f(x)$ irreducible. Then $F[x] / (f(x))$ is a field and a vector space over $F$ with basis

	\[ B := \{\bar{1}, \bar{x}, \bar{x}^2, \ldots, \bar{x}^{n - 1} \} \] where $n = \deg f$.

	That is, every element of $F[x] / (f(x))$ can be uniquely written as

	\[ \overline{a_0 1 + a_1 x + \cdots + a_{n - 1} x^{n - 1}} \]
\end{theorem}

\begin{proof}
	Since $f(x)$ is irreducible, $F[x] / (f(x))$ is a field. $F[x]/(f(x))$ is a vector space over $F$ and an abelian group with respect to addition and scalar multiplication with elements of $F$: if $\overline{g(x)} \in F[x] / (f(x))$ and $\alpha \in F$ then $\alpha \overline{g(x)} = \overline{\alpha g(x)} \in F[x] / (f(x))$.

	We must prove $B$ spans $F[x] / (f(x))$. For every $\overline{g(x)} \in F[x] / (f(x))$, $g(x) = q(x) f(x) + r(x)$ with $\deg(r) < \deg(f) = n \Rightarrow g(x) - r(x) = q(x) f(x) \in (f(x)) \Rightarrow \overline{g(x)} = \overline{r(x)}$, $\deg(r) < n$. Hence $\overline{g(x)} = \overline{r(x)} = a_0 + a_1 \bar{x} + \cdots + a_{n - 1} \bar{x}^{n - 1}$ with $a_i \in F$. Hence $B$ spans $F[x] / (f(x))$.

	We must show $B$ is linearly independent over F, i.e. show if $\sum_{i = 0}^{n - 1} a_i \bar{x}^i = \bar{0}$ then $\forall i, a_i = 0$.

	$\sum_{i = 0}^{n - 1} a_i \bar{x}^i = \bar{0} \Leftrightarrow \sum_{i = 0}^{n - 1} a_i x^i \in (f(x)) \Rightarrow f(x) | \sum_{i = 0}^{n - 1} a_i x^i$. But $\deg(f) = n$ and $\deg(\sum_{i = 0}^{n - 1} a_i x^i) < n$ so $\sum_{i = 0}^{n - 1} a_i x^i$ is the zero polynomial so $\forall i, a_i = 0$. Therefore $B$ is linearly independent.

	So $B$ is a basis.
\end{proof}

\section{Finite fields}

\begin{theorem}
	For every prime $p$ and $n \in \mathbb{N}$, for some irreducible polynomial $f(x) \in (\mathbb{Z} / p)[x]$, $\deg(f) = n$. Thus $(\mathbb{Z} / p)[x] / (f(x))$ is a field with $p^n$ elements (since there are $p$ choices for each $a_i$ in $a_0 + a_1 \bar{x} + \cdots + a_{n - 1}\bar{x}^{n - 1}$).

	Any two such fields are isomorphic and we denote the unique, up to isomorphism, field with $p^n$ elements with $\mathbb{F}_{p^n}$.
\end{theorem}

\begin{proof}
	Not examinable.
\end{proof}

\begin{remark}
	If $n = 1$ then $\mathbb{F}_p \cong \mathbb{Z} / p$ with $p$ prime. However if $n > 1$ then $\mathbb{F}_{p^n} \not\cong \mathbb{Z} / p^n$ since $\mathbb{Z} / p^n$ is not a field.
\end{remark}

\begin{example}
	Find an irreducible polynomial $f$ in $(\mathbb{Z} / 3)[x]$ of degree $3$.
	
	$f(x) = x^3 + x^2 + x + \bar{2}$. This has no roots in $\mathbb{Z} / 3$ so $f(x)$ is irreducible since $\deg(f) = 3$. Then $\mathbb{F}_{27} = \mathbb{F}_{3^3} \cong (\mathbb{Z} / 3) [x] / (f(x))$. All elements can be written as $a_0 + a_1 \bar{x} + a_2 \bar{x}^2$, $a_i \in \mathbb{Z} / 3$.

	$\overline{f(x)} = \bar{0} = \overline{x^3 + x^2 + x + \bar{2}} \Rightarrow \bar{x}^3 = - \bar{x}^2 - \bar{x} - \bar{2}$.
\end{example}

\subsection{The Chinese Remainder Theorem (CRT)}

\begin{definition}
	Let $a, b \in R$. $a$ and $b$ are \textbf{coprime} if $\not\exists r$ irreducible in $R$ such that $r | a$ and $r | b$.
\end{definition}

\begin{lemma}
	Let $R$ be a PID and $a, b \in R$ be coprime. Then $(a, b) = R$ and hence $\exists x, y \in R$ such that $xa + yb = 1$.
\end{lemma}

\begin{proof}
	Since $R$ is a PID, $(a, b) = (r)$ for some $r \in R$. So $a, b \in (r) \Rightarrow r | a \text{ and } r | b$. So $a = rn$ and $b = rm$ for some $n, m \in R$. $r$ must be a unit in $R$ since otherwise, $r = p_1 \cdots p_k$ for some $p_i$ irreducible, but then $a = p_1 \cdots p_k n$, $b = p_k \cdot p_k m$, which would contradict $a$ and $b$ being coprime.

	So $r \in R^{\times} \Rightarrow (r) = R \Rightarrow (a, b) = R$.
\end{proof}

\begin{corollary}
	For $a, b \in R$ coprime, any $\gcd(a, b) \in R^{\times}$.
\end{corollary}

\begin{proof}
	In a PID, $(a, b) = (\gcd(a, b))$. By the lemma above, if $a$ and $b$ are coprime, $(a, b) = R \Rightarrow (\gcd(a, b)) = R = (1) \Rightarrow \gcd(a, b) \in R^{\times}$.
\end{proof}

\begin{theorem}
	(CRT for PID's) Let $R$ be a PID and let $a_1, \ldots, a_k \in R$ be pairwise coprime elements. Then the map from $R / (a_1, \ldots, a_k) \rightarrow R / (a_1) \times \cdots \times R / (a_k)$ given by $r + (a_1, \dots, a_k) \rightarrow (r + (a_1), \dots, r + (a_k))$ is a ring isomorphism.
\end{theorem}

\begin{proof}
	Let $\psi: R \rightarrow R / (a_1) \times \cdots \times R / (a_k)$, $\psi(r) = (r + (a_1), \dots, r + (a_k))$. Clearly, $\psi$ is a ring homomorphism. 
	
	For every $i = 1, 2, \dots, k$, the elements $a_i$ and $a_1 \dots a_{i - 1} a_{i + 1} \dots a_k$ are coprime. (If not, there exists an irreducible $p$ such that $p | a_i$ and $p | a_1 \dots a_{i - 1} a_{i + 1} \dots a_k$. But then $p \text{irreducible} \Leftrightarrow p \text{ prime}$ hence $p | a_j$ for some $j \ne i$, but this contradicts that $a_i$ and $a_j$ are coprime).

	By the above lemma, for some $x_i, y_i \in R$, $x_i a_i + y_i (a_1 \dots a_{i - 1} a_{i + 1} \dots a_k) = 1$. Set $e_i := 1 - a_i x_i$ for each $i = 1, \dots, k$. Then $e_i = 1 + (a_i)$ and $e_i = 0 + (a_j)$ for $j \ne i$, since $e_i = 1 - a_i x_i = y_i (a_1 \dots a_{i - 1} a_{i + 1} \dots a_k)$.

	Let $(r_1 + (a_1), \dots, r_k + (a_k))$ be any element in $R / (a_1) \times \cdots \times R / (a_k)$. We claim that

	\[ \psi \left( \sum_{i = 1}^k r_i e_i \right) = (r_1 + (a_1), \dots, r_k + (a_k)) \]

	\[ \psi \left( \sum_{i = 1}^k r_i e_i \right) = \sum_{i = 1}^k \psi(r_i e_i) = \sum_{i = 1}^k \psi(r_i) \psi(e_i) \]

	\[ \psi(e_1) = (0 + (a_1), \dots, 1 + (a_i), 0 + (a_{i + 1}), \dots, 0 + (a_k)) \]
	since $e_i = 1 + (a_i)$ and $e_i = 0 + (a_j)$ for $j \ne i$ and

	\[ \psi(r_i) = (r_i + (a_1), \dots r_i + (a_k)) \] so

	\[ \psi(e_i) \psi(r_i) = TODO finish and check this proof \]
	Thus $\psi$ is surjective.
	$\ker \psi = \{ r \in R: r \in (a_i), i = 1, \dots, k \} = \{ r \in R: a_i | r, i = 1, \dots, k \} = \{ r \in R: a_1 \dots a_k | r \}$ since $a_i$ and $a_j$ are coprime.
	$\ker \psi = (a_1 a_2 \dots a_k)$. Then by the FIT, $R / \ker \psi \cong R / (a_1) \times \cdots \times R / (a_k)$.
\end{proof}

\end{document}